%% Generated by Sphinx.
\def\sphinxdocclass{jupyterBook}
\documentclass[letterpaper,10pt,english]{jupyterBook}
\ifdefined\pdfpxdimen
   \let\sphinxpxdimen\pdfpxdimen\else\newdimen\sphinxpxdimen
\fi \sphinxpxdimen=.75bp\relax
\ifdefined\pdfimageresolution
    \pdfimageresolution= \numexpr \dimexpr1in\relax/\sphinxpxdimen\relax
\fi
%% let collapsible pdf bookmarks panel have high depth per default
\PassOptionsToPackage{bookmarksdepth=5}{hyperref}
%% turn off hyperref patch of \index as sphinx.xdy xindy module takes care of
%% suitable \hyperpage mark-up, working around hyperref-xindy incompatibility
\PassOptionsToPackage{hyperindex=false}{hyperref}
%% memoir class requires extra handling
\makeatletter\@ifclassloaded{memoir}
{\ifdefined\memhyperindexfalse\memhyperindexfalse\fi}{}\makeatother

\PassOptionsToPackage{booktabs}{sphinx}
\PassOptionsToPackage{colorrows}{sphinx}

\PassOptionsToPackage{warn}{textcomp}

\catcode`^^^^00a0\active\protected\def^^^^00a0{\leavevmode\nobreak\ }
\usepackage{cmap}
\usepackage{fontspec}
\defaultfontfeatures[\rmfamily,\sffamily,\ttfamily]{}
\usepackage{amsmath,amssymb,amstext}
\usepackage{polyglossia}
\setmainlanguage{english}



\setmainfont{FreeSerif}[
  Extension      = .otf,
  UprightFont    = *,
  ItalicFont     = *Italic,
  BoldFont       = *Bold,
  BoldItalicFont = *BoldItalic
]
\setsansfont{FreeSans}[
  Extension      = .otf,
  UprightFont    = *,
  ItalicFont     = *Oblique,
  BoldFont       = *Bold,
  BoldItalicFont = *BoldOblique,
]
\setmonofont{FreeMono}[
  Extension      = .otf,
  UprightFont    = *,
  ItalicFont     = *Oblique,
  BoldFont       = *Bold,
  BoldItalicFont = *BoldOblique,
]



\usepackage[Bjarne]{fncychap}
\usepackage[,numfigreset=1,mathnumfig]{sphinx}

\fvset{fontsize=\small}
\usepackage{geometry}


% Include hyperref last.
\usepackage{hyperref}
% Fix anchor placement for figures with captions.
\usepackage{hypcap}% it must be loaded after hyperref.
% Set up styles of URL: it should be placed after hyperref.
\urlstyle{same}

\addto\captionsenglish{\renewcommand{\contentsname}{Preliminaries}}

\usepackage{sphinxmessages}



        % Start of preamble defined in sphinx-jupyterbook-latex %
         \usepackage[Latin,Greek]{ucharclasses}
        \usepackage{unicode-math}
        % fixing title of the toc
        \addto\captionsenglish{\renewcommand{\contentsname}{Contents}}
        \hypersetup{
            pdfencoding=auto,
            psdextra
        }
        % End of preamble defined in sphinx-jupyterbook-latex %
        

\title{Linear Algebra}
\date{Sep 17, 2025}
\release{}
\author{Dr Katie Steckles (based on original book by Dr Jon Shiach)}
\newcommand{\sphinxlogo}{\vbox{}}
\renewcommand{\releasename}{}
\makeindex
\begin{document}

\pagestyle{empty}
\sphinxmaketitle
\pagestyle{plain}
\sphinxtableofcontents
\pagestyle{normal}
\phantomsection\label{\detokenize{_pages/0_intro::doc}}


\sphinxAtStartPar
These are the materials that accompany the linear algebra part of the module Linear Algebra and Programming skills, as part of the first year of the Mathematics course at Manchester Metropolitan University. While these notes cover all of the material necessary for students to successfully pass the unit, it is advisable to seek out other sources of information \sphinxhyphen{} either on the internet, or in textbooks. Mathematical notation can differ between authors, and these notes have been written using notation that is commonly found elsewhere, but might have some differences.

\begin{sphinxuseclass}{sd-container-fluid}
\begin{sphinxuseclass}{sd-sphinx-override}
\begin{sphinxuseclass}{sd-mb-4}
\begin{sphinxuseclass}{text-center}
\begin{sphinxuseclass}{sd-row}
\begin{sphinxuseclass}{sd-row-cols-1}
\begin{sphinxuseclass}{sd-row-cols-xs-1}
\begin{sphinxuseclass}{sd-row-cols-sm-1}
\begin{sphinxuseclass}{sd-row-cols-md-2}
\begin{sphinxuseclass}{sd-row-cols-lg-3}
\begin{sphinxuseclass}{sd-g-3}
\begin{sphinxuseclass}{sd-g-xs-3}
\begin{sphinxuseclass}{sd-g-sm-3}
\begin{sphinxuseclass}{sd-g-md-3}
\begin{sphinxuseclass}{sd-g-lg-3}
\begin{sphinxuseclass}{sd-col}
\begin{sphinxuseclass}{sd-d-flex-row}
\begin{sphinxuseclass}{sd-col-4}
\begin{sphinxuseclass}{sd-col-xs-4}
\begin{sphinxuseclass}{sd-col-sm-4}
\begin{sphinxuseclass}{sd-col-md-4}
\begin{sphinxuseclass}{sd-col-lg-4}
\begin{sphinxuseclass}{sd-card}
\begin{sphinxuseclass}{sd-sphinx-override}
\begin{sphinxuseclass}{sd-w-100}
\begin{sphinxuseclass}{sd-shadow-sm}
\begin{sphinxuseclass}{sd-card-hover}
\begin{sphinxuseclass}{sd-card-header}
\begin{sphinxuseclass}{bg-light}
\sphinxAtStartPar
Matrices

\end{sphinxuseclass}
\end{sphinxuseclass}
\begin{sphinxuseclass}{sd-card-body}
\begin{figure}[htbp]
\centering

\noindent\sphinxincludegraphics{{0_matrix}.png}
\end{figure}

\end{sphinxuseclass}{\hyperref[\detokenize{_pages/1.0_Matrices::doc}]{\sphinxcrossref{\DUrole{doc}{Matrices}}}}
\end{sphinxuseclass}
\end{sphinxuseclass}
\end{sphinxuseclass}
\end{sphinxuseclass}
\end{sphinxuseclass}
\end{sphinxuseclass}
\end{sphinxuseclass}
\end{sphinxuseclass}
\end{sphinxuseclass}
\end{sphinxuseclass}
\end{sphinxuseclass}
\end{sphinxuseclass}
\begin{sphinxuseclass}{sd-col}
\begin{sphinxuseclass}{sd-d-flex-row}
\begin{sphinxuseclass}{sd-col-4}
\begin{sphinxuseclass}{sd-col-xs-4}
\begin{sphinxuseclass}{sd-col-sm-4}
\begin{sphinxuseclass}{sd-col-md-4}
\begin{sphinxuseclass}{sd-col-lg-4}
\begin{sphinxuseclass}{sd-card}
\begin{sphinxuseclass}{sd-sphinx-override}
\begin{sphinxuseclass}{sd-w-100}
\begin{sphinxuseclass}{sd-shadow-sm}
\begin{sphinxuseclass}{sd-card-hover}
\begin{sphinxuseclass}{sd-card-header}
\begin{sphinxuseclass}{bg-light}
\sphinxAtStartPar
Systems of Linear Equations

\end{sphinxuseclass}
\end{sphinxuseclass}
\begin{sphinxuseclass}{sd-card-body}
\begin{figure}[htbp]
\centering

\noindent\sphinxincludegraphics[width=250\sphinxpxdimen]{{2_consistent_systems}.png}
\end{figure}

\end{sphinxuseclass}{\hyperref[\detokenize{_pages/2.0_Linear_systems::doc}]{\sphinxcrossref{\DUrole{doc}{Systems of Linear Equations}}}}
\end{sphinxuseclass}
\end{sphinxuseclass}
\end{sphinxuseclass}
\end{sphinxuseclass}
\end{sphinxuseclass}
\end{sphinxuseclass}
\end{sphinxuseclass}
\end{sphinxuseclass}
\end{sphinxuseclass}
\end{sphinxuseclass}
\end{sphinxuseclass}
\end{sphinxuseclass}
\begin{sphinxuseclass}{sd-col}
\begin{sphinxuseclass}{sd-d-flex-row}
\begin{sphinxuseclass}{sd-col-4}
\begin{sphinxuseclass}{sd-col-xs-4}
\begin{sphinxuseclass}{sd-col-sm-4}
\begin{sphinxuseclass}{sd-col-md-4}
\begin{sphinxuseclass}{sd-col-lg-4}
\begin{sphinxuseclass}{sd-card}
\begin{sphinxuseclass}{sd-sphinx-override}
\begin{sphinxuseclass}{sd-w-100}
\begin{sphinxuseclass}{sd-shadow-sm}
\begin{sphinxuseclass}{sd-card-hover}
\begin{sphinxuseclass}{sd-card-header}
\begin{sphinxuseclass}{bg-light}
\sphinxAtStartPar
Vectors

\end{sphinxuseclass}
\end{sphinxuseclass}
\begin{sphinxuseclass}{sd-card-body}
\begin{figure}[htbp]
\centering

\noindent\sphinxincludegraphics[width=250\sphinxpxdimen]{{3_dot_product}.png}
\end{figure}

\end{sphinxuseclass}{\hyperref[\detokenize{_pages/3.0_Vectors::doc}]{\sphinxcrossref{\DUrole{doc}{Vectors}}}}
\end{sphinxuseclass}
\end{sphinxuseclass}
\end{sphinxuseclass}
\end{sphinxuseclass}
\end{sphinxuseclass}
\end{sphinxuseclass}
\end{sphinxuseclass}
\end{sphinxuseclass}
\end{sphinxuseclass}
\end{sphinxuseclass}
\end{sphinxuseclass}
\end{sphinxuseclass}
\begin{sphinxuseclass}{sd-col}
\begin{sphinxuseclass}{sd-d-flex-row}
\begin{sphinxuseclass}{sd-col-4}
\begin{sphinxuseclass}{sd-col-xs-4}
\begin{sphinxuseclass}{sd-col-sm-4}
\begin{sphinxuseclass}{sd-col-md-4}
\begin{sphinxuseclass}{sd-col-lg-4}
\begin{sphinxuseclass}{sd-card}
\begin{sphinxuseclass}{sd-sphinx-override}
\begin{sphinxuseclass}{sd-w-100}
\begin{sphinxuseclass}{sd-shadow-sm}
\begin{sphinxuseclass}{sd-card-hover}
\begin{sphinxuseclass}{sd-card-header}
\begin{sphinxuseclass}{bg-light}
\sphinxAtStartPar
Co\sphinxhyphen{}ordinate Geometry

\end{sphinxuseclass}
\end{sphinxuseclass}
\begin{sphinxuseclass}{sd-card-body}
\begin{figure}[htbp]
\centering

\noindent\sphinxincludegraphics[width=250\sphinxpxdimen]{{4_line_line_distance}.png}
\end{figure}

\end{sphinxuseclass}{\hyperref[\detokenize{_pages/4.0_Coordinate_geometry::doc}]{\sphinxcrossref{\DUrole{doc}{Co\sphinxhyphen{}ordinate Geometry}}}}
\end{sphinxuseclass}
\end{sphinxuseclass}
\end{sphinxuseclass}
\end{sphinxuseclass}
\end{sphinxuseclass}
\end{sphinxuseclass}
\end{sphinxuseclass}
\end{sphinxuseclass}
\end{sphinxuseclass}
\end{sphinxuseclass}
\end{sphinxuseclass}
\end{sphinxuseclass}
\begin{sphinxuseclass}{sd-col}
\begin{sphinxuseclass}{sd-d-flex-row}
\begin{sphinxuseclass}{sd-col-4}
\begin{sphinxuseclass}{sd-col-xs-4}
\begin{sphinxuseclass}{sd-col-sm-4}
\begin{sphinxuseclass}{sd-col-md-4}
\begin{sphinxuseclass}{sd-col-lg-4}
\begin{sphinxuseclass}{sd-card}
\begin{sphinxuseclass}{sd-sphinx-override}
\begin{sphinxuseclass}{sd-w-100}
\begin{sphinxuseclass}{sd-shadow-sm}
\begin{sphinxuseclass}{sd-card-hover}
\begin{sphinxuseclass}{sd-card-header}
\begin{sphinxuseclass}{bg-light}
\sphinxAtStartPar
Vector Spaces

\end{sphinxuseclass}
\end{sphinxuseclass}
\begin{sphinxuseclass}{sd-card-body}
\begin{figure}[htbp]
\centering

\noindent\sphinxincludegraphics[width=250\sphinxpxdimen]{{5_subspaces}.png}
\end{figure}

\end{sphinxuseclass}{\hyperref[\detokenize{_pages/5.0_Vector_spaces::doc}]{\sphinxcrossref{\DUrole{doc}{Vector spaces}}}}
\end{sphinxuseclass}
\end{sphinxuseclass}
\end{sphinxuseclass}
\end{sphinxuseclass}
\end{sphinxuseclass}
\end{sphinxuseclass}
\end{sphinxuseclass}
\end{sphinxuseclass}
\end{sphinxuseclass}
\end{sphinxuseclass}
\end{sphinxuseclass}
\end{sphinxuseclass}
\begin{sphinxuseclass}{sd-col}
\begin{sphinxuseclass}{sd-d-flex-row}
\begin{sphinxuseclass}{sd-col-4}
\begin{sphinxuseclass}{sd-col-xs-4}
\begin{sphinxuseclass}{sd-col-sm-4}
\begin{sphinxuseclass}{sd-col-md-4}
\begin{sphinxuseclass}{sd-col-lg-4}
\begin{sphinxuseclass}{sd-card}
\begin{sphinxuseclass}{sd-sphinx-override}
\begin{sphinxuseclass}{sd-w-100}
\begin{sphinxuseclass}{sd-shadow-sm}
\begin{sphinxuseclass}{sd-card-hover}
\begin{sphinxuseclass}{sd-card-header}
\begin{sphinxuseclass}{bg-light}
\sphinxAtStartPar
Linear Transformations

\end{sphinxuseclass}
\end{sphinxuseclass}
\begin{sphinxuseclass}{sd-card-body}
\begin{figure}[htbp]
\centering

\noindent\sphinxincludegraphics[width=250\sphinxpxdimen]{{6_linear_transformation}.png}
\end{figure}

\end{sphinxuseclass}{\hyperref[\detokenize{_pages/6.0_Linear_transformations::doc}]{\sphinxcrossref{\DUrole{doc}{Linear Transformations}}}}
\end{sphinxuseclass}
\end{sphinxuseclass}
\end{sphinxuseclass}
\end{sphinxuseclass}
\end{sphinxuseclass}
\end{sphinxuseclass}
\end{sphinxuseclass}
\end{sphinxuseclass}
\end{sphinxuseclass}
\end{sphinxuseclass}
\end{sphinxuseclass}
\end{sphinxuseclass}
\end{sphinxuseclass}
\end{sphinxuseclass}
\end{sphinxuseclass}
\end{sphinxuseclass}
\end{sphinxuseclass}
\end{sphinxuseclass}
\end{sphinxuseclass}
\end{sphinxuseclass}
\end{sphinxuseclass}
\end{sphinxuseclass}
\end{sphinxuseclass}
\end{sphinxuseclass}
\end{sphinxuseclass}
\end{sphinxuseclass}
\end{sphinxuseclass}
\sphinxAtStartPar
Students are advised to read through the relevant section of the lecture notes prior to attending the lecture. Do not worry about trying to understand everything when you first read through it. Reading mathematics is not like reading a novel: it often requires repeated reading of a passage before you fully grasp the concepts that are being explained. In the lectures we will explain the various topics and provide more insight to complement what is written in the notes.

\sphinxAtStartPar
The examples and exercises contained in this book have the solutions hidden from the reader by default but can be revealed by clicking on the drop down link. Readers are advised to avoid the temptation of accessing the solutions before an attempt has been made to answer the questions.


\bigskip\hrule\bigskip


\begin{sphinxuseclass}{sd-container-fluid}
\begin{sphinxuseclass}{sd-sphinx-override}
\begin{sphinxuseclass}{sd-mb-4}
\begin{sphinxuseclass}{sd-row}
\begin{sphinxuseclass}{sd-col}
\begin{sphinxuseclass}{sd-d-flex-column}
\begin{sphinxuseclass}{sd-col-2}
\begin{sphinxuseclass}{sd-col-xs-2}
\begin{sphinxuseclass}{sd-col-sm-2}
\begin{sphinxuseclass}{sd-col-md-2}
\begin{sphinxuseclass}{sd-col-lg-2}
\noindent{\sphinxincludegraphics[width=125\sphinxpxdimen]{{katie_steckles}.jpg}\hspace*{\fill}}

\end{sphinxuseclass}
\end{sphinxuseclass}
\end{sphinxuseclass}
\end{sphinxuseclass}
\end{sphinxuseclass}
\end{sphinxuseclass}
\end{sphinxuseclass}
\begin{sphinxuseclass}{sd-col}
\begin{sphinxuseclass}{sd-d-flex-column}
\sphinxAtStartPar
Dr Katie Steckles 
Homepage: MMU website profile 
Email: k.steckles@mmu.ac.uk 
Office: 3.27 (Dalton Building)

\end{sphinxuseclass}
\end{sphinxuseclass}
\end{sphinxuseclass}
\end{sphinxuseclass}
\end{sphinxuseclass}
\end{sphinxuseclass}
\sphinxstepscope


\part{Preliminaries}

\sphinxstepscope


\chapter{Teaching schedule}
\label{\detokenize{_pages/0.1_Teaching_schedule:teaching-schedule}}\label{\detokenize{_pages/0.1_Teaching_schedule::doc}}
\sphinxAtStartPar
The teaching schedule for linear algebra is given below. If you happen to miss a lecture or tutorial, you will need to make sure you catch up with the work before the following week.


\begin{savenotes}\sphinxattablestart
\sphinxthistablewithglobalstyle
\centering
\begin{tabulary}{\linewidth}[t]{TTTT}
\sphinxtoprule
\sphinxstyletheadfamily 
\sphinxAtStartPar
Week
&\sphinxstyletheadfamily 
\sphinxAtStartPar
Date (w/c)
&\sphinxstyletheadfamily 
\sphinxAtStartPar
Lecture
&\sphinxstyletheadfamily 
\sphinxAtStartPar
Tutorial
\\
\sphinxmidrule
\sphinxtableatstartofbodyhook
\sphinxAtStartPar
1
&
\sphinxAtStartPar
29/09/2025
&
\sphinxAtStartPar
{\hyperref[\detokenize{_pages/1.0_Matrices:matrices-chapter}]{\sphinxcrossref{\DUrole{std,std-ref}{\sphinxstylestrong{Matrices}}}}}: {\hyperref[\detokenize{_pages/1.0_Matrices:matrices-chapter}]{\sphinxcrossref{\DUrole{std,std-ref}{definition}}}}, {\hyperref[\detokenize{_pages/1.0_Matrices:indexing-a-matrix-section}]{\sphinxcrossref{\DUrole{std,std-ref}{indexing a matrix}}}}, {\hyperref[\detokenize{_pages/1.1_Matrix_operations:matrix-operations-section}]{\sphinxcrossref{\DUrole{std,std-ref}{basic arithmetic operations}}}}, {\hyperref[\detokenize{_pages/1.2_Matrix_multiplication:matrix-multiplication-section}]{\sphinxcrossref{\DUrole{std,std-ref}{matrix multiplication}}}}
&
\sphinxAtStartPar
\hyperref[\detokenize{_pages/1.8_Matrices_exercises:matrices-ex1}]{Exercise \ref{\detokenize{_pages/1.8_Matrices_exercises:matrices-ex1}}} to \hyperref[\detokenize{_pages/1.8_Matrices_exercises:matrices-ex3}]{Exercise \ref{\detokenize{_pages/1.8_Matrices_exercises:matrices-ex3}}}
\\
\sphinxhline
\sphinxAtStartPar
2
&
\sphinxAtStartPar
06/10/2025
&
\sphinxAtStartPar
{\hyperref[\detokenize{_pages/1.4_Determinants:determinant-section}]{\sphinxcrossref{\DUrole{std,std-ref}{\sphinxstylestrong{Matrices (cont.)}}}}}: {\hyperref[\detokenize{_pages/1.4_Determinants:determinant-section}]{\sphinxcrossref{\DUrole{std,std-ref}{determinants}}}}, {\hyperref[\detokenize{_pages/1.5_Inverse_matrix:inverse-matrix-section}]{\sphinxcrossref{\DUrole{std,std-ref}{inverse matrix}}}}, {\hyperref[\detokenize{_pages/1.6_Matrix_algebra:matrix-algebra-section}]{\sphinxcrossref{\DUrole{std,std-ref}{matrix algebra}}}}
&
\sphinxAtStartPar
\hyperref[\detokenize{_pages/1.8_Matrices_exercises:matrices-ex-determinants}]{Exercise \ref{\detokenize{_pages/1.8_Matrices_exercises:matrices-ex-determinants}}} to \hyperref[\detokenize{_pages/1.8_Matrices_exercises:matrices-ex7}]{Exercise \ref{\detokenize{_pages/1.8_Matrices_exercises:matrices-ex7}}}
\\
\sphinxhline
\sphinxAtStartPar
3
&
\sphinxAtStartPar
13/10/2025
&
\sphinxAtStartPar
{\hyperref[\detokenize{_pages/2.0_Linear_systems:systems-of-linear-equations-chapter}]{\sphinxcrossref{\DUrole{std,std-ref}{\sphinxstylestrong{Systems of linear equations}}}}}: {\hyperref[\detokenize{_pages/2.0_Linear_systems:systems-of-linear-equations-chapter}]{\sphinxcrossref{\DUrole{std,std-ref}{definition}}}}, solving using {\hyperref[\detokenize{_pages/2.1_Solving_using_inverse_matrix:solving-systems-using-inverse-section}]{\sphinxcrossref{\DUrole{std,std-ref}{inverse matrices}}}} and {\hyperref[\detokenize{_pages/2.2_Cramers_rule:cramers-rule-section}]{\sphinxcrossref{\DUrole{std,std-ref}{Cramer’s rule}}}}
&
\sphinxAtStartPar
\hyperref[\detokenize{_pages/2.9_Linear_systems_exercises:systems-ex-inverse-sol}]{Exercise \ref{\detokenize{_pages/2.9_Linear_systems_exercises:systems-ex-inverse-sol}}} and \hyperref[\detokenize{_pages/2.9_Linear_systems_exercises:systems-ex-cramer}]{Exercise \ref{\detokenize{_pages/2.9_Linear_systems_exercises:systems-ex-cramer}}}
\\
\sphinxhline
\sphinxAtStartPar
4
&
\sphinxAtStartPar
20/10/2025
&
\sphinxAtStartPar
{\hyperref[\detokenize{_pages/2.3_Gaussian_elimination:gaussian-elimination-section}]{\sphinxcrossref{\DUrole{std,std-ref}{\sphinxstylestrong{Systems of linear equations (cont.)}}}}}: {\hyperref[\detokenize{_pages/2.3_Gaussian_elimination:gaussian-elimination-section}]{\sphinxcrossref{\DUrole{std,std-ref}{Gaussian elimination}}}} and {\hyperref[\detokenize{_pages/2.4_Partial_pivoting:partial-pivoting-section}]{\sphinxcrossref{\DUrole{std,std-ref}{partial pivoting}}}}
&
\sphinxAtStartPar
\hyperref[\detokenize{_pages/2.9_Linear_systems_exercises:systems-ex-gelim}]{Exercise \ref{\detokenize{_pages/2.9_Linear_systems_exercises:systems-ex-gelim}}} and \hyperref[\detokenize{_pages/2.9_Linear_systems_exercises:systems-ex-gelimpp}]{Exercise \ref{\detokenize{_pages/2.9_Linear_systems_exercises:systems-ex-gelimpp}}}
\\
\sphinxhline
\sphinxAtStartPar
5
&
\sphinxAtStartPar
27/10/2025
&
\sphinxAtStartPar
{\hyperref[\detokenize{_pages/2.5_Gauss_Jordan_elimination:gauss-jordan-elimination-section}]{\sphinxcrossref{\DUrole{std,std-ref}{\sphinxstylestrong{Systems of linear equations (cont.)}}}}}: {\hyperref[\detokenize{_pages/2.5_Gauss_Jordan_elimination:gauss-jordan-elimination-section}]{\sphinxcrossref{\DUrole{std,std-ref}{Gauss\sphinxhyphen{}Jordan elimination}}}}, {\hyperref[\detokenize{_pages/2.6_Consistent_systems:consistent-inconsistent-and-indeterminate-systems-section}]{\sphinxcrossref{\DUrole{std,std-ref}{consistent, inconsistent and indeterminate systems}}}}, {\hyperref[\detokenize{_pages/2.7_Homogeneous_systems:homogeneous-systems-section}]{\sphinxcrossref{\DUrole{std,std-ref}{homogeneous systems}}}}
&
\sphinxAtStartPar
\hyperref[\detokenize{_pages/2.9_Linear_systems_exercises:systems-ex-gjelim}]{Exercise \ref{\detokenize{_pages/2.9_Linear_systems_exercises:systems-ex-gjelim}}} to \hyperref[\detokenize{_pages/2.9_Linear_systems_exercises:systems-ex-consistency}]{Exercise \ref{\detokenize{_pages/2.9_Linear_systems_exercises:systems-ex-consistency}}}
\\
\sphinxhline
\sphinxAtStartPar
6
&
\sphinxAtStartPar
03/11/2025
&
\sphinxAtStartPar
{\hyperref[\detokenize{_pages/3.0_Vectors:vectors-chapter}]{\sphinxcrossref{\DUrole{std,std-ref}{\sphinxstylestrong{Vectors}}}}}: {\hyperref[\detokenize{_pages/3.0_Vectors:vectors-definition-section}]{\sphinxcrossref{\DUrole{std,std-ref}{definition}}}}, {\hyperref[\detokenize{_pages/3.1_Vector_arithmetic:arithmetic-operations-on-vectors-section}]{\sphinxcrossref{\DUrole{std,std-ref}{arithmetic operations on vectors}}}}, {\hyperref[\detokenize{_pages/3.2_Vector_magnitude:vector-magnitude-section}]{\sphinxcrossref{\DUrole{std,std-ref}{vector magnitude}}}}, {\hyperref[\detokenize{_pages/3.3_Dot_and_cross_products:dot-product-section}]{\sphinxcrossref{\DUrole{std,std-ref}{dot}}}} and {\hyperref[\detokenize{_pages/3.3_Dot_and_cross_products:cross-product-section}]{\sphinxcrossref{\DUrole{std,std-ref}{cross}}}} products, {\hyperref[\detokenize{_pages/3.4_Linear_combinations:linear-combination-of-vectors-section}]{\sphinxcrossref{\DUrole{std,std-ref}{linear combination of vectors}}}}
&
\sphinxAtStartPar
\hyperref[\detokenize{_pages/3.5_Vectors_exercises:vectors-ex-arithmetic}]{Exercise \ref{\detokenize{_pages/3.5_Vectors_exercises:vectors-ex-arithmetic}}} to \hyperref[\detokenize{_pages/3.5_Vectors_exercises:vectors-ex-angle}]{Exercise \ref{\detokenize{_pages/3.5_Vectors_exercises:vectors-ex-angle}}}
\\
\sphinxhline
\sphinxAtStartPar
7
&
\sphinxAtStartPar
10/11/2025
&
\sphinxAtStartPar
{\hyperref[\detokenize{_pages/4.0_Coordinate_geometry:co-ordinate-geometry-chapter}]{\sphinxcrossref{\DUrole{std,std-ref}{\sphinxstylestrong{Co\sphinxhyphen{}ordinate Geometry}}}}}: {\hyperref[\detokenize{_pages/4.0_Coordinate_geometry:points-section}]{\sphinxcrossref{\DUrole{std,std-ref}{points}}}}, {\hyperref[\detokenize{_pages/4.1_Lines:lines-section}]{\sphinxcrossref{\DUrole{std,std-ref}{lines}}}} and {\hyperref[\detokenize{_pages/4.2_Planes:planes-section}]{\sphinxcrossref{\DUrole{std,std-ref}{planes}}}}, vector equation of {\hyperref[\detokenize{_pages/4.1_Lines:lines-section}]{\sphinxcrossref{\DUrole{std,std-ref}{lines}}}} and {\hyperref[\detokenize{_pages/4.2_Planes:planes-section}]{\sphinxcrossref{\DUrole{std,std-ref}{planes}}}}, the {\hyperref[\detokenize{_pages/4.2_Planes:point-normal-definition}]{\sphinxcrossref{point\sphinxhyphen{}normal}}} equation of a plane, {\hyperref[\detokenize{_pages/4.3_Shortest_distance_problems:id1}]{\sphinxcrossref{\DUrole{std,std-ref}{shortest distance problems}}}}
&
\sphinxAtStartPar
\hyperref[\detokenize{_pages/4.4_Coordinate_geometry_exercises:geometry-ex-line-plane-equations}]{Exercise \ref{\detokenize{_pages/4.4_Coordinate_geometry_exercises:geometry-ex-line-plane-equations}}} to \hyperref[\detokenize{_pages/4.4_Coordinate_geometry_exercises:geometry-ex-point-plane-distance}]{Exercise \ref{\detokenize{_pages/4.4_Coordinate_geometry_exercises:geometry-ex-point-plane-distance}}}
\\
\sphinxhline
\sphinxAtStartPar
8
&
\sphinxAtStartPar
10/11/2025
&
\sphinxAtStartPar
{\hyperref[\detokenize{_pages/5.0_Vector_spaces:vector-spaces-chapter}]{\sphinxcrossref{\DUrole{std,std-ref}{\sphinxstylestrong{Vector Spaces}}}}}: {\hyperref[\detokenize{_pages/5.1_Vector_spaces_definitions:vector-spaces-definitions-section}]{\sphinxcrossref{\DUrole{std,std-ref}{definitions}}}}, {\hyperref[\detokenize{_pages/5.2_Subspaces:subspaces-section}]{\sphinxcrossref{\DUrole{std,std-ref}{subspaces}}}}, {\hyperref[\detokenize{_pages/5.3_Linear_dependence:linear-dependence-section}]{\sphinxcrossref{\DUrole{std,std-ref}{linear dependence}}}}, {\hyperref[\detokenize{_pages/5.4_Basis:basis-section}]{\sphinxcrossref{\DUrole{std,std-ref}{basis of a vector space}}}}
&
\sphinxAtStartPar
\hyperref[\detokenize{_pages/5.5_Vector_spaces_exercises:vector-spaces-ex-R3-axioms}]{Exercise \ref{\detokenize{_pages/5.5_Vector_spaces_exercises:vector-spaces-ex-R3-axioms}}} to \hyperref[\detokenize{_pages/5.5_Vector_spaces_exercises:vector-spaces-ex-R4-basis-2}]{Exercise \ref{\detokenize{_pages/5.5_Vector_spaces_exercises:vector-spaces-ex-R4-basis-2}}}
\\
\sphinxhline
\sphinxAtStartPar
9
&
\sphinxAtStartPar
17/11/2025
&
\sphinxAtStartPar
{\hyperref[\detokenize{_pages/6.0_Linear_transformations:linear-transformations-chapter}]{\sphinxcrossref{\DUrole{std,std-ref}{\sphinxstylestrong{Linear Transformations}}}}}: {\hyperref[\detokenize{_pages/6.0_Linear_transformations:linear-transformations-chapter}]{\sphinxcrossref{\DUrole{std,std-ref}{definition}}}}, {\hyperref[\detokenize{_pages/6.1_Transformation_matrices:transformation-matrices-section}]{\sphinxcrossref{\DUrole{std,std-ref}{transformation matrices}}}}, {\hyperref[\detokenize{_pages/6.2_Composite_transformations:composite-linear-transformations-section}]{\sphinxcrossref{\DUrole{std,std-ref}{composite linear transformations}}}}
&
\sphinxAtStartPar
\hyperref[\detokenize{_pages/6.5_Linear_transforation_exercises:transformations-ex-linear-transformations}]{Exercise \ref{\detokenize{_pages/6.5_Linear_transforation_exercises:transformations-ex-linear-transformations}}} to \hyperref[\detokenize{_pages/6.5_Linear_transforation_exercises:transformations-ex-R3}]{Exercise \ref{\detokenize{_pages/6.5_Linear_transforation_exercises:transformations-ex-R3}}}
\\
\sphinxhline
\sphinxAtStartPar
10
&
\sphinxAtStartPar
24/11/2025
&
\sphinxAtStartPar
{\hyperref[\detokenize{_pages/6.3_Rotation_reflection_and_translation:rotation-section}]{\sphinxcrossref{\DUrole{std,std-ref}{\sphinxstylestrong{Linear Transformations (cont.)}}}}}: {\hyperref[\detokenize{_pages/6.3_Rotation_reflection_and_translation:rotation-section}]{\sphinxcrossref{\DUrole{std,std-ref}{rotation}}}}, {\hyperref[\detokenize{_pages/6.3_Rotation_reflection_and_translation:reflection-section}]{\sphinxcrossref{\DUrole{std,std-ref}{reflection}}}}, {\hyperref[\detokenize{_pages/6.3_Rotation_reflection_and_translation:scaling-section}]{\sphinxcrossref{\DUrole{std,std-ref}{scaling}}}} and {\hyperref[\detokenize{_pages/6.4_Translation:translation-section}]{\sphinxcrossref{\DUrole{std,std-ref}{translation}}}} transformations
&
\sphinxAtStartPar
\hyperref[\detokenize{_pages/6.5_Linear_transforation_exercises:transformations-ex-rotation}]{Exercise \ref{\detokenize{_pages/6.5_Linear_transforation_exercises:transformations-ex-rotation}}} to \hyperref[\detokenize{_pages/6.5_Linear_transforation_exercises:transformations-ex-transform-square}]{Exercise \ref{\detokenize{_pages/6.5_Linear_transforation_exercises:transformations-ex-transform-square}}}
\\
\sphinxbottomrule
\end{tabulary}
\sphinxtableafterendhook\par
\sphinxattableend\end{savenotes}

\sphinxstepscope


\chapter{Mathematical preliminaries}
\label{\detokenize{_pages/0.3_Mathematical_preliminaries:mathematical-preliminaries}}\label{\detokenize{_pages/0.3_Mathematical_preliminaries::doc}}
\sphinxAtStartPar
It is assumed that readers are familiar with the following.


\bigskip\hrule\bigskip



\section{Sets of numbers}
\label{\detokenize{_pages/0.3_Mathematical_preliminaries:sets-of-numbers}}
\sphinxAtStartPar
Numbers are classified into different sets based on their properties.


\begin{savenotes}\sphinxattablestart
\sphinxthistablewithglobalstyle
\centering
\begin{tabulary}{\linewidth}[t]{TTTT}
\sphinxtoprule
\sphinxstyletheadfamily 
\sphinxAtStartPar
Type
&\sphinxstyletheadfamily 
\sphinxAtStartPar
Symbol
&\sphinxstyletheadfamily 
\sphinxAtStartPar
Explanation
&\sphinxstyletheadfamily 
\sphinxAtStartPar
Example
\\
\sphinxmidrule
\sphinxtableatstartofbodyhook
\sphinxAtStartPar
Natural
&
\sphinxAtStartPar
\(\mathbb{N}\)
&
\sphinxAtStartPar
Positive whole numbers
&
\sphinxAtStartPar
\(0, 1, 2\)
\\
\sphinxhline
\sphinxAtStartPar
Integer
&
\sphinxAtStartPar
\(\mathbb{Z}\)
&
\sphinxAtStartPar
All whole numbers including negatives
&
\sphinxAtStartPar
\(-1, 0, 1, 2\)
\\
\sphinxhline
\sphinxAtStartPar
Rational
&
\sphinxAtStartPar
\(\mathbb{Q}\)
&
\sphinxAtStartPar
Numbers expressed as a fraction of two integers, \(\dfrac{a}{b}\) where \(b \neq 0\)
&
\sphinxAtStartPar
\(\dfrac{1}{2}\)
\\
\sphinxhline
\sphinxAtStartPar
Real
&
\sphinxAtStartPar
\(\mathbb{R}\)
&
\sphinxAtStartPar
Any number that can be expressed on a number line
&
\sphinxAtStartPar
\(\sqrt{2}, \pi\)
\\
\sphinxhline
\sphinxAtStartPar
Complex
&
\sphinxAtStartPar
\(\mathbb{C}\)
&
\sphinxAtStartPar
A real number added to a multiple of an {\hyperref[\detokenize{_pages/0.3_Mathematical_preliminaries:imaginary-numbers-section}]{\sphinxcrossref{\DUrole{std,std-ref}{imaginary number}}}}
&
\sphinxAtStartPar
\(1 + 2i\)
\\
\sphinxbottomrule
\end{tabulary}
\sphinxtableafterendhook\par
\sphinxattableend\end{savenotes}

\sphinxAtStartPar
The sets of numbers are related by
\begin{equation*}
\begin{split} \mathbb{N} \subset \mathbb{Z} \subset \mathbb{Q} \subset \mathbb{R} \subset \mathbb{C}. \end{split}
\end{equation*}\phantomsection\label{\detokenize{_pages/0.3_Mathematical_preliminaries:axioms-of-addition-and-multiplication-section}}

\bigskip\hrule\bigskip



\section{Axioms of addition and multiplication}
\label{\detokenize{_pages/0.3_Mathematical_preliminaries:axioms-of-addition-and-multiplication}}
\sphinxAtStartPar
An \sphinxstylestrong{axiom} is a rule that we assume is true for a starting point for further reasoning and arguments. The following nine axioms apply to addition and multiplication, denoted by \(+\) and \(\times\) respectively, over a field \(F\) (e.g., \(F\) could be the set of all numbers). For every \(x,y,z \in F\):
\begin{itemize}
\item {} 
\sphinxAtStartPar
\sphinxstylestrong{Commutativity of addition}: \(x + y = y + x\)

\item {} 
\sphinxAtStartPar
\sphinxstylestrong{Associativity of addition}: \(x + (y + z) = (x + y) + z\)

\item {} 
\sphinxAtStartPar
\sphinxstylestrong{Additive neutral element}: there exists an element \(0\) such that \(x + 0 = x\)

\item {} 
\sphinxAtStartPar
\sphinxstylestrong{Additive inverse}: there exists an element \(-x\) such that \(x + (-x) = 0\)

\item {} 
\sphinxAtStartPar
\sphinxstylestrong{Commutativity of multiplication}: \(xy = yz\)

\item {} 
\sphinxAtStartPar
\sphinxstylestrong{Associativity of multiplication}: \(x(yz) = (xy)z\)

\item {} 
\sphinxAtStartPar
\sphinxstylestrong{Multiplicative identity}: there exists an element \(1\) such that \(1 \times x = x\)

\item {} 
\sphinxAtStartPar
\sphinxstylestrong{Multiplicative inverse}: there exists an element \(1/x\) such that \(x \times \dfrac{1}{x} = 1\)

\item {} 
\sphinxAtStartPar
\sphinxstylestrong{Distributive law}: \(x(y + z) = xy + xz\)

\end{itemize}
\phantomsection\label{\detokenize{_pages/0.3_Mathematical_preliminaries:imaginary-numbers-section}}

\bigskip\hrule\bigskip



\section{Imaginary numbers}
\label{\detokenize{_pages/0.3_Mathematical_preliminaries:imaginary-numbers}}
\sphinxAtStartPar
The imaginary number is denoted by \(i\) (some disciplines use \(j\) to denote the imaginary number) and defined to be \(i^2 = -1\). Complex numbers are numbers of the form \(a + bi\) where \(a,b\in \mathbb{R}\) and \(a\) is called the \sphinxstylestrong{real part} and \(b\) is called the \sphinxstylestrong{imaginary part}.

\sphinxAtStartPar
The following arithmetic operations are defined on complex numbers:
\begin{itemize}
\item {} 
\sphinxAtStartPar
addition: \((a + bi) + (c + di) = (a + c) + (b + d)i\);

\item {} 
\sphinxAtStartPar
multiplication: \((a + bi) (c + di) = ac - bd + (ad + bc)i\);

\item {} 
\sphinxAtStartPar
modulus: \(|a + bi| = \sqrt{a^2 + b^2}\);

\item {} 
\sphinxAtStartPar
complex conjugate: if \(z = a + bi\) then the complex conjugate of \(z\) is \(\bar{z} = a - bi\);

\end{itemize}


\bigskip\hrule\bigskip



\section{Summation}
\label{\detokenize{_pages/0.3_Mathematical_preliminaries:summation}}
\sphinxAtStartPar
The sum of a set of objects \(a_1, a_2, \ldots, a_n\) is denoted using
\begin{equation*}
\begin{split}\sum_{i=1}^n a_i = a_1 + a_2 + \cdots + a_n,\end{split}
\end{equation*}
\sphinxAtStartPar
where the uppercase Greek letter sigma \(\Sigma\) denotes the sum, the lower delimiter \(i=1\) indicates that the summation begins at the element \(i=1\), and the upper delimiter \(n\) indicates that the summation ends at the element indexed \(i=n\). For example,
\begin{equation*}
\begin{split} \sum_{i=1}^5 i = 1 + 2 + 3 + 4 + 5 = 15. \end{split}
\end{equation*}

\bigskip\hrule\bigskip



\section{Definitions, theorems, proofs and examples}
\label{\detokenize{_pages/0.3_Mathematical_preliminaries:definitions-theorems-proofs-and-examples}}
\sphinxAtStartPar
These notes use standard constructs to present mathematical ideas.


\subsection{Definitions}
\label{\detokenize{_pages/0.3_Mathematical_preliminaries:definitions}}
\sphinxAtStartPar
A \sphinxstylestrong{definition} is a statement that precisely describes the meaning of a mathematical concept or term. It provides clarity and establishes a common understanding of a specific mathematical object, property, or relationship. Here definitions are presented like the following:
\label{_pages/0.3_Mathematical_preliminaries:definition-0}
\begin{sphinxadmonition}{note}{Definition}



\sphinxAtStartPar
A prime number is a natural number greater than 1 whose only factors are 1 and itself.
\end{sphinxadmonition}


\subsection{Theorems}
\label{\detokenize{_pages/0.3_Mathematical_preliminaries:theorems}}
\sphinxAtStartPar
A mathematical \sphinxstylestrong{theorem} is a statement that has been proven to be true based on a set of axioms and previously established results or assumptions. Here theorems are presented like the following:
\label{_pages/0.3_Mathematical_preliminaries:theorem-1}
\begin{sphinxadmonition}{note}{Theorem}



\sphinxAtStartPar
There are an infinite number of primes.
\end{sphinxadmonition}


\subsection{Proofs}
\label{\detokenize{_pages/0.3_Mathematical_preliminaries:proofs}}
\sphinxAtStartPar
A mathematical \sphinxstylestrong{proof} is a logical argument or demonstration that establishes the truth or validity of a mathematical statement or theorem. It is a rigorous and systematic process used to justify mathematical claims and establish certainty about mathematical assertions. Here proofs are presented like the following

\begin{sphinxadmonition}{note}
\sphinxAtStartPar
Proof. Assume there are a finite number of primes \(p_1, p_2, \ldots, p_n\). Let
\begin{equation*}
\begin{split}N = (p_1 \cdot p_2 \cdot \ldots \cdot p_n) + 1, \end{split}
\end{equation*}
\sphinxAtStartPar
then either \(N\) is prime or not prime.
\begin{itemize}
\item {} 
\sphinxAtStartPar
If \(N\) is prime then since \(N\) is not in the finite list of primes then this contradicts the assumption that there are a finite number of primes.

\item {} 
\sphinxAtStartPar
If \(N\) is not prime then it must have at least one prime factor which is not in the finite list of primes because that would leave a remainder of 1. This contradicts the assumption that there are a finite number of primes.

\end{itemize}

\sphinxAtStartPar
Since \(N\) cannot be prime or not prime then the assumption that there are a finite number of primes is false, so there must be an infinite number of primes.
\end{sphinxadmonition}


\subsection{Examples}
\label{\detokenize{_pages/0.3_Mathematical_preliminaries:examples}}
\sphinxAtStartPar
A mathematical \sphinxstylestrong{example} refers to an instance or case that illustrates a concept, property, or theorem in mathematics. It serves as a specific instance that helps to understand and apply mathematical ideas and principles. Here examples are used to help you understand and apply the concepts that have been covered. These are designed for you to following in class but the solutions of which are also given in a dropdown box.
\label{_pages/0.3_Mathematical_preliminaries:example-2}
\begin{sphinxadmonition}{note}{Example}



\sphinxAtStartPar
Determine the prime factors of 84.
\subsubsection*{Solution}

\sphinxAtStartPar
Repeatedly divide 84 by the first prime number 2:
\begin{equation*}
\begin{split} \begin{align*}
    \frac{84}{2} &= 42, && \therefore \text{2 is a prime factor of 84}, \\
    \frac{42}{2} &= 21, && \therefore \text{2 is another prime factor of 84}
\end{align*} \end{split}
\end{equation*}
\sphinxAtStartPar
21 cannot be evenly divided by 2, so the first two prime factors are \(2\) and \(2\). Next, divide 21 by the next prime number 3:
\begin{equation*}
\begin{split} \begin{align*}
    \frac{21}{3} &= 7, && \therefore \text{3 is another prime factor of 84}
\end{align*} \end{split}
\end{equation*}
\sphinxAtStartPar
7 is itself a prime number, so this must be the remaining prime factor of 84. The prime factorisation of 84 is \(2 \cdot 2 \cdot 3 \cdot 7 = 84\).
\end{sphinxadmonition}

\sphinxstepscope


\chapter{How to Pass the Module}
\label{\detokenize{_pages/0.5_How_to_pass:how-to-pass-the-module}}\label{\detokenize{_pages/0.5_How_to_pass::doc}}
\sphinxAtStartPar
Some general advice to students:
\begin{itemize}
\item {} 
\sphinxAtStartPar
\sphinxstylestrong{Attend all the classes} – the key to successfully passing the unit is to attend all of your classes. Mathematics is not a subject that can be learned easily in isolation just by reading the lecture notes. You will get much more out of the unit by attending, and more importantly, actively engaging in the classes.

\item {} 
\sphinxAtStartPar
\sphinxstylestrong{Complete all the exercises} – you would not expect an athlete to get faster or stronger without exercising and the same applies to studying mathematics. The tutorial exercises are designed to give you practice at using the various techniques and to help you to fully grasp the content. Try to make sure that you complete all the exercises before the following week’s lectures. Full worked solutions are provided but do try to be disciplined and avoid looking up the answers before you have attempted the questions.

\item {} 
\sphinxAtStartPar
\sphinxstylestrong{Catch up on missed work} – for whatever reason there may be a day when you cannot attend your classes, if this happens make sure you make the effort to catch up on missed work. Read through the appropriate chapter in the notes (as specified in the teaching schedule), complete the examples and attempt the tutorial exercises. It is very easy to start falling behind and the longer you leave it the more difficult it will be to catch up.

\item {} 
\sphinxAtStartPar
\sphinxstylestrong{Start the coursework as soon as you are able} – the coursework is released to students early in the teaching block, so you are not expected to be able to answer all the questions right away. As we progress through the unit you will be told which questions you should now be able to answer. Try to start these questions as soon as you can and not leave it to the last minute.

\item {} 
\sphinxAtStartPar
\sphinxstylestrong{Ask questions!} – perhaps the most important piece of advice here, there will be times when you are not quite sure about a concept, application or question. You can ask for help from the teaching staff and your fellow students. Mathematics is a hierarchical subject which means it requires full understanding at a fundamental level before moving onto more advanced topics, so if there are any gaps in your knowledge don’t be afraid to ask questions.

\end{itemize}

\sphinxstepscope


\part{Chapters}

\sphinxstepscope


\chapter{Matrices}
\label{\detokenize{_pages/1.0_Matrices:matrices}}\label{\detokenize{_pages/1.0_Matrices:matrices-chapter}}\label{\detokenize{_pages/1.0_Matrices::doc}}
\index{Matrix@\spxentry{Matrix}}\ignorespaces 
\sphinxAtStartPar
A \sphinxstylestrong{matrix} is a rectangular array of elements which can be numbers, mathematical expressions, symbols, or even other matrices. Matrices are arranged in rows and columns and enclosed by parentheses. For example:
\begin{equation*}
\begin{split} \begin{pmatrix}
    1 & 2 & 3 \\
    4 & 5 & 6
\end{pmatrix}.\end{split}
\end{equation*}
\sphinxAtStartPar
This matrix contains 6 elements arranged in 2 (horizontal) rows and 3 (vertical) columns. Sometimes matrices are shown using square brackets, or even no brackets at all i.e.,
\begin{equation*}
\begin{split} \begin{align*}
    &\begin{bmatrix}
        1 & 2 & 3 \\
        4 & 5 & 6
    \end{bmatrix}, &
    &\begin{matrix}
        1 & 2 & 3 \\
        4 & 5 & 6
    \end{matrix}
\end{align*} \end{split}
\end{equation*}
\sphinxAtStartPar
It is more useful to enclose a matrix in brackets, since we often write them next to other matrices or numbers, and the brackets make it clear where the matrix starts and ends, and what size it is.

\index{Matrix@\spxentry{Matrix}!dimension@\spxentry{dimension}}\ignorespaces \label{_pages/1.0_Matrices:matrix-dimension-definition}
\begin{sphinxadmonition}{note}{Definition 1.1 (Dimension of a matrix)}



\sphinxAtStartPar
The dimension or size of a matrix is written as \(m \times n\), where \(m\) and \(n\) are the number of \sphinxstylestrong{rows} and \sphinxstylestrong{columns} of the matrix respectively. We might say it is an “\(m\) by \(n\) matrix”. If \(m = n\), so the number of rows and columns are equal we say that the matrix is a \sphinxstylestrong{square matrix}.
\begin{equation*}
\begin{split} \begin{align*}
    &\,\,\,\begin{matrix} \leftarrow & \!\!\text{columns}\!\! & \rightarrow \end{matrix} \\
    \begin{matrix} \uparrow \\ \text{rows} \\ \downarrow \end{matrix}
    &\begin{pmatrix}
        \square & \square & \cdots & \square \\
        \square & \square & \cdots & \square \\
        \vdots & \vdots & \ddots & \vdots \\
        \square & \square & \cdots & \square
    \end{pmatrix}
\end{align*} \end{split}
\end{equation*}\end{sphinxadmonition}

\sphinxAtStartPar
For example, given the following matrices
\begin{equation*}
\begin{split} \begin{align*}
    A &= \begin{pmatrix} 1 & 2 \\ 3 & 4 \end{pmatrix}, &
    B &= \begin{pmatrix} a & b \\ c & d \\ e & f \end{pmatrix}, &
    C &= \begin{pmatrix} x & y & z & w \end{pmatrix}, &
    D &= \begin{pmatrix} \alpha & \beta & \gamma \\ \delta & \epsilon & \zeta \end{pmatrix},
\end{align*} \end{split}
\end{equation*}
\sphinxAtStartPar
we see that \(A\) is a \(2\times 2\) matrix, \(B\) is a \(3 \times 2\) matrix, \(C\) is a \(1 \times 4\) matrix and \(D\) is a \(2 \times 3\) matrix.


\bigskip\hrule\bigskip



\section{Labelling entries in a matrix}
\label{\detokenize{_pages/1.0_Matrices:labelling-entries-in-a-matrix}}\label{\detokenize{_pages/1.0_Matrices:indexing-a-matrix-section}}
\sphinxAtStartPar
Matrices are typically given names that are single uppercase characters, e.g., \(A\), and the elements of a matrix are labelled with the corresponding lowercase character, e.g. \(a\). The individual entries of a matrix can be indicated using two subscript indices: an entry would be denoted \(a_{ij}\) where \(i\) is the row number reading from top to bottom and \(j\) is the column number reading from left to right.

\begin{sphinxadmonition}{note}{Note:}
\sphinxAtStartPar
It will be generally useful to remember that the numbers represent \sphinxstylestrong{rows}, then \sphinxstylestrong{columns}. Come up with your own way of remembering that it’s rows first, then columns.
\end{sphinxadmonition}

\sphinxAtStartPar
If \(A\) is an \(m \times n\) matrix, then
\begin{equation*}
\begin{split} A =
\begin{pmatrix}
    a_{11} & a_{12} & \cdots & a_{1n} \\
    a_{21} & a_{22} & \cdots & a_{2n} \\
    \vdots & \vdots & \ddots & \vdots \\
    a_{m1} & a_{m2} & \cdots & a_{mn}
\end{pmatrix}. \end{split}
\end{equation*}
\sphinxAtStartPar
Some alternative notation used for matrix indexing are given below
\begin{equation*}
\begin{split} a_{ij} = [A]_{ij} = A(i,j). \end{split}
\end{equation*}
\sphinxAtStartPar
For example, given the matrix
\begin{equation*}
\begin{split}A = \begin{pmatrix} 2 & 0 & -3 \\ 1 & 7 & 4 \end{pmatrix},\end{split}
\end{equation*}
\sphinxAtStartPar
then we could write \(a_{12} = 0\), \(a_{21} = 1\), \([A]_{13} = -3\) and \(A(2,2) = 7\).

\sphinxstepscope


\section{Arithmetic operations on matrices}
\label{\detokenize{_pages/1.1_Matrix_operations:arithmetic-operations-on-matrices}}\label{\detokenize{_pages/1.1_Matrix_operations:matrix-operations-section}}\label{\detokenize{_pages/1.1_Matrix_operations::doc}}
\sphinxAtStartPar
So far, we have given a fancy name to a rectangular array of objects and showed how we can index its elements. Matrices are important objects in mathematics, and much like numbers themselves, matrices can be added together, multiplied, and modified in different ways. We will develop an \sphinxstylestrong{algebra} for matrices \sphinxhyphen{} a system the matrices inhabit, with all the necessary rules and definitions. This system resembles that of real numbers, but we will see some differences and new concepts.

\sphinxAtStartPar
For simplicity, we are going to assume that the entries of our matrices are numbers \sphinxhyphen{} however the developed theory applies to a broader range of objects, and it’s possible to define matrices with other types of object as the entries.

\index{Matrix@\spxentry{Matrix}!equality@\spxentry{equality}}\ignorespaces 

\subsection{Matrix equality}
\label{\detokenize{_pages/1.1_Matrix_operations:matrix-equality}}\label{\detokenize{_pages/1.1_Matrix_operations:index-0}}\label{_pages/1.1_Matrix_operations:matrix-equality-definition}
\begin{sphinxadmonition}{note}{Definition 1.2.1 (Matrix equality)}



\sphinxAtStartPar
We say that an \(m \times n\) matrix \(A\) and an \(p \times q\) matrix \(B\) are \sphinxstylestrong{equal} and write \(A = B\) if and only if \sphinxstylestrong{both} of the following conditions are satisfied:
\begin{itemize}
\item {} 
\sphinxAtStartPar
they have the same dimensions, i.e., \(m = p\) and \(n = q\)

\item {} 
\sphinxAtStartPar
for all \(1 \leq i \leq m\) and \(1 \leq j \leq n\), \(a_{ij} = b_{ij}\)

\end{itemize}
\end{sphinxadmonition}

\sphinxAtStartPar
For example, consider the following matrices
\begin{equation*}
\begin{split} \begin{align*}
    A &= \begin{pmatrix} 1 & 2 \\ 3 & 4 \end{pmatrix}, &
    B &= \begin{pmatrix} 1 & 2 & 5 \\ 3 & 4 & 6 \end{pmatrix}, &
    C &= \begin{pmatrix} 3^0 & \sqrt{4} \\ 1 + 2 & 2^2 \end{pmatrix}.
\end{align*} \end{split}
\end{equation*}
\sphinxAtStartPar
Here \(A \neq B\) since \(A\) has 2 columns and \(B\) has 3 columns. However, \(A=C\) because both \(A\) and \(C\) have the same number of rows and columns and all of the corresponding elements are equal.

\index{Matrix@\spxentry{Matrix}!addition@\spxentry{addition}}\ignorespaces 

\subsection{Matrix addition}
\label{\detokenize{_pages/1.1_Matrix_operations:matrix-addition}}\label{\detokenize{_pages/1.1_Matrix_operations:index-1}}\label{\detokenize{_pages/1.1_Matrix_operations:matrix-addition-section}}\label{_pages/1.1_Matrix_operations:matrix-addition-definition}
\begin{sphinxadmonition}{note}{Definition 1.2.2 (Matrix addition and subtraction)}



\sphinxAtStartPar
Let \(A\) and \(B\) be two \(m \times n\) matrices. The sum of two \(m \times n\) matrices \(A\) and \(B\) is an \(m \times n\) matrix \(A \pm B\) defined by:
\begin{equation}\label{equation:_pages/1.1_Matrix_operations:matrix-addition-equation}
\begin{split} [A + B]_{ij} = a_{ij} + b_{ij}, \end{split}
\end{equation}
\sphinxAtStartPar
where \(1 \leq i \leq m\) and \(1 \leq j \leq n\). For example, the sum of two \(2 \times 2\) matrices is given by matrix addition:
\begin{equation*}
\begin{split} \begin{align*}
    \begin{pmatrix}
        a_{11} & a_{12} \\
        a_{21} & a_{22}
    \end{pmatrix} +
    \begin{pmatrix}
        b_{11} & b_{12} \\
        b_{21} & b_{22}
    \end{pmatrix} =
    \begin{pmatrix}
        a_{11} + b_{11} & a_{12} + b_{12} \\
        a_{21} + b_{21} & a_{22} + b_{22}
    \end{pmatrix}.
\end{align*}  \end{split}
\end{equation*}
\sphinxAtStartPar
Similarly, we can define the difference of two matrices using matrix subtraction:
\begin{equation}\label{equation:_pages/1.1_Matrix_operations:matrix-subtraction-equation}
\begin{split} [A - B]_{ij} = a_{ij} - b_{ij}, \end{split}
\end{equation}
\sphinxAtStartPar
where \(1 \leq i \leq m\) and \(1 \leq j \leq n\).
\begin{equation*}
\begin{split} \begin{align*}
    \begin{pmatrix}
        a_{11} & a_{12} \\
        a_{21} & a_{22}
    \end{pmatrix} -
    \begin{pmatrix}
        b_{11} & b_{12} \\
        b_{21} & b_{22}
    \end{pmatrix} =
    \begin{pmatrix}
        a_{11} - b_{11} & a_{12} - b_{12} \\
        a_{21} - b_{21} & a_{22} - b_{22}
    \end{pmatrix}.
\end{align*} \end{split}
\end{equation*}\end{sphinxadmonition}

\sphinxAtStartPar
The addition and subtraction of two matrices of different sizes is \sphinxstylestrong{not defined}.
\label{_pages/1.1_Matrix_operations:properties-of-matrix-addition-theorem}
\begin{sphinxadmonition}{note}{Theorem 1.2.1 (Properties of matrix addition)}



\sphinxAtStartPar
For all \(m \times n\) matrices \(A,B\) and \(C\), the following conditions are satisfied:
\begin{itemize}
\item {} 
\sphinxAtStartPar
\(A + B = B + A\)    (commutative)

\item {} 
\sphinxAtStartPar
\(A + (B + C) = (A + B) + C\)    (associative)

\end{itemize}
\end{sphinxadmonition}
\label{_pages/1.1_Matrix_operations:matrix-addition-example}
\begin{sphinxadmonition}{note}{Example 1.2.1}



\sphinxAtStartPar
Evaluate the following:

\sphinxAtStartPar
(i)   \(\begin{pmatrix} 1 & 2 \\ 3 & 4 \end{pmatrix} + \begin{pmatrix} 5 & 6 \\ 7 & 8 \end{pmatrix}\);

\sphinxAtStartPar
(ii)   \(\begin{pmatrix} 2 \\ 3 \\ 5 \end{pmatrix} - \begin{pmatrix} 7 \\ -11 \\ -13 \end{pmatrix}\);

\sphinxAtStartPar
(iii)   \(\begin{pmatrix} 1 & 3 & 5 \\ 7 & 9 & 11 \end{pmatrix} + \begin{pmatrix}2 & 3 \\ 5 & 7 \end{pmatrix}\).
\subsubsection*{Solution}

\sphinxAtStartPar
(i)   \( \begin{pmatrix} 1 & 2 \\ 3 & 4 \end{pmatrix} + \begin{pmatrix} 5 & 6 \\ 7 & 8 \end{pmatrix} = \begin{pmatrix} 1 + 5 & 2 + 6 \\ 3 + 7 & 4 + 8 \end{pmatrix}= \begin{pmatrix}6 & 8 \\ 10 & 12 \end{pmatrix} \)

\sphinxAtStartPar
(ii)   \( \begin{pmatrix} 2 \\ 3 \\ 5 \end{pmatrix} - \begin{pmatrix} 7 \\ -11 \\ -13 \end{pmatrix} = \begin{pmatrix}2 - 7 \\ 3 + 11 \\ 5 + 13 \end{pmatrix} = \begin{pmatrix} -5 \\ 14 \\ 18\end{pmatrix} \)

\sphinxAtStartPar
(iii)   Undefined since the left matrix is \(2\times 3\) and the right matrix is \(2\times 2\)
\end{sphinxadmonition}

\index{Matrix@\spxentry{Matrix}!scalar multiplication@\spxentry{scalar multiplication}}\ignorespaces 

\subsection{Scalar multiplication}
\label{\detokenize{_pages/1.1_Matrix_operations:scalar-multiplication}}\label{\detokenize{_pages/1.1_Matrix_operations:index-2}}
\sphinxAtStartPar
\sphinxstylestrong{Scalar} is a term used to denote a value drawn from the same set we use for the entries in a matrix. Here, since we’re working with matrices of numbers, we can think of a scalar as simply a single number.
\label{_pages/1.1_Matrix_operations:scalar-multiplication-definition}
\begin{sphinxadmonition}{note}{Definition 1.2.3 (Scalar multiplication)}



\sphinxAtStartPar
Let \(A\) be a matrix and \(k \in \mathbb{R}\) be a scalar. The product of the scalar and the matrix, \(kA\), can be found using \sphinxstylestrong{scalar multiplication}, defined as:
\begin{equation}\label{equation:_pages/1.1_Matrix_operations:matrix-scalar-multiplication-equation}
\begin{split} [kA]_{ij} = k\cdot a_{ij}, \end{split}
\end{equation}
\sphinxAtStartPar
i.e., we multiply each element in the matrix by the scalar.
\begin{equation*}
\begin{split} \begin{align*}
    k
    \begin{pmatrix}
        a_{11} & a_{12} & \cdots & a_{1n} \\
        a_{21} & a_{22} & \cdots & a_{2n} \\
        \vdots & \vdots & \ddots & \vdots \\
        a_{m1} & a_{m2} & \cdots & a_{mn}
    \end{pmatrix} =
    \begin{pmatrix}
        ka_{11} & ka_{12} & \cdots & ka_{1n} \\
        ka_{21} & ka_{22} & \cdots & ka_{2n} \\
        \vdots & \vdots & \ddots & \vdots \\
        ka_{m1} & ka_{m2} & \cdots & ka_{mn}
    \end{pmatrix}.
\end{align*} \end{split}
\end{equation*}\end{sphinxadmonition}
\label{_pages/1.1_Matrix_operations:properties-of-scalar-multiplication-theorem}
\begin{sphinxadmonition}{note}{Theorem 1.2.2 (Properties of scalar multiplication)}



\sphinxAtStartPar
Let \(A\) and \(B\) be two \(m \times n\) matrices, and \(k\) and \(\ell\) be two scalars. Then
\begin{itemize}
\item {} 
\sphinxAtStartPar
\(kA = Ak\)    (commutative)

\item {} 
\sphinxAtStartPar
\(k (A + B) = kA + kB\)    (distributive over matrix addition)

\item {} 
\sphinxAtStartPar
\((k + \ell)A = kA +\ell A\)    (distributive over scalar addition)

\item {} 
\sphinxAtStartPar
\(k(\ell A) = (k \ell) A = \ell(kA)\)    (associative)

\item {} 
\sphinxAtStartPar
\(-1 \times A = -A\)    (multiplication by \(-1\) gives the additive inverse)

\end{itemize}
\end{sphinxadmonition}
\label{_pages/1.1_Matrix_operations:scalar-multiplication-example}
\begin{sphinxadmonition}{note}{Example 1.2.2}



\sphinxAtStartPar
Evaluate the following:

\sphinxAtStartPar
(i)   \(2 \begin{pmatrix} 1 & 2 \\ 3 & 4 \end{pmatrix}\)

\sphinxAtStartPar
(ii)   \(\dfrac{1}{2} \begin{pmatrix} 0 & -1 \\ 3 & 2 \\ 4  &-2  \end{pmatrix}\)

\sphinxAtStartPar
(iii)   \(a \begin{pmatrix} 1 & 6 & 4 \\ 0 & 3 & -1 \end{pmatrix}\)

\sphinxAtStartPar
(iv)   \(101 \begin{pmatrix} 1 & 2 \\ 0 & 1 \end{pmatrix} - 99 \begin{pmatrix} 1 & 2 \\ 0 & 1 \end{pmatrix}\)
\subsubsection*{Solution}

\sphinxAtStartPar
(i)   \( 2 \begin{pmatrix} 1 & 2 \\ 3 & 4 \end{pmatrix} = \begin{pmatrix} 2 & 4 \\ 6 & 8 \end{pmatrix} \)

\sphinxAtStartPar
(ii)   \(\dfrac{1}{2} \begin{pmatrix} 0 & -1 \\ 3 & 2 \\ 4  &-2  \end{pmatrix} = \begin{pmatrix} 0 & -1/2 \\ 3/2 & 1 \\ 2 & -1 \end{pmatrix} \)

\sphinxAtStartPar
(iii)   \(a \begin{pmatrix} 1 & 6 & 4 \\ 0 & 3 & -1 \end{pmatrix} = \begin{pmatrix} a & 6a & 4a \\ 0 & 3a & -a \end{pmatrix} \)

\sphinxAtStartPar
(iv)   \(101 \begin{pmatrix} 1 & 2 \\ 0 & 1 \end{pmatrix} - 99 \begin{pmatrix} 1 & 2 \\ 0 & 1 \end{pmatrix} = (101 - 99) \begin{pmatrix} 1 & 2 \\ 0 & 1 \end{pmatrix} = \begin{pmatrix} 2 & 4 \\ 0 & 2 \end{pmatrix}\)
\end{sphinxadmonition}


\bigskip\hrule\bigskip


\index{Matrix@\spxentry{Matrix}!transpose@\spxentry{transpose}}\ignorespaces 

\subsection{Matrix transpose}
\label{\detokenize{_pages/1.1_Matrix_operations:matrix-transpose}}\label{\detokenize{_pages/1.1_Matrix_operations:index-3}}\label{_pages/1.1_Matrix_operations:matrix-transpose-definition}
\begin{sphinxadmonition}{note}{Definition 1.2.4 (Matrix transpose)}



\sphinxAtStartPar
The \sphinxstylestrong{transpose} of an \(m \times n\) matrix \(A\) is an \(n \times m\) matrix denoted by \(A^\mathsf{T}\) formed by switching the rows and columns of \(A\), i.e.,
\begin{equation}\label{equation:_pages/1.1_Matrix_operations:matrix-transpose-equation}
\begin{split} [A^\mathsf{T}]_{ij}=a_{ji}. \end{split}
\end{equation}\end{sphinxadmonition}

\sphinxAtStartPar
Transposing a matrix switches the rows and columns around so that row \(i\) becomes column \(i\) and column \(j\) becomes row \(j\), i.e.,
\begin{equation*}
\begin{split} \begin{align*}
    \begin{pmatrix}
        a_{11} & a_{12} & \cdots & a_{1n} \\
        a_{21} & a_{22} & \cdots & a_{2n} \\
        \vdots & \vdots & \ddots & \vdots \\
        a_{m1} & a_{m2} & \cdots & a_{mn}
    \end{pmatrix}^\mathsf{T} =
    \begin{pmatrix}
        a_{11} & a_{21} & \cdots & a_{m1} \\
        a_{12} & a_{22} & \cdots & a_{m2} \\
        \vdots & \vdots & \ddots & \vdots \\
        a_{1n} & a_{2n} & \cdots & a_{mn}
    \end{pmatrix}.
\end{align*} \end{split}
\end{equation*}
\sphinxAtStartPar
For example:
\begin{equation*}
\begin{split} \begin{align*}
    \begin{pmatrix}
        \textcolor{blue}{2} & \textcolor{blue}{0} & \textcolor{blue}{3} \\
        4 & 1 & 5 \\
    \end{pmatrix}^\mathsf{T} =
    \begin{pmatrix}
        \textcolor{blue}{2} & 4 \\
        \textcolor{blue}{0} & 1 \\
        \textcolor{blue}{3} & 5
    \end{pmatrix}.
\end{align*} \end{split}
\end{equation*}
\sphinxAtStartPar
Note that what was previously the first row of the matrix is now the first column, and so on. The number of rows and columns in the matrix have been swapped.
\label{_pages/1.1_Matrix_operations:properties-of-matrix-transpose-theorem}
\begin{sphinxadmonition}{note}{Theorem 1.2.3 (Properties of matrix transpose)}



\sphinxAtStartPar
Let \(A\) and \(B\) be two square \(n \times n\) matrices and \(k\) a scalar, then
\begin{itemize}
\item {} 
\sphinxAtStartPar
\((A^\mathsf{T})^\mathsf{T} = A\)

\item {} 
\sphinxAtStartPar
\((A + B)^\mathsf{T} = A^\mathsf{T} + B^\mathsf{T}\)

\item {} 
\sphinxAtStartPar
\((k A)^\mathsf{T} = k (A^\mathsf{T})\)

\end{itemize}
\end{sphinxadmonition}
\label{_pages/1.1_Matrix_operations:matrix-transpose-example}
\begin{sphinxadmonition}{note}{Example 1.2.3}



\sphinxAtStartPar
Evaluate the following:

\sphinxAtStartPar
(i)   \(\begin{pmatrix} 1 & 3 \\ 2 & 4 \end{pmatrix}^\mathsf{T}\)

\sphinxAtStartPar
(ii)   \(\begin{pmatrix} 1 & 0 & -2 \\ 3 & -4 & 1 \end{pmatrix}^\mathsf{T}\)

\sphinxAtStartPar
(iii)   \(\begin{pmatrix}2 & 3 & 5 \end{pmatrix}^\mathsf{T}\)
\subsubsection*{Solution}

\sphinxAtStartPar
(i)   \( \begin{pmatrix} 1 & 3 \\ 2 & 4 \end{pmatrix}^\mathsf{T} = \begin{pmatrix} 1 & 2 \\ 3 & 4 \end{pmatrix} \)

\sphinxAtStartPar
(i)i   \( \begin{pmatrix} 1 & 0 & -2 \\ 3 & -4 & 1 \end{pmatrix}^\mathsf{T} = \begin{pmatrix} 1 & 3 \\ 0 & -4 \\ -2 & 1 \end{pmatrix} \)

\sphinxAtStartPar
(iii)   \( \begin{pmatrix}2 & 3 & 5 \end{pmatrix}^\mathsf{T} = \begin{pmatrix} 2 \\ 3 \\ 5 \end{pmatrix} \)
\end{sphinxadmonition}

\sphinxstepscope

\index{Matrix@\spxentry{Matrix}!matrix multiplication@\spxentry{matrix multiplication}}\ignorespaces 

\section{Matrix multiplication}
\label{\detokenize{_pages/1.2_Matrix_multiplication:matrix-multiplication}}\label{\detokenize{_pages/1.2_Matrix_multiplication:index-0}}\label{\detokenize{_pages/1.2_Matrix_multiplication:matrix-multiplication-section}}\label{\detokenize{_pages/1.2_Matrix_multiplication::doc}}
\sphinxAtStartPar
We saw in the {\hyperref[\detokenize{_pages/1.1_Matrix_operations:matrix-operations-section}]{\sphinxcrossref{\DUrole{std,std-ref}{previous section}}}} that we can perform the arithmetic operations of addition and subtraction on matrices, and multiply a matrix by a scalar. Here we will look at multiplication of two matrices.
\label{_pages/1.2_Matrix_multiplication:matrix-multiplication-definition}
\begin{sphinxadmonition}{note}{Definition 1.3.1 (Matrix multiplication)}



\sphinxAtStartPar
Let \(A\) be an \(m \times n\) matrix and \(B\) a \(p \times q\) matrix, the product \(AB\) is defined as
\begin{equation}\label{equation:_pages/1.2_Matrix_multiplication:matrix-multiplication-equation}
\begin{split} [AB]_{ij} = \sum_{k=1}^n a_{ik}b_{kj}. \end{split}
\end{equation}
\sphinxAtStartPar
The product \(AB\) is only defined if the number of columns in the first matrix is the same as the number of rows in the second matrix, i.e., \(n = p\), and the resulting matrix has the same number of rows as the first matrix and the same number of columns of the second matrix, i.e., a \(m \times q\) matrix.
\end{sphinxadmonition}

\sphinxAtStartPar
For example, the following pairs of matrices can be multiplied together:
\begin{equation*}
\begin{split} (2 \times \textcolor{red}{2}) \cdot (\textcolor{red}{2} \times 2) \qquad (4 \times \textcolor{red}{3}) \cdot (\textcolor{red}{3} \times 2) \qquad (3 \times \textcolor{red}{3}) \cdot (\textcolor{red}{3} \times 5) \end{split}
\end{equation*}
\sphinxAtStartPar
These pairs cannot:
\begin{equation*}
\begin{split} (2 \times \textcolor{red}{2}) \cdot (\textcolor{blue}{4} \times 3) \qquad (4 \times \textcolor{red}{5}) \cdot (\textcolor{blue}{3} \times 1) \qquad (1 \times \textcolor{red}{3}) \cdot (\textcolor{blue}{2} \times 2) \end{split}
\end{equation*}
\sphinxAtStartPar
To check whether matrix multiplication is defined for a given pair of matrices, one easy check is to write the dimensions of the matrices next to each other, e.g. underneath the matrices, and if the two inside numbers are the same, the matrix multiplication can be performed.
\begin{equation*}
\begin{split} \begin{align*}
    \underset{2 \times \textcolor{red}{2}}{\begin{pmatrix} a & b \\ c & d \end{pmatrix}}
    \underset{\textcolor{red}{2} \times 2}{\begin{pmatrix} e & f \\ g & h \end{pmatrix}}
\qquad
    \underset{3 \times \textcolor{red}{2}}{\begin{pmatrix} a & b \\ c & d \\ e & f \end{pmatrix}}
    \underset{\textcolor{red}{2} \times 4}{\begin{pmatrix} g & h & i & j \\  k & l & m & n \end{pmatrix}}
\end{align*} \end{split}
\end{equation*}
\sphinxAtStartPar
Furthermore, the dimensions of the product of these two matrices is given by the two outside numbers \sphinxhyphen{} in the example above, we’d get a \(2 \times 2\) matrix and a \(3 \times 4\) matrix.


\subsection{Multiplying two matrices}
\label{\detokenize{_pages/1.2_Matrix_multiplication:multiplying-two-matrices}}
\sphinxAtStartPar
The technique used to multiply two matrices together requires us to work across the horizontal rows of the first matrix (the \(i\) index) and down the vertical columns of the second matrix (the \(j\) index). We multiply the corresponding elements together, and sum the result. For example, consider the multiplication of the two \(2\times 2\) matrices \(A\) and \(B\) given below
\begin{equation*}
\begin{split} \begin{align*}
    A &= \begin{pmatrix} 1 & 2 \\ 3 & 4 \end{pmatrix}, &
    B &= \begin{pmatrix} 5 & 6 \\ 7 & 8 \end{pmatrix}.
\end{align*} \end{split}
\end{equation*}
\sphinxAtStartPar
The first thing we need to do is check whether matrix multiplication is defined for these matrices. They are \(2 \times \textcolor{red}{2}\) and \(\textcolor{red}{2} \times 2\), so this works.

\sphinxAtStartPar
Using the rule given in equation \eqref{equation:_pages/1.2_Matrix_multiplication:matrix-multiplication-equation}, to calculate \([AB]_{11}\) (the \((1,1)\)th entry in the product \(AB\)) we use row \(i=1\) and column \(j=1\). We move through the elements from row 1 of \(A\) and column 1 of \(B\), multiplying the corresponding values and summing the result:
\begin{equation*}
\begin{split} \begin{align*}
    \begin{pmatrix}
        {\color{blue}{1}} & {\color{blue}{2}} \\
        {\color{lightgray}{3}} & {\color{lightgray}{4}}
    \end{pmatrix}
    \begin{pmatrix}
        {\color{red}{5}} & {\color{lightgray}{6}} \\
        {\color{red}{7}} & {\color{lightgray}{8}}
    \end{pmatrix} =
    \begin{pmatrix}
        {\color{blue}{1}} \cdot {\color{red}{5}} + {\color{blue}{2}} \cdot {\color{red}{7}} & {\color{lightgray}{\square}} \\
        {\color{lightgray}{\square}} & {\color{lightgray}{\square}}
    \end{pmatrix}
    = \begin{pmatrix} 5 + 14 & {\color{lightgray}{\square}} \\ {\color{lightgray}{\square}} & {\color{lightgray}{\square}} \end{pmatrix}
    = \begin{pmatrix} 19 & {\color{lightgray}{\square}} \\ {\color{lightgray}{\square}} & {\color{lightgray}{\square}} \end{pmatrix}.
\end{align*} \end{split}
\end{equation*}
\sphinxAtStartPar
We can write this as:
\begin{equation*}
\begin{split} [AB]_{11} = a_{11}b_{11} + a_{12}b_{21} = 1 \cdot 5 + 2 \cdot 7 = 5 + 14 = 19\end{split}
\end{equation*}
\sphinxAtStartPar
Carrying on across the top row of our product matrix \(AB\): next we need to calculate the value of \([AB]_{12}\). Now our row number is \(i = 1\) and column number is \(j = 2\), so we multiply the values from row 1 of \(A\) and column 2 of \(B\) and sum the result.
\begin{equation*}
\begin{split} \begin{align*}
    \begin{pmatrix}
        \textcolor{blue}{1} & \textcolor{blue}{2} \\
        \textcolor{lightgray}{3} & \textcolor{lightgray}{4}
    \end{pmatrix}
    \begin{pmatrix}
        \textcolor{lightgray}{5} & \textcolor{red}{6} \\
        \textcolor{lightgray}{7} & \textcolor{red}{8}
    \end{pmatrix} =
    \begin{pmatrix}
        19 & \textcolor{blue}{1} \cdot \textcolor{red}{6} + \textcolor{blue}{2} \cdot \textcolor{red}{8} \\
        \textcolor{lightgray}{\square} & \textcolor{lightgray}{\square}
    \end{pmatrix}
    = \begin{pmatrix} 19 & 6 + 16 \\ \textcolor{lightgray}{\square} & \textcolor{lightgray}{\square} \end{pmatrix}
    = \begin{pmatrix} 19 & 22 \\ \textcolor{lightgray}{\square} & \textcolor{lightgray}{\square} \end{pmatrix}.
\end{align*} \end{split}
\end{equation*}
\sphinxAtStartPar
Or in other words,
\begin{equation*}
\begin{split} [AB]_{12} = a_{11}b_{12} + a_{12}b_{22} = 1 \cdot 6 + 2 \cdot 8 = 6 + 16 = 22\end{split}
\end{equation*}
\sphinxAtStartPar
Now that we have finished the first row of \(AB\) we now move down to the second row and back to the first column to calculate \([AB]_{21}\). Since \(i = 2\) and \(j = 1\), we multiply the values from row 2 of \(A\) and column 1 of \(B\) and sum the result
\begin{equation*}
\begin{split} \begin{align*}
    \begin{pmatrix}
        \textcolor{lightgray}{1} & \textcolor{lightgray}{2} \\
        \textcolor{blue}{3} & \textcolor{blue}{4}
    \end{pmatrix}
    \begin{pmatrix}
        \textcolor{red}{5} & \textcolor{lightgray}{6} \\
        \textcolor{red}{7} & \textcolor{lightgray}{8}
    \end{pmatrix} =
    \begin{pmatrix}
        19 & 22 \\
        \textcolor{blue}{3} \cdot \textcolor{red}{5} + \textcolor{blue}{4} \cdot \textcolor{red}{7} & \textcolor{lightgray}{\square}
    \end{pmatrix}
    = \begin{pmatrix} 19 & 22 \\ 15 + 28 & \textcolor{lightgray}{\square} \end{pmatrix}
    = \begin{pmatrix} 19 & 22 \\ 43 & \textcolor{lightgray}{\square} \end{pmatrix}.
\end{align*} \end{split}
\end{equation*}
\sphinxAtStartPar
That is,
\begin{equation*}
\begin{split}[AB]_{21} = a_{21} b_{11} + a_{22} b_{21} = 3 \cdot 5 + 4 \cdot 7 = 15 + 28 = 43\end{split}
\end{equation*}
\sphinxAtStartPar
Finally we calculate the last element of the product matrix, \([AB]_{22}\). Since \(i = 2\) and \(j = 2\), we multiply the values from row 2 of \(A\) and column 2 of \(B\) and sum the result.
\begin{equation*}
\begin{split} \begin{align*}
    \begin{pmatrix}
        \textcolor{lightgray}{1} & \textcolor{lightgray}{2} \\
        \textcolor{blue}{3} & \textcolor{blue}{4}
    \end{pmatrix}
    \begin{pmatrix}
        \textcolor{lightgray}{5} & \textcolor{red}{6} \\
        \textcolor{lightgray}{7} & \textcolor{red}{8}
    \end{pmatrix} =
    \begin{pmatrix}
        19 & 22 \\
        43 & \textcolor{blue}{3} \cdot \textcolor{red}{6} + \textcolor{blue}{4} \cdot \textcolor{red}{8}
    \end{pmatrix}
    = \begin{pmatrix} 19 & 22 \\ 43 & 18 + 32 \end{pmatrix}
    = \begin{pmatrix} 19 & 22 \\ 43 & 50 \end{pmatrix}.
\end{align*} \end{split}
\end{equation*}
\sphinxAtStartPar
In other words,
\begin{equation*}
\begin{split}[AB]_{22} = a_{21} b_{12} + a_{22} b_{22} = 3 \cdot 6 + 4 \cdot 8 = 18 + 32 = 50\end{split}
\end{equation*}


\sphinxAtStartPar
Multiplying together larger matrices simply involves adding together the products of all the terms in each row and column, and repeating this for each entry in the product matrix. This is why we require the rows in the first matrix to be the same length as the rows in the second: we need to multiply together one from each, and add the products.
\label{_pages/1.2_Matrix_multiplication:matrix-multiplication-example}
\begin{sphinxadmonition}{note}{Example 1.3.1}



\sphinxAtStartPar
Given the matrices
\begin{equation*}
\begin{split} \begin{align*}
    A &= \begin{pmatrix} 1 & 0 \\ -2 & 3 \end{pmatrix}, &
    B &= \begin{pmatrix} 2 & 3 \\ 1 & 5 \end{pmatrix}, &
    C &= \begin{pmatrix} 1 & 1 & 0 \\ 3 & -2 & 1 \end{pmatrix}, &
    D &= \begin{pmatrix} 1 \\ 3 \end{pmatrix}.
\end{align*} \end{split}
\end{equation*}
\sphinxAtStartPar
calculate the following (where possible):

\sphinxAtStartPar
(i)   \(AB\);  
(ii)   \(BC\);  
(iii)   \(CD\);  
(iv)   \(CC^\mathsf{T}\)
\subsubsection*{Solution}

\sphinxAtStartPar
(i)  
\begin{equation*}
\begin{split} \begin{align*}
    AB &= \begin{pmatrix} 1 & 0 \\ -2 & 3 \end{pmatrix} \begin{pmatrix} 2 & 3 \\ 1 & 5 \end{pmatrix}
    = \begin{pmatrix} 1 \cdot 2 + 0 \cdot 1 & 1 \cdot 3 + 0 \cdot 5\\ -2 \cdot 2 + 3 \cdot 1 & -2 \cdot 3 + 3 \cdot 5 \end{pmatrix} \\
    &= \begin{pmatrix} 2 + 0 & 3 + 0 \\ -4 + 3 & -6 + 15\end{pmatrix}
    = \begin{pmatrix} 2 & 3 \\ -1 & 9 \end{pmatrix}
\end{align*} \end{split}
\end{equation*}
\sphinxAtStartPar
(ii)  
\begin{equation*}
\begin{split} \begin{align*}
    BC &= \begin{pmatrix} 2 & 3 \\ 1 & 5 \end{pmatrix} \begin{pmatrix} 1 & 1 & 0 \\ 3 & -2 & 1 \end{pmatrix}
    = \begin{pmatrix}
        2 \cdot 1 + 3 \cdot 3 & 2 \cdot 1 + 3 \cdot -2 & 2 \cdot 0 + 3 \cdot 1 \\
        1 \cdot 1 + 5 \cdot 3 & 1 \cdot 1 + 5 \cdot -2 & 1 \cdot 0 + 5 \cdot 1
    \end{pmatrix} \\
    &= \begin{pmatrix} 2+9 & 2-6 & 0+3 \\ 1+15 & 1-10 & 0+5 \end{pmatrix}
    = \begin{pmatrix}11 & -4 & 3 \\ 16 & -9 & 5 \end{pmatrix}
\end{align*} \end{split}
\end{equation*}
\sphinxAtStartPar
(iii)   \(CD\) is undefined since \(C\) has 3 columns and \(D\) only has 2 rows.

\sphinxAtStartPar
(iv)
\begin{equation*}
\begin{split}\begin{align*}
    CC^\mathsf{T} &=
    \begin{pmatrix} 1 & 1 & 0 \\ 3 & -2 & 1 \end{pmatrix}
    \begin{pmatrix} 1 & 3 \\ 1 & -2 \\ 0 & 1 \end{pmatrix}
    =
    \begin{pmatrix}
        1 \cdot 1 + 1 \cdot 1 + 0 \cdot 0 & 1 \cdot 3 + 1 \cdot -2 + 0 \cdot 1 \\
        3 \cdot 1 + -2 \cdot 1 + 1 \cdot 0 & 3 \cdot 3 + -2 \cdot -2 + 1 \cdot 1
    \end{pmatrix} \\
    &=
    \begin{pmatrix}
        1 + 1 + 0 & 3 - 2 + 0 \\
        3 - 2 + 0 & 9 + 4 + 1
    \end{pmatrix}
    =
    \begin{pmatrix}
        2 & 1 \\
        1 & 14
    \end{pmatrix}
\end{align*} \end{split}
\end{equation*}\end{sphinxadmonition}

\sphinxAtStartPar
The method of multiplying matrices together may seem overly complicated, but doing it this way gives us a consistent definition of multiplication that (mostly) behaves the way we expect it to.
\label{_pages/1.2_Matrix_multiplication:theorem-2}
\begin{sphinxadmonition}{note}{Theorem 1.3.1 (Properties of matrix multiplication)}



\sphinxAtStartPar
The following properties hold for matrix multiplication
\begin{itemize}
\item {} 
\sphinxAtStartPar
\(AB \neq BA\)   (matrix multiplication is not commutative)

\item {} 
\sphinxAtStartPar
\(A(BC) = (AB)C\)   (associative)

\item {} 
\sphinxAtStartPar
\(A(B + C) = AB + AC\)   (left distributive law)

\item {} 
\sphinxAtStartPar
\((A + B)C = AC + BC\)   (right distributive law)

\item {} 
\sphinxAtStartPar
\((AB)^\mathsf{T} = B^\mathsf{T}A^\mathsf{T}\)

\end{itemize}
\end{sphinxadmonition}


\bigskip\hrule\bigskip


\index{Matrix@\spxentry{Matrix}!exponents@\spxentry{exponents}}\ignorespaces 

\subsection{Matrix exponents}
\label{\detokenize{_pages/1.2_Matrix_multiplication:matrix-exponents}}\label{\detokenize{_pages/1.2_Matrix_multiplication:index-1}}
\sphinxAtStartPar
Just like with scalar quantities we can calculate the exponent of a number \(a^n\) by multiplying by itself \(n\) times, i.e., \(a^3 = a \cdot a \cdot a\), we can also do this for square matrices that have the same number of rows and columns.
\label{_pages/1.2_Matrix_multiplication:matrix-exponents-definition}
\begin{sphinxadmonition}{note}{Definition 1.3.2 (Matrix exponents)}



\sphinxAtStartPar
Let \(A\) be a square \(n \times n\) matrix. Then we write \(A^2=AA\) and more generally:
\begin{equation}\label{equation:_pages/1.2_Matrix_multiplication:matrix-exponent-equation}
\begin{split} \begin{align*}
    A^n = \underbrace{A A \cdots A}_{n \textsf{ times}}.
\end{align*} \end{split}
\end{equation}\end{sphinxadmonition}
\label{_pages/1.2_Matrix_multiplication:matrix-exponents-example}
\begin{sphinxadmonition}{note}{Example 1.3.2}



\sphinxAtStartPar
Given the matrix
\begin{equation*}
\begin{split} A = \begin{pmatrix} 1 & 2 \\ 3 & 4 \end{pmatrix}, \end{split}
\end{equation*}
\sphinxAtStartPar
evaluate:

\sphinxAtStartPar
(i)   \(A^2\);   (ii)   \(A^3\);   (iii)   \(A^5\)
\subsubsection*{Solution}

\sphinxAtStartPar
(i)
\begin{equation*}
\begin{split} \begin{align*}
    A^2 = AA &= \begin{pmatrix} 1 & 2 \\ 3 & 4 \end{pmatrix}
    \begin{pmatrix} 1 & 2 \\ 3 & 4 \end{pmatrix}
    = \begin{pmatrix} 1 + 6 & 2 + 8 \\ 3 + 12 & 6 + 16 \end{pmatrix}
    = \begin{pmatrix} 7 & 10 \\ 15 & 22 \end{pmatrix}
\end{align*} \end{split}
\end{equation*}
\sphinxAtStartPar
(ii)
\begin{equation*}
\begin{split} \begin{align*}
    A^3 = AA^2 &=
    \begin{pmatrix} 1 & 2 \\ 3 & 4 \end{pmatrix}
    \begin{pmatrix} 7 & 10 \\ 15 & 22 \end{pmatrix}
    = \begin{pmatrix} 7 + 30 & 10 + 44 \\ 21 + 60 & 30 + 88 \end{pmatrix}
    = \begin{pmatrix} 37 & 54 \\ 81 & 118 \end{pmatrix}
\end{align*} \end{split}
\end{equation*}
\sphinxAtStartPar
(iii)
\begin{equation*}
\begin{split} \begin{align*}
    A^5 = A^2A^3 &=
    \begin{pmatrix} 7 & 10 \\ 15 & 22 \end{pmatrix}
    \begin{pmatrix} 37 & 54 \\ 81 & 118 \end{pmatrix}
    = \begin{pmatrix} 259 + 810 & 378 + 1180 \\ 555 + 1782 & 810 + 2596 \end{pmatrix}
    = \begin{pmatrix} 1069 & 1558 \\ 2337 & 3406 \end{pmatrix}
\end{align*} \end{split}
\end{equation*}\end{sphinxadmonition}

\sphinxstepscope

\index{Matrix@\spxentry{Matrix}!special matrices@\spxentry{special matrices}}\ignorespaces 

\section{Special matrices}
\label{\detokenize{_pages/1.3_Special_matrices:special-matrices}}\label{\detokenize{_pages/1.3_Special_matrices:index-0}}\label{\detokenize{_pages/1.3_Special_matrices::doc}}
\sphinxAtStartPar
Some matrices have certain properties which makes them useful for various mathematical applications. Understanding special matrices and their properties is important for gaining a deeper insight into linear algebra and its practical applications.

\index{Matrix@\spxentry{Matrix}!square matrix@\spxentry{square matrix}}\ignorespaces 

\begin{savenotes}\sphinxattablestart
\sphinxthistablewithglobalstyle
\centering
\phantomsection\label{\detokenize{_pages/1.3_Special_matrices:index-1}}\nobreak
\begin{tabulary}{\linewidth}[t]{TTT}
\sphinxtoprule
\sphinxstyletheadfamily 
\sphinxAtStartPar
Name
&\sphinxstyletheadfamily 
\sphinxAtStartPar
Definition
&\sphinxstyletheadfamily 
\sphinxAtStartPar
Example
\\
\sphinxmidrule
\sphinxtableatstartofbodyhook
\sphinxAtStartPar
Square matrix
&
\sphinxAtStartPar
An \(m \times n\) matrix where \(m = n\)
&
\sphinxAtStartPar
\(\begin{pmatrix} 1 & 2 \\ 3 & 4 \end{pmatrix}\)
\\
\sphinxhline
\sphinxAtStartPar
Zero matrix
&
\sphinxAtStartPar
An \(m \times n\) matrix of zeros
&
\sphinxAtStartPar
\(\begin{pmatrix} 0 & 0 & 0 \\ 0 & 0 & 0 \end{pmatrix}\)
\\
\sphinxhline
\sphinxAtStartPar
Diagonal matrix
&
\sphinxAtStartPar
A non\sphinxhyphen{}zero square matrix where \(a_{ij} = 0\) where \(i \neq j\)
&
\sphinxAtStartPar
\(\begin{pmatrix} 1 & 0 & 0 \\ 0 & 2 & 0 \\ 0 & 0 & 3 \end{pmatrix}\)
\\
\sphinxhline
\sphinxAtStartPar
Identity matrix
&
\sphinxAtStartPar
A square matrix \(I\) such that \([I]_{ij} = \begin{cases} 1, &i = j, \\ 0, & \text{otherwise} \end{cases}\)
&
\sphinxAtStartPar
\(\begin{pmatrix} 1 & 0 & 0 \\ 0 & 1 & 0 \\ 0 & 0 & 1 \end{pmatrix} \)
\\
\sphinxhline
\sphinxAtStartPar
Symmetric matrix
&
\sphinxAtStartPar
A square matrix \(A\) such that \(a_{ij} = a_{ji}\)
&
\sphinxAtStartPar
\(\begin{pmatrix} 1 & 2 & 3 \\ 2 & 1 & 4 \\ 3 & 4 & 1 \end{pmatrix}\)
\\
\sphinxhline
\sphinxAtStartPar
Upper triangular matrix
&
\sphinxAtStartPar
A square matrix \(A\) such that \(a_{ij} = 0\) where \(i > j\)
&
\sphinxAtStartPar
\(\begin{pmatrix} 1 & 2 & 3 \\ 0 & 4 & 5 \\ 0 & 0 & 6 \end{pmatrix}\)
\\
\sphinxhline
\sphinxAtStartPar
Lower triangular matrix
&
\sphinxAtStartPar
A square matrix \(A\) such that \(a_{ij} = 0\) where \(i < j\)
&
\sphinxAtStartPar
\(\begin{pmatrix} 1 & 0 & 0 \\ 2 & 3 & 0 \\ 4 & 5 & 6 \end{pmatrix}\)
\\
\sphinxbottomrule
\end{tabulary}
\sphinxtableafterendhook\par
\sphinxattableend\end{savenotes}

\index{Matrix@\spxentry{Matrix}!square matrix@\spxentry{square matrix}}\ignorespaces 
\index{Matrix@\spxentry{Matrix}!main diagonal@\spxentry{main diagonal}}\ignorespaces 
\index{Matrix@\spxentry{Matrix}!diagonal matrix@\spxentry{diagonal matrix}}\ignorespaces 
\index{Matrix@\spxentry{Matrix}!zero matrix@\spxentry{zero matrix}}\ignorespaces 
\index{Matrix@\spxentry{Matrix}!identity matrix@\spxentry{identity matrix}}\ignorespaces 
\index{Matrix@\spxentry{Matrix}!symmetric matrix@\spxentry{symmetric matrix}}\ignorespaces \label{_pages/1.3_Special_matrices:theorem-0}
\begin{sphinxadmonition}{note}{Theorem 1.4.1 (Properties of special matrices)}


\begin{itemize}
\item {} 
\sphinxAtStartPar
If \(A = A^\mathsf{T}\) then \(A\) is a symmetric matrix

\item {} 
\sphinxAtStartPar
If \(A\) is an upper triangular matrix, then \(A^\mathsf{T}\) is a lower triangular matrix, and vice versa.

\end{itemize}

\sphinxAtStartPar
We say the zero matrix is the \sphinxstylestrong{identity element} or sometimes a \sphinxstylestrong{neutral element}, with respect to matrix addition, since it doesn’t change the matrix it’s added to: i.e., for any \(m \times n\) matrix \(A\):
\begin{itemize}
\item {} 
\sphinxAtStartPar
\( A + \mathbf{0} = \mathbf{0} + A = A\)

\end{itemize}

\sphinxAtStartPar
However, for multiplication, it is not neutral:
\begin{itemize}
\item {} 
\sphinxAtStartPar
\( A \mathbf{0} = \mathbf{0} A = \mathbf{0}\)

\end{itemize}

\sphinxAtStartPar
The identity matrix is called this because it’s the identity element for multiplication:
\begin{itemize}
\item {} 
\sphinxAtStartPar
\(I A = A I = A\)

\end{itemize}

\sphinxAtStartPar
It also allows us to define {\hyperref[\detokenize{_pages/1.5_Inverse_matrix:inverse-matrix-section}]{\sphinxcrossref{\DUrole{std,std-ref}{inverse matrices}}}}, which we’ll cover next:
\begin{itemize}
\item {} 
\sphinxAtStartPar
\(AA^{-1} = I\) where \(A^{-1}\) is the inverse of \(A\)

\end{itemize}
\end{sphinxadmonition}

\sphinxstepscope

\index{Determinants@\spxentry{Determinants}}\ignorespaces 
\index{Matrix@\spxentry{Matrix}!determinant@\spxentry{determinant}}\ignorespaces 

\section{Determinants}
\label{\detokenize{_pages/1.4_Determinants:determinants}}\label{\detokenize{_pages/1.4_Determinants:index-1}}\label{\detokenize{_pages/1.4_Determinants:index-0}}\label{\detokenize{_pages/1.4_Determinants:determinant-section}}\label{\detokenize{_pages/1.4_Determinants::doc}}
\sphinxAtStartPar
A \sphinxstylestrong{determinant} is a scalar value that is calculated using the elements of a square matrix (non\sphinxhyphen{}square matrices do not have a determinant).  Determinants play a very important role in linear algebra, and tell us about the properties of the matrix. One of their uses is they enable us to determine if a {\hyperref[\detokenize{_pages/2.0_Linear_systems:systems-of-linear-equations-chapter}]{\sphinxcrossref{\DUrole{std,std-ref}{system of linear equations}}}} has a unique solution (as we will see later).

\sphinxAtStartPar
The determinant of a square matrix \(A\) is denoted by \(\det(A)\) or \(|A|\), and is a scalar value that can be computed from the values of its elements.

\index{Determinants@\spxentry{Determinants}!2 x 2@\spxentry{2 x 2}}\ignorespaces 

\bigskip\hrule\bigskip



\subsection{Calculating the determinant of a \protect\(2 \times 2\protect\) matrix}
\label{\detokenize{_pages/1.4_Determinants:calculating-the-determinant-of-a-2-times-2-matrix}}\label{_pages/1.4_Determinants:2x2-determinant-definition}
\begin{sphinxadmonition}{note}{Definition 1.5.1 (Determinant of a \protect\(2 \times 2\protect\) matrix)}



\sphinxAtStartPar
The determinant of the \(2 \times 2\) matrix \(A = \begin{pmatrix}a & b \\ c & d \end{pmatrix}\) is
\begin{equation}\label{equation:_pages/1.4_Determinants:2x2-determinant-equation}
\begin{split} \det(A) = |A| = \det \begin{pmatrix} a & b \\ c & d \end{pmatrix} =
\begin{vmatrix} a & b \\ c & d \end{vmatrix} = ad - bc, \end{split}
\end{equation}
\sphinxAtStartPar
i.e., the product of the elements on the main diagonal minus the product of the other two elements.
\end{sphinxadmonition}

\begin{sphinxadmonition}{note}{Note:}
\sphinxAtStartPar
The formula for the determinant, \(ad-bc\), will be a useful thing to remember. Come up with your own way of remembering it!
\end{sphinxadmonition}
\label{_pages/1.4_Determinants:2x2-determinant-example}
\begin{sphinxadmonition}{note}{Example 1.5.1}



\sphinxAtStartPar
Calculate the following determinants

\sphinxAtStartPar
(i)   \(\begin{vmatrix} 5 & 2 \\ 3 & 4 \end{vmatrix}\);  
(ii)   \(\det \begin{pmatrix} a & b \\ ka & kb \end{pmatrix}\);  
(iii)   \(\begin{vmatrix} 2-\lambda & 3 \\ 5 & 6 - \lambda \end{vmatrix}\)
\subsubsection*{Solution}

\sphinxAtStartPar
(i)   \( \begin{vmatrix} 5 & 2 \\ 3 & 4 \end{vmatrix} = 5 \cdot 4 - 2 \cdot 3 = 20 - 6 = 14 \)

\sphinxAtStartPar
(ii)   \( \det \begin{pmatrix} a & b \\ ka & kb \end{pmatrix} = akb - akb = 0 \)

\sphinxAtStartPar
(iii)   \( \begin{vmatrix} 2-\lambda & 3 \\ 5 & 6 - \lambda \end{vmatrix} = (2-\lambda)(6-\lambda) - 3 \cdot 5 = \lambda^2 - 8\lambda - 3 \)
\end{sphinxadmonition}


\bigskip\hrule\bigskip


\index{Determinants@\spxentry{Determinants}!n x n@\spxentry{n x n}}\ignorespaces 

\subsection{Determinant of \protect\(n \times n\protect\) matrices}
\label{\detokenize{_pages/1.4_Determinants:determinant-of-n-times-n-matrices}}\label{\detokenize{_pages/1.4_Determinants:index-3}}
\sphinxAtStartPar
Computing the determinant of a \(2\times 2\) matrix is relatively straightforward, but computing the determinant of a matrix larger than \(2\times 2\) gets more complicated.

\sphinxAtStartPar
Essentially, we split the larger matrix up into multiple \(2\times 2\) matrices, so we can use equation \eqref{equation:_pages/1.4_Determinants:2x2-determinant-equation} repeatedly to obtain the overall determinant. This is done in a specific way explained below which uses \sphinxstylestrong{minors} and \sphinxstylestrong{cofactors}.


\subsubsection{Minors}
\label{\detokenize{_pages/1.4_Determinants:minors}}
\index{Determinants@\spxentry{Determinants}!minor@\spxentry{minor}}\ignorespaces \label{_pages/1.4_Determinants:minor-definition}
\begin{sphinxadmonition}{note}{Definition 1.5.2 (Minor)}



\sphinxAtStartPar
The \sphinxstylestrong{minor} of an element of an \(n \times n\) square matrix is denoted by \(M_{ij}\) and is the determinant of the \((n-1) \times (n-1)\) square matrix that is formed by removing row \(i\) and column \(j\) from \(A\).
\end{sphinxadmonition}

\sphinxAtStartPar
For example, given the matrix \(A\)
\begin{equation*}
\begin{split} A = \begin{pmatrix}
    a_{11} & a_{12} & a_{13} \\
    a_{21} & a_{22} & a_{23} \\
    a_{31} & a_{32} & a_{33}
\end{pmatrix}, \end{split}
\end{equation*}
\sphinxAtStartPar
then the minor \(M_{21}\) is the determinant of the matrix \(A\) with row 2 and column 1 removed
\begin{equation*}
\begin{split} M_{21} = \begin{vmatrix}
    \textcolor{lightgray}{\square} & a_{12} & a_{13} \\
    \textcolor{lightgray}{\square} & \textcolor{lightgray}{\square} & \textcolor{lightgray}{\square} \\
    \textcolor{lightgray}{\square} & a_{32} & a_{33}
\end{vmatrix}
= \begin{vmatrix}
    a_{12} & a_{13} \\
    a_{32} & a_{33}
\end{vmatrix}
= a_{12}a_{33} - a_{13}a_{32}. \end{split}
\end{equation*}\label{_pages/1.4_Determinants:minor-example}
\begin{sphinxadmonition}{note}{Example 1.5.2}



\sphinxAtStartPar
Given the matrix
\begin{equation*}
\begin{split} A = \begin{pmatrix} 1 & 2 & 3 \\ 4 & 5 & 6 \\ 7 & 8 & 9 \end{pmatrix},\end{split}
\end{equation*}
\sphinxAtStartPar
calculate:

\sphinxAtStartPar
(i)   \(M_{11}\);  
(ii)   \(M_{12}\);  
(iii)   \(M_{13}\)
\subsubsection*{Solution}

\sphinxAtStartPar
(i)   \( M_{11} = \begin{vmatrix} 5 & 6 \\ 8 & 9 \end{vmatrix} = 5(9) - 6(8) = 45 - 48 = -3 \)

\sphinxAtStartPar
(ii)   \( M_{12} = \begin{vmatrix} 4 & 6 \\ 7 & 9 \end{vmatrix} = 4(9) - 6(7) = 36 - 42 = -6 \)

\sphinxAtStartPar
(iii)   \( M_{13} = \begin{vmatrix} 4 & 5 \\ 7 & 8 \end{vmatrix} = 4(8) - 5(7) = 32 - 35 = -3 \)
\end{sphinxadmonition}


\subsubsection{Cofactors}
\label{\detokenize{_pages/1.4_Determinants:cofactors}}
\index{Determinants@\spxentry{Determinants}!cofactor@\spxentry{cofactor}}\ignorespaces \label{_pages/1.4_Determinants:cofactor-definition}
\begin{sphinxadmonition}{note}{Definition 1.5.3 (Cofactor)}



\sphinxAtStartPar
The \sphinxstylestrong{cofactor} of an element of a square matrix is denoted by \(C_{ij}\) and is defined by
\begin{equation}\label{equation:_pages/1.4_Determinants:cofactor-equation}
\begin{split} C_{ij} = (-1)^{i+j}M_{ij}. \end{split}
\end{equation}\end{sphinxadmonition}

\sphinxAtStartPar
The \((-1)^{i+j}\) term in equation \eqref{equation:_pages/1.4_Determinants:cofactor-equation} is positive when \(i + j\) is even and negative when \(i + j\) is odd which results in the following pattern of signs
\begin{equation*}
\begin{split} \begin{pmatrix}
        + & - & + & \cdots \\
        - & + & - & \cdots \\
        + & - & + & \cdots \\
        \vdots & \vdots & \vdots & \ddots
\end{pmatrix}. \end{split}
\end{equation*}

\subsubsection{Calculating the determinant of an \protect\(n \times n\protect\) matrix}
\label{\detokenize{_pages/1.4_Determinants:calculating-the-determinant-of-an-n-times-n-matrix}}
\sphinxAtStartPar
Now we have the terminology of minors and cofactors, we can use them to calculate determinants. For a larger matrix, we choose a single row or column of the matrix, and work our way along it.
\label{_pages/1.4_Determinants:nxn-determinant-definition}
\begin{sphinxadmonition}{note}{Definition 1.5.4 (Determinant of an \protect\(n \times n\protect\) matrix)}



\sphinxAtStartPar
The determinant of an \(n\times n\) matrix \(A\) is defined by
\begin{equation}\label{equation:_pages/1.4_Determinants:nxn-determinant-equation}
\begin{split} \det(A) = \sum_{i=1}^n a_{ik} C_{ik} = \sum_{j=1}^n a_{kj} C_{kj}, \end{split}
\end{equation}
\sphinxAtStartPar
for some fixed value in the range \(1 \leq k \leq n\) which represents a single row or column of \(A\).
\end{sphinxadmonition}

\sphinxAtStartPar
Equation \eqref{equation:_pages/1.4_Determinants:nxn-determinant-equation} allows us to express the determinant of an \(n \times n\) matrix in terms of determinants of \((n-1) \times (n-1)\) matrices. We can then apply the formula again to the sub\sphinxhyphen{}matrices. Continuing in this fashion we will eventually just be calculating \(2\times 2\) matrices, which we know how to do from equation \eqref{equation:_pages/1.4_Determinants:2x2-determinant-equation}.

\sphinxAtStartPar
For example, to calculate the determinant of the matrix
\begin{equation*}
\begin{split} A = \begin{pmatrix} a & b & c \\ d & e & f \\ g & h & i \end{pmatrix}, \end{split}
\end{equation*}
\sphinxAtStartPar
we can expand across row 1 \sphinxhyphen{} this means we are using \(k = 1\) in the first summation in equation \eqref{equation:_pages/1.4_Determinants:nxn-determinant-equation}.
\begin{equation*}
\begin{split} \begin{align*}
    \det(A) &= a C_{11} + b C_{12} + cC_{13} \\
    &= (-1)^{(1+1)} a \begin{vmatrix} e & f \\ h & i \end{vmatrix}
    + (-1)^{(1+2)} b \begin{vmatrix}d & f \\ g & i \end{vmatrix}
    + (-1)^{(1+3)} c \begin{vmatrix}d & e \\ g & h \end{vmatrix} \\
    &= a(ei - fh) - b(di - fg) + c(dh - eg)
\end{align*} \end{split}
\end{equation*}
\sphinxAtStartPar
This is just the alternating sum of the determinants of three \(2 \times 2\) submatrices of \(A\), each multiplied by the entry defining which rows and columns are not used in that matrix.

\sphinxAtStartPar
We could also have chosen to expand along column 2 using \(k = 2\) in equation \eqref{equation:_pages/1.4_Determinants:nxn-determinant-equation} and the second summation in equation \eqref{equation:_pages/1.4_Determinants:nxn-determinant-equation}.
\begin{equation*}
\begin{split} \begin{align*}
    \det(A) &= b C_{12} + e C_{22} + h C_{32} \\
    &= (-1)^{(1+2)} b\begin{vmatrix} d & f \\ g & i \end{vmatrix}
    + (-1)^{(2+2)}e \begin{vmatrix} a & c \\ g & i \end{vmatrix}
    + (-1)^{(3+2)}h \begin{vmatrix} a & c \\ d & f \end{vmatrix} \\
    &= -b(di-fg) + e(ai-cg) - h(af-cd) \\
    &= -bdi + bfg + aei - ceg - afh + cdh.
\end{align*} \end{split}
\end{equation*}
\sphinxAtStartPar
In our first example, the final line can be multiplied out to give \(aei - afh - bdi + bfg + cdh - ceg\), and this is the same (after some rearrangement) as the result when we expand along column 2.

\sphinxAtStartPar
It does not matter which row or column we expand along to compute the determinant, we will always get the same answer. It is usually preferable to expand along the row or columns with the most zero elements, or smallest integer values, to simplify the calculations.


\label{_pages/1.4_Determinants:nxn-determinant-example}
\begin{sphinxadmonition}{note}{Example 1.5.3}



\sphinxAtStartPar
Calculate the determinant of the matrix
\begin{equation*}
\begin{split} \begin{pmatrix} 1 & 0 & 4 \\ 2 & 5 & 6 \\ 4 & 5 & 2 \end{pmatrix}, \end{split}
\end{equation*}
\sphinxAtStartPar
by expanding along:

\sphinxAtStartPar
(i)   row 1;  
(ii)   column 3
\subsubsection*{Solution}

\sphinxAtStartPar
(i)
\begin{equation*}
\begin{split} \begin{align*}
    \begin{vmatrix}1 & 0 & 4 \\ 2 & 5 & 6 \\ 4 & 5 & 2 \end{vmatrix}
    &= 1\begin{vmatrix}5 & 6 \\ 5 & 2 \end{vmatrix} - 0 \begin{vmatrix} 2 & 6 \\ 4 & 2 \end{vmatrix} + 4 \begin{vmatrix}2 & 5 \\ 4 & 5 \end{vmatrix} \\
    &= (10-30) + 4(10-20)
    = -60
\end{align*} \end{split}
\end{equation*}
\sphinxAtStartPar
(ii)
\begin{equation*}
\begin{split} \begin{align*}
    \begin{vmatrix}1 & 0 & 4 \\ 2 & 5 & 6 \\ 4 & 5 & 2 \end{vmatrix}
    &= 4 \begin{vmatrix} 2 & 5 \\ 4 & 5 \end{vmatrix} - 6\begin{vmatrix}1 & 0 \\ 4 & 5 \end{vmatrix} + 2\begin{vmatrix}1 & 0 \\ 2 & 5 \end{vmatrix} \\
    &= 4(10-20) - 6(5 - 0) + 2(5 - 0)
    = -60
\end{align*} \end{split}
\end{equation*}\end{sphinxadmonition}

\sphinxAtStartPar
For larger matrices we have to apply equation \eqref{equation:_pages/1.4_Determinants:nxn-determinant-equation} recursively until we get to \(2 \times 2\) determinants where we can use \eqref{equation:_pages/1.4_Determinants:2x2-determinant-equation}.
\label{_pages/1.4_Determinants:4x4-determinant-example}
\begin{sphinxadmonition}{note}{Example 1.5.4}



\sphinxAtStartPar
Calculate the determinant of the \(4 \times 4\) matrix
\begin{equation*}
\begin{split} \begin{pmatrix} 1 & -1 & 4 & 3 \\ 2 & 0 & 5 & -3 \\ 1 & 2 & 4 & 5 \\ 2 & 0 & -2 & 4 \end{pmatrix}. \end{split}
\end{equation*}\subsubsection*{Solution}

\sphinxAtStartPar
Here column 2 has two zero elements so it would be more efficient to expand along this column
\begin{equation*}
\begin{split} \begin{align*}
    \begin{vmatrix} 1 & -1 & 4 & 3 \\ 2 & 0 & 5 & -3 \\ 1 & 2 & 4 & 5 \\ 2 & 0 & -2 & 4 \end{vmatrix} &=
    - (-1) \begin{vmatrix}2 & 5 & -3 \\ 1 & 4 & 5 \\ 2 & -2 & 4 \end{vmatrix}
    - 2 \begin{vmatrix} 1 & 4 & 3 \\ 2 & 5 & -3 \\ 2 & -2 & 4 \end{vmatrix} \\
    &= 1\left(
        2 \begin{vmatrix} 4 & 5 \\ -2 & 4 \end{vmatrix}
      - 5\begin{vmatrix}1 & 5 \\ 2 & 4 \end{vmatrix}
      - 3 \begin{vmatrix} 1 & 4 \\ 2 & -2 \end{vmatrix}
    \right) \\
    & \qquad - 2 \left(
        \begin{vmatrix} 5 & -3 \\ -2 & 4 \end{vmatrix}
      - 4 \begin{vmatrix} 2 & -3 \\ 2 & 4 \end{vmatrix}
      + 3 \begin{vmatrix} 2 & 5 \\ 2 & -2 \end{vmatrix}
    \right) \\
    &= 2(16 + 10) - 5(4 - 10) - 3(-2-8) \\
    & \qquad - 2(20-6) + 8(8+6) - 6(-4-10) \\
    &= 280.
\end{align*} \end{split}
\end{equation*}\end{sphinxadmonition}

\index{Determinants@\spxentry{Determinants}!properties of@\spxentry{properties of}}\ignorespaces \label{_pages/1.4_Determinants:properties-of-determinants-theorem}
\begin{sphinxadmonition}{note}{Theorem 1.5.1 (Properties of determinants)}



\sphinxAtStartPar
Determinants have the following properties:
\begin{itemize}
\item {} 
\sphinxAtStartPar
\(\det(AB) = \det(A)\det(B)\)

\item {} 
\sphinxAtStartPar
\(\det(A) = \det(A^\mathsf{T})\)

\item {} 
\sphinxAtStartPar
If a matrix has a row or column with all zero elements then its determinant is zero

\item {} 
\sphinxAtStartPar
Interchanging any two rows of a matrix changes the sign of the determinant

\item {} 
\sphinxAtStartPar
If all elements in a row are multiplied by a scalar \(k\) then the determinant is also multiplied by \(k\)

\item {} 
\sphinxAtStartPar
If one row of a matrix is a multiple of another row then the matrix has a determinant of zero

\item {} 
\sphinxAtStartPar
The value of a determinant is unchanged by adding a multiple of one row to another row

\end{itemize}
\end{sphinxadmonition}

\sphinxstepscope

\index{Matrix@\spxentry{Matrix}!inverse@\spxentry{inverse}}\ignorespaces 

\section{Inverse matrix}
\label{\detokenize{_pages/1.5_Inverse_matrix:inverse-matrix}}\label{\detokenize{_pages/1.5_Inverse_matrix:index-0}}\label{\detokenize{_pages/1.5_Inverse_matrix:inverse-matrix-section}}\label{\detokenize{_pages/1.5_Inverse_matrix::doc}}
\sphinxAtStartPar
We have seen that {\hyperref[\detokenize{_pages/1.2_Matrix_multiplication:matrix-multiplication-definition}]{\sphinxcrossref{matrix multiplication}}} is defined for matrices of the right sizes. There is no equivalent definition of division for matrices. However, for certain square matrices it may be possible to compute the \sphinxstylestrong{inverse matrix}, which can be multiplied by the matrix to achieve the same effect.

\sphinxAtStartPar
In the case of standard numerical multiplication, we say the {\hyperref[\detokenize{_pages/0.3_Mathematical_preliminaries:axioms-of-addition-and-multiplication-section}]{\sphinxcrossref{\DUrole{std,std-ref}{multiplicative inverse}}}} of a number \(x\) is a number denoted by \(\dfrac{1}{x}\), or sometimes \(x^{-1}\), which when multiplied by \(x\) results in the multiplicative identity 1. In general across mathematics, using \(x^{-1}\) to denote the inverse of an object \(x\) is standard, and it is also used to denote inverse matrices.

\sphinxAtStartPar
For matrices, the multiplicative identity is the identity matrix, so we can define a multiplicative inverse with respect to matrix multiplication, and for a matrix \(A\) its inverse is denoted \(A^{-1}\).
\label{_pages/1.5_Inverse_matrix:inverse-matrix-definition}
\begin{sphinxadmonition}{note}{Definition 1.6.1 (Inverse matrix)}



\sphinxAtStartPar
Let \(A\) be a non\sphinxhyphen{}zero \(n \times n\) matrix. Then, if there exists an \(n \times n\) matrix \(A^{-1}\) such that \(A^{-1}A = I_n\), we say that \(A^{-1}\) is the \sphinxstylestrong{inverse} of the matrix \(A\).
\end{sphinxadmonition}

\sphinxAtStartPar
For example, given the matrix \(A = \begin{pmatrix} 1 & 1 \\ 1 & 2 \end{pmatrix}\) then its inverse is \(A^{-1} = \begin{pmatrix} 2 & -1 \\ -1 & 1 \end{pmatrix}\). Indeed,
\begin{equation*}
\begin{split} \begin{align*}
    AA^{-1} = \begin{pmatrix} 2 & -1 \\ -1 & 1 \end{pmatrix}\begin{pmatrix} 1 & 1 \\ 1 & 2 \end{pmatrix} = \begin{pmatrix} 1 & 0 \\ 0 & 1 \end{pmatrix} = I.
\end{align*}\end{split}
\end{equation*}
\sphinxAtStartPar
We could also show this by calculating \(A^{-1}A\):
\begin{equation*}
\begin{split} \begin{align*}
    A^{-1}A = \begin{pmatrix} 1 & 1 \\ 1 & 2 \end{pmatrix}\begin{pmatrix} 2 & -1 \\ -1 & 1 \end{pmatrix} = \begin{pmatrix} 1 & 0 \\ 0 & 1 \end{pmatrix} = I
\end{align*}\end{split}
\end{equation*}
\sphinxAtStartPar
In order for a matrix to be invertible, it needs:
\begin{itemize}
\item {} 
\sphinxAtStartPar
to be a square matrix

\item {} 
\sphinxAtStartPar
to have a non\sphinxhyphen{}zero determinant.

\end{itemize}

\begin{sphinxadmonition}{note}{Note:}\begin{itemize}
\item {} 
\sphinxAtStartPar
Not all square matrices have an inverse.

\item {} 
\sphinxAtStartPar
If a matrix does not have an inverse it has a determinant of zero and is said to be \sphinxstylestrong{singular}.

\item {} 
\sphinxAtStartPar
If a matrix has an inverse, it has a non\sphinxhyphen{}zero determinant and it is said to be \sphinxstylestrong{non\sphinxhyphen{}singular} or \sphinxstylestrong{invertible}.

\item {} 
\sphinxAtStartPar
The inverse of a square matrix is unique.

\item {} 
\sphinxAtStartPar
Even though matrix multiplication does not in general commute, we can always say that \(A^{-1}A = AA^{-1} = I_n\).

\end{itemize}
\end{sphinxadmonition}

\sphinxAtStartPar
Calculating the inverse of a matrix (called ‘inverting the matrix’) can be done using a variety of methods. One involves using the adjoint of a matrix.


\bigskip\hrule\bigskip


\index{Matrix@\spxentry{Matrix}!adjoint (adjugate)@\spxentry{adjoint}\spxextra{adjugate}}\ignorespaces 

\subsection{The adjoint of a matrix}
\label{\detokenize{_pages/1.5_Inverse_matrix:the-adjoint-of-a-matrix}}\label{\detokenize{_pages/1.5_Inverse_matrix:index-1}}\label{_pages/1.5_Inverse_matrix:adjoint-definition}
\begin{sphinxadmonition}{note}{Definition 1.6.2 (Adjoint of a matrix)}



\sphinxAtStartPar
The \sphinxstylestrong{adjoint} (also known as the \sphinxstylestrong{adjugate}) of a square matrix \(A\) is denoted by \(\operatorname{adj}(A)\) and is the transpose of the matrix of {\hyperref[\detokenize{_pages/1.4_Determinants:cofactor-definition}]{\sphinxcrossref{cofactors}}} of \(A\):
\begin{equation}\label{equation:_pages/1.5_Inverse_matrix:adjoint-equation}
\begin{split} \begin{align*}
    \operatorname{adj}(A) &= C^\mathsf{T}.
\end{align*} \end{split}
\end{equation}\end{sphinxadmonition}

\sphinxAtStartPar
Recall that the cofactor of the \((i,j)\)th element in a matrix \(A\) is found by removing the \(i\)th row and \(j\)th column from the matrix, taking the determinant of the remaining matrix (called the \(i,j\)th minor of \(A\)), and giving it a sign of \((-1)^{i+j}\).

\sphinxAtStartPar
The matrix of cofactors will be the same size as the original matrix, and each entry will be the cofactor for the corresponding entry in the original matrix. This can then be transposed to obtain the adjoint.

\begin{sphinxadmonition}{note}{Note:}
\sphinxAtStartPar
In the case of a \(2 \times 2\) matrix, the minors will each consist of a \(1 \times 1\) matrix, whose determinant is just the value in the matrix.
\end{sphinxadmonition}
\label{_pages/1.5_Inverse_matrix:adjoint-example}
\begin{sphinxadmonition}{note}{Example 1.6.1}



\sphinxAtStartPar
Calculate the adjoint of the following matrices:

\sphinxAtStartPar
  (i)   \(\begin{pmatrix} a & b \\ c & d \end{pmatrix}\);  
  (ii)   \(\begin{pmatrix} 5 & 2 \\ 3 & 4 \end{pmatrix}\);  
  (iii)   \(\begin{pmatrix} 1 & 0 & 3 \\ 4 & -2 & 1 \\ 2 & 1 & 3 \end{pmatrix}\).
\subsubsection*{Solution}

\sphinxAtStartPar
(i)
\begin{equation*}
\begin{split} \begin{align*}
    \operatorname{adj}\begin{pmatrix} a & b \\ c & d \end{pmatrix} = \begin{pmatrix} d & -c \\ -b & a \end{pmatrix}^\mathsf{T}
= \begin{pmatrix} d & -b \\ -c & a \end{pmatrix}
\end{align*} \end{split}
\end{equation*}
\sphinxAtStartPar
Note that this is the general formula for the adjoint of a \(2 \times 2\) matrix. Make a note of it \sphinxhyphen{} you may need it again!

\sphinxAtStartPar
(ii)
\begin{equation*}
\begin{split} \begin{align*}
    \operatorname{adj}\begin{pmatrix} 5 & 2 \\ 3 & 4 \end{pmatrix} = \begin{pmatrix} 4 & -3 \\ -2 & 5 \end{pmatrix}^\mathsf{T}
= \begin{pmatrix} 4 & -2 \\ -3 & 5 \end{pmatrix}
\end{align*} \end{split}
\end{equation*}
\sphinxAtStartPar
(iii)
\begin{equation*}
\begin{split} \begin{align*}
    \operatorname{adj}\begin{pmatrix} 1 & 0 & 3 \\ 4 & -2 & 1 \\ 2 & 1 & 3 \end{pmatrix}
    &= \begin{pmatrix}
        \begin{vmatrix} -2 & 1 \\ 1 & 3 \end{vmatrix} &
        -\begin{vmatrix} 4 & 1 \\ 2 & 3 \end{vmatrix} &
        \begin{vmatrix} 4 & -2 \\ 2 & 1 \end{vmatrix} \\
        -\begin{vmatrix} 0 & 3 \\ 1 & 3 \end{vmatrix} &
        \begin{vmatrix} 1 & 3 \\ 2 & 3 \end{vmatrix} &
        -\begin{vmatrix} 1 & 0 \\ 2 & 1 \end{vmatrix} \\
        \begin{vmatrix} 0 & 3 \\ -2 & 1 \end{vmatrix} &
        -\begin{vmatrix} 1 & 3 \\ 4 & 1 \end{vmatrix} &
        \begin{vmatrix} 1 & 0 \\ 4 & -2 \end{vmatrix}
    \end{pmatrix}^\mathsf{T} \\
    &= \begin{pmatrix} -7 & -10 & 8 \\ 3 & -3 & -1 \\ 6 & 11 & -2 \end{pmatrix}^\mathsf{T}
    = \begin{pmatrix} -7 & 3 & 6 \\ -10 & -3 & 11 \\ 8 & -1 & -2 \end{pmatrix}.
\end{align*} \end{split}
\end{equation*}\end{sphinxadmonition}


\bigskip\hrule\bigskip


\index{Matrix@\spxentry{Matrix}!inverse using adjoint\sphinxhyphen{}determinant formula@\spxentry{inverse using adjoint\sphinxhyphen{}determinant formula}}\ignorespaces 

\subsection{Calculating a matrix inverse using the adjoint\sphinxhyphen{}determinant formula}
\label{\detokenize{_pages/1.5_Inverse_matrix:calculating-a-matrix-inverse-using-the-adjoint-determinant-formula}}\label{\detokenize{_pages/1.5_Inverse_matrix:index-2}}
\sphinxAtStartPar
There are several methods used to calculate the inverse of a matrix. For smaller matrices we can use the adjoint\sphinxhyphen{}determinant formula.
\label{_pages/1.5_Inverse_matrix:adjoint-determinant-formula-theorem}
\begin{sphinxadmonition}{note}{Theorem 1.6.1 (Adjoint\sphinxhyphen{}determinant formula)}



\sphinxAtStartPar
The inverse of a non\sphinxhyphen{}singular square matrix \(A\) can be calculated as:
\begin{equation}\label{equation:_pages/1.5_Inverse_matrix:adjoint-determinant-formula-equation}
\begin{split} A^{-1} = \frac{1}{\det(A)} \cdot \operatorname{adj}(A). \end{split}
\end{equation}
\sphinxAtStartPar
Recall that \(A\) is invertible if and only if \(\det(A)\neq 0\).
\end{sphinxadmonition}

\sphinxAtStartPar
It is useful to calculate the determinant first to check that it is non\sphinxhyphen{}zero before calculating the adjoint. For example, let’s calculate the inverse of
\begin{equation*}
\begin{split} A = \begin{pmatrix} 1 & 2 \\ 3 & 4 \end{pmatrix}. \end{split}
\end{equation*}
\sphinxAtStartPar
Here \(\det(A) = 1\cdot 4 - 2 \cdot3 = 4 - 6 = -2\) which is non\sphinxhyphen{}zero so we know that an inverse exists for \(A\). Next we calculate the adjoint
\begin{equation*}
\begin{split} \begin{align*}
    \operatorname{adj}(A) =
    \begin{pmatrix} 4 & -3 \\ -2 & 1 \end{pmatrix}^\mathsf{T}
    =
    \begin{pmatrix} 4 & -2 \\ -3 & 1 \end{pmatrix}.
\end{align*} \end{split}
\end{equation*}
\sphinxAtStartPar
So using equation \eqref{equation:_pages/1.5_Inverse_matrix:adjoint-determinant-formula-equation} the inverse of \(A\) is
\begin{equation*}
\begin{split} \begin{align*}
    A^{-1} &= \frac{1}{-2}\begin{pmatrix} 4 & -2 \\ -3 & 1 \end{pmatrix}
    = \begin{pmatrix} -2 & 1 \\ \frac{3}{2} & -\frac{1}{2} \end{pmatrix}.
\end{align*} \end{split}
\end{equation*}
\sphinxAtStartPar
We can check that this is the correct inverse by calculating \(AA^{-1}\)
\begin{equation*}
\begin{split} \begin{align*}
    AA^{-1} &=
    \begin{pmatrix} 1 & 2 \\ 3 & 4 \end{pmatrix}
    \begin{pmatrix} -2 & 1 \\ \frac{3}{2} & -\frac{1}{2} \end{pmatrix}
    =
    \begin{pmatrix} -2 + 3 & 1 - 1 \\ -6 + 6 & 3 - 2 \end{pmatrix}
    =
    \begin{pmatrix} 1 & 0 \\ 0 & 1 \end{pmatrix}.
\end{align*} \end{split}
\end{equation*}
\sphinxAtStartPar
So we know our inverse matrix is correct.


\label{_pages/1.5_Inverse_matrix:matrix-inverse-example-2}
\begin{sphinxadmonition}{note}{Example 1.6.2}



\sphinxAtStartPar
Calculate the inverses of the following matrices if they exist:

\sphinxAtStartPar
  (i)   \(A = \begin{pmatrix}1 & 0 \\ 3 & 2\end{pmatrix}\);  
  (ii)   \(B = \begin{pmatrix} 1 & 2 & 0 \\ -2 & 1 & 1 \\ 1 & 0 & 3 \end{pmatrix}\);  
  (iii)   \(C = \begin{pmatrix}1 & 2 & 3 \\ 4 & 5 & 6 \\ 7 & 8 & 9 \end{pmatrix}\)
\subsubsection*{Solution}

\sphinxAtStartPar
(i)
\begin{equation*}
\begin{split} \begin{align*}
    \det(A) &= 2, \\
    \operatorname{adj}(A) &= \begin{pmatrix} 2 & -3 \\ 0 & 1 \end{pmatrix}^\mathsf{T}  = \begin{pmatrix} 2 & 0 \\ -3 & 1\end{pmatrix} \\
    \therefore A^{-1} &= \frac{1}{2}\begin{pmatrix} 2 & 0 \\ -3 & 1\end{pmatrix}.
\end{align*} \end{split}
\end{equation*}
\sphinxAtStartPar
Check answer:
\begin{equation*}
\begin{split} \begin{align*}
    A^{-1}A = \frac{1}{2}\begin{pmatrix} 2 & 0 \\ -3 & 1\end{pmatrix}\begin{pmatrix}1 & 0 \\ 3 & 2\end{pmatrix} = \frac{1}{2} \begin{pmatrix} 2 & 0 \\ 0 & 2 \end{pmatrix} = \begin{pmatrix} 1 & 0 \\ 0 & 1 \end{pmatrix} = I \qquad \checkmark
\end{align*} \end{split}
\end{equation*}
\sphinxAtStartPar
(ii)
\begin{equation*}
\begin{split} \begin{align*}
    \det\begin{pmatrix} 1 & 2 & 0 \\ -2 & 1 & 1 \\ 1 & 0 & 3 \end{pmatrix} &=
    \begin{vmatrix} 1 & 1 \\ 0 & 3 \end{vmatrix}  - 2 \begin{vmatrix}-2 & 1 \\ 1 & 3 \end{vmatrix} =
    3 - 2(-7) = 17, \\
    \operatorname{adj}\begin{pmatrix} 1 & 2 & 0 \\ -2 & 1 & 1 \\ 1 & 0 & 3 \end{pmatrix} &=
    \begin{pmatrix} 3 & 7 & -1 \\ -6 & 3 & 2 \\ 2 & -1 & 5 \end{pmatrix}^\mathsf{T} =
    \begin{pmatrix} 3 & -6 & 2 \\ 7 & 3 & -1 \\ -1 & 2 & 5 \end{pmatrix}, \\
    \therefore B^{-1} &= \frac{1}{17}\begin{pmatrix} 3 & -6 & 2 \\ 7 & 3 & -1 \\ -1 & 2 & 5 \end{pmatrix}
\end{align*} \end{split}
\end{equation*}
\sphinxAtStartPar
Check answer:
\begin{equation*}
\begin{split} \begin{align*}
    B^{-1}B &= \frac{1}{17}\begin{pmatrix} 3 & -6 & 2 \\ 7 & 3 & -1 \\ -1 & 2 & 5 \end{pmatrix}
    \begin{pmatrix} 1 & 2 & 0 \\ -2 & 1 & 1 \\ 1 & 0 & 3 \end{pmatrix} \\
    &= \frac{1}{17}\begin{pmatrix} 17 & 0 & 0 \\ 0 & 17 & 0 \\ 0 & 0 & 17 \end{pmatrix} =
    \begin{pmatrix} 1 & 0 & 0 \\ 0 & 1 & 0 \\ 0 & 0 & 1 \end{pmatrix} = I \qquad \checkmark
\end{align*} \end{split}
\end{equation*}
\sphinxAtStartPar
(iii)
\begin{equation*}
\begin{split} \begin{align*}
    \det\begin{pmatrix}1 & 2 & 3 \\ 4 & 5 & 6 \\ 7 & 8 & 9 \end{pmatrix} &=
    \begin{vmatrix} 5 & 6 \\ 8 & 9 \end{vmatrix} - 2
    \begin{vmatrix} 4 & 6 \\ 7 & 9 \end{vmatrix} + 3
    \begin{vmatrix} 4 & 5 \\ 7 & 8 \end{vmatrix} \\
    &= -3 - 2 \cdot -6 + 3 \cdot -3 = 0.
\end{align*} \end{split}
\end{equation*}
\sphinxAtStartPar
Since \(\det(C)=0\) then \(C\) is singular and does not have an inverse.
\end{sphinxadmonition}

\sphinxAtStartPar
In general, the inverse of a \(2 \times 2\) matrix is given by:
\begin{align*}
    \begin{pmatrix} a & b \\ c & d \end{pmatrix}^{-1} = \frac{1}{ad - bc}\begin{pmatrix} d & -c \\ -b & a \end{pmatrix}^\mathsf{T}
= \frac{1}{ad - bc}\begin{pmatrix} d & -b \\ -c & a \end{pmatrix}
\end{align*} 
\index{Matrix@\spxentry{Matrix}!inverse properties@\spxentry{inverse properties}}\ignorespaces \label{_pages/1.5_Inverse_matrix:inverse-matrix-properties-theorem}
\begin{sphinxadmonition}{note}{Theorem 1.6.2 (Properties of inverse matrices)}



\sphinxAtStartPar
Let \(A\) and \(B\) be two invertible \(n \times n\) matrices, then the following hold:
\begin{itemize}
\item {} 
\sphinxAtStartPar
\(AB\) is also invertible and \((AB)^{-1} = B^{-1}A^{-1}.\)

\item {} 
\sphinxAtStartPar
\((A^m)^{-1} = (A^{-1})^m\) for all positive integers \(m\).

\item {} 
\sphinxAtStartPar
\(A^\mathsf{T}\) is also invertible and \((A^\mathsf{T})^{-1} = (A^{-1})^\mathsf{T}\).

\end{itemize}
\end{sphinxadmonition}

\sphinxstepscope


\section{Matrix algebra}
\label{\detokenize{_pages/1.6_Matrix_algebra:matrix-algebra}}\label{\detokenize{_pages/1.6_Matrix_algebra:matrix-algebra-section}}\label{\detokenize{_pages/1.6_Matrix_algebra::doc}}
\index{Matrix@\spxentry{Matrix}!algebra@\spxentry{algebra}}\ignorespaces 
\sphinxAtStartPar
Having defined the rules for addition, scalar multiplication and matrix multiplication, we can now consider solving equations involving matrices. Let \(X\) be a \(m \times n\) matrix whose value we want to find. The basic procedure for solving \(X\) is similar to standard algebra: for any operation we apply to one side of the equation we must apply the equivalent operation to the other side.

\sphinxAtStartPar
This is easy for addition, subtraction and scalar multiplication, since the rules governing matrices are the same as those for variables. For example, consider the algebraic equation:
\begin{equation*}
\begin{split} 2x + a = b, \end{split}
\end{equation*}
\sphinxAtStartPar
where \(a, b \in \mathbb{R}\). To solve for \(x\) we subtract \(a\) from both sides of the equation and multiply both sides by \(\dfrac{1}{2}\), i.e.,
\begin{equation*}
\begin{split} x = \frac{1}{2}(b - a). \end{split}
\end{equation*}
\sphinxAtStartPar
Now consider a similar matrix equation
\begin{equation*}
\begin{split} 2 X + A = B, \end{split}
\end{equation*}
\sphinxAtStartPar
where \(A\) and \(B\) are \(m \times n\) matrices. Solving for \(X\) is the same procedure as before, i.e.,
\begin{equation*}
\begin{split} X = \frac{1}{2}(B - A). \end{split}
\end{equation*}
\sphinxAtStartPar
When the unknown matrix is multiplied by another matrix things get slightly more complicated. Using an algebraic example
\begin{equation*}
\begin{split} c x + a = b, \end{split}
\end{equation*}
\sphinxAtStartPar
where \(a,b,c \in \mathbb{R}\), then the solution is
\begin{equation*}
\begin{split} x = \frac{1}{c}(b - a). \end{split}
\end{equation*}
\sphinxAtStartPar
However, consider the equivalent matrix equation
\begin{equation*}
\begin{split} C X + A = B, \end{split}
\end{equation*}
\sphinxAtStartPar
where \(A, B, C\) are \(n \times n\) matrices. You cannot divide by a matrix \sphinxhyphen{} but you can multiply by an inverse matrix, if one exists.
\begin{equation*}
\begin{split} \begin{align*}
    C X + A &= B \\
    CX &= B - A \\
    C^{-1} C X &= C^{-1} (B - A) \textrm{ (multiplying by }C^{-1} \textrm{ on the left on both sides)}\\
    I X &= C^{-1}(B - A) \textrm{(the matrices cancel out to give the identity matrix)}\\
    X &= C^{-1}(B - A).
\end{align*} \end{split}
\end{equation*}
\sphinxAtStartPar
Note that for a solution to exist, \(C\) must be invertible.
\label{_pages/1.6_Matrix_algebra:matrix-algebra-example}
\begin{sphinxadmonition}{note}{Example 1.7.1}



\sphinxAtStartPar
Solve the following matrix equations to find the possible matrices \(X\) satisfying each:

\sphinxAtStartPar
(i)   \(2X + \begin{pmatrix} 1 & 3 \\ 6 & 4 \end{pmatrix} = \begin{pmatrix} 1 & 0 \\ 0 & 1 \end{pmatrix}\)

\sphinxAtStartPar
(ii)   \(\begin{pmatrix} 4 & 1 \\ 2 & 5 \end{pmatrix} X - \begin{pmatrix} 5 & -3 \\ 2 & 4 \end{pmatrix} = \begin{pmatrix} 1 & 1 \\ 3 & 2 \end{pmatrix}\)

\sphinxAtStartPar
(iii)   \(X^2 = \begin{pmatrix} 4 & 3 \\ 0 & 1 \end{pmatrix}\)
\subsubsection*{Solution}

\sphinxAtStartPar
(i)
\begin{equation*}
\begin{split} \begin{align*}
    2X +
    \begin{pmatrix} 1 & 3 \\ 6 & 4 \end{pmatrix}
    &=
    \begin{pmatrix} 1 & 0 \\ 0 & 1 \end{pmatrix} \\
    2X &=
    \begin{pmatrix} 1 & 0 \\ 0 & 1 \end{pmatrix} -
    \begin{pmatrix} 1 & 3 \\ 6 & 4 \end{pmatrix} \\
    X &= \frac{1}{2}
    \begin{pmatrix} 0 & -3 \\ -6 & -3 \end{pmatrix} =
    \begin{pmatrix} 0 & -3/2 \\ -3 & -3/2 \end{pmatrix}
\end{align*} \end{split}
\end{equation*}
\sphinxAtStartPar
(ii)
\begin{equation*}
\begin{split} \begin{align*}
    \begin{pmatrix} 4 & 1 \\ 2 & 5 \end{pmatrix} X -
    \begin{pmatrix} 5 & -3 \\ 2 & 4 \end{pmatrix}
    &=
    \begin{pmatrix} 1 & 1 \\ 3 & 2 \end{pmatrix} \\
    \begin{pmatrix} 4 & 1 \\ 2 & 5 \end{pmatrix} X
    &=
    \begin{pmatrix} 1 & 1 \\ 3 & 2 \end{pmatrix} +
    \begin{pmatrix} 5 & -3 \\ 2 & 4 \end{pmatrix} \\
    X &=
    \begin{pmatrix} 4 & 1 \\ 2 & 5 \end{pmatrix}^{-1}
    \begin{pmatrix} 6 & -2 \\ 5 & 6 \end{pmatrix} \\
    &=
    \frac{1}{18} \begin{pmatrix} 5 & -1 \\ -2 & 4 \end{pmatrix}
    \begin{pmatrix} 6 & -2 \\ 5 & 6 \end{pmatrix} \\
    &= \begin{pmatrix} 25 & -16 \\ 8 & 28 \end{pmatrix}
\end{align*} \end{split}
\end{equation*}
\sphinxAtStartPar
(iii)   Let \(X = \begin{pmatrix} a & b \\ c & d \end{pmatrix}\) then
\begin{equation*}
\begin{split} \begin{align*}
    X^2 &=
    \begin{pmatrix} a & b \\ c & d \end{pmatrix}
    \begin{pmatrix} a & b \\ c & d \end{pmatrix}
    =
    \begin{pmatrix} a^2 + bc & b(a + d) \\ c(a + d) & bc + d^2 \end{pmatrix}
    =
    \begin{pmatrix} 4 & 3 \\ 0 & 1 \end{pmatrix}.
\end{align*} \end{split}
\end{equation*}
\sphinxAtStartPar
Taking the element in row 2 column 1 we have \(c(a+d)=0\) so either \(c=0\) or \(a + d = 0\)

\sphinxAtStartPar
When \(c = 0\)
\begin{equation*}
\begin{split} \begin{align*}
    X^2 = \begin{pmatrix} a^2 & b(a + d) \\ 0 & d^2 \end{pmatrix}
    =
    \begin{pmatrix} 4 & 3 \\ 0 & 1 \end{pmatrix},
\end{align*} \end{split}
\end{equation*}
\sphinxAtStartPar
so \(a = \pm 2\) and \(d = \pm 1\) and solving top\sphinxhyphen{}right element for \(b\) gives \(b = \dfrac{3}{a + b}\) so
\begin{equation*}
\begin{split}X = \begin{pmatrix} a & \dfrac{3}{a + b} \\ c & d \end{pmatrix}.\end{split}
\end{equation*}
\sphinxAtStartPar
Since when \(c=0\), \(a = \pm 2\) and \(d = \pm 1\) we have four possible solutions
\begin{equation*}
\begin{split} \begin{align*}
    a &= 2, d = 1: &
    X &= 
    \begin{pmatrix} 2 & 1 \\ 0 & 1 \end{pmatrix}, \\
    a &= 2, d = -1: &
    X &=
    \begin{pmatrix} 2 & 3 \\ 0 & -1 \end{pmatrix}, \\
    a &= -2, d = 1: &
    X &=
    \begin{pmatrix} -2 & -3 \\ 0 & 1 \end{pmatrix}, \\
    a &= -2, d = -1: &
    X &=
    \begin{pmatrix} -2 & -1 \\ 0 &  -1 \end{pmatrix}.
\end{align*} \end{split}
\end{equation*}\begin{equation*}
\begin{split} \begin{align*}
    X^2 = \begin{pmatrix} a^2 + bc & b(a + d) \\ c(a + d) & bc + d^2 \end{pmatrix}
    =
    \begin{pmatrix} 4 & 3 \\ 0 & 1 \end{pmatrix}.
\end{align*} \end{split}
\end{equation*}
\sphinxAtStartPar
When \(a + d = 0\), the element in row 1 column 2 gives
\begin{equation*}
\begin{split} \begin{align*}
    b(a + d) &= 3 \\
    0 &\neq 3,
\end{align*} \end{split}
\end{equation*}
\sphinxAtStartPar
which is a contradiction so \(a + d=0\) yields no solutions.
\end{sphinxadmonition}

\sphinxstepscope


\section{Matrices Exercises}
\label{\detokenize{_pages/1.8_Matrices_exercises:matrices-exercises}}\label{\detokenize{_pages/1.8_Matrices_exercises::doc}}
\sphinxAtStartPar
Answer the following exercises based on the content from this chapter. The solutions can be found in the {\hyperref[\detokenize{_pages/A1_Matrices_exercises_solutions:matrices-exercises-solutions}]{\sphinxcrossref{\DUrole{std,std-ref}{appendices}}}}.
\phantomsection \label{exercise:matrices-ex1}

\begin{sphinxadmonition}{note}{Exercise 1.8.1}



\sphinxAtStartPar
(a)   Write down the \(3 \times 3\) matrix \(A\) whose entries are given by \(a_{ij} = i+j.\)

\sphinxAtStartPar
(b)   Write down the \(4 \times 4\) matrix \(B\) whose entries are given by \(b_{ij} = (-1)^{i+j}.\)

\sphinxAtStartPar
(c)   Write down the \(4 \times 4\) matrix \(C\) whose entries are given by \(
    c_{ij} = 
    \begin{cases}
    -1, & i>j, \\
    0, & i=j, \\
    1, & i<j. \\
    \end{cases} \)
\end{sphinxadmonition}
\phantomsection \label{exercise:matrices-ex-hilbert}

\begin{sphinxadmonition}{note}{Exercise 1.8.2}



\sphinxAtStartPar
The Hilbert matrix is the \(n \times n\) matrix \(H\) where the value of its elements are \(h_{ij} = \dfrac{1}{i + j - 1}\).

\sphinxAtStartPar
(a)   Write down the \(4 \times 4\) Hilbert matrix.

\sphinxAtStartPar
(b)   Show that an \(n \times n\) Hilbert matrix is symmetric.
\end{sphinxadmonition}
\phantomsection \label{exercise:matrices-ex2}

\begin{sphinxadmonition}{note}{Exercise 1.8.3}



\sphinxAtStartPar
Given the matrices
\begin{equation*}
\begin{split} \begin{align*}
    A &= \begin{pmatrix} 1 & -3 \\ 4 & 2 \end{pmatrix}, &
    B &= \begin{pmatrix} 3 & 0 \\ -1 & 5 \end{pmatrix}, &
    C &= \begin{pmatrix} 5 \\ 9 \end{pmatrix}, &
    D &= \begin{pmatrix} 1 & 1 & 3 \\ 4 & -2 & 3 \end{pmatrix}, \\
    E &= \begin{pmatrix} 1 & 2 \\ 0 & 6 \\ -2 & 3 \end{pmatrix} &
    F &= \begin{pmatrix} 1 & -2 & 4 \end{pmatrix}, &
    G &= \begin{pmatrix} 4 & 2 & 3 \\ -2 & 6 & 0 \\ 0 & 7 & 1 \end{pmatrix}, &
    H &= \begin{pmatrix} 1 & 0 & 1 \\ 5 & 2 & -2 \\ 2 & -3 & 4 \end{pmatrix}.
\end{align*} \end{split}
\end{equation*}
\sphinxAtStartPar
Calculate the following where possible:

\begin{sphinxuseclass}{sd-container-fluid}
\begin{sphinxuseclass}{sd-sphinx-override}
\begin{sphinxuseclass}{sd-mb-4}
\begin{sphinxuseclass}{sd-row}
\begin{sphinxuseclass}{sd-col}
\begin{sphinxuseclass}{sd-d-flex-column}
\begin{sphinxuseclass}{sd-col-3}
\begin{sphinxuseclass}{sd-col-xs-3}
\begin{sphinxuseclass}{sd-col-sm-3}
\begin{sphinxuseclass}{sd-col-md-3}
\begin{sphinxuseclass}{sd-col-lg-3}
\sphinxAtStartPar
(a)   \(A + B\)

\end{sphinxuseclass}
\end{sphinxuseclass}
\end{sphinxuseclass}
\end{sphinxuseclass}
\end{sphinxuseclass}
\end{sphinxuseclass}
\end{sphinxuseclass}
\begin{sphinxuseclass}{sd-col}
\begin{sphinxuseclass}{sd-d-flex-column}
\begin{sphinxuseclass}{sd-col-3}
\begin{sphinxuseclass}{sd-col-xs-3}
\begin{sphinxuseclass}{sd-col-sm-3}
\begin{sphinxuseclass}{sd-col-md-3}
\begin{sphinxuseclass}{sd-col-lg-3}
\sphinxAtStartPar
(b)   \(B + C\)

\end{sphinxuseclass}
\end{sphinxuseclass}
\end{sphinxuseclass}
\end{sphinxuseclass}
\end{sphinxuseclass}
\end{sphinxuseclass}
\end{sphinxuseclass}
\begin{sphinxuseclass}{sd-col}
\begin{sphinxuseclass}{sd-d-flex-column}
\begin{sphinxuseclass}{sd-col-3}
\begin{sphinxuseclass}{sd-col-xs-3}
\begin{sphinxuseclass}{sd-col-sm-3}
\begin{sphinxuseclass}{sd-col-md-3}
\begin{sphinxuseclass}{sd-col-lg-3}
\sphinxAtStartPar
(c)   \(A^\mathsf{T}\)

\end{sphinxuseclass}
\end{sphinxuseclass}
\end{sphinxuseclass}
\end{sphinxuseclass}
\end{sphinxuseclass}
\end{sphinxuseclass}
\end{sphinxuseclass}
\begin{sphinxuseclass}{sd-col}
\begin{sphinxuseclass}{sd-d-flex-column}
\begin{sphinxuseclass}{sd-col-3}
\begin{sphinxuseclass}{sd-col-xs-3}
\begin{sphinxuseclass}{sd-col-sm-3}
\begin{sphinxuseclass}{sd-col-md-3}
\begin{sphinxuseclass}{sd-col-lg-3}
\sphinxAtStartPar
(d)   \(C^\mathsf{T}\)

\end{sphinxuseclass}
\end{sphinxuseclass}
\end{sphinxuseclass}
\end{sphinxuseclass}
\end{sphinxuseclass}
\end{sphinxuseclass}
\end{sphinxuseclass}
\begin{sphinxuseclass}{sd-col}
\begin{sphinxuseclass}{sd-d-flex-column}
\begin{sphinxuseclass}{sd-col-3}
\begin{sphinxuseclass}{sd-col-xs-3}
\begin{sphinxuseclass}{sd-col-sm-3}
\begin{sphinxuseclass}{sd-col-md-3}
\begin{sphinxuseclass}{sd-col-lg-3}
\sphinxAtStartPar
(e)   \(3B - A\)

\end{sphinxuseclass}
\end{sphinxuseclass}
\end{sphinxuseclass}
\end{sphinxuseclass}
\end{sphinxuseclass}
\end{sphinxuseclass}
\end{sphinxuseclass}
\begin{sphinxuseclass}{sd-col}
\begin{sphinxuseclass}{sd-d-flex-column}
\begin{sphinxuseclass}{sd-col-3}
\begin{sphinxuseclass}{sd-col-xs-3}
\begin{sphinxuseclass}{sd-col-sm-3}
\begin{sphinxuseclass}{sd-col-md-3}
\begin{sphinxuseclass}{sd-col-lg-3}
\sphinxAtStartPar
(f)   \((F^\mathsf{T})^\mathsf{T}\)

\end{sphinxuseclass}
\end{sphinxuseclass}
\end{sphinxuseclass}
\end{sphinxuseclass}
\end{sphinxuseclass}
\end{sphinxuseclass}
\end{sphinxuseclass}
\begin{sphinxuseclass}{sd-col}
\begin{sphinxuseclass}{sd-d-flex-column}
\begin{sphinxuseclass}{sd-col-3}
\begin{sphinxuseclass}{sd-col-xs-3}
\begin{sphinxuseclass}{sd-col-sm-3}
\begin{sphinxuseclass}{sd-col-md-3}
\begin{sphinxuseclass}{sd-col-lg-3}
\sphinxAtStartPar
(g)   \(A^\mathsf{T} + B^\mathsf{T}\)

\end{sphinxuseclass}
\end{sphinxuseclass}
\end{sphinxuseclass}
\end{sphinxuseclass}
\end{sphinxuseclass}
\end{sphinxuseclass}
\end{sphinxuseclass}
\begin{sphinxuseclass}{sd-col}
\begin{sphinxuseclass}{sd-d-flex-column}
\begin{sphinxuseclass}{sd-col-3}
\begin{sphinxuseclass}{sd-col-xs-3}
\begin{sphinxuseclass}{sd-col-sm-3}
\begin{sphinxuseclass}{sd-col-md-3}
\begin{sphinxuseclass}{sd-col-lg-3}
\sphinxAtStartPar
(h)   \((A + B)^\mathsf{T}\)

\end{sphinxuseclass}
\end{sphinxuseclass}
\end{sphinxuseclass}
\end{sphinxuseclass}
\end{sphinxuseclass}
\end{sphinxuseclass}
\end{sphinxuseclass}
\end{sphinxuseclass}
\end{sphinxuseclass}
\end{sphinxuseclass}
\end{sphinxuseclass}\end{sphinxadmonition}
\phantomsection \label{exercise:matrices-ex3}

\begin{sphinxadmonition}{note}{Exercise 1.8.4}



\sphinxAtStartPar
Using the matrices from \hyperref[exercise:matrices-ex2]{Exercise 1.8.3} calculate the following where possible:

\begin{sphinxuseclass}{sd-container-fluid}
\begin{sphinxuseclass}{sd-sphinx-override}
\begin{sphinxuseclass}{sd-mb-4}
\begin{sphinxuseclass}{sd-row}
\begin{sphinxuseclass}{sd-col}
\begin{sphinxuseclass}{sd-d-flex-column}
\begin{sphinxuseclass}{sd-col-3}
\begin{sphinxuseclass}{sd-col-xs-3}
\begin{sphinxuseclass}{sd-col-sm-3}
\begin{sphinxuseclass}{sd-col-md-3}
\begin{sphinxuseclass}{sd-col-lg-3}
\sphinxAtStartPar
(a)    \(AB\)

\end{sphinxuseclass}
\end{sphinxuseclass}
\end{sphinxuseclass}
\end{sphinxuseclass}
\end{sphinxuseclass}
\end{sphinxuseclass}
\end{sphinxuseclass}
\begin{sphinxuseclass}{sd-col}
\begin{sphinxuseclass}{sd-d-flex-column}
\begin{sphinxuseclass}{sd-col-3}
\begin{sphinxuseclass}{sd-col-xs-3}
\begin{sphinxuseclass}{sd-col-sm-3}
\begin{sphinxuseclass}{sd-col-md-3}
\begin{sphinxuseclass}{sd-col-lg-3}
\sphinxAtStartPar
(b)    \(BA\)

\end{sphinxuseclass}
\end{sphinxuseclass}
\end{sphinxuseclass}
\end{sphinxuseclass}
\end{sphinxuseclass}
\end{sphinxuseclass}
\end{sphinxuseclass}
\begin{sphinxuseclass}{sd-col}
\begin{sphinxuseclass}{sd-d-flex-column}
\begin{sphinxuseclass}{sd-col-3}
\begin{sphinxuseclass}{sd-col-xs-3}
\begin{sphinxuseclass}{sd-col-sm-3}
\begin{sphinxuseclass}{sd-col-md-3}
\begin{sphinxuseclass}{sd-col-lg-3}
\sphinxAtStartPar
(c)    \(AC\)

\end{sphinxuseclass}
\end{sphinxuseclass}
\end{sphinxuseclass}
\end{sphinxuseclass}
\end{sphinxuseclass}
\end{sphinxuseclass}
\end{sphinxuseclass}
\begin{sphinxuseclass}{sd-col}
\begin{sphinxuseclass}{sd-d-flex-column}
\begin{sphinxuseclass}{sd-col-3}
\begin{sphinxuseclass}{sd-col-xs-3}
\begin{sphinxuseclass}{sd-col-sm-3}
\begin{sphinxuseclass}{sd-col-md-3}
\begin{sphinxuseclass}{sd-col-lg-3}
\sphinxAtStartPar
(d)    \(CA\)

\end{sphinxuseclass}
\end{sphinxuseclass}
\end{sphinxuseclass}
\end{sphinxuseclass}
\end{sphinxuseclass}
\end{sphinxuseclass}
\end{sphinxuseclass}
\begin{sphinxuseclass}{sd-col}
\begin{sphinxuseclass}{sd-d-flex-column}
\begin{sphinxuseclass}{sd-col-3}
\begin{sphinxuseclass}{sd-col-xs-3}
\begin{sphinxuseclass}{sd-col-sm-3}
\begin{sphinxuseclass}{sd-col-md-3}
\begin{sphinxuseclass}{sd-col-lg-3}
\sphinxAtStartPar
(e)    \(C^\mathsf{T}C\)

\end{sphinxuseclass}
\end{sphinxuseclass}
\end{sphinxuseclass}
\end{sphinxuseclass}
\end{sphinxuseclass}
\end{sphinxuseclass}
\end{sphinxuseclass}
\begin{sphinxuseclass}{sd-col}
\begin{sphinxuseclass}{sd-d-flex-column}
\begin{sphinxuseclass}{sd-col-3}
\begin{sphinxuseclass}{sd-col-xs-3}
\begin{sphinxuseclass}{sd-col-sm-3}
\begin{sphinxuseclass}{sd-col-md-3}
\begin{sphinxuseclass}{sd-col-lg-3}
\sphinxAtStartPar
(f)    \(CC^\mathsf{T}\)

\end{sphinxuseclass}
\end{sphinxuseclass}
\end{sphinxuseclass}
\end{sphinxuseclass}
\end{sphinxuseclass}
\end{sphinxuseclass}
\end{sphinxuseclass}
\begin{sphinxuseclass}{sd-col}
\begin{sphinxuseclass}{sd-d-flex-column}
\begin{sphinxuseclass}{sd-col-3}
\begin{sphinxuseclass}{sd-col-xs-3}
\begin{sphinxuseclass}{sd-col-sm-3}
\begin{sphinxuseclass}{sd-col-md-3}
\begin{sphinxuseclass}{sd-col-lg-3}
\sphinxAtStartPar
(g)    \(DE\)

\end{sphinxuseclass}
\end{sphinxuseclass}
\end{sphinxuseclass}
\end{sphinxuseclass}
\end{sphinxuseclass}
\end{sphinxuseclass}
\end{sphinxuseclass}
\begin{sphinxuseclass}{sd-col}
\begin{sphinxuseclass}{sd-d-flex-column}
\begin{sphinxuseclass}{sd-col-3}
\begin{sphinxuseclass}{sd-col-xs-3}
\begin{sphinxuseclass}{sd-col-sm-3}
\begin{sphinxuseclass}{sd-col-md-3}
\begin{sphinxuseclass}{sd-col-lg-3}
\sphinxAtStartPar
(h)    \(GH\)

\end{sphinxuseclass}
\end{sphinxuseclass}
\end{sphinxuseclass}
\end{sphinxuseclass}
\end{sphinxuseclass}
\end{sphinxuseclass}
\end{sphinxuseclass}
\begin{sphinxuseclass}{sd-col}
\begin{sphinxuseclass}{sd-d-flex-column}
\begin{sphinxuseclass}{sd-col-3}
\begin{sphinxuseclass}{sd-col-xs-3}
\begin{sphinxuseclass}{sd-col-sm-3}
\begin{sphinxuseclass}{sd-col-md-3}
\begin{sphinxuseclass}{sd-col-lg-3}
\sphinxAtStartPar
(i)    \(A(DE)\)

\end{sphinxuseclass}
\end{sphinxuseclass}
\end{sphinxuseclass}
\end{sphinxuseclass}
\end{sphinxuseclass}
\end{sphinxuseclass}
\end{sphinxuseclass}
\begin{sphinxuseclass}{sd-col}
\begin{sphinxuseclass}{sd-d-flex-column}
\begin{sphinxuseclass}{sd-col-3}
\begin{sphinxuseclass}{sd-col-xs-3}
\begin{sphinxuseclass}{sd-col-sm-3}
\begin{sphinxuseclass}{sd-col-md-3}
\begin{sphinxuseclass}{sd-col-lg-3}
\sphinxAtStartPar
(j)    \((AD)E\)

\end{sphinxuseclass}
\end{sphinxuseclass}
\end{sphinxuseclass}
\end{sphinxuseclass}
\end{sphinxuseclass}
\end{sphinxuseclass}
\end{sphinxuseclass}
\begin{sphinxuseclass}{sd-col}
\begin{sphinxuseclass}{sd-d-flex-column}
\begin{sphinxuseclass}{sd-col-3}
\begin{sphinxuseclass}{sd-col-xs-3}
\begin{sphinxuseclass}{sd-col-sm-3}
\begin{sphinxuseclass}{sd-col-md-3}
\begin{sphinxuseclass}{sd-col-lg-3}
\sphinxAtStartPar
(k)    \(A^3\)

\end{sphinxuseclass}
\end{sphinxuseclass}
\end{sphinxuseclass}
\end{sphinxuseclass}
\end{sphinxuseclass}
\end{sphinxuseclass}
\end{sphinxuseclass}
\begin{sphinxuseclass}{sd-col}
\begin{sphinxuseclass}{sd-d-flex-column}
\begin{sphinxuseclass}{sd-col-3}
\begin{sphinxuseclass}{sd-col-xs-3}
\begin{sphinxuseclass}{sd-col-sm-3}
\begin{sphinxuseclass}{sd-col-md-3}
\begin{sphinxuseclass}{sd-col-lg-3}
\sphinxAtStartPar
(l)    \(G^4\)

\end{sphinxuseclass}
\end{sphinxuseclass}
\end{sphinxuseclass}
\end{sphinxuseclass}
\end{sphinxuseclass}
\end{sphinxuseclass}
\end{sphinxuseclass}
\end{sphinxuseclass}
\end{sphinxuseclass}
\end{sphinxuseclass}
\end{sphinxuseclass}\end{sphinxadmonition}
\phantomsection \label{exercise:matrices-ex-determinants}

\begin{sphinxadmonition}{note}{Exercise 1.8.5}



\sphinxAtStartPar
Calculate the determinants of the square matrices from \hyperref[exercise:matrices-ex2]{Exercise 1.8.3}.
\end{sphinxadmonition}
\phantomsection \label{exercise:matrices-ex-inverse}

\begin{sphinxadmonition}{note}{Exercise 1.8.6}



\sphinxAtStartPar
For each non\sphinxhyphen{}singular matrix from \hyperref[exercise:matrices-ex2]{Exercise 1.8.3} calculate its inverse. Show that your answers are correct.
\end{sphinxadmonition}
\phantomsection \label{exercise:matrices-ex-AAT-symmetric}

\begin{sphinxadmonition}{note}{Exercise 1.8.7}



\sphinxAtStartPar
Show that \(AA^\mathsf{T}\) is a symmetric matrix. Hint: use the properties of matrix transposes with respect to addition and multiplication.
\end{sphinxadmonition}
\phantomsection \label{exercise:matrices-ex-invAB}

\begin{sphinxadmonition}{note}{Exercise 1.8.8}



\sphinxAtStartPar
Show that \((AB)^{-1} = B^{-1}A^{-1}\). Hint: use the associativity law.
\end{sphinxadmonition}
\phantomsection \label{exercise:matrices-ex-AplusBsquared}

\begin{sphinxadmonition}{note}{Exercise 1.8.9}



\sphinxAtStartPar
If \(A\) and \(B\) are \(n \times n\) matrices is the following equation true?
\begin{equation*}
\begin{split}(A + B)^2 = A^2 + 2AB + B^2\end{split}
\end{equation*}
\sphinxAtStartPar
If not, under what conditions would it be true?
\end{sphinxadmonition}
\phantomsection \label{exercise:matrices-ex-involutory-matrix}

\begin{sphinxadmonition}{note}{Exercise 1.8.10}



\sphinxAtStartPar
An \sphinxstylestrong{involutory matrix} is a matrix that is its own inverse, i.e., it satisfies the equation \(A^2 = I\). Under what conditions is the following matrix an involutory matrix?
\begin{equation*}
\begin{split}A = \begin{pmatrix} a & b \\ c & -a \end{pmatrix} \end{split}
\end{equation*}\end{sphinxadmonition}
\phantomsection \label{exercise:matrices-ex-trueorfalse}

\begin{sphinxadmonition}{note}{Exercise 1.8.11}



\sphinxAtStartPar
Which of the following statements are true? For the false statements, give one counter example where the statement doesn’t hold.

\sphinxAtStartPar
(a)   If \(A = B\) then \(AC = BC\).

\sphinxAtStartPar
(b)   If \(AC = BC\) then \(A = B\).

\sphinxAtStartPar
(c)   Let \(O\) denote the matrix consisting of all zeroes \sphinxhyphen{} thatv is, \([O]_{ij} = 0\) for all \(i,j\). If \(AB = O\) then \(A = O\) or \(B = O\).

\sphinxAtStartPar
(d)   If \(A + C = B + C\) then \(A = B\).

\sphinxAtStartPar
(e)   If \(A^2 = I\) then \(A = \pm I\).

\sphinxAtStartPar
(f)   If \(B = A^2\) and if \(A\) is an \(n \times n\) symmetric matrix then \(b_{ii} \geq 0\) for \(i = 1, 2, \ldots, n\).

\sphinxAtStartPar
(g)   If \(AB = C\) and if two of the matrices are square then so is the third.

\sphinxAtStartPar
(h)   If \(AB = C\) and if \(C\) has a single column then so does \(B\).

\sphinxAtStartPar
(i)   If \(A^2 = I\) then \(A^n = I\) for all integers \(n \geq 2\).
\end{sphinxadmonition}
\phantomsection \label{exercise:matrices-ex7}

\begin{sphinxadmonition}{note}{Exercise 1.8.12}



\sphinxAtStartPar
Given the matrices
\begin{equation*}
\begin{split} \begin{align*}
    A &= \begin{pmatrix} 1 & -3 \\ 4 & 2 \end{pmatrix}, &
    B &= \begin{pmatrix} 3 & 0 \\ -1 & 5 \end{pmatrix},
\end{align*} \end{split}
\end{equation*}
\sphinxAtStartPar
solve the following equations for \(X\).

\begin{sphinxuseclass}{sd-container-fluid}
\begin{sphinxuseclass}{sd-sphinx-override}
\begin{sphinxuseclass}{sd-mb-4}
\begin{sphinxuseclass}{sd-row}
\begin{sphinxuseclass}{sd-col}
\begin{sphinxuseclass}{sd-d-flex-column}
\begin{sphinxuseclass}{sd-col-4}
\begin{sphinxuseclass}{sd-col-xs-4}
\begin{sphinxuseclass}{sd-col-sm-4}
\begin{sphinxuseclass}{sd-col-md-4}
\begin{sphinxuseclass}{sd-col-lg-4}
\sphinxAtStartPar
(a)   \(5X = A\)

\end{sphinxuseclass}
\end{sphinxuseclass}
\end{sphinxuseclass}
\end{sphinxuseclass}
\end{sphinxuseclass}
\end{sphinxuseclass}
\end{sphinxuseclass}
\begin{sphinxuseclass}{sd-col}
\begin{sphinxuseclass}{sd-d-flex-column}
\begin{sphinxuseclass}{sd-col-4}
\begin{sphinxuseclass}{sd-col-xs-4}
\begin{sphinxuseclass}{sd-col-sm-4}
\begin{sphinxuseclass}{sd-col-md-4}
\begin{sphinxuseclass}{sd-col-lg-4}
\sphinxAtStartPar
(b)   \(X + A = I\)

\end{sphinxuseclass}
\end{sphinxuseclass}
\end{sphinxuseclass}
\end{sphinxuseclass}
\end{sphinxuseclass}
\end{sphinxuseclass}
\end{sphinxuseclass}
\begin{sphinxuseclass}{sd-col}
\begin{sphinxuseclass}{sd-d-flex-column}
\begin{sphinxuseclass}{sd-col-4}
\begin{sphinxuseclass}{sd-col-xs-4}
\begin{sphinxuseclass}{sd-col-sm-4}
\begin{sphinxuseclass}{sd-col-md-4}
\begin{sphinxuseclass}{sd-col-lg-4}
\sphinxAtStartPar
(c)   \(2X - B = A\)

\end{sphinxuseclass}
\end{sphinxuseclass}
\end{sphinxuseclass}
\end{sphinxuseclass}
\end{sphinxuseclass}
\end{sphinxuseclass}
\end{sphinxuseclass}
\begin{sphinxuseclass}{sd-col}
\begin{sphinxuseclass}{sd-d-flex-column}
\begin{sphinxuseclass}{sd-col-4}
\begin{sphinxuseclass}{sd-col-xs-4}
\begin{sphinxuseclass}{sd-col-sm-4}
\begin{sphinxuseclass}{sd-col-md-4}
\begin{sphinxuseclass}{sd-col-lg-4}
\sphinxAtStartPar
(d)   \(XA = I\)

\end{sphinxuseclass}
\end{sphinxuseclass}
\end{sphinxuseclass}
\end{sphinxuseclass}
\end{sphinxuseclass}
\end{sphinxuseclass}
\end{sphinxuseclass}
\begin{sphinxuseclass}{sd-col}
\begin{sphinxuseclass}{sd-d-flex-column}
\begin{sphinxuseclass}{sd-col-4}
\begin{sphinxuseclass}{sd-col-xs-4}
\begin{sphinxuseclass}{sd-col-sm-4}
\begin{sphinxuseclass}{sd-col-md-4}
\begin{sphinxuseclass}{sd-col-lg-4}
\sphinxAtStartPar
(e)   \(BX = A\)

\end{sphinxuseclass}
\end{sphinxuseclass}
\end{sphinxuseclass}
\end{sphinxuseclass}
\end{sphinxuseclass}
\end{sphinxuseclass}
\end{sphinxuseclass}
\begin{sphinxuseclass}{sd-col}
\begin{sphinxuseclass}{sd-d-flex-column}
\begin{sphinxuseclass}{sd-col-4}
\begin{sphinxuseclass}{sd-col-xs-4}
\begin{sphinxuseclass}{sd-col-sm-4}
\begin{sphinxuseclass}{sd-col-md-4}
\begin{sphinxuseclass}{sd-col-lg-4}
\sphinxAtStartPar
(f)   \(A^2 = X\)

\end{sphinxuseclass}
\end{sphinxuseclass}
\end{sphinxuseclass}
\end{sphinxuseclass}
\end{sphinxuseclass}
\end{sphinxuseclass}
\end{sphinxuseclass}
\begin{sphinxuseclass}{sd-col}
\begin{sphinxuseclass}{sd-d-flex-column}
\begin{sphinxuseclass}{sd-col-4}
\begin{sphinxuseclass}{sd-col-xs-4}
\begin{sphinxuseclass}{sd-col-sm-4}
\begin{sphinxuseclass}{sd-col-md-4}
\begin{sphinxuseclass}{sd-col-lg-4}
\sphinxAtStartPar
(g)   \(X^2 = B\)

\end{sphinxuseclass}
\end{sphinxuseclass}
\end{sphinxuseclass}
\end{sphinxuseclass}
\end{sphinxuseclass}
\end{sphinxuseclass}
\end{sphinxuseclass}
\begin{sphinxuseclass}{sd-col}
\begin{sphinxuseclass}{sd-d-flex-column}
\begin{sphinxuseclass}{sd-col-4}
\begin{sphinxuseclass}{sd-col-xs-4}
\begin{sphinxuseclass}{sd-col-sm-4}
\begin{sphinxuseclass}{sd-col-md-4}
\begin{sphinxuseclass}{sd-col-lg-4}
\sphinxAtStartPar
(h)   \((X + A)B = I\)

\end{sphinxuseclass}
\end{sphinxuseclass}
\end{sphinxuseclass}
\end{sphinxuseclass}
\end{sphinxuseclass}
\end{sphinxuseclass}
\end{sphinxuseclass}
\end{sphinxuseclass}
\end{sphinxuseclass}
\end{sphinxuseclass}
\end{sphinxuseclass}\end{sphinxadmonition}

\sphinxstepscope

\index{Linear systems@\spxentry{Linear systems}}\ignorespaces 

\chapter{Systems of Linear Equations}
\label{\detokenize{_pages/2.0_Linear_systems:systems-of-linear-equations}}\label{\detokenize{_pages/2.0_Linear_systems:systems-of-linear-equations-chapter}}\label{\detokenize{_pages/2.0_Linear_systems:index-0}}\label{\detokenize{_pages/2.0_Linear_systems::doc}}
\index{Systems of linear equations@\spxentry{Systems of linear equations}}\ignorespaces 
\sphinxAtStartPar
Systems of linear equations, and the methods used to solve them, are a fundamental part of linear algebra. Systems of equations can be used to describe relationships between sets of variables, and have many applications including co\sphinxhyphen{}ordinate geometry, numerical methods, computer science, engineering and statistics. There are many techniques we can use to compute the solutions to systems of linear equations, and we will be covering the most common of these.

\sphinxAtStartPar
A linear equation is an equation that involves a set of \sphinxstylestrong{variables} multiplied by scalar \sphinxstylestrong{coefficients}, where the sum of these is equal to some \sphinxstylestrong{constant} scalar value. Generally, if \(x_1, x_2, \ldots, x_n\) are variables, \(a_1, a_2, \ldots, a_n\) are coefficients and \(b\) is a constant value then a linear equation can take the form
\begin{equation*}
\begin{split} a_1 x_1 + a_2 x_2 + \cdots + a_n x_n = b. \end{split}
\end{equation*}
\sphinxAtStartPar
Examples of linear equations:
\begin{equation*}
\begin{split} 3x + 2y = 5 \qquad x = 2 \qquad x + y + z = 2 \end{split}
\end{equation*}
\sphinxAtStartPar
A \sphinxstylestrong{nonlinear equation} is an equation where one or more of the variables appear with exponents not equal to 1 (e.g. a polynomial equation like \(x^2 = 4\)), or two or more of the variables are multiplied together (e.g. an equation in which \(x\) and \(y\) are variables, that contains a term like \(xy\)). The field of linear algebra is only concerned with linear equations.

\sphinxAtStartPar
If we were to plot all the points that satisfy a linear equation, we would see a straight line. (Nonlinear equations generally define curved lines.) The solutions to systems of linear equations are points which satisfy all the equations at once, which notionally correspond to points where the lines intersect in \(n\)\sphinxhyphen{}dimensional space, where \(n\) is the number of variables in the system.
\label{_pages/2.0_Linear_systems:system-of-linear-equation-definition}
\begin{sphinxadmonition}{note}{Definition 2.1 (System of linear equations)}



\sphinxAtStartPar
A \sphinxstylestrong{system of linear equations} is a collection of one or more linear equations expressed using the same set of variables. For example,
\begin{equation*}
\begin{split} \begin{align*}
    a_{11} x_1+a_{12} x_2+\cdots +a_{1n}x_n &=b_1, \\
    a_{21} x_1+a_{22} x_2+\cdots+a_{2n}x_n &=b_2, \\
    &\vdots \\
    a_{m1} x_1+a_{m2} x_2+\cdots+a_{mn}x_n &=b_m,
\end{align*} \end{split}
\end{equation*}
\sphinxAtStartPar
where \(x_1, x_2, \ldots, x_n\) are \sphinxstylestrong{variables}, \(a_{11}, a_{12}, \ldots, a_{mn}\) are \sphinxstylestrong{coefficients} and \(b_1, b_2, \ldots, b_n\) are \sphinxstylestrong{constants}.
\end{sphinxadmonition}

\sphinxAtStartPar
In general we would know the values of \(a_{ij}\) and \(b_i\) and we would like to find out what the values of \(x_i\) are. Note: It is possible for some of the coefficients \(a_{ij}\) to be zero, in which case not all variables will be visible in all the equations.

\sphinxAtStartPar
Under the right conditions, a system of linear equations will have a single unique solution \sphinxhyphen{} one value that each of the variables can take if all the equations are true. For this to happen, we need the number of equations in the system to be the same as the number of unknowns, as well as some conditions on the equations themselves. We will discuss this more in the section on {\hyperref[\detokenize{_pages/2.6_Consistent_systems:consistent-inconsistent-and-indeterminate-systems-section}]{\sphinxcrossref{\DUrole{std,std-ref}{Consistent, Inconsistent and Indeterminate systems}}}}.


\bigskip\hrule\bigskip


\index{Systems of linear equations@\spxentry{Systems of linear equations}!matrix equation@\spxentry{matrix equation}}\ignorespaces 

\section{Representing systems of linear equations using matrix equations}
\label{\detokenize{_pages/2.0_Linear_systems:representing-systems-of-linear-equations-using-matrix-equations}}\label{\detokenize{_pages/2.0_Linear_systems:index-2}}
\sphinxAtStartPar
Systems of linear equations are often represented using a matrix equation \(A \mathbf{x} = \mathbf{b}\), where \(A\) is an \(m \times n\) \sphinxstylestrong{coefficient matrix} containing the values of \(a_{ij}\), \(\mathbf{x}\) is an \(m \times 1\) \sphinxstylestrong{variable vector} containing the variables \(x_i\) and \(\mathbf{b}\) is a \(m \times 1\) \sphinxstylestrong{constant vector} containing the constant terms \(b_i\). (Vectors are more formally introduced in the {\hyperref[\detokenize{_pages/3.0_Vectors:vectors-chapter}]{\sphinxcrossref{\DUrole{std,std-ref}{next chapter}}}}, but for now simply consider them as matrices with a single column.)
\begin{equation*}
\begin{split} \begin{array}{cccc}
    A & \mathbf{x} & = & \mathbf{b} \\[4pt]
    \underbrace{\begin{pmatrix}
        a_{11} & a_{12} & \cdots & a_{1n} \\
        a_{21} & a_{22} & \cdots & a_{2n} \\
        \vdots & \vdots & \ddots & \vdots \\
        a_{m1} & a_{m2} & \cdots & a_{mn}
    \end{pmatrix}}_{\textsf{coefficient matrix}} &
    \underbrace{\begin{pmatrix} x_1 \\ x_2 \\ \vdots \\ x_n \end{pmatrix}}_{\textsf{variable vector}} & = &
    \underbrace{\begin{pmatrix} b_1 \\ b_2 \\ \vdots \\ b_m \end{pmatrix}}_{\textsf{constant vector}}
\end{array} \end{split}
\end{equation*}
\sphinxAtStartPar
For example, the system of linear equations
\begin{equation*}
\begin{split} \begin{align*}
    2x_1 + x_2 &= 4 \\
    4x_1 + 3x_2 &= 10
\end{align*} \end{split}
\end{equation*}
\sphinxAtStartPar
can be written as the matrix equation
\begin{equation*}
\begin{split} \begin{align*}
    \begin{pmatrix}
        2 & 1 \\
        4 & 3
    \end{pmatrix}
    \begin{pmatrix}
        x_1 \\ x_2
    \end{pmatrix} =
    \begin{pmatrix}
        4 \\ 10
    \end{pmatrix}
\end{align*} \end{split}
\end{equation*}

\bigskip\hrule\bigskip



\section{Solving systems of linear equations using algebra}
\label{\detokenize{_pages/2.0_Linear_systems:solving-systems-of-linear-equations-using-algebra}}\label{\detokenize{_pages/2.0_Linear_systems:solving-systems-of-linear-equations-using-algebra-section}}
\sphinxAtStartPar
One way to solve systems of linear equations is by manipulating them using standard algebraic methods.

\sphinxAtStartPar
The usual approach is to try to \sphinxstylestrong{solve for each unknown} in turn by \sphinxstylestrong{eliminating all of the other unknowns}. For example, consider the system of linear equations
\begin{equation*}
\begin{split} \begin{align*}
    2x_1 + x_2 &= 4, \\
    4x_1 + 3x_2 &= 10.
\end{align*} \end{split}
\end{equation*}
\sphinxAtStartPar
If we wish to eliminate the variable \(x_2\), we might note that the first equation would still be true if we multiplied both sides by \(3\), to get
\begin{equation*}
\begin{split} 6x_1 + 3x_2 = 12 \end{split}
\end{equation*}
\sphinxAtStartPar
We could then eliminate \(x_2\) by subtracting this from the second equation, to give
\begin{equation*}
\begin{split} -2 x_1 = -2,\end{split}
\end{equation*}
\sphinxAtStartPar
which is easily solved to give \(x_1 = 1\). Now that we know the value of \(x_1\) we can substitute it into any equation in the system to find the value of \(x_2\). Substituting \(x_1 = 1\) into the first equation gives
\begin{equation*}
\begin{split} 2 \cdot 1 + x_2 = 4, \end{split}
\end{equation*}
\sphinxAtStartPar
so \(x_2 = 2\). The solution should satisfy all equations in the system, so we can also check that these values also satisfy the second equation:
\begin{equation*}
\begin{split} 4x_1 + 3x_2  = 4 \cdot 1 + 3 \cdot 2 = 4 + 6 = 10 \end{split}
\end{equation*}
\sphinxAtStartPar
So we know that \(x_1 = 1\) and \(x_2 = 2\) is the solution to this system. While this approach is simple to implement for small systems of linear equations, the algebra will soon become unwieldy for larger systems, which is why we use the methods that are covered in this chapter.


\section{Requirements for a system to be solvable}
\label{\detokenize{_pages/2.0_Linear_systems:requirements-for-a-system-to-be-solvable}}\label{\detokenize{_pages/2.0_Linear_systems:requirements-for-systems-of-linear-equations-to-be-solvable-section}}
\sphinxAtStartPar
Consider this system of two linear equations:
\begin{equation*}
\begin{split} \begin{align*}
    ax + by &= e, \\
    cx + dy &= f.
\end{align*} \end{split}
\end{equation*}
\sphinxAtStartPar
To solve for \(x\), we could look to eliminate \(y\) by multiplying the second equation by \(\dfrac{b}{d}\) to give
\begin{equation*}
\begin{split} \frac{b}{d}c x + by = \frac{b}{d}f,\end{split}
\end{equation*}
\sphinxAtStartPar
Subtract this from the first equation
\begin{equation*}
\begin{split} \begin{align*}
    ax - \frac{b}{d}c x &= e - \frac{b}{d}f \\
    (ad - bc)x &= (de - bf) \\
    x &= \frac{de - bf}{ad - bc}.
\end{align*} \end{split}
\end{equation*}
\sphinxAtStartPar
We can also solve for \(y\) by multiplying the first equation by \(\dfrac{c}{a}\) to give
\begin{equation*}
\begin{split} c x + \frac{c}{a} b y = \frac{c}{a} e. \end{split}
\end{equation*}
\sphinxAtStartPar
Subtract this from the second equation
\begin{equation*}
\begin{split} \begin{align*}
    d y - \frac{c}{a} b y &= f - \frac{c}{a} e \\
    (ad - bc) y &= af - ce \\
    y &= \frac{af - ce}{ad - bc}.
\end{align*} \end{split}
\end{equation*}
\sphinxAtStartPar
The denominators (bottom half of the fractions) in the solutions to \(x\) and \(y\) are both \(a d - b c\), so if this value is zero then the system of equations does not have a solution. If we write the system using matrices:
\begin{equation*}
\begin{split} \begin{align*}
    \begin{pmatrix} a & b \\ c & d \end{pmatrix}
    \begin{pmatrix} x \\ y \end{pmatrix} =
    \begin{pmatrix} e \\ f \end{pmatrix} \quad \implies \quad
    \begin{matrix}
        ax + by \!\!\! &= e, \\
        cx + dy \!\!\! &= f.
    \end{matrix}
\end{align*} \end{split}
\end{equation*}
\sphinxAtStartPar
We see that the requirement for the system to have a solution is the same as the requirement for matrix to be invertible, i.e, that the value of \(ad-bc\) is non\sphinxhyphen{}zero. As we will see, inverse matrices are an important tool in solving systems of linear equations.

\sphinxstepscope

\index{Systems of linear equations@\spxentry{Systems of linear equations}!solution using inverse matrix@\spxentry{solution using inverse matrix}}\ignorespaces 

\section{Solving systems of linear equations using inverse matrices}
\label{\detokenize{_pages/2.1_Solving_using_inverse_matrix:solving-systems-of-linear-equations-using-inverse-matrices}}\label{\detokenize{_pages/2.1_Solving_using_inverse_matrix:index-0}}\label{\detokenize{_pages/2.1_Solving_using_inverse_matrix:solving-systems-using-inverse-section}}\label{\detokenize{_pages/2.1_Solving_using_inverse_matrix::doc}}
\sphinxAtStartPar
Given the system of linear equations \(A \mathbf{x} = \mathbf{b}\), we have:
\begin{equation*}
\begin{split} \begin{align*}
    A\mathbf{x} &= \mathbf{b}\\
    A^{-1}A \mathbf{x} &= A^{-1}\mathbf{b}\\
    \mathbf{x} &= A^{-1}\mathbf{b} \qquad (\text{since } A^{-1}A = I).
\end{align*} \end{split}
\end{equation*}
\sphinxAtStartPar
So we can use the inverse of the coefficient matrix \(A\) to solve for \(\mathbf{x}\).
\label{_pages/2.1_Solving_using_inverse_matrix:solution-using-inverse-matrix-theorem}
\begin{sphinxadmonition}{note}{Theorem 2.4.1 (Solution of a linear system of equations using the inverse matrix)}



\sphinxAtStartPar
The solution to a system of linear equations \(A\mathbf{x} = \mathbf{b}\) can be calculated using \(\mathbf{x} = A^{-1}\mathbf{b}\).
\end{sphinxadmonition}

\sphinxAtStartPar
Consider the system of linear equations from the {\hyperref[\detokenize{_pages/2.0_Linear_systems:systems-of-linear-equations-chapter}]{\sphinxcrossref{\DUrole{std,std-ref}{previous section}}}}:
\begin{equation*}
\begin{split} \begin{pmatrix}
    2 & 1 \\
    4 & 3
\end{pmatrix}
\begin{pmatrix}
    x_1 \\ x_2
\end{pmatrix} =
\begin{pmatrix}
    4 \\ 10
\end{pmatrix}. \end{split}
\end{equation*}
\sphinxAtStartPar
The coefficient matrix and constant vector are
\begin{equation*}
\begin{split} \begin{align*}
    A &= \begin{pmatrix}
        2 & 1 \\
        4 & 3
    \end{pmatrix}, &
    \mathbf{b} &= \begin{pmatrix} 4 \\ 10 \end{pmatrix},
\end{align*} \end{split}
\end{equation*}
\sphinxAtStartPar
and the inverse of this coefficient matrix can be easily calculated using the {\hyperref[\detokenize{_pages/1.5_Inverse_matrix:adjoint-determinant-formula-theorem}]{\sphinxcrossref{adjoint\sphinxhyphen{}determinant formula}}} to give
\begin{equation*}
\begin{split} A^{-1} = \begin{pmatrix}
    2 & 1 \\
    4 & 3
\end{pmatrix}^{-1}
= \frac{1}{2} \begin{pmatrix}
    3 & -1 \\
    -4 & 2
\end{pmatrix}. \end{split}
\end{equation*}
\sphinxAtStartPar
Substituting this inverse matrix into the equation \(\mathbf{x} = A^{-1}\mathbf{b}\), we find
\begin{equation*}
\begin{split} \mathbf{x} = \frac{1}{2}
\begin{pmatrix}
    3 & -1 \\
    -4 & 2
\end{pmatrix}
    \begin{pmatrix} 4 \\ 10 \end{pmatrix} =
    \begin{pmatrix} 1 \\ 2 \end{pmatrix},\end{split}
\end{equation*}
\sphinxAtStartPar
so \(x_1 = 1\) and \(x_2 = 2\) (which matches what we saw when we used {\hyperref[\detokenize{_pages/2.0_Linear_systems:solving-systems-of-linear-equations-using-algebra-section}]{\sphinxcrossref{\DUrole{std,std-ref}{algebra to solve this system}}}}). We can check whether this is the correct solution by substituting \(\mathbf{x}\) into the matrix equation \(A\mathbf{x} = \mathbf{b}\):
\begin{equation*}
\begin{split}A \mathbf{x} = \begin{pmatrix} 2 & 1 \\ 4 & 3 \end{pmatrix}
    \begin{pmatrix} 1 \\ 2 \end{pmatrix} =
    \begin{pmatrix} 4 \\ 10 \end{pmatrix} = \mathbf{b}.\end{split}
\end{equation*}

\label{_pages/2.1_Solving_using_inverse_matrix:solution-by-inverse-example}
\begin{sphinxadmonition}{note}{Example 2.4.1}



\sphinxAtStartPar
Solve the following systems of linear equations using the inverse of the coefficient matrix:

\sphinxAtStartPar
(i)   \(\begin{array}{rl}
        x_1 - 2x_2  \!\!\!\! &= 11, \\
        2x_1 + 4x_2  \!\!\!\! &= -18.
    \end{array}\);  
(ii)   \(\begin{array}{rl}
        x_1 - 2x_2 + 3x_3 \!\!\!\! &= -7, \\
        2x_2 - 4x_3  \!\!\!\! &= 8, \\
        3x_1 + x_2 - 4x_3  \!\!\!\! &= 7.
    \end{array}\)
\subsubsection*{Solution}

\sphinxAtStartPar
(i)   Here \(A = \begin{pmatrix} 1 & -2 \\ 2 & 4 \end{pmatrix}\) and \(\mathbf{b} = \begin{pmatrix} 11 \\ -18 \end{pmatrix}\). Calculating \(A^{-1}\):
\begin{equation*}
\begin{split} \begin{align*}
    \det\begin{pmatrix} 1 & -2 \\ 2 & 4 \end{pmatrix} &= 8, \\
    \operatorname{adj}\begin{pmatrix} 1 & -2 \\ 2 & 4 \end{pmatrix} &=
    \begin{pmatrix} 4 & -2 \\ 2 & 1 \end{pmatrix}^\mathsf{T} =
    \begin{pmatrix} 4 & 2 \\ -2 & 1 \end{pmatrix},\\
    \therefore A^{-1} &= \frac{1}{8} \begin{pmatrix} 4 & 2 \\ -2 & 1 \end{pmatrix}.
\end{align*} \end{split}
\end{equation*}
\sphinxAtStartPar
So the solution is
\begin{equation*}
\begin{split} \begin{align*}
    \mathbf{x} &= \frac{1}{8} \begin{pmatrix} 4 & 2 \\ -2 & 1 \end{pmatrix}
    \begin{pmatrix} 11 \\ -18 \end{pmatrix} =
    \frac{1}{8} \begin{pmatrix} 8 \\ -40 \end{pmatrix} =
    \begin{pmatrix} 1 \\ -5 \end{pmatrix}.
\end{align*} \end{split}
\end{equation*}
\sphinxAtStartPar
Checking the solution
\begin{equation*}
\begin{split}A\mathbf{x} = \begin{pmatrix} 1 & -2 \\ 2 & 4 \end{pmatrix} \begin{pmatrix} 1 \\ -5 \end{pmatrix} = \begin{pmatrix}  11 \\  -18    \end{pmatrix}  = \mathbf{b} \qquad \checkmark\end{split}
\end{equation*}
\sphinxAtStartPar
(ii)   Here \(A = \begin{pmatrix} 1 & -2 & 3 \\ 0 & 2 & -4 \\ 3 & 1 & -4 \end{pmatrix}\) and \(\mathbf{b} = \begin{pmatrix}-7 \\ 8 \\ 7 \end{pmatrix}\). To calculate \(A^{-1}\), we can start by finding the determinant \sphinxhyphen{} expanding down the first column, to take advantage of the zero entry:
\begin{equation*}
\begin{split} \begin{align*}
    \det \begin{pmatrix} 1 & -2 & 3 \\ 0 & 2 & -4 \\ 3 & 1 & -4 \end{pmatrix} &=
    1 \begin{vmatrix} 2 & -4 \\ 1 & -4 \end{vmatrix} +
    3 \begin{vmatrix} -2 & 3 \\ 2 & -4 \end{vmatrix} = -4 + 3 \cdot 2 = 2
  \end{align*} 
\end{split}
\end{equation*}
\sphinxAtStartPar
Then we find the adjoint matrix, by calculating the matrix of cofactors and transposing it:
\begin{equation*}
\begin{split} \begin{align*}
    \operatorname{adj} \begin{pmatrix} 1 & -2 & 3 \\ 0 & 2 & -4 \\ 3 & 1 & -4 \end{pmatrix}
    &= \begin{pmatrix} -4 & -12 & -6 \\ -5 & -13 & -7 \\ 2 & 4 & 2 \end{pmatrix}^\mathsf{T}
    = \begin{pmatrix} -4 & -5 & 2 \\ -12 & -13 & 4 \\ -6 & -7 & 2 \end{pmatrix}, \\
    \therefore A^{-1} &= \frac{1}{2}
    \begin{pmatrix}
        -4 & -5 & 2 \\
        -12 & -13 & 4 \\
        -6 & -7 & 2
    \end{pmatrix}.
\end{align*} \end{split}
\end{equation*}
\sphinxAtStartPar
So the solution is
\begin{equation*}
\begin{split} \begin{align*}
    \mathbf{x} &= \frac{1}{2}
    \begin{pmatrix}
        -4 & -5 & 2 \\
        -12 & -13 & 4 \\
        -6 & -7 & 2
    \end{pmatrix}
    \begin{pmatrix}-7 \\ 8 \\ 7 \end{pmatrix}
    = \frac{1}{2} \begin{pmatrix} 2 \\ 8 \\ 0 \end{pmatrix}
    = \begin{pmatrix} 1 \\4 \\ 0 \end{pmatrix}.
\end{align*} \end{split}
\end{equation*}
\sphinxAtStartPar
Checking the solution:
\begin{equation*}
\begin{split}A \mathbf{x} = \begin{pmatrix} 1 & -2 &  3 \\ 0 & 2 & -4 \\ 3 & 1 & -4 \end{pmatrix} \begin{pmatrix} 1 \\ 4 \\ 0 \end{pmatrix} = \begin{pmatrix} -7 \\ 8 \\ 7 \end{pmatrix} = \mathbf{b} \qquad \checkmark\end{split}
\end{equation*}\end{sphinxadmonition}

\sphinxstepscope

\index{Cramer's rule@\spxentry{Cramer's rule}}\ignorespaces 
\index{Systems of linear equations@\spxentry{Systems of linear equations}!solution using Cramer's rule@\spxentry{solution using Cramer's rule}}\ignorespaces 

\section{Cramer’s rule}
\label{\detokenize{_pages/2.2_Cramers_rule:cramer-s-rule}}\label{\detokenize{_pages/2.2_Cramers_rule:index-1}}\label{\detokenize{_pages/2.2_Cramers_rule:index-0}}\label{\detokenize{_pages/2.2_Cramers_rule:cramers-rule-section}}\label{\detokenize{_pages/2.2_Cramers_rule::doc}}
\begin{figure}[htbp]
\centering
\capstart

\noindent\sphinxincludegraphics[width=200\sphinxpxdimen]{{220px-Gabriel_Cramer}.jpg}
\caption{Gabriel Cramer (1704 \sphinxhyphen{} 1752)}\label{\detokenize{_pages/2.2_Cramers_rule:id1}}\end{figure}

\sphinxAtStartPar
\sphinxstylestrong{Cramer’s rule}, named after Swiss mathematician Gabriel Cramer, is an explicit rule for calculating the solution to a system of linear equations using determinants. We saw in the section on {\hyperref[\detokenize{_pages/2.0_Linear_systems:requirements-for-systems-of-linear-equations-to-be-solvable-section}]{\sphinxcrossref{\DUrole{std,std-ref}{when systems are solvable}}}}) that the solution to the system of linear equations
\begin{equation*}
\begin{split} \begin{pmatrix} a & b \\ c & d \end{pmatrix}
    \begin{pmatrix} x_1 \\ x_2 \end{pmatrix} =
    \begin{pmatrix} e \\ f \end{pmatrix}\end{split}
\end{equation*}
\sphinxAtStartPar
is
\begin{equation*}
\begin{split} \begin{align*}
    x_1 &= \frac{de - bf}{ad - bc}, & \qquad
    x_2 &= \frac{af - ce}{ad - bc}.
\end{align*} \end{split}
\end{equation*}
\sphinxAtStartPar
The solution to both variables includes the determinant of the coefficient matrix, \(ad - bc\), in the denominator. But what about the numerator?

\sphinxAtStartPar
If we consider the numerator of \(x_1\), which is \(de - bf\), we can see that it includes the constant values \(e\) and \(f\), and it is the difference of two products \sphinxhyphen{} similar to the determinant of a \(2 \times 2\) matrix. In fact, we can write down the matrix this is the determinant of:
\begin{equation*}
\begin{split} \begin{align*}
    de - bf = \begin{vmatrix} e & b \\ f & d \end{vmatrix}.
\end{align*} \end{split}
\end{equation*}
\sphinxAtStartPar
Performing a similar analysis for \(x_2\), we see that the numerator is also the determinant of a matrix:
\begin{equation*}
\begin{split} \begin{align*}
    af - ce = \begin{vmatrix} a & e \\ c & f \end{vmatrix}.
\end{align*} \end{split}
\end{equation*}
\sphinxAtStartPar
On inspection, these matrices are simply the coefficient matrix \(\begin{pmatrix} a & b \\ c & d \end{pmatrix}\) with columns 1 and 2 replaced by the constant vector \(\begin{pmatrix} e \\ f \end{pmatrix}\) for \(x_1\) and \(x_2\) respectively. This idea can be extended to larger systems, and is known as Cramer’s rule.
\label{None:cramers-rule-theorem}
\begin{sphinxadmonition}{note}{Theorem  (Cramer’s rule)}



\sphinxAtStartPar
The solution to a non\sphinxhyphen{}singular linear system of equations of the form \(A\mathbf{x}=\mathbf{b}\) can be calculated using Cramer’s rule which is
\begin{equation}\label{equation:_pages/2.2_Cramers_rule:cramers-rule-equation}
\begin{split} x_i = \frac{\det(A_i)}{\det(A)}, \end{split}
\end{equation}
\sphinxAtStartPar
where \(A_i\) is the matrix obtained by replacing column \(i\) of \(A\) with \(\mathbf{b}\).
\end{sphinxadmonition}

\begin{sphinxadmonition}{note}
\sphinxAtStartPar
Proof. The solution to a system of linear equations \(A \mathbf{x} = \mathbf{b}\) can be calculated as \(\mathbf{x} = A^{-1} \mathbf{b}\), where \(A^{-1}\) is the inverse of the coefficient matrix \(A\). The {\hyperref[\detokenize{_pages/1.5_Inverse_matrix:adjoint-determinant-formula-theorem}]{\sphinxcrossref{adjoint\sphinxhyphen{}determinant formula}}} for calculating the inverse is
\begin{equation*}
\begin{split} A^{-1} = \frac{\operatorname{adj}(A)}{\det(A)}, \end{split}
\end{equation*}
\sphinxAtStartPar
so
\begin{equation*}
\begin{split} \mathbf{x} = \frac{1}{\det(A)} \operatorname{adj}(A) \mathbf{b}. \end{split}
\end{equation*}
\sphinxAtStartPar
The matrix \(\operatorname{adj}(A) = C^\mathsf{T}\) \sphinxhyphen{} it is the transpose of the matrix of {\hyperref[\detokenize{_pages/1.4_Determinants:cofactor-definition}]{\sphinxcrossref{cofactors}}}:
\begin{equation*}
\begin{split} \operatorname{adj}(A) = C^\mathsf{T} = \begin{pmatrix}
    C_{11} & C_{21} & \cdots & C_{n1} \\
    C_{12} & C_{22} & \cdots & C_{n2} \\
    \vdots & \vdots & \ddots & \vdots \\
    C_{1n} & C_{2n} & \cdots & C_{nn}
\end{pmatrix}. \end{split}
\end{equation*}
\sphinxAtStartPar
then for the \(i\)th element of \(\mathbf{x}\), using the {\hyperref[\detokenize{_pages/1.2_Matrix_multiplication:matrix-multiplication-definition}]{\sphinxcrossref{definition of matrix multiplication}}} we have
\begin{equation*}
\begin{split} x_i = \frac{1}{\det(A)} \sum_{j=1}^n C_{ji} b_j \end{split}
\end{equation*}
\sphinxAtStartPar
In Cramer’s rule, \(A_i\) is the matrix formed by replacing the \(i\)th column of \(A\) with \(\mathbf{b}\):
\begin{equation*}
\begin{split} A_i = \begin{pmatrix}
        a_{11} & \cdots & a_{1,i-1} & b_1 & a_{1,i+1} & \cdots & a_{1n} \\
        a_{21} & \cdots & a_{2,i-1} & b_2 & a_{2,i+1} & \cdots & a_{2n} \\
        \vdots & \ddots & \vdots & \vdots & \vdots & \ddots & \vdots \\
        a_{n1} & \cdots & a_{n,i-1} & b_n & a_{n,i+1} & \cdots & a_{nn}
    \end{pmatrix}. \end{split}
\end{equation*}
\sphinxAtStartPar
Since removing the \(i\)th column from both \(A\) and \(A_i\) (the one with the \(b\) values) results in the same matrix, then the cofactors of \(A_i\) are the same as the cofactors of \(A\). If we calculate \(\det(A_i)\) by expanding along the \(i\)th column of \(A_i\) then
\begin{equation*}
\begin{split} \det(A_i) = b_1 C_{1i} + b_2 C_{2i} + \cdots + b_n C_{ni} = \sum_{j=1}^n C_{ji} b_j, \end{split}
\end{equation*}
\sphinxAtStartPar
so \(x_i = \dfrac{\det(A_i)}{\det(A)}\).
\end{sphinxadmonition}


\label{None:cramers-rule-example}
\begin{sphinxadmonition}{note}{Example }



\sphinxAtStartPar
Solve the following systems of linear equations using Cramer’s rule

\sphinxAtStartPar
(i)   \(\begin{array}{rl}
    3x_1 - 2x_2  \!\!\!\! &= -4, \\
    x_1 - 3x_2  \!\!\!\! &= 1.
\end{array}\);  
(ii)   \(\begin{array}{rl}
    -2x_1 - 3x_2 - x_3 \!\!\!\!  &= -5, \\
    -4x_1 + 4x_2 + 3x_3  \!\!\!\! &= -20, \\
    -3x_1  \!\!\!\! &= -12.
\end{array}\)
\subsubsection*{Solution}

\sphinxAtStartPar
(i)   Here \(A = \begin{pmatrix} 3 & -2 \\ 1 & -3 \end{pmatrix}\) and \(\mathbf{b} = \begin{pmatrix} -4 \\ 1 \end{pmatrix}\)
\begin{equation*}
\begin{split} \begin{align*}
    x_1 &= \frac{
    \begin{vmatrix} -4 & -2 \\ 1 & -3 \end{vmatrix}}
    {\begin{vmatrix} 3 & -2 \\ 1 & -3 \end{vmatrix}} = \frac{14}{-7} = -2,  \\
    \\
    x_2 &= \frac{\begin{vmatrix} 3 & -4 \\ 1 & 1 \end{vmatrix}}{-7} =
    \frac{7}{-7} = -1.
\end{align*} \end{split}
\end{equation*}
\sphinxAtStartPar
Checking the solution
\begin{equation*}
\begin{split} \begin{align*}
    A\mathbf{x} = \begin{pmatrix} 3 & -2 \\ 1 & -3 \end{pmatrix}
    \begin{pmatrix} -2 \\ -1 \end{pmatrix} =
    \begin{pmatrix} -4 \\ 1 \end{pmatrix} = \mathbf{b} \qquad \checkmark
\end{align*} \end{split}
\end{equation*}
\sphinxAtStartPar
(ii)   Here \(A = \begin{pmatrix} -2 & -3 & -1 \\ -4 & 4 & 3 \\ -3 & 0 & 0 \end{pmatrix}\) and \(\mathbf{b} = \begin{pmatrix} -5 \\ -20 \\ -12 \end{pmatrix}\)
\begin{equation*}
\begin{split} \begin{align*}
    x_1 &= \frac{
    \begin{vmatrix}-5 & -3 & -1 \\ -20 & 4 & 3 \\ -12 & 0 & 0 \end{vmatrix}}
    {\begin{vmatrix} -2 & -3 & -1 \\ -4 & 4 & 3 \\ -3 & 0 & 0 \end{vmatrix}}
    = \frac{-12
    \begin{vmatrix} -3 & -1 \\ 4 & 3 \end{vmatrix}}
    {-3\begin{vmatrix} -3 & -1 \\ 4 & 3 \end{vmatrix}}
    = \frac{-12 \cdot -5}{-3 \cdot -5} = \frac{60}{15} = 4,\\ \\
    x_2 &= \frac{
    \begin{vmatrix} -2 & -5 & -1 \\ -4 & -20 & 3 \\ -3 & -12 & 0 \end{vmatrix}}{15}
    = \frac{-\begin{vmatrix} -4 & -20 \\ -3 & -12 \end{vmatrix}
    - 3\begin{vmatrix}-2 & -5 \\ -3 & -12 \end{vmatrix}}{15}
    = \frac{12 - 3 \cdot 9}{15} = \frac{-15}{15} = -1, \\ \\
    x_3 &= \frac{
    \begin{vmatrix} -2 & -3 & - 5 \\ -4 & 4 & -20 \\ -3 & 0 & -12 \end{vmatrix}}{15}
    =\frac{-3\begin{vmatrix} -3 & -5 \\ 4 & -20 \end{vmatrix} -
    12\begin{vmatrix} -2 & -3 \\ -4 & 4 \end{vmatrix}}{15}
     = \frac{-3 \cdot 80 - 12 \cdot -20}{15} = \frac{0}{15} = 0.
\end{align*} \end{split}
\end{equation*}
\sphinxAtStartPar
Checking the solution
\begin{equation*}
\begin{split} \begin{align*}
    A\mathbf{x} =
    \begin{pmatrix} -2 & -3 & -1 \\ -4 & 4 & 3 \\ -3 & 0 & 0 \end{pmatrix}
    \begin{pmatrix} 4 \\ -1 \\ 0 \end{pmatrix} =
    \begin{pmatrix} -5 \\ -20 \\ -12 \end{pmatrix} = \mathbf{b} \qquad \checkmark
\end{align*} \end{split}
\end{equation*}\end{sphinxadmonition}

\sphinxstepscope

\index{Gaussian elimination@\spxentry{Gaussian elimination}}\ignorespaces 
\index{Systems of linear equations@\spxentry{Systems of linear equations}!solution using Gaussian elimination@\spxentry{solution using Gaussian elimination}}\ignorespaces 

\section{Gaussian elimination}
\label{\detokenize{_pages/2.3_Gaussian_elimination:gaussian-elimination}}\label{\detokenize{_pages/2.3_Gaussian_elimination:index-1}}\label{\detokenize{_pages/2.3_Gaussian_elimination:index-0}}\label{\detokenize{_pages/2.3_Gaussian_elimination:gaussian-elimination-section}}\label{\detokenize{_pages/2.3_Gaussian_elimination::doc}}
\begin{figure}[htbp]
\centering
\capstart

\noindent\sphinxincludegraphics[width=200\sphinxpxdimen]{{3f33ffb4c899d95865b215561fd9005b4e59b3ff}.jpg}
\caption{Carl Friedrich Gauss (1777 \sphinxhyphen{} 1855)}\label{\detokenize{_pages/2.3_Gaussian_elimination:id1}}\end{figure}

\sphinxAtStartPar
\sphinxstylestrong{Gaussian elimination (GE)}, named after German mathematician Carl Friedrich Gauss is an algorithm for solving systems of linear equations. It is the most common method used in practice, since it can by easily implemented by computer and applied to larger systems.

\sphinxAtStartPar
Consider the following method for solving a linear system of three equations in three unknowns:
\begin{equation*}
\begin{split} \begin{align*}
    3x_1 + x_2 - 2x_3 &= 1, \\
    x_1 - x_2 + 2x_3 &= 3, \\
    2x_1 - 3x_2 + 7x_3 &= 4.
\end{align*} \end{split}
\end{equation*}
\sphinxAtStartPar
Swap the first two equations around:
\begin{equation*}
\begin{split} \begin{align*}
    x_1 - x_2 + 2x_3 &= 3, \\
    3x_1 + x_2 - 2x_3 &= 1, \\
    2x_1 - 3x_2 + 7x_3 &= 4.
\end{align*} \end{split}
\end{equation*}
\sphinxAtStartPar
Subtract 3 times the first equation from the second equation, and 2 times the first equation from the third equation:
\begin{equation*}
\begin{split} \begin{align*}
    x_1 - x_2 + 2x_3 &= 3, \\
    4x_2 - 8x_3 &= -8, \\
    - x_2 + 3x_3 &= -2.
\end{align*} \end{split}
\end{equation*}
\sphinxAtStartPar
Multiply the second equation by \(\dfrac{1}{4}\)
\begin{equation*}
\begin{split} \begin{align*}
    x_1 - x_2 + 2x_3 &= 3, \\
    x_2 - 2 x_3 &= -2, \\
    - x_2 + 3x_3 &= -2.
\end{align*} \end{split}
\end{equation*}
\sphinxAtStartPar
Add the second equation to the third
\begin{equation*}
\begin{split} \begin{align*}
    x_1 - x_2 + 2x_3 &= 3, \\
    x_2 - 2 x_3 &= -2, \\
    x_3 &= -4.
\end{align*} \end{split}
\end{equation*}
\sphinxAtStartPar
Here, the third equation gives the solution to \(x_3=-4\). We can substitute the value of \(x_3\) into the other two to find the solutions of \(x_2\) and \(x_1\):
\begin{equation*}
\begin{split} \begin{align*}
    x_2 - 2 \cdot -4 &= -2, \\
    \therefore x_2 &= -2 + (2 \cdot -4) = -10, \\
    x_1 - (-10) + (-8) &= 3 \\
    \therefore x_1 &= 3 + (-10) - (-8) = 1.
\end{align*} \end{split}
\end{equation*}
\sphinxAtStartPar
In this method, we used three types of operations on the equations in the system. These operations are known as elementary row operations.

\index{Elementary row operations@\spxentry{Elementary row operations}}\ignorespaces \label{_pages/2.3_Gaussian_elimination:ero-definition}
\begin{sphinxadmonition}{note}{Definition 2.6.1 (Elementary Row Operations (EROs))}



\sphinxAtStartPar
The three \sphinxstylestrong{elementary row operations} that can be applied to a linear system of equations without changing the solution are
\begin{itemize}
\item {} 
\sphinxAtStartPar
Type I: swap any two rows of the system

\item {} 
\sphinxAtStartPar
Type II: multiply one row by a non\sphinxhyphen{}zero scalar

\item {} 
\sphinxAtStartPar
Type III: replace a single row by itself plus a multiple of another row

\end{itemize}
\end{sphinxadmonition}

\sphinxAtStartPar
In the solution to the linear system of equations shown above, we used a type I row operation in step 1, a type II row operation in step 3 and type III row operations in steps 2 and 4.

\sphinxAtStartPar
We can represent the linear system using matrices for convenience. We begin by expressing the linear system using an \sphinxstylestrong{augmented matrix}, made by concatenating together \(A\) and \(\mathbf{x}\) \sphinxhyphen{} so any EROs that are applied to the augmented matrix are applied to the coefficients and the constant terms at the same time.

\index{Augmented matrix@\spxentry{Augmented matrix}}\ignorespaces \label{_pages/2.3_Gaussian_elimination:augmented-matrix-definition}
\begin{sphinxadmonition}{note}{Definition 2.6.2 (Augmented matrix)}



\sphinxAtStartPar
The \sphinxstylestrong{augmented matrix} is a representation of a system of linear equations \(A\mathbf{x}=\mathbf{b}\) such that the \(m\times n\) coefficient matrix \(A\) and right\sphinxhyphen{}hand side constant vector \(\mathbf{b}\) are combined into a single \(m\times (n+1)\) matrix \((A \mid \mathbf{b})\).
\end{sphinxadmonition}

\sphinxAtStartPar
When writing the augmented matrix we often draw a partition separating \(A\) and \(\mathbf{b}\) (although this is not strictly necessary), i.e.,
\begin{equation*}
\begin{split} \left(
\begin{array}{cccc|c}
    a_{11} & a_{12} & \cdots & a_{1n} & b_1 \\
    a_{21} & a_{22} & \cdots & a_{2n} & b_2 \\
    \vdots & \vdots & \ddots & \vdots & \vdots \\
    a_{m1} & a_{m2} & \cdots & a_{mn} & b_n
\end{array}
\right). \end{split}
\end{equation*}
\sphinxAtStartPar
Elementary row operations are applied to the augmented matrix, to convert it to a state called \sphinxstylestrong{row echelon form}, from which the solution of the system can be easily calculated. We often call this process \sphinxstylestrong{reduction}, or sometimes \sphinxstylestrong{row reduction}.

\index{Gaussian elimination@\spxentry{Gaussian elimination}!pivot element@\spxentry{pivot element}}\ignorespaces 
\index{Row echelon form@\spxentry{Row echelon form}}\ignorespaces \label{_pages/2.3_Gaussian_elimination:ref-definition}
\begin{sphinxadmonition}{note}{Definition 2.6.3 (Row Echelon Form (REF))}



\sphinxAtStartPar
A matrix is said to be in \sphinxstylestrong{Row Echelon Form (REF)} if the following conditions are satisfied:
\begin{itemize}
\item {} 
\sphinxAtStartPar
Any non\sphinxhyphen{}zero rows are above any all\sphinxhyphen{}zero rows

\item {} 
\sphinxAtStartPar
In each non\sphinxhyphen{}zero row the \sphinxstylestrong{pivot element}, the first non\sphinxhyphen{}zero element in the row, is to the right of the pivot element in the row above

\end{itemize}
\end{sphinxadmonition}

\sphinxAtStartPar
For example, the following matrices are in row echelon form and the red numbers are the pivot elements
\begin{equation*}
\begin{split} \begin{align*}
    &\begin{pmatrix} \color{red}{1} & 2 \\ 0 & \color{red}{3} \end{pmatrix} &
    &\begin{pmatrix} \color{red}{1} & 2 & 3 \\ 0 & 0 & \color{red}{4} \end{pmatrix} &
    &\begin{pmatrix} 0 & \color{red}{1} & 3\\ 0 & 0 & \color{red}{4} \\ 0 & 0 & 0\end{pmatrix}.
\end{align*} \end{split}
\end{equation*}
\sphinxAtStartPar
Note that the elements below the pivot elements are all zero. We say that a system of linear equations is in row echelon form if the corresponding augmented matrix is in row echelon form.


\bigskip\hrule\bigskip


\index{Gaussian elimination@\spxentry{Gaussian elimination}!row reduction@\spxentry{row reduction}}\ignorespaces 

\subsection{Row reduction}
\label{\detokenize{_pages/2.3_Gaussian_elimination:row-reduction}}\label{\detokenize{_pages/2.3_Gaussian_elimination:index-6}}
\sphinxAtStartPar
The process of transforming a matrix into row echelon form using elementary row operations is known as \sphinxstylestrong{row reduction}. For example, we will use Gaussian elimination to solve the same system of linear equations from earlier:
\begin{equation*}
\begin{split} \begin{align*}
    3x_1 + x_2 - 2x_3 &= 1, \\
    x_1 - x_2 + 2x_3 &= 3, \\
    2x_1 - 3x_2 + 7x_3 &= 4.
\end{align*} \end{split}
\end{equation*}
\sphinxAtStartPar
Expressing this using an augmented matrix, we have
\begin{equation*}
\begin{split} \left( \begin{array}{ccc|c}
        \color{red}{3} & 1 & -2 & 1 \\
        \color{blue}{1} & -1 & 2 & 3 \\
        \color{blue}{2} & -3 & 7 & 4
    \end{array} \right). \end{split}
\end{equation*}
\sphinxAtStartPar
The first pivot element is in row 1, column 1, and has a value of 3. We need to apply row operations so that the elements in the column beneath the pivot element are zero. To do so we add a multiple of the pivot row to each of the rows beneath (a {\hyperref[\detokenize{_pages/2.3_Gaussian_elimination:ero-definition}]{\sphinxcrossref{type III}}} row operation).

\sphinxAtStartPar
Since the pivot element is 3 and the first non\sphinxhyphen{}zero element in row 2 is 1, to reduce this to zero we subtract row 1 multiplied by \(\dfrac{1}{3}\) from row 2.
\begin{equation*}
\begin{split} \begin{align*}
  \left( \begin{array}{ccc|c}
      \color{red}{ 3} & 1 & -2 & 1 \\
      \color{blue}{ 1 - (\frac{1}{3}\cdot 3)} & -1 - (\frac{1}{3}\cdot 1) & 2 - (\frac{1}{3}\cdot -2) & 3 - (\frac{1}{3}\cdot 1) \\
      \color{blue}{ 2} & -3 & 7 & 4
  \end{array} \right) &
  \longrightarrow
  \left( \begin{array}{ccc|c}
      \color{red}{ 3} & 1 & -2 & 1 \\
      \color{blue}{ 0} & -\frac{4}{3} & \frac{8}{3} & \frac{8}{3} \\
      \color{blue}{ 2} & -3 & 7 & 4
  \end{array} \right).
\end{align*} \end{split}
\end{equation*}
\sphinxAtStartPar
We also need to do the same to row 3. Since the element row 3 column 1 has a value of 2, we need to subtract row 1 multiplied by \(\dfrac{2}{3}\) from row 3.
\begin{equation*}
\begin{split} \left( \begin{array}{ccc|c}
    \color{red}{3} & 1 & -2 & 1 \\
    \color{blue}{0} & -\frac{4}{3} & \frac{8}{3} & \frac{8}{3} \\
    \color{blue}{2 - (\frac{2}{3}\cdot 3)} & -3 - (\frac{2}{3}\cdot 1) & 7 - (\frac{2}{3}\cdot -2) & 4 - (\frac{2}{3}\cdot 1)
\end{array} \right)
\longrightarrow  
\left( \begin{array}{ccc|c}
    \color{red}{3} & 1 & -2 & 1 \\
    \color{blue}{0} & -\frac{4}{3} & \frac{8}{3} & \frac{8}{3} \\
    \color{blue}{0} & -\frac{11}{3} & \frac{25}{3} & \frac{10}{3}
\end{array} \right). \end{split}
\end{equation*}
\sphinxAtStartPar
Note that these two row operations could have been done simultaneously, since changing the values in row 2 does not affect row 3 and vice\sphinxhyphen{}versa. Column 1 is now in row echelon form, so we move to the next pivot element in row 2, which is \(-\dfrac{4}{3}\).
\begin{equation*}
\begin{split} \left( \begin{array}{ccc|c}
    3 & 1 & -2 & 1 \\
    0 & \color{red}{-\frac{4}{3}} & \frac{8}{3} & \frac{8}{3} \\
    0 & \color{blue}{-\frac{11}{3}} & \frac{25}{3} & \frac{10}{3}
\end{array} \right). \end{split}
\end{equation*}
\sphinxAtStartPar
The element in row 3 column 2 has a value of \(-\dfrac{11}{3}\) and the pivot element has a value of \(-\dfrac{4}{3}\), so we need to subtract row 2 multiplied by \(-\dfrac{11}{3} \div \left(-\dfrac{4}{3}\right) = \dfrac{11}{4}\) from row 3.
\begin{equation*}
\begin{split} \left( \begin{array}{ccc|c}
    3 & 1 & -2 & 1 \\
    0 & \color{red}{-\frac{4}{3}} & \frac{8}{3} & \frac{8}{3} \\
    0 & \color{blue}{-\frac{11}{3} - (\frac{11}{4}\cdot -\frac{4}{3})} & \frac{25}{3} - (\frac{11}{4}\cdot \frac{8}{3}) & \frac{10}{3} - (\frac{11}{4}\cdot \frac{8}{3})
\end{array} \right)
\longrightarrow
\left( \begin{array}{ccc|c}
    3 & 1 & -2 & 1 \\
    0 & \color{red}{-\frac{4}{3}} & \frac{8}{3} & \frac{8}{3} \\
    0 & \color{blue}{0} & 1 & -4
\end{array} \right). \end{split}
\end{equation*}
\sphinxAtStartPar
Now the augmented matrix is in row echelon form. We can convert back to matrix form and express the linear system as three separate equations.
\begin{equation*}
\begin{split} \begin{align*}
  \begin{pmatrix}
      3 & 1 & -2 \\
      0 & -\frac{4}{3} & \frac{8}{3} \\
      0 & 0 & 1
  \end{pmatrix}
  \begin{pmatrix} x_1 \\ x_2 \\ x_3 \end{pmatrix} 
  = \begin{pmatrix} 1 \\ \frac{8}{3} \\ -4 \end{pmatrix}  
  && \therefore &&
  \begin{array}{rcl}
      3 x_1 + x_2 - 2 x_2 &=& 1, \\
      -\dfrac{4}{3} x_2 - \dfrac{8}{3} x_3 &=& \dfrac{8}{3}, \\
      x_3 &=& -4.
  \end{array}
\end{align*} \end{split}
\end{equation*}
\sphinxAtStartPar
(Note that in the earlier calculation, at this stage, we had our second equation as \(x_2 - 2 x_3 = -2\). This is obtained by multiplying the second equation here by \(-3\), and gives an equivalent statement.)

\index{Back substitution@\spxentry{Back substitution}}\ignorespaces 
\sphinxAtStartPar
Since we have reduced the coefficient matrix to row echelon form, we have a single value for the final variable. We can then substitute known values of the variables into the preceding equations to solve for the preceding variables. We continue in this way until we have solutions for all of the variables in the system. This step is known as \sphinxstylestrong{back substitution}.

\sphinxAtStartPar
So, for our system, the final equation gives \(x_3=-4\), which we can substitute into the second equation to give:
\begin{equation*}
\begin{split} \begin{align*}
    -\frac{4}{3} x_2 + \frac{8}{3} (-4) &= \frac{8}{3} \\
    -\frac{4}{3} x_2 &= \frac{40}{3} \\
    \therefore x_2 &= -10,
\end{align*} \end{split}
\end{equation*}
\sphinxAtStartPar
Then substituting \(x_2\) and \(x_3\) into the first equation gives:
\begin{equation*}
\begin{split} \begin{align*}
    3x_1 + (-10) - 2(-4) &= 1 \\
    3x_1 &= 3 \\
    \therefore x_1 &= 1.
\end{align*} \end{split}
\end{equation*}
\sphinxAtStartPar
In the interest of brevity, the following notation is used to denote the three types of EROs
\begin{itemize}
\item {} 
\sphinxAtStartPar
Swap row \(i\) and row \(j\):   \(R_i \leftrightarrow R_j\)

\item {} 
\sphinxAtStartPar
Multiply row \(i\) by the scalar \(k\):     \(kR_i\)

\item {} 
\sphinxAtStartPar
Add \(k\) times row \(j\) to row \(i\):    \(R_i + kR_j\)

\end{itemize}

\sphinxAtStartPar
Since the EROs do not change the solution to the system of equations, it does not matter which EROs are applied to reduce the augmented matrix, or in what order \sphinxhyphen{} although you can make decisions about this to make your calculations easier.

\sphinxAtStartPar
A common approach is to ensure the pivot elements have a value of 1, which can decrease the number of fractional values, thus simplifying the calculations. For example, consider the following row reduction of the same augmented matrix as before.
\begin{equation*}
\begin{split} \begin{align*}
    & \left( \begin{array}{ccc|c}
        3 & 1 & -2 & 1 \\
        1 & -1 & 2 & 3 \\
        2 & -3 & 7 & 4
    \end{array} \right)
    \\ \\
    R_1 \leftrightarrow R_2 \longrightarrow &
    \left( \begin{array}{ccc|c}
        1 & -1 & 2 & 3 \\
        3 & 1 & -2 & 1 \\
        2 & -3 & 7 & 4
    \end{array} \right) 
    \\ \\
    
        \begin{array}{l} \\ R_2 - 3 R_1 \\ R_3 - 2 R_1 \end{array} \longrightarrow &
    \left( \begin{array}{ccc|c}
        1 & -1 & 2 & 3 \\
        0 & 4 & -8 & -8 \\
        0 & -1 & 3 & -2
    \end{array} \right)
	  \\ \\
    \dfrac{1}{4}R_2  \longrightarrow &
    \left( \begin{array}{ccc|c}
        1 & -1 & 2 & 3 \\
        0 & 1 & -2 & -2 \\
        0 & -1 & 3 & -2
    \end{array} \right)
	\\ \\
     R_3 + R_2 \longrightarrow &
    \left( \begin{array}{ccc|c}
        1 & -1 & 2 & 3 \\
        0 & 1 & -2 & -2 \\
        0 & 0 & 1 & -4
    \end{array} \right)
\end{align*} \end{split}
\end{equation*}
\sphinxAtStartPar
This gives us a much simpler set of equations to perform back substitution with. Solving this set of equations gives \(x_1 = 1\), \(x_2 = -10\) and \(x_3 = -4\) which was the same solution as we saw before.

\sphinxAtStartPar
\sphinxstylestrong{Gaussian elimination} is a specific way of implementing this process that reduces a matrix to row echelon form in order to solve the associated system of equations. The steps used in Gaussian elimination are summarised in {\hyperref[\detokenize{_pages/2.3_Gaussian_elimination:ge-algorithm}]{\sphinxcrossref{Algorithm 2.6.1}}}.

\index{Gaussian elimination@\spxentry{Gaussian elimination}!algorithm@\spxentry{algorithm}}\ignorespaces \label{_pages/2.3_Gaussian_elimination:ge-algorithm}
\begin{sphinxadmonition}{note}{Algorithm 2.6.1 (Gaussian elimination)}



\sphinxAtStartPar
\sphinxstylestrong{Inputs:} An \(m \times n\) coefficient matrix \(A\) and an \(m\)\sphinxhyphen{}element constant vector \(\mathbf{b}\).

\sphinxAtStartPar
\sphinxstylestrong{Outputs:} An augmented matrix in row echelon form.
\begin{itemize}
\item {} 
\sphinxAtStartPar
Form the augmented matrix \(( A \mid \mathbf{b} )\)

\item {} 
\sphinxAtStartPar
Initialise pivot row \(k\) to 1

\item {} 
\sphinxAtStartPar
For each column \(j\) from 1 to \(n\):
\begin{itemize}
\item {} 
\sphinxAtStartPar
If the pivot element \(a_{kj} = 0\):
\begin{itemize}
\item {} 
\sphinxAtStartPar
Identify a row \(i\) where \(i > k\) and \(a_{ij} \neq 0\); if no such row exists, increment \(j\) and try again

\item {} 
\sphinxAtStartPar
Swap row \(i\) with row \(k\)

\end{itemize}

\end{itemize}

\item {} 
\sphinxAtStartPar
For each row \(i\) from \(k+1\) to \(m\):
\begin{itemize}
\item {} 
\sphinxAtStartPar
Subtract \(\dfrac{a_{ij}}{a_{jj}}\) times row \(j\) from row \(i\)

\end{itemize}

\item {} 
\sphinxAtStartPar
Move to the next pivot row, by incrementing \(k\) to \(k+1\), and repeat until you have done every row

\end{itemize}
\end{sphinxadmonition}

\sphinxAtStartPar
This starts by finding the pivot in each row (looking for the first entry that is non\sphinxhyphen{}zero, and making sure it’s further left than all the ones below it by swapping rows) then subtracting appropriate multiples of that row from all the others, repeating until complete.


\label{_pages/2.3_Gaussian_elimination:ge-example}
\begin{sphinxadmonition}{note}{Example 2.6.1}



\sphinxAtStartPar
Use Gaussian elimination to solve the following systems of linear equations:

\sphinxAtStartPar
(i)   \( \begin{align*}
    x_1 + 2x_2 &= 7, \\
    3x_1 - 4x_2 &= 1.
    \end{align*} \);  
(ii)   \(\begin{align*}
    x_1 + x_3 &= 3, \\
    -2x_1 + x_2 + 3x_3 &= 3, \\
    -x_1 + 2x_2 + 4x_3 &= 5.
\end{align*}\)
\subsubsection*{Solution}

\sphinxAtStartPar
(i)   Row reduce augmented matrix to row echelon form
\begin{equation*}
\begin{split} \begin{align*}
    & \left( \begin{array}{cc|c} 1 & 2 & 7 \\ 3 & -4 & 1 \end{array} \right)
    \begin{array}{l} \\ \end{array} &
    R_2 - 3R_1 \longrightarrow &
    & \left( \begin{array}{cc|c} 1 & 2 & 7 \\ 0 & -10 & -20 \end{array} \right)
\end{align*} \end{split}
\end{equation*}
\sphinxAtStartPar
Solving for \(x_1\) and \(x_2\) using back substitution
\begin{equation*}
\begin{split} \begin{align*}
    -10x_2 &= -20 & \therefore x_2 &= \frac{-20}{-10} = 2, \\
    x_1 + 2x_2 &= 7 & \therefore x_1 &= 7 - 2x_2 = 7 - 2 \cdot 2 = 3.
\end{align*} \end{split}
\end{equation*}
\sphinxAtStartPar
This solution can be easily verified by substituting it back into the original system.

\sphinxAtStartPar
(ii)   Row reduce augmented matrix to row echelon form
\begin{equation*}
\begin{split} \begin{align*}
    & \left( \begin{array}{ccc|c}
        1 & 0 & 1 & 3 \\
        -2 & 1 & 3 & 3 \\
        -1 & 2 & 4 & 5
    \end{array} \right)
     \\ \\
    \begin{array}{l} R_2 + 2R_1 \\ R_3 + R_1 \end{array} \longrightarrow &
    \left( \begin{array}{ccc|c}
        1 & 0 & 1 & 3 \\
        0 & 1 & 5 & 9 \\
        0 & 2 & 5 & 8
    \end{array} \right)
    \\ \\
     R_3 - 2R_2  \longrightarrow &
    \left( \begin{array}{ccc|c}
        1 & 0 & 1 & 3 \\
        0 & 1 & 5 & 9 \\
        0 & 0 & -5 & -10
    \end{array} \right)
\end{align*} \end{split}
\end{equation*}
\sphinxAtStartPar
Solving for \(x_1\), \(x_2\) and \(x_3\) using back substitution
\begin{equation*}
\begin{split} \begin{align*}
    -5x_3 &= -10 & \therefore x_3 &= \dfrac{-10}{-5} = 2, \\
    x_2 + 5x_3 &= 9 & \therefore x_2 &= 9 - 5x_3 = 9 - 5 \cdot 2 = -1, \\
    x_1 + x_3 &= 3 & \therefore x_1 &= 3 - x_3 = 3 - 2 = 1.
\end{align*} \end{split}
\end{equation*}\end{sphinxadmonition}

\sphinxstepscope

\index{Partial pivoting@\spxentry{Partial pivoting}}\ignorespaces 

\section{Partial pivoting}
\label{\detokenize{_pages/2.4_Partial_pivoting:partial-pivoting}}\label{\detokenize{_pages/2.4_Partial_pivoting:index-0}}\label{\detokenize{_pages/2.4_Partial_pivoting:partial-pivoting-section}}\label{\detokenize{_pages/2.4_Partial_pivoting::doc}}
\sphinxAtStartPar
In most practical applications of row reduction to solve a linear system, we use computers to perform the calculations. Computers use floating point numbers to compute arithmetic operations \sphinxhyphen{} which are not exact, and can be prone to rounding errors. Consider the following linear system of equations:
\begin{equation*}
\begin{split} \begin{pmatrix}0.001 & 1 \\ 1 & 1 \end{pmatrix}
\begin{pmatrix} x_1 \\ x_2 \end{pmatrix} =
\begin{pmatrix}1 \\ 2 \end{pmatrix}. \end{split}
\end{equation*}
\sphinxAtStartPar
Using Gaussian elimination to solve this system
\begin{equation*}
\begin{split} \begin{align*}
    & \left( \begin{array}{cc|c}
        0.001 & 1 & 1 \\
        1 & 1 & 2
    \end{array} \right)
    \\ \\
     R_2 - 1000R_1 \longrightarrow &
    \left( \begin{array}{cc|c}
        0.001 & 1 & 1 \\
        0 & -999 & -998
    \end{array} \right)
    \\ \\
    -\dfrac{1}{999}R_2 \longrightarrow &
    \left( \begin{array}{cc|c}
        0.01 & 1 & 1 \\
        0 & 1 & 0.9999\ldots
    \end{array} \right).
\end{align*} \end{split}
\end{equation*}
\sphinxAtStartPar
So \(x_2 = 0.9999\ldots\) which we can round to \(x_2 = 1\) and using back substitution we have \(x_1 = \dfrac{1 - 1 \cdot 1}{0.001} = 0\). However, let’s check our solution to see if we are correct
\begin{equation*}
\begin{split} \begin{align*}
    0.01 \cdot 0 + 1 \cdot 1 &= 1, \\
    1 \cdot 0 + 1 \cdot 1 &= 1 \neq 2.
\end{align*} \end{split}
\end{equation*}
\sphinxAtStartPar
So the solution of \(x_1 = 0\) and \(x_2 = 1\) does not satisfy the original system, and is clearly wrong. Lets solve the system again, but perform a row swap on the two rows before eliminating the value beneath the pivot.
\begin{equation*}
\begin{split} \begin{align*}
    & \left(\begin{array}{cc|c}
        0.001 & 1 & 1 \\
        1 & 1 & 2
    \end{array} \right)
    \\ \\
    R_1 \leftrightarrow R_2  \ \longrightarrow &
    \left( \begin{array}{cc|c}
        1 & 1 & 2 \\
        0.001 & 1 & 1
    \end{array} \right)
	 \\ \\
     R_2 - 0.001 R_1  \longrightarrow &
    \left( \begin{array}{cc|c}
        1 & 1 & 2 \\
        0 & 0.999 & 0.998
    \end{array} \right)
\end{align*} \end{split}
\end{equation*}
\sphinxAtStartPar
Solving using back substitution we have \(x_2 = \dfrac{0.998}{0.999} = 0.998998\ldots \approx 1\) as before, and using back substitution we now have \(x_1 = 2 - 1 \cdot 1 = 1\). Let’s check this solution to see if it is correct:
\begin{equation*}
\begin{split} \begin{align*}
    0.001 \cdot 1 + 1 \cdot 1 &= 1.001 \approx 1,
\end{align*} \end{split}
\end{equation*}
\sphinxAtStartPar
which is consistent with the original system when rounding is applied. This process is called \sphinxstylestrong{partial pivoting}, and involves performing a row swap to ensure that the pivot element has a larger absolute value than the elements in the column below the pivot.

\index{Partial pivoting@\spxentry{Partial pivoting}!algorithm@\spxentry{algorithm}}\ignorespaces \label{_pages/2.4_Partial_pivoting:ge-pp-algorithm}
\begin{sphinxadmonition}{note}{Algorithm 2.7.1 (Gaussian elimination with partial pivoting)}



\sphinxAtStartPar
\sphinxstylestrong{Inputs:} An \(m \times n\) coefficient matrix \(A\) and an \(m\)\sphinxhyphen{}element constant vector \(\mathbf{b}\).

\sphinxAtStartPar
\sphinxstylestrong{Outputs:} An augmented matrix in row echelon form.
\begin{itemize}
\item {} 
\sphinxAtStartPar
Form the augmented matrix \(( A \mid \mathbf{b} )\)

\item {} 
\sphinxAtStartPar
Initialise pivot row \(k\) to 1

\item {} 
\sphinxAtStartPar
For each column \(j\) from 1 to \(n\):
\begin{itemize}
\item {} 
\sphinxAtStartPar
Identify the row \(i\), where \(i \geq k\), with the largest absolute value in column \(j\)

\item {} 
\sphinxAtStartPar
If the largest absolute value is zero, skip to next column

\item {} 
\sphinxAtStartPar
Swap row \(i\) with row \(k\)

\end{itemize}

\item {} 
\sphinxAtStartPar
For each row \(i\) from \(k+1\) to \(m\):
\begin{itemize}
\item {} 
\sphinxAtStartPar
Subtract \(\dfrac{a_{ij}}{a_{jj}}\) times row \(j\) from row \(i\)

\end{itemize}

\item {} 
\sphinxAtStartPar
Increment pivot row \(k\) to \(k+1\) and repeat until you have done every row

\end{itemize}
\end{sphinxadmonition}


\label{_pages/2.4_Partial_pivoting:partial-pivoting-example}
\begin{sphinxadmonition}{note}{Example 2.7.1}



\sphinxAtStartPar
Solve the system of linear equations using Gaussian elimination with partial pivoting.
\begin{equation*}
\begin{split} \begin{align*}
   x_1 - x_2 + 3x_3 &= 13, \\
   4x_1 - 2x_2 + x_3 &= 15, \\
   -3x_1 - x_2 + 4x_3 &= 8.
\end{align*} \end{split}
\end{equation*}\subsubsection*{Solution}

\sphinxAtStartPar
Row reduce augmented matrix to row echelon form using partial pivoting
\begin{equation*}
\begin{split} \begin{align*}
    & \left(\begin{array}{ccc|c}
        1 & -1 & 3 & 13 \\
        4 & -2 & 1 & 15\\
        -3 & -1 & 4 & 8
    \end{array} \right)
	\\ \\    
	\begin{array}{l} R_1 \leftrightarrow R_2 \end{array} 
    \longrightarrow &
    \left( \begin{array}{ccc|c}
        4 & -2 & 1 & 15\\
        1 & -1 & 3 & 13 \\
        -3 & -1 & 4 & 8
    \end{array} \right)
    \\ \\
    \begin{array}{l} \\ R_2 - \frac{1}{4}R_1 \\[1pt] R_3 + \frac{3}{4}R_1 \end{array}  \longrightarrow &
    \left( \begin{array}{ccc|c}
        4 & -2 & 1 & 15 \\
        0 & -\frac{1}{2} & \frac{11}{4} & \frac{37}{4} \\
        0 & -\frac{5}{2} & \frac{19}{4} & \frac{77}{4}
    \end{array} \right)
   \\ \\
     R_2 \leftrightarrow R_3 \longrightarrow &
    \left( \begin{array}{ccc|c}
        4 & -2 & 1 & 15\\
        0 & -\frac{5}{2} & \frac{19}{4} & \frac{77}{4} \\
        0 & -\frac{1}{2} & \frac{11}{4} & \frac{37}{4}
    \end{array} \right)
     \\ \\
    R_3 - \frac{1}{5}R_2  \longrightarrow &
    \left( \begin{array}{ccc|c}
        4 & -2 & 1 & 15\\
        0 & -\frac{5}{2} & \frac{19}{4} & \frac{77}{4} \\
        0 & 0 & \frac{9}{5} & \frac{27}{5}
    \end{array} \right)
\end{align*} \end{split}
\end{equation*}
\sphinxAtStartPar
Solving for \(x_3\), \(x_2\) and \(x_1\) using back substitution
\begin{equation*}
\begin{split} \begin{align*}
    x_3 &= \frac{5}{9}\left( \frac{27}{5} \right) = 3, \\
    x_2 &= -\frac{2}{5} \left( \frac{77}{4} - \frac{19}{4}(3)\right) = -2, \\
    x_1 &= \frac{1}{4}( 15 + (2 \cdot -2) - 3) = 2.
\end{align*} \end{split}
\end{equation*}\end{sphinxadmonition}

\sphinxstepscope

\index{Gauss\sphinxhyphen{}Jordan elimination@\spxentry{Gauss\sphinxhyphen{}Jordan elimination}}\ignorespaces 
\index{Systems of linear equations@\spxentry{Systems of linear equations}!solution using Gauss\sphinxhyphen{}Jordan elimination@\spxentry{solution using Gauss\sphinxhyphen{}Jordan elimination}}\ignorespaces 

\section{Gauss\sphinxhyphen{}Jordan elimination}
\label{\detokenize{_pages/2.5_Gauss_Jordan_elimination:gauss-jordan-elimination}}\label{\detokenize{_pages/2.5_Gauss_Jordan_elimination:index-1}}\label{\detokenize{_pages/2.5_Gauss_Jordan_elimination:index-0}}\label{\detokenize{_pages/2.5_Gauss_Jordan_elimination:gauss-jordan-elimination-section}}\label{\detokenize{_pages/2.5_Gauss_Jordan_elimination::doc}}
\begin{figure}[htbp]
\centering
\capstart

\noindent\sphinxincludegraphics[width=200\sphinxpxdimen]{{Wilhelm_Jordan}.png}
\caption{Carl Wilhelm Jordan (1842 \sphinxhyphen{} 1899)}\label{\detokenize{_pages/2.5_Gauss_Jordan_elimination:id1}}\end{figure}

\sphinxAtStartPar
\sphinxstylestrong{Gauss\sphinxhyphen{}Jordan elimination (GJE)}, named after Gauss and German geodesist Wilhelm Jordan, is similar to {\hyperref[\detokenize{_pages/2.3_Gaussian_elimination:gaussian-elimination-section}]{\sphinxcrossref{\DUrole{std,std-ref}{Gaussian elimination}}}} with the difference that the augmented matrix is reduced using elementary row operations so that the values of the pivot elements are 1 and are the only non\sphinxhyphen{}zero element in the column. This allows the solution to be read from the final augmented matrix without the need to perform back substitution. A matrix in this form is said to be in reduced row echelon form.

\index{Reduced row echelon form@\spxentry{Reduced row echelon form}}\ignorespaces \label{_pages/2.5_Gauss_Jordan_elimination:rref-definition}
\begin{sphinxadmonition}{note}{Definition 2.8.1 (Reduced Row Echelon Form (RREF))}



\sphinxAtStartPar
A matrix is said to be in \sphinxstylestrong{Reduced Row Echelon Form (RREF)} if it satisfies the following conditions:
\begin{itemize}
\item {} 
\sphinxAtStartPar
It is in row echelon form

\item {} 
\sphinxAtStartPar
The leading entry in each non\sphinxhyphen{}zero row has a value of 1

\item {} 
\sphinxAtStartPar
The leading entry in each non\sphinxhyphen{}zero row is the only non\sphinxhyphen{}zero element in its column

\end{itemize}
\end{sphinxadmonition}

\sphinxAtStartPar
For example the following matrices are in reduced row echelon form:
\begin{equation*}
\begin{split} \begin{align*}
    &\begin{pmatrix}
        {\color{red}{1}} & 0 \\
        0 & {\color{red}{1}}
    \end{pmatrix}, &
    &\begin{pmatrix}
        {\color{red}{1}} & 2 & 0 \\
        0 & 0 & {\color{red}{1}}
    \end{pmatrix}, &
    &\begin{pmatrix}
        {\color{red}{1}} & 0 & 2 & 0 \\
        0 & {\color{red}{1}} & 3 & 0 \\
        0 & 0 & 0 & {\color{red}{1}}
    \end{pmatrix}
\end{align*} \end{split}
\end{equation*}
\sphinxAtStartPar
The method of Gauss\sphinxhyphen{}Jordan elimination is summarised in {\hyperref[\detokenize{_pages/2.5_Gauss_Jordan_elimination:gje-algorithm}]{\sphinxcrossref{Algorithm 2.8.1}}}.

\index{Gauss\sphinxhyphen{}Jordan elimination@\spxentry{Gauss\sphinxhyphen{}Jordan elimination}!algorithm@\spxentry{algorithm}}\ignorespaces \label{_pages/2.5_Gauss_Jordan_elimination:gje-algorithm}
\begin{sphinxadmonition}{note}{Algorithm 2.8.1 (Gauss\sphinxhyphen{}Jordan elimination)}



\sphinxAtStartPar
\sphinxstylestrong{Inputs:} An \(m \times n\) coefficient matrix \(A\) and an \(m\)\sphinxhyphen{}element constant vector \(\mathbf{b}\).

\sphinxAtStartPar
\sphinxstylestrong{Outputs:} An augmented matrix in row echelon form.
\begin{itemize}
\item {} 
\sphinxAtStartPar
Form the augmented matrix \(( A \mid \mathbf{b} )\)

\item {} 
\sphinxAtStartPar
Initialise pivot row \(k\) to 1

\item {} 
\sphinxAtStartPar
For each column \(j\) from 1 to \(n\):
\begin{itemize}
\item {} 
\sphinxAtStartPar
Perform a row swap of the pivot row if necessary (e.g. for partial pivoting, or if pivot element is zero)

\item {} 
\sphinxAtStartPar
If pivot element is zero, and all elements beneath the pivot element are zero, skip to the next column by incrementing \(j\)

\end{itemize}

\item {} 
\sphinxAtStartPar
Divide row \(k\) by pivot \(a_{kj}\)

\item {} 
\sphinxAtStartPar
For each row \(i\) from \(1\) to \(m\) where \(i \neq k\):
\begin{itemize}
\item {} 
\sphinxAtStartPar
Subtract \(\dfrac{a_{ij}}{a_{jj}}\) times row \(j\) from row \(i\)

\end{itemize}

\item {} 
\sphinxAtStartPar
Increment \(k\) to \(k+1\) and repeat until you have done all the rows

\end{itemize}
\end{sphinxadmonition}


\label{_pages/2.5_Gauss_Jordan_elimination:gje-example}
\begin{sphinxadmonition}{note}{Example 2.8.1}



\sphinxAtStartPar
Use Gauss\sphinxhyphen{}Jordan elimination to solve the following system of linear equations
\begin{equation*}
\begin{split} \begin{align*}
    3x_1 + x_2 - 2 x_3 &= 1, \\
    x_1 - x_2 + 2x_3 &= 3, \\
    2x_1 - 3x_2 + 7x_3 &= 4.
\end{align*} \end{split}
\end{equation*}\subsubsection*{Solution}

\sphinxAtStartPar
Row reduce the augmented matrix to reduced row echelon form
\begin{equation*}
\begin{split} \begin{align*}
    & \left( \begin{array}{ccc|c}
        3 & 1 & -2 & 1 \\
        1 & -1 & 2 & 3 \\
        2 & -3 & 7 & 4
    \end{array} \right)
    \\ \\
     R_1 \leftrightarrow R_2 \longrightarrow
    & \left( \begin{array}{ccc|c}
        1 & -1 & 2 & 3 \\
        3 & 1 & -2 & 1 \\
        2 & -3 & 7 & 4
    \end{array} \right)
    \\ \\
    \begin{array}{l} \\ R_2 - 3R_1 \\ R_3 - 2R_1 \end{array}  \longrightarrow
    & \left( \begin{array}{ccc|c}
        1 & -1 & 2 & 3 \\
        0 & 4 & -8 & -8 \\
        0 & -1 & 3 & -2
    \end{array} \right)
    \\ \\
    \frac{1}{4} R_2 \longrightarrow
    & \left( \begin{array}{ccc|c}
        1 & -1 & 2 & 3 \\
        0 & 1 & -2 & -2 \\
        0 & -1 & 3 & -2
    \end{array} \right)
 	\\ \\
    \begin{array}{l} R_1 + R_2 \\ \\ R_3 + R_2 \end{array} \longrightarrow
    & \left( \begin{array}{ccc|c}
        1 & 0 & 0 & 1 \\
        0 & 1 & -2 & -2 \\
        0 & 0 & 1 & -4
    \end{array} \right)
    \\ \\
     R_2 + 2R_3 \longrightarrow
    & \left( \begin{array}{ccc|c}
        1 & 0 & 0 & 1 \\
        0 & 1 & 0 & -10 \\
        0 & 0 & 1 & -4
    \end{array} \right)
\end{align*} \end{split}
\end{equation*}
\sphinxAtStartPar
Therefore the solution is \(x_1 = 1\), \(x_2 = -10\) and \(x_3 = -4\).
\end{sphinxadmonition}

\sphinxAtStartPar
For any given matrix, its reduced row echelon form is unique. If the matrix is square (\(n \times n\)) and non\sphinxhyphen{}singular (invertible), the reduced row echelon form of the matrix will be \(I_n\), the \(n \times n\) identity matrix.


\bigskip\hrule\bigskip


\index{Elementary matrices@\spxentry{Elementary matrices}}\ignorespaces 

\subsection{Elementary matrices}
\label{\detokenize{_pages/2.5_Gauss_Jordan_elimination:elementary-matrices}}\label{\detokenize{_pages/2.5_Gauss_Jordan_elimination:index-4}}
\sphinxAtStartPar
Gauss\sphinxhyphen{}Jordan elimination allows us to calculate the inverse of matrices, and is much more computationally efficient than the {\hyperref[\detokenize{_pages/1.5_Inverse_matrix:adjoint-definition}]{\sphinxcrossref{adjoint\sphinxhyphen{}determinant formula}}}. To show how we can use row operations to calculate the inverse of a matrix, we first need to consider elementary matrices.
\label{_pages/2.5_Gauss_Jordan_elimination:elementary-matrices-definition}
\begin{sphinxadmonition}{note}{Definition 2.8.2 (Elementary matrices)}



\sphinxAtStartPar
An \sphinxstylestrong{elementary matrix} is an \(n \times n\) square matrix which can be obtained by performing a single elementary row operation on the identity matrix \(I_n\)
\end{sphinxadmonition}

\sphinxAtStartPar
Since we have {\hyperref[\detokenize{_pages/2.3_Gaussian_elimination:ero-definition}]{\sphinxcrossref{three types}}} of elementary row operations, there are three corresponding types of elementary matrices. Consider examples of each of the three types for \(I_3\), the \(3 \times 3\) identity matrix:
\begin{itemize}
\item {} 
\sphinxAtStartPar
Swap row 1 and row 2:

\end{itemize}
\begin{equation*}
\begin{split} \begin{align*}
    \begin{pmatrix} 1 & 0 & 0 \\ 0 & 1 & 0 \\ 0 & 0 & 1 \end{pmatrix}
    \qquad R_1 \leftrightarrow R_2 \longrightarrow
    \begin{pmatrix} 0 & 1 & 0 \\ 1 & 0 & 0 \\ 0 & 0 & 1 \end{pmatrix}
\end{align*} \end{split}
\end{equation*}\begin{itemize}
\item {} 
\sphinxAtStartPar
Multiply row 2 by \(k\):

\end{itemize}
\begin{equation*}
\begin{split} \begin{align*}
    \begin{pmatrix} 1 & 0 & 0 \\ 0 & 1 & 0 \\ 0 & 0 & 1 \end{pmatrix}
    \qquad kR_2 \longrightarrow
    \begin{pmatrix} 1 & 0 & 0 \\ 0 & k & 0 \\ 0 & 0 & 1 \end{pmatrix}
\end{align*} \end{split}
\end{equation*}\begin{itemize}
\item {} 
\sphinxAtStartPar
Add \(k\) multiples of row 1 to row 3:

\end{itemize}
\begin{equation*}
\begin{split} \begin{align*}
    \begin{pmatrix} 1 & 0 & 0 \\ 0 & 1 & 0 \\ 0 & 0 & 1 \end{pmatrix}
    \qquad R_3 + kR_1 \longrightarrow
    \begin{pmatrix} 1 & 0 & 0 \\ 0 & 1 & 0 \\ k & 0 & 1 \end{pmatrix}
\end{align*} \end{split}
\end{equation*}
\sphinxAtStartPar
Elementary matrices have an inverse, which is obtained by \sphinxstylestrong{inverting the elementary row operation} and applying it to the identity matrix. The operations used above have the following inverse operations :
\begin{itemize}
\item {} 
\sphinxAtStartPar
Swapping row 1 and row 2 can be inverted by simply performing the same swap again:

\end{itemize}
\begin{equation*}
\begin{split} \begin{align*}
    \begin{pmatrix} 1 & 0 & 0 \\ 0 & 1 & 0 \\ 0 & 0 & 1 \end{pmatrix}
    \qquad R_1 \leftrightarrow R_2 \longrightarrow
    \begin{pmatrix} 0 & 1 & 0 \\ 1 & 0 & 0 \\ 0 & 0 & 1 \end{pmatrix}
\end{align*} \end{split}
\end{equation*}\begin{itemize}
\item {} 
\sphinxAtStartPar
Multiplying row 2 by \(k\) can be inverted by dividing row 2 by \(k\):

\end{itemize}
\begin{equation*}
\begin{split} \begin{align*}
    \begin{pmatrix} 1 & 0 & 0 \\ 0 & 1 & 0 \\ 0 & 0 & 1 \end{pmatrix}
    \qquad \frac{1}{k} R_2  \longrightarrow
    \begin{pmatrix} 1 & 0 & 0 \\ 0 & \frac{1}{k} & 0 \\ 0 & 0 & 1 \end{pmatrix}
\end{align*} \end{split}
\end{equation*}\begin{itemize}
\item {} 
\sphinxAtStartPar
Adding \(k\) multiples of row 1 to row 3 can be inverted by subtracting \(k\) multiples of row 1 from row 3:

\end{itemize}
\begin{equation*}
\begin{split} \begin{align*}
    \begin{pmatrix} 1 & 0 & 0 \\ 0 & 1 & 0 \\ 0 & 0 & 1 \end{pmatrix}
    \qquad R_3 - kR_1 \longrightarrow
    \begin{pmatrix} 1 & 0 & 0 \\ 0 & 1 & 0 \\ -k & 0 & 1 \end{pmatrix}
\end{align*} \end{split}
\end{equation*}\label{_pages/2.5_Gauss_Jordan_elimination:elementary-matrix-multiplication-theorem}
\begin{sphinxadmonition}{note}{Theorem 2.8.1 (Multiplying by an elementary matrix)}



\sphinxAtStartPar
If \(A\) is an \(n \times n\) matrix and \(E\) is an elementary matrix, the product \(EA\) is equivalent to performing the elementary row operation used to obtain \(E\) on \(A\).
\end{sphinxadmonition}

\sphinxAtStartPar
For example, let \(A = \begin{pmatrix} 1 & 0 & 4 \\ 2 & -1 & 3 \\ 0 & 5 & 1 \end{pmatrix}\) and consider the following row operations:
\begin{itemize}
\item {} 
\sphinxAtStartPar
\(R_1 \leftrightarrow R_2\):   The corresponding elementrary matrix is  \(E_1 = \begin{pmatrix} 0 & 1 & 0 \\ 1 & 0 & 0 \\ 0 & 0 & 1 \end{pmatrix}\) so

\end{itemize}
\begin{equation*}
\begin{split}E_1A = \begin{pmatrix} 0 & 1 & 0 \\ 1 & 0 & 0 \\ 0 & 0 & 1 \end{pmatrix} \begin{pmatrix} 1 & 0 & 4 \\ 2 & -1 & 3 \\ 0 & 5 & 1 \end{pmatrix} =  \begin{pmatrix} 2 & -1 & 3 \\ 1 & 0 & 4 \\ 0 & 5 & 1 \end{pmatrix}\end{split}
\end{equation*}\begin{itemize}
\item {} 
\sphinxAtStartPar
\(-2R_2\):   The elementrary matrix is \(E_2 = \begin{pmatrix} 1 & 0 & 0 \\ 0 & -2 & 0 \\ 0 & 0 & 1 \end{pmatrix}\) so

\end{itemize}
\begin{equation*}
\begin{split}E_2A = \begin{pmatrix} 1 & 0 & 0 \\ 0 & -2 & 0 \\ 0 & 0 & 1 \end{pmatrix} \begin{pmatrix} 1 & 0 & 4 \\ 2 & -1 & 3 \\ 0 & 5 & 1 \end{pmatrix}=\begin{pmatrix} 1 & 0 & 4 \\ -4 & 2 & -6 \\ 0 & 5 & 1 \end{pmatrix}\end{split}
\end{equation*}\begin{itemize}
\item {} 
\sphinxAtStartPar
\(R_3 + 3R_2\):   The elementrary matrix is  \(E_3 = \begin{pmatrix} 1 & 0 & 0 \\ 0 & 1 & 0 \\ 0 & 3 & 1 \end{pmatrix}\) so

\end{itemize}
\begin{equation*}
\begin{split}E_3A = \begin{pmatrix} 1 & 0 & 0 \\ 0 & 1 & 0 \\ 0 & 3 & 1 \end{pmatrix} \begin{pmatrix} 1 & 0 & 4 \\ 2 & -1 & 3 \\ 0 & 5 & 1 \end{pmatrix} = \begin{pmatrix} 1 & 0 & 4 \\ 2 & -1 & 3 \\ 6 & 2 & 10 \end{pmatrix}\end{split}
\end{equation*}

\bigskip\hrule\bigskip



\subsection{Calculating the inverse of a matrix using Gauss\sphinxhyphen{}Jordan elimination}
\label{\detokenize{_pages/2.5_Gauss_Jordan_elimination:calculating-the-inverse-of-a-matrix-using-gauss-jordan-elimination}}
\index{Matrix@\spxentry{Matrix}!inverse using Gauss\sphinxhyphen{}Jordan elimination@\spxentry{inverse using Gauss\sphinxhyphen{}Jordan elimination}}\ignorespaces \label{_pages/2.5_Gauss_Jordan_elimination:theorem-5}
\begin{sphinxadmonition}{note}{Theorem 2.8.2}



\sphinxAtStartPar
If \(A\) is a non\sphinxhyphen{}singular (invertible) square matrix, then the inverse of \(A\) can be calculated as:
\begin{equation*}
\begin{split}A^{-1} = E_kE_{k-1}\cdots E_2E_1\end{split}
\end{equation*}
\sphinxAtStartPar
Where \(E_1, E_2, \ldots, E_k\) are the elementary matrices corresponding to the \(k\) elementary row operations needed to reduce \(A\) to reduced row echelon form.
\end{sphinxadmonition}

\begin{sphinxadmonition}{note}
\sphinxAtStartPar
Proof. Since \(A\) is square and non\sphinxhyphen{}singular, its reduced row echelon form will be the identity matrix. If we can apply \(k\) elementary row operations with corresponding elementary matrices \(E_1, E_2, \ldots, E_k\) to row reduce \(A\) to reduced row echelon form, then we have
\begin{equation*}
\begin{split}E_k E_{k-1} \cdots E_2 E_1 A = I\end{split}
\end{equation*}
\sphinxAtStartPar
That is, if we multiply \(A\) on the left by all of the elementary matrices in turn, we will get the identity matrix.  Multiplying \(A^{-1}\) to the right of both sides gives
\begin{equation*}
\begin{split} \begin{align*}
 E_k E_{k-1} \cdots E_2 E_1 A A^{-1} &= I A^{-1} \\
 E_k E_{k-1} \cdots E_2 E_1 &= A^{-1}
\end{align*} \end{split}
\end{equation*}
\sphinxAtStartPar
as required.
\end{sphinxadmonition}

\sphinxAtStartPar
So we can calculate the inverse of \(A\) by applying the same row operations to the identity matrix as we do when we reduce \(A\) to reduced row echelon form. In practice, we can do this by forming an augmented matrix \((A \mid I)\) and performing Gauss\sphinxhyphen{}Jordan elimination on the augmented matrix. Once this has been converted to reduced row echelon form, the matrix to the right of the partition will then be the inverse of \(A\).

\sphinxAtStartPar
The calculation of a matrix inverse using Gauss\sphinxhyphen{}Jordan elimination is more efficient than using the adjoint\sphinxhyphen{}determinant formula when dealing with larger matrices (i.e., when \(n > 3\)) since the steps can be easily programmed into a computer, and it does not require the calculation of determinants \sphinxhyphen{} which can be computationally expensive.


\label{_pages/2.5_Gauss_Jordan_elimination:gauss-jordan-inverse-example}
\begin{sphinxadmonition}{note}{Example 2.8.2}



\sphinxAtStartPar
Use Gauss\sphinxhyphen{}Jordan elimination to calculate the inverse of
\begin{equation*}
\begin{split} A =
\begin{pmatrix}
  1 & 0 & 2 \\
  2 & -1 & 3 \\
  1 & 4 & 4
\end{pmatrix}.\end{split}
\end{equation*}\subsubsection*{Solution}

\sphinxAtStartPar
Row reduce the augmented matrix \((A \mid I)\) to reduced row echelon form
\begin{equation*}
\begin{split} \begin{align*}
    &\left( \begin{array}{ccc|ccc}
      1 & 0 & 2 & 1 & 0 & 0 \\
      2 & -1 & 3 & 0 & 1 & 0 \\
      1 & 4 & 4 & 0 & 0 & 1
    \end{array} \right)
     \\ \\
    \begin{array}{l} R_2 - 2R_1 \\ R_3 - R_1 \end{array} \longrightarrow
    &\left( \begin{array}{ccc|ccc}
      1 & 0 & 2 & 1 & 0 & 0 \\
      0 & -1 & -1 & -2 & 1 & 0 \\
      0 & 4 & 2 & -1 & 0 & 1
    \end{array} \right)
    \\ \\
    -R_2\longrightarrow
    &\left( \begin{array}{ccc|ccc}
      1 & 0 & 2 & 1 & 0 & 0 \\
      0 & 1 & 1 & 2 & -1 & 0 \\
      0 & 4 & 2 & -1 & 0 & 1
    \end{array} \right)
    \\ \\
    R_3 - 4R_2 \longrightarrow
    &\left( \begin{array}{ccc|ccc}
      1 & 0 & 2 & 1 & 0 & 0 \\
      0 & 1 & 1 & 2 & -1 & 0 \\
      0 & 0 & -2 & -9 & 4 & 1
    \end{array} \right)
    \\ \\
    -\frac{1}{2}R_3 \longrightarrow
    &\left( \begin{array}{ccc|ccc}
      1 & 0 & 2 & 1 & 0 & 0 \\
      0 & 1 & 1 & 2 & -1 & 0 \\
      0 & 0 & 1 & \frac{9}{2} & -2 & -\frac{1}{2}
    \end{array} \right)
    \\ \\
    \begin{array}{l} R_1 - 2R_3 \\ R_2 - R_3 \end{array}  \longrightarrow
    &\left( \begin{array}{ccc|ccc}
      1 & 0 & 0 & -8 & 4 & 1 \\
      0 & 1 & 1 & -\frac{5}{2} & 1 & \frac{1}{2} \\
      0 & 0 & 1 & \frac{9}{2} & -2 & -\frac{1}{2}
    \end{array} \right)  
\end{align*} \end{split}
\end{equation*}
\sphinxAtStartPar
The augmented matrix is now in reduced row echelon form and the inverse of \(A\) is
\begin{equation*}
\begin{split} A^{-1} =
\begin{pmatrix} -8 & 4 & 1 \\ -\frac{5}{2} & 1 & \frac{1}{2} \\ \frac{9}{2} & -2 & -\frac{1}{2} \end{pmatrix}. \end{split}
\end{equation*}
\sphinxAtStartPar
Checking whether this is correct
\begin{equation*}
\begin{split} \begin{align*}
  A^{-1} A &=
  \begin{pmatrix}
    -8 & 4 & 1 \\
    -\frac{5}{2} & 1 & \frac{1}{2} \\
    \frac{9}{2} & -2 & -\frac{1}{2} \end{pmatrix}
  \end{align*}
  \begin{pmatrix}
    1 & 0 & 2 \\
    2 & -1 & 3 \\
    1 & 4 & 4
  \end{pmatrix} =
  \begin{pmatrix}
    1 & 0 & 0 \\
    0 & 1 & 0 \\
    0 & 0 & 1
  \end{pmatrix} \checkmark\end{split}
\end{equation*}\end{sphinxadmonition}

\sphinxstepscope


\section{Consistent, inconsistent and indeterminate systems}
\label{\detokenize{_pages/2.6_Consistent_systems:consistent-inconsistent-and-indeterminate-systems}}\label{\detokenize{_pages/2.6_Consistent_systems:consistent-inconsistent-and-indeterminate-systems-section}}\label{\detokenize{_pages/2.6_Consistent_systems::doc}}
\sphinxAtStartPar
So far all the examples of systems of linear equations we have seen have had a single unique solution. This will not always be the case, and it is important that we consider the different situations that can arise when dealing with systems of equations.

\sphinxAtStartPar
For example, there might not be enough information given by the equations to be able to fully determine the solution. Or it may be that there are no possible solutions, as the equations contradict each other. While it is possible to just look at the equations to determine if a system is solvable or not, an easier way is to examine the corresponding matrix of coefficients and augmented matrix.

\sphinxAtStartPar
One measure that can be helpful for this is the \sphinxstylestrong{rank} of a matrix.

\index{Matrix@\spxentry{Matrix}!rank@\spxentry{rank}}\ignorespaces 
\index{Rank@\spxentry{Rank}}\ignorespaces \label{_pages/2.6_Consistent_systems:rank-definition}
\begin{sphinxadmonition}{note}{Definition 2.9.1 (Rank of a matrix)}



\sphinxAtStartPar
The \sphinxstylestrong{rank} of a matrix \(A\) denoted by \(\operatorname{rank}(A)\) is the number of non\sphinxhyphen{}zero rows of \(A\) when it has been converted to row echelon form.
\end{sphinxadmonition}

\sphinxAtStartPar
To compute the rank of a matrix, we can use elementary row operations to row reduce to row echelon form, and then count the number of non\sphinxhyphen{}zero rows. Comparing the rank to the overall number of rows in the matrix will help us determine if the system has solutions, as outlined below.

\sphinxAtStartPar
Since linear equations represent straight lines, we can visualise the solutions \sphinxhyphen{} and whether they will exist or not \sphinxhyphen{} by considering the intersections of the lines. We can categorise linear systems of equations in three ways:
\label{_pages/2.6_Consistent_systems:consistent-inconsistent-and-indeterminate-systems-definition}
\begin{sphinxadmonition}{note}{Definition 2.9.2 (Consistent, inconsistent and indeterminate systems)}



\sphinxAtStartPar
A system of equations is said to be \sphinxstylestrong{consistent} if it has a solution, and otherwise it is said to be \sphinxstylestrong{inconsistent}. If a system of equations has more than one solution, it is said to be \sphinxstylestrong{indeterminate}.
\end{sphinxadmonition}

\index{Linear systems@\spxentry{Linear systems}!consistent system@\spxentry{consistent system}}\ignorespaces 

\bigskip\hrule\bigskip



\subsection{Consistent systems}
\label{\detokenize{_pages/2.6_Consistent_systems:consistent-systems}}
\index{Systems of linear equations@\spxentry{Systems of linear equations}!consistent systems@\spxentry{consistent systems}}\ignorespaces 
\sphinxAtStartPar
Consider the plot of a system with two variables, \(x_1\) and \(x_2\). If the lines that represent each equation in a system intersect at a single point, then the coordinates of the point of intersection provide a unique solution to the system. Consider the following system of linear equations:
\begin{equation*}
\begin{split} \begin{align*}
    x_1 - x_2 &= 1, \\
    x_1 + x_2 &= 3.
\end{align*} \end{split}
\end{equation*}
\sphinxAtStartPar
The lines for these two equations have been plotted in \hyperref[\detokenize{_pages/2.6_Consistent_systems:consistent-system-plot}]{Fig.\@ \ref{\detokenize{_pages/2.6_Consistent_systems:consistent-system-plot}}} below.

\begin{figure}[htbp]
\centering
\capstart

\noindent\sphinxincludegraphics{{161e3b8c2482006a811277b33dfad69d85d0691b3ccd3f3375cdf0ce4ae51f3d}.png}
\caption{Plots of the lines \(x_1 + x_2 = 3\) and \(x_1 - x_2 = 1\).}\label{\detokenize{_pages/2.6_Consistent_systems:consistent-system-plot}}\end{figure}

\sphinxAtStartPar
The two lines intersect at \((2,1)\) so this system is \sphinxstylestrong{consistent} and has a unique solution. This can easily be confirmed using Gaussian elimination:
\begin{equation*}
\begin{split} \begin{align*}
    & \left( \begin{array}{cc|c}
        1 & -1 & 1 \\
        1 & 1 & 3
    \end{array} \right)
    \begin{array}{l} \\ R_2 - R_1 \end{array}
    \longrightarrow  &
    \left( \begin{array}{cc|c}
        1 & -1 & 1 \\
        0 & 2 & 2
    \end{array} \right)
\end{align*} \end{split}
\end{equation*}
\sphinxAtStartPar
Solving by back substitution gives \(x_1 = 2\) and \(x_2 = 1.\)
\label{_pages/2.6_Consistent_systems:consistent-system-theorem}
\begin{sphinxadmonition}{note}{Theorem 2.9.1 (Consistent systems)}



\sphinxAtStartPar
A system of linear equations \(A\mathbf{x} = \mathbf{b}\) is consistent if \(\operatorname{rank}(A) = \operatorname{rank}(A \mid \mathbf{b})\).
\end{sphinxadmonition}


\bigskip\hrule\bigskip


\index{Systems of linear equations@\spxentry{Systems of linear equations}!inconsistent systems@\spxentry{inconsistent systems}}\ignorespaces 

\subsection{Inconsistent systems}
\label{\detokenize{_pages/2.6_Consistent_systems:inconsistent-systems}}\label{\detokenize{_pages/2.6_Consistent_systems:index-4}}
\sphinxAtStartPar
If the lines that represent each equation in a system with two variables never intersect, then they must be parallel \sphinxhyphen{} and we have an \sphinxstylestrong{inconsistent system}. Consider the following system of linear equations
\begin{equation*}
\begin{split} \begin{align*}
    x_1 - x_2 &= 1, \\
    x_1 - x_2 & = -1.
\end{align*} \end{split}
\end{equation*}
\sphinxAtStartPar
The lines for these two equations have been plotted in \hyperref[\detokenize{_pages/2.6_Consistent_systems:inconsistent-system-plot}]{Fig.\@ \ref{\detokenize{_pages/2.6_Consistent_systems:inconsistent-system-plot}}} below.

\begin{figure}[htbp]
\centering
\capstart

\noindent\sphinxincludegraphics{{fbeddc91ddd83b2a8d2028c84b24151e32a43edf2fbfb9ef47180ed2ee99b821}.png}
\caption{Plots of the lines \(x_1 - x_2 = 1\) and \(x_1 - x_2 = -1\).}\label{\detokenize{_pages/2.6_Consistent_systems:inconsistent-system-plot}}\end{figure}

\sphinxAtStartPar
The two lines do not intersect, so this system does not have a solution. If we attempt to solve this using Gaussian elimination, we find:
\begin{equation*}
\begin{split} \begin{align*}
    & \left( \begin{array}{cc|c}
        1 & -1 & 1 \\
        1 & -1 & -1
    \end{array} \right)
    \begin{array}{l} \\ R_2 - R_1 \end{array} &
    \longrightarrow &
    \left( \begin{array}{cc|c}
        1 & -1 & 1 \\
        0 & 0 & -2
    \end{array} \right),
\end{align*} \end{split}
\end{equation*}
\sphinxAtStartPar
Here, the second equation is \(0x_1 + 0x_2 = -2\), which is clearly impossible.
\label{_pages/2.6_Consistent_systems:inconsistent-system-theorem}
\begin{sphinxadmonition}{note}{Theorem 2.9.2 (Inconsistent systems)}



\sphinxAtStartPar
A system of linear equations is inconsistent if \(A\mathbf{x} = \mathbf{b}\) is inconsistent if \(\operatorname{rank}(A) < \operatorname{rank}(A \mid \mathbf{b})\).
\end{sphinxadmonition}


\bigskip\hrule\bigskip


\index{Systems of linear equations@\spxentry{Systems of linear equations}!indeterminate systems@\spxentry{indeterminate systems}}\ignorespaces 

\subsection{Indeterminate systems}
\label{\detokenize{_pages/2.6_Consistent_systems:indeterminate-systems}}\label{\detokenize{_pages/2.6_Consistent_systems:index-5}}
\sphinxAtStartPar
If the lines that represent each equation in a system with two variables are the same line, than we have an \sphinxstylestrong{indeterminate system}, where we have an infinite number of solutions. Consider the following system of linear equations
\begin{equation*}
\begin{split} \begin{align*}
    x_1 - x_2 &= 1, \\
    2x_1 - 2x_2 &= 2.
\end{align*} \end{split}
\end{equation*}
\sphinxAtStartPar
The lines for these two equations have been plotted in \hyperref[\detokenize{_pages/2.6_Consistent_systems:indeterminate-system-plot}]{Fig.\@ \ref{\detokenize{_pages/2.6_Consistent_systems:indeterminate-system-plot}}} below.

\begin{figure}[htbp]
\centering
\capstart

\noindent\sphinxincludegraphics{{d869a9dc27b612dadc240a288090676a332ffa7d44523d4ed81ee711b3488b81}.png}
\caption{Plots of the lines \(x_1 - x_2 = 1\) and \(2x_1 - 2x_2 = 2\).}\label{\detokenize{_pages/2.6_Consistent_systems:indeterminate-system-plot}}\end{figure}

\sphinxAtStartPar
The two lines are the same \sphinxhyphen{} they are defined by different equations, but the set of points satisfying each equation is the same, so any point along either line represents a solution to the system. If we attempt to solve this using Gaussian elimination, we get:
\begin{equation*}
\begin{split} \begin{align*}
    \left( \begin{array}{cc|c}
        1 & -1 & 1 \\
        2 & -2 & 2
    \end{array} \right)
    \begin{array}{l} \\ R_2 - 2R_1 \end{array} 
    \longrightarrow \quad
    \left( \begin{array}{cc|c}
        1 & -1 & 1 \\
        0 & 0 & 0 
    \end{array} \right),
\end{align*} \end{split}
\end{equation*}
\sphinxAtStartPar
Here, the second equation has been eliminated, so we have a single equation \(x_1 - x_2 = 1\). This is not enough to define a unique pair of points \(x_1\) and \(x_2\), so any set of points satisfying this one equation will be a valid solution to the system.
\label{_pages/2.6_Consistent_systems:indeterminate-system-theorem}
\begin{sphinxadmonition}{note}{Theorem 2.9.3 (indeterminate systems)}



\sphinxAtStartPar
A system of linear equations with \(n\) unknowns \(A\mathbf{x} = \mathbf{b}\) is indeterminate if \(\operatorname{rank}(A) < n\).
\end{sphinxadmonition}

\sphinxAtStartPar
If we have an indeterminate system, the variables that correspond to the columns in the coefficient matrix without pivots (i.e., variables we cannot solve for) are called \sphinxstylestrong{free variables}. We express the solution of indeterminate systems by assigning a parameter (\(r\), \(s\), \(t\) etc.) to the free variables, and proceed as normal. For example, in our system, \(x_2\) is a free variable \sphinxhyphen{} so let \(x_2 = r\) where \(r \in \mathbb{R}\) and
\begin{equation*}
\begin{split} x_1 - r = 1, \end{split}
\end{equation*}
\sphinxAtStartPar
so the solution is \(x_1 = 1 + r\) and \(x_2 = r\). Choosing any value of \(r\) gives a solution to the system, representing a point on the line of possible solutions.

\sphinxAtStartPar
For example, if \(r=1\), then \(x_1 = 2\) and \(x_2 = 1\). Alternatively if \(r = -1\) then \(x_1 = 0\) and \(x_2 = -1\).


\label{_pages/2.6_Consistent_systems:rank-example}
\begin{sphinxadmonition}{note}{Example 2.9.1}



\sphinxAtStartPar
Determine the rank of the coefficient matrix and the augmented matrix for the following systems of linear equations and classify them as either consistent, inconsistent or indeterminate systems.

\sphinxAtStartPar
(i)   \(\begin{array}{rl}
    3x_1 + x_2 - 2x_3  \!\!\!\! &= 1, \\
    x_1 - x_2 + 2x_3  \!\!\!\! &= 3, \\
    2x_1 - 3x_2 + 7x_3  \!\!\!\! &= 4.
\end{array}\);  
(ii)   \(\begin{array}{rl}
    x_1 - x_2 + 2x_3  \!\!\!\! &= 3, \\
    2x_1 - 3x_2 + 7x_3  \!\!\!\! &= 4, \\
    -x_1 + 3x_2 - 8x_3  \!\!\!\! &= 1.
\end{array}\);   \textbackslash{} \textbackslash{}
(iii)    \(\begin{array}{rcl}
    x_1 + x_2 - 2x_3  \!\!\!\! &= 1, \\
    2x_1 - x_2 + x_3  \!\!\!\! &= 9, \\
    x_1 + 4x_2 - 7x_3  \!\!\!\! &= 2.
\end{array}\)
\subsubsection*{Solution}

\sphinxAtStartPar
(i)  
\begin{equation*}
\begin{split} \begin{align*}
    & \left( \begin{array}{ccc|c}
        3 & 1 & -2 & 1 \\
        1 & -1 & 2 & 3 \\
        2 & -3 & 7 & 4
    \end{array} \right)
    \\ \\
    R_1 \leftrightarrow R_2 \longrightarrow &
    \left( \begin{array}{ccc|c}
        1 & -1 & 2 & 3 \\
        3 & 1 & -2 & 1 \\
        2 & -3 & 7 & 4
    \end{array} \right)
    \\ \\
    \begin{array}{l} R_2 - 3R_1 \\ R_3 - 2R_1 \end{array} \longrightarrow &
    \left( \begin{array}{ccc|c}
        1 & -1 & 2 & 3 \\
        0 & 4 & -8 & -8 \\
        0 & -1 & 3 & -2
    \end{array} \right)
    \\ \\
    R_3 + \frac{1}{4}R_2 \longrightarrow &
    \left( \begin{array}{ccc|c}
        1 & -1 & 2 & 3 \\
        0 & 4 & -8 & -8 \\
        0 & 0 & 1 & -4
    \end{array} \right)
\end{align*} \end{split}
\end{equation*}
\sphinxAtStartPar
The augmented matrix is now in row echelon form. Since \(\operatorname{rank}(A) = \operatorname{rank}(A \mid \mathbf{b}) = 3\), this is a consistent system by the {\hyperref[\detokenize{_pages/2.6_Consistent_systems:consistent-system-theorem}]{\sphinxcrossref{theorem for consistent systems}}}. Using back\sphinxhyphen{}substitution, the solution is \(x_1 = 1\), \(x_2 = -10\) and \(x_3 = -4\).

\sphinxAtStartPar
(ii)
\begin{equation*}
\begin{split} \begin{align*}
    & \left( \begin{array}{ccc|c}
        1 & -1 & 2 & 3 \\
        2 & -3 & 7 & 4 \\
        -1 & 3 & -8 & 1 
    \end{array} \right)
    \\ \\
    \begin{array}{l} R_2 - 2R_1 \\ R_3 + R_1 \end{array} \longrightarrow &
    \left( \begin{array}{ccc|c}
        1 & -1 & 2 & 3 \\
        0 & -1 & 3 & -2 \\
        0 & 2 & -6 & 4 
    \end{array} \right)
    \\ \\
     R_3 + 2R_2 \longrightarrow &
    \left( \begin{array}{ccc|c}
        1 & -1 & 2 & 3 \\
        0 & -1 & 3 & -2 \\
        0 & 0 & 0 & 0
    \end{array} \right)
\end{align*} \end{split}
\end{equation*}
\sphinxAtStartPar
The augmented matrix is now in row echelon form. Since \(\operatorname{rank}(A) = \operatorname{rank}(A \mid \mathbf{b}) = 2\), this is a consistent system by the {\hyperref[\detokenize{_pages/2.6_Consistent_systems:consistent-system-theorem}]{\sphinxcrossref{theorem for consistent systems}}} \sphinxhyphen{} that is, it has a solution. Furthermore, since \(\operatorname{rank}(A) \) is less than the number of unknowns, this is an indeterminate system by the {\hyperref[\detokenize{_pages/2.6_Consistent_systems:indeterminate-system-theorem}]{\sphinxcrossref{theorem for indeterminate systems}}}, so it has infinitely many solutions. Let us define a parameter by \(x_3 = r\), then using back\sphinxhyphen{}substitution we have \(x_1 = 5 + r\) and \(x_2 = 2 + 3r\).

\sphinxAtStartPar
(iii)
\begin{equation*}
\begin{split} \begin{align*}
    & \left( \begin{array}{ccc|c}
        1 & 1 & -2 & 1 \\
        2 & -1 & 1 & 9 \\
        1 & 4 & -7 & 2 
    \end{array} \right) 
    \\ \\
    \begin{array}{l} \\ R_2 - 2R_1 \\ R_3 - R_1 \end{array} \longrightarrow &
    \left( \begin{array}{ccc|c}
        1 & 1 & -2 & 1 \\
        0 & -3 & 5 & 7 \\
        0 & 3 & -5 & 1 
    \end{array} \right)
    \\ \\
    R_3 + R_2 \longrightarrow &
    \left( \begin{array}{ccc|c}
        1 & 1 & -2 & 1 \\
        0 & -3 & 5 & 7 \\
        0 & 0 & 0 & 8
    \end{array} \right)
\end{align*} \end{split}
\end{equation*}
\sphinxAtStartPar
The augmented matrix is now in row echelon form, and we have \(\operatorname{rank}(A) = 2\) and \(\operatorname{rank}(A \mid \mathbf{b}) = 3\). Since \(\operatorname{rank}(A) < \operatorname{rank}(A \mid \mathbf{b})\) then this is an inconsistent system by the {\hyperref[\detokenize{_pages/2.6_Consistent_systems:inconsistent-system-theorem}]{\sphinxcrossref{theorem for inconsistent systems}}}, and does not have a solution.
\end{sphinxadmonition}

\sphinxstepscope

\index{Systems of linear equations@\spxentry{Systems of linear equations}!homogeneous systems@\spxentry{homogeneous systems}}\ignorespaces 
\index{Homogeneous systems@\spxentry{Homogeneous systems}}\ignorespaces 

\section{Homogeneous systems}
\label{\detokenize{_pages/2.7_Homogeneous_systems:homogeneous-systems}}\label{\detokenize{_pages/2.7_Homogeneous_systems:index-1}}\label{\detokenize{_pages/2.7_Homogeneous_systems:index-0}}\label{\detokenize{_pages/2.7_Homogeneous_systems:homogeneous-systems-section}}\label{\detokenize{_pages/2.7_Homogeneous_systems::doc}}\label{_pages/2.7_Homogeneous_systems:homogeneous-system-definition}
\begin{sphinxadmonition}{note}{Definition 2.10.1 (Homogeneous systems)}



\sphinxAtStartPar
A \sphinxstylestrong{homogeneous} system of linear equations is of the form \(A\mathbf{x} = \mathbf{0}\), i.e.,
\begin{equation*}
\begin{split} \begin{align*}
    \begin{pmatrix}
        a_{11} & a_{12} & \cdots & a_{1n} \\
        a_{21} & a_{22} & \cdots & a_{2n} \\
        \vdots & \vdots & \ddots & \vdots \\
        a_{m1} & a_{m2} & \cdots & a_{mn}
    \end{pmatrix}
    \begin{pmatrix} x_1 \\ x_2 \\ \vdots \\ x_n \end{pmatrix} = 
    \begin{pmatrix} 0 \\ 0 \\ \vdots \\ 0 \end{pmatrix}.
\end{align*} \end{split}
\end{equation*}\end{sphinxadmonition}

\sphinxAtStartPar
A homogeneous system always has the \sphinxstylestrong{trivial} solution \(\mathbf{x} = \mathbf{0}\), so a homogeneous system is always consistent. If \(\mathbf{x} = \mathbf{0}\) is not the only solution then the system is indeterminate, and there will be infinitely many non\sphinxhyphen{}trivial solutions.

\sphinxAtStartPar
When a homogeneous system is solved by Gaussian elimination, the column of zeros on the right of the partition in the augmented matrix remains unchanged at each stage, since none of the three types of elementary row operations can introduce a non\sphinxhyphen{}zero entry in this column. So we only need to perform row operations on the coefficient matrix \(A\).
\label{_pages/2.7_Homogeneous_systems:homogeneous-systems-example}
\begin{sphinxadmonition}{note}{Example 2.10.1}



\sphinxAtStartPar
Solve the following homogeneous system of linear equations using Gauss\sphinxhyphen{}Jordan elimination
\begin{equation*}
\begin{split} \begin{align*}
    x_1 + 2x_3 - x_5 &= 0, \\
    2x_1 + 4x_3 - 2x_4 + 4x_5 &= 0, \\
    x_2 + x_3 +2x_4 &= 0.
\end{align*} \end{split}
\end{equation*}\subsubsection*{Solution}

\sphinxAtStartPar
Applying elementary row operations to the coefficient matrix
\begin{equation*}
\begin{split} \begin{align*}
    & \left( \begin{array}{ccccc}
        1 & 0 & 2 & 0 & -1 \\
        2 & 0 & 4 & -2 & 4 \\
        0 & 1 & 1 & 2 & 0
    \end{array} \right)
	\\ \\
    R_2 - 2R_1 \longrightarrow &
    \left( \begin{array}{ccccc}
        1 & 0 & 2 & 0 & -1 \\
        0 & 0 & 0 & -2 & 6 \\
        0 & 1 & 1 & 2 & 0
    \end{array} \right)
    \\ \\
    R_2 \leftrightarrow R_3 \longrightarrow &
    \left( \begin{array}{ccccc}
        1 & 0 & 2 & 0 & -1 \\
        0 & 1 & 1 & 2 & 0 \\
        0 & 0 & 0 & -2 & 6
    \end{array} \right)
    \\ \\
    -\frac{1}{2} R_3 \longrightarrow &
    \left( \begin{array}{ccccc}
        1 & 0 & 2 & 0 & -1 \\
        0 & 1 & 1 & 2 & 0 \\
        0 & 0 & 0 & 1 & -3
    \end{array} \right)
	\\ \\
    R_2 - 2R_3  \longrightarrow &
    \left( \begin{array}{ccccc}
        1 & 0 & 2 & 0 & -1 \\
        0 & 1 & 1 & 0 & 6 \\
        0 & 0 & 0 & 1 & -3
    \end{array} \right)
\end{align*} \end{split}
\end{equation*}
\sphinxAtStartPar
Here we have 5 variables and, since columns 3 and 5 do not have pivot elements, 2 free parameters \(x_3\) and \(x_5\). Let \(r = x_3 \) and \(s = x_5\) then solving for \(x_1\), \(x_2\) and \(x_4\) by back substitution we have
\begin{equation*}
\begin{split} \begin{align*}
    x_4 &= 3s, \\
    x_2 &= -6s - r, \\
    x_1 &= s - 2r.
\end{align*} \end{split}
\end{equation*}\end{sphinxadmonition}

\sphinxstepscope


\section{Systems of Linear Equations Exercises}
\label{\detokenize{_pages/2.9_Linear_systems_exercises:systems-of-linear-equations-exercises}}\label{\detokenize{_pages/2.9_Linear_systems_exercises::doc}}
\sphinxAtStartPar
Answer the following exercises based on the content from this chapter. The solutions can be found in the {\hyperref[\detokenize{_pages/A2_Linear_systems_exercises_solutions:systems-exercises-solutions}]{\sphinxcrossref{\DUrole{std,std-ref}{appendices}}}}.
\phantomsection \label{exercise:systems-ex-inverse-sol}

\begin{sphinxadmonition}{note}{Exercise 2.11.1}



\sphinxAtStartPar
Solve the following linear system of equations using the inverse of the coefficient matrix.

\begin{sphinxuseclass}{sd-container-fluid}
\begin{sphinxuseclass}{sd-sphinx-override}
\begin{sphinxuseclass}{sd-mb-4}
\begin{sphinxuseclass}{sd-row}
\begin{sphinxuseclass}{sd-col}
\begin{sphinxuseclass}{sd-d-flex-column}
\begin{sphinxuseclass}{sd-col-6}
\begin{sphinxuseclass}{sd-col-xs-6}
\begin{sphinxuseclass}{sd-col-sm-6}
\begin{sphinxuseclass}{sd-col-md-6}
\begin{sphinxuseclass}{sd-col-lg-6}
\sphinxAtStartPar
(a)  
\( \begin{align*}
     - 4 x_{1} + 2 x_{2} &= -22, \\
     3 x_{1} + 4 x_{2} &= 11.
\end{align*} \)

\end{sphinxuseclass}
\end{sphinxuseclass}
\end{sphinxuseclass}
\end{sphinxuseclass}
\end{sphinxuseclass}
\end{sphinxuseclass}
\end{sphinxuseclass}
\begin{sphinxuseclass}{sd-col}
\begin{sphinxuseclass}{sd-d-flex-column}
\begin{sphinxuseclass}{sd-col-6}
\begin{sphinxuseclass}{sd-col-xs-6}
\begin{sphinxuseclass}{sd-col-sm-6}
\begin{sphinxuseclass}{sd-col-md-6}
\begin{sphinxuseclass}{sd-col-lg-6}
\sphinxAtStartPar
(b)  
\( \begin{align*}
     - 4 x_{1} + 2 x_{2} &= 6, \\
     - x_{1} - 3 x_{2} &= -2.
\end{align*} \)

\end{sphinxuseclass}
\end{sphinxuseclass}
\end{sphinxuseclass}
\end{sphinxuseclass}
\end{sphinxuseclass}
\end{sphinxuseclass}
\end{sphinxuseclass}
\begin{sphinxuseclass}{sd-col}
\begin{sphinxuseclass}{sd-d-flex-column}
\begin{sphinxuseclass}{sd-col-12}
\begin{sphinxuseclass}{sd-col-xs-12}
\begin{sphinxuseclass}{sd-col-sm-12}
\begin{sphinxuseclass}{sd-col-md-12}
\begin{sphinxuseclass}{sd-col-lg-12}
\sphinxAtStartPar
 

\end{sphinxuseclass}
\end{sphinxuseclass}
\end{sphinxuseclass}
\end{sphinxuseclass}
\end{sphinxuseclass}
\end{sphinxuseclass}
\end{sphinxuseclass}
\begin{sphinxuseclass}{sd-col}
\begin{sphinxuseclass}{sd-d-flex-column}
\begin{sphinxuseclass}{sd-col-6}
\begin{sphinxuseclass}{sd-col-xs-6}
\begin{sphinxuseclass}{sd-col-sm-6}
\begin{sphinxuseclass}{sd-col-md-6}
\begin{sphinxuseclass}{sd-col-lg-6}
\sphinxAtStartPar
(c)  
\( \begin{align*}
     - 4 x_{1} - 4 x_{2} - 2 x_{3} &= 16, \\
     3 x_{1} + 4 x_{3} &= -8, \\
     x_{1} &= 0.
\end{align*} \)

\end{sphinxuseclass}
\end{sphinxuseclass}
\end{sphinxuseclass}
\end{sphinxuseclass}
\end{sphinxuseclass}
\end{sphinxuseclass}
\end{sphinxuseclass}
\begin{sphinxuseclass}{sd-col}
\begin{sphinxuseclass}{sd-d-flex-column}
\begin{sphinxuseclass}{sd-col-6}
\begin{sphinxuseclass}{sd-col-xs-6}
\begin{sphinxuseclass}{sd-col-sm-6}
\begin{sphinxuseclass}{sd-col-md-6}
\begin{sphinxuseclass}{sd-col-lg-6}
\sphinxAtStartPar
(d)  
\( \begin{align*}
     4 x_{1} - 4 x_{3} &= 8, \\
     4 x_{1} -  x_{2} +  x_{3} &= -4, \\
     3 x_{1} +  x_{2} + 2 x_{3} &= -12.
\end{align*} \)

\end{sphinxuseclass}
\end{sphinxuseclass}
\end{sphinxuseclass}
\end{sphinxuseclass}
\end{sphinxuseclass}
\end{sphinxuseclass}
\end{sphinxuseclass}
\end{sphinxuseclass}
\end{sphinxuseclass}
\end{sphinxuseclass}
\end{sphinxuseclass}\end{sphinxadmonition}
\phantomsection \label{exercise:systems-ex-cramer}

\begin{sphinxadmonition}{note}{Exercise 2.11.2}



\sphinxAtStartPar
Solve the following linear system of equations using Cramer’s rule.

\begin{sphinxuseclass}{sd-container-fluid}
\begin{sphinxuseclass}{sd-sphinx-override}
\begin{sphinxuseclass}{sd-mb-4}
\begin{sphinxuseclass}{sd-row}
\begin{sphinxuseclass}{sd-col}
\begin{sphinxuseclass}{sd-d-flex-column}
\begin{sphinxuseclass}{sd-col-6}
\begin{sphinxuseclass}{sd-col-xs-6}
\begin{sphinxuseclass}{sd-col-sm-6}
\begin{sphinxuseclass}{sd-col-md-6}
\begin{sphinxuseclass}{sd-col-lg-6}
\sphinxAtStartPar
(a)  
\( \begin{align*}
     x_{1} + 4 x_{2} &= -20, \\
     - 4 x_{1} +  x_{2} &= -5.
\end{align*} \)

\end{sphinxuseclass}
\end{sphinxuseclass}
\end{sphinxuseclass}
\end{sphinxuseclass}
\end{sphinxuseclass}
\end{sphinxuseclass}
\end{sphinxuseclass}
\begin{sphinxuseclass}{sd-col}
\begin{sphinxuseclass}{sd-d-flex-column}
\begin{sphinxuseclass}{sd-col-6}
\begin{sphinxuseclass}{sd-col-xs-6}
\begin{sphinxuseclass}{sd-col-sm-6}
\begin{sphinxuseclass}{sd-col-md-6}
\begin{sphinxuseclass}{sd-col-lg-6}
\sphinxAtStartPar
(b)  
\( \begin{align*}
     x_{1} +  x_{2} &= 4, \\
     4 x_{2} &= 12.
\end{align*} \)

\end{sphinxuseclass}
\end{sphinxuseclass}
\end{sphinxuseclass}
\end{sphinxuseclass}
\end{sphinxuseclass}
\end{sphinxuseclass}
\end{sphinxuseclass}
\begin{sphinxuseclass}{sd-col}
\begin{sphinxuseclass}{sd-d-flex-column}
\begin{sphinxuseclass}{sd-col-12}
\begin{sphinxuseclass}{sd-col-xs-12}
\begin{sphinxuseclass}{sd-col-sm-12}
\begin{sphinxuseclass}{sd-col-md-12}
\begin{sphinxuseclass}{sd-col-lg-12}
\sphinxAtStartPar
 

\end{sphinxuseclass}
\end{sphinxuseclass}
\end{sphinxuseclass}
\end{sphinxuseclass}
\end{sphinxuseclass}
\end{sphinxuseclass}
\end{sphinxuseclass}
\begin{sphinxuseclass}{sd-col}
\begin{sphinxuseclass}{sd-d-flex-column}
\begin{sphinxuseclass}{sd-col-6}
\begin{sphinxuseclass}{sd-col-xs-6}
\begin{sphinxuseclass}{sd-col-sm-6}
\begin{sphinxuseclass}{sd-col-md-6}
\begin{sphinxuseclass}{sd-col-lg-6}
\sphinxAtStartPar
(c)  
\( \begin{align*}
     3 x_{1} - 4 x_{2} - 4 x_{3} &= 21, \\
     - 2 x_{1} -  x_{2} -  x_{3} &= 8, \\
     4 x_{1} -  x_{2} + 3 x_{3} &= -14.
\end{align*} \)

\end{sphinxuseclass}
\end{sphinxuseclass}
\end{sphinxuseclass}
\end{sphinxuseclass}
\end{sphinxuseclass}
\end{sphinxuseclass}
\end{sphinxuseclass}
\begin{sphinxuseclass}{sd-col}
\begin{sphinxuseclass}{sd-d-flex-column}
\begin{sphinxuseclass}{sd-col-6}
\begin{sphinxuseclass}{sd-col-xs-6}
\begin{sphinxuseclass}{sd-col-sm-6}
\begin{sphinxuseclass}{sd-col-md-6}
\begin{sphinxuseclass}{sd-col-lg-6}
\sphinxAtStartPar
(d)   
\( \begin{align*}
     4 x_{1} + 4 x_{2} +  x_{3} &= 5, \\
     - 2 x_{1} +  x_{2} +  x_{3} &= -1, \\
     - 5 x_{1} - 4 x_{2} + 2 x_{3} &= -14.
\end{align*} \)

\end{sphinxuseclass}
\end{sphinxuseclass}
\end{sphinxuseclass}
\end{sphinxuseclass}
\end{sphinxuseclass}
\end{sphinxuseclass}
\end{sphinxuseclass}
\end{sphinxuseclass}
\end{sphinxuseclass}
\end{sphinxuseclass}
\end{sphinxuseclass}\end{sphinxadmonition}
\phantomsection \label{exercise:systems-ex-gelim}

\begin{sphinxadmonition}{note}{Exercise 2.11.3}



\sphinxAtStartPar
Solve the following linear system of equations using Gaussian elimination.

\begin{sphinxuseclass}{sd-container-fluid}
\begin{sphinxuseclass}{sd-sphinx-override}
\begin{sphinxuseclass}{sd-mb-4}
\begin{sphinxuseclass}{sd-row}
\begin{sphinxuseclass}{sd-col}
\begin{sphinxuseclass}{sd-d-flex-column}
\begin{sphinxuseclass}{sd-col-6}
\begin{sphinxuseclass}{sd-col-xs-6}
\begin{sphinxuseclass}{sd-col-sm-6}
\begin{sphinxuseclass}{sd-col-md-6}
\begin{sphinxuseclass}{sd-col-lg-6}
\sphinxAtStartPar
(a)  
\( \begin{align*}
     - x_{1} + 3 x_{2} &= -2, \\
     - 2 x_{1} +  x_{2} &= 1.
\end{align*} \)

\end{sphinxuseclass}
\end{sphinxuseclass}
\end{sphinxuseclass}
\end{sphinxuseclass}
\end{sphinxuseclass}
\end{sphinxuseclass}
\end{sphinxuseclass}
\begin{sphinxuseclass}{sd-col}
\begin{sphinxuseclass}{sd-d-flex-column}
\begin{sphinxuseclass}{sd-col-6}
\begin{sphinxuseclass}{sd-col-xs-6}
\begin{sphinxuseclass}{sd-col-sm-6}
\begin{sphinxuseclass}{sd-col-md-6}
\begin{sphinxuseclass}{sd-col-lg-6}
\sphinxAtStartPar
(b)  
\( \begin{align*}
     3 x_{1} +  x_{2} + 2 x_{3} &= 11, \\
     4 x_{1} - 4 x_{3} &= -4, \\
     4 x_{1} - 2 x_{2} +  x_{3} &= 13.
\end{align*} \)

\end{sphinxuseclass}
\end{sphinxuseclass}
\end{sphinxuseclass}
\end{sphinxuseclass}
\end{sphinxuseclass}
\end{sphinxuseclass}
\end{sphinxuseclass}
\begin{sphinxuseclass}{sd-col}
\begin{sphinxuseclass}{sd-d-flex-column}
\begin{sphinxuseclass}{sd-col-12}
\begin{sphinxuseclass}{sd-col-xs-12}
\begin{sphinxuseclass}{sd-col-sm-12}
\begin{sphinxuseclass}{sd-col-md-12}
\begin{sphinxuseclass}{sd-col-lg-12}
\sphinxAtStartPar
 

\end{sphinxuseclass}
\end{sphinxuseclass}
\end{sphinxuseclass}
\end{sphinxuseclass}
\end{sphinxuseclass}
\end{sphinxuseclass}
\end{sphinxuseclass}
\begin{sphinxuseclass}{sd-col}
\begin{sphinxuseclass}{sd-d-flex-column}
\begin{sphinxuseclass}{sd-col-6}
\begin{sphinxuseclass}{sd-col-xs-6}
\begin{sphinxuseclass}{sd-col-sm-6}
\begin{sphinxuseclass}{sd-col-md-6}
\begin{sphinxuseclass}{sd-col-lg-6}
\sphinxAtStartPar
(c)  
\( \begin{align*}
     - x_{1} - 5 x_{2} - 2 x_{3} &= -17, \\
     2 x_{1} - 2 x_{2} - 3 x_{3} &= -14, \\
     3 x_{1} -  x_{2} + 4 x_{3} &= -13.
\end{align*} \)

\end{sphinxuseclass}
\end{sphinxuseclass}
\end{sphinxuseclass}
\end{sphinxuseclass}
\end{sphinxuseclass}
\end{sphinxuseclass}
\end{sphinxuseclass}
\begin{sphinxuseclass}{sd-col}
\begin{sphinxuseclass}{sd-d-flex-column}
\begin{sphinxuseclass}{sd-col-6}
\begin{sphinxuseclass}{sd-col-xs-6}
\begin{sphinxuseclass}{sd-col-sm-6}
\begin{sphinxuseclass}{sd-col-md-6}
\begin{sphinxuseclass}{sd-col-lg-6}
\sphinxAtStartPar
(d)  
\( \begin{align*}
     - x_{1} - 5 x_{2} - 2 x_{3} &= -26, \\
     2 x_{1} - 2 x_{2} - 3 x_{3} &= -19, \\
     3 x_{1} -  x_{2} - 4 x_{3} &= -20.
\end{align*} \)

\end{sphinxuseclass}
\end{sphinxuseclass}
\end{sphinxuseclass}
\end{sphinxuseclass}
\end{sphinxuseclass}
\end{sphinxuseclass}
\end{sphinxuseclass}
\begin{sphinxuseclass}{sd-col}
\begin{sphinxuseclass}{sd-d-flex-column}
\begin{sphinxuseclass}{sd-col-12}
\begin{sphinxuseclass}{sd-col-xs-12}
\begin{sphinxuseclass}{sd-col-sm-12}
\begin{sphinxuseclass}{sd-col-md-12}
\begin{sphinxuseclass}{sd-col-lg-12}
\sphinxAtStartPar
 

\end{sphinxuseclass}
\end{sphinxuseclass}
\end{sphinxuseclass}
\end{sphinxuseclass}
\end{sphinxuseclass}
\end{sphinxuseclass}
\end{sphinxuseclass}
\begin{sphinxuseclass}{sd-col}
\begin{sphinxuseclass}{sd-d-flex-column}
\begin{sphinxuseclass}{sd-col-6}
\begin{sphinxuseclass}{sd-col-xs-6}
\begin{sphinxuseclass}{sd-col-sm-6}
\begin{sphinxuseclass}{sd-col-md-6}
\begin{sphinxuseclass}{sd-col-lg-6}
\sphinxAtStartPar
(e)  
\( \begin{align*}
     3 x_{1} - 5 x_{2} - 4 x_{3} -  x_{4} &= 28, \\
     - 4 x_{2} + 3 x_{3} - 4 x_{4} &= 41, \\
     2 x_{1} + 3 x_{2} + 3 x_{3} - 3 x_{4} &= 11, \\
     - 2 x_{1} + 2 x_{2} - 5 x_{3} - 4 x_{4} &= -21.
\end{align*} \)

\end{sphinxuseclass}
\end{sphinxuseclass}
\end{sphinxuseclass}
\end{sphinxuseclass}
\end{sphinxuseclass}
\end{sphinxuseclass}
\end{sphinxuseclass}
\begin{sphinxuseclass}{sd-col}
\begin{sphinxuseclass}{sd-d-flex-column}
\begin{sphinxuseclass}{sd-col-6}
\begin{sphinxuseclass}{sd-col-xs-6}
\begin{sphinxuseclass}{sd-col-sm-6}
\begin{sphinxuseclass}{sd-col-md-6}
\begin{sphinxuseclass}{sd-col-lg-6}
\sphinxAtStartPar
(f)  
\( \begin{align*}
     2 x_{1} - 3 x_{2} - 3 x_{3} + 4 x_{4} &= -1, \\
     4 x_{1} - 5 x_{2} +  x_{3} - 5 x_{4} &= 42, \\
     3 x_{1} + 3 x_{2} -  x_{3} - 5 x_{4} &= 20, \\
     x_{1} +  x_{3} + 3 x_{4} &= -4.
\end{align*} \)

\end{sphinxuseclass}
\end{sphinxuseclass}
\end{sphinxuseclass}
\end{sphinxuseclass}
\end{sphinxuseclass}
\end{sphinxuseclass}
\end{sphinxuseclass}
\end{sphinxuseclass}
\end{sphinxuseclass}
\end{sphinxuseclass}
\end{sphinxuseclass}\end{sphinxadmonition}
\phantomsection \label{exercise:systems-ex-gelimpp}

\begin{sphinxadmonition}{note}{Exercise 2.11.4}



\sphinxAtStartPar
Solve the linear system of equations from \hyperref[exercise:systems-ex-gelim]{Exercise 2.11.3} using Gaussian elimination with partial pivoting.
\end{sphinxadmonition}
\phantomsection \label{exercise:systems-ex-gjelim}

\begin{sphinxadmonition}{note}{Exercise 2.11.5}



\sphinxAtStartPar
Solve the linear system of equations from \hyperref[exercise:systems-ex-gelim]{Exercise 2.11.3} using Gauss\sphinxhyphen{}Jordan elimination.
\end{sphinxadmonition}
\phantomsection \label{exercise:systems-ex-gj-inverse}

\begin{sphinxadmonition}{note}{Exercise 2.11.6}



\sphinxAtStartPar
Use Gauss\sphinxhyphen{}Jordan elimination to calculate the inverse of the following matrices.

\begin{sphinxuseclass}{sd-container-fluid}
\begin{sphinxuseclass}{sd-sphinx-override}
\begin{sphinxuseclass}{sd-mb-4}
\begin{sphinxuseclass}{sd-row}
\begin{sphinxuseclass}{sd-col}
\begin{sphinxuseclass}{sd-d-flex-column}
\begin{sphinxuseclass}{sd-col-6}
\begin{sphinxuseclass}{sd-col-xs-6}
\begin{sphinxuseclass}{sd-col-sm-6}
\begin{sphinxuseclass}{sd-col-md-6}
\begin{sphinxuseclass}{sd-col-lg-6}
\sphinxAtStartPar
(a)  
\( \begin{pmatrix} -4 & 2 \\ 3 & 4 \end{pmatrix} \)

\end{sphinxuseclass}
\end{sphinxuseclass}
\end{sphinxuseclass}
\end{sphinxuseclass}
\end{sphinxuseclass}
\end{sphinxuseclass}
\end{sphinxuseclass}
\begin{sphinxuseclass}{sd-col}
\begin{sphinxuseclass}{sd-d-flex-column}
\begin{sphinxuseclass}{sd-col-6}
\begin{sphinxuseclass}{sd-col-xs-6}
\begin{sphinxuseclass}{sd-col-sm-6}
\begin{sphinxuseclass}{sd-col-md-6}
\begin{sphinxuseclass}{sd-col-lg-6}
\sphinxAtStartPar
(b)  
\( \begin{pmatrix} -4 & 2 \\ -1 & -3 \end{pmatrix} \)

\end{sphinxuseclass}
\end{sphinxuseclass}
\end{sphinxuseclass}
\end{sphinxuseclass}
\end{sphinxuseclass}
\end{sphinxuseclass}
\end{sphinxuseclass}
\begin{sphinxuseclass}{sd-col}
\begin{sphinxuseclass}{sd-d-flex-column}
\begin{sphinxuseclass}{sd-col-12}
\begin{sphinxuseclass}{sd-col-xs-12}
\begin{sphinxuseclass}{sd-col-sm-12}
\begin{sphinxuseclass}{sd-col-md-12}
\begin{sphinxuseclass}{sd-col-lg-12}
\sphinxAtStartPar
 

\end{sphinxuseclass}
\end{sphinxuseclass}
\end{sphinxuseclass}
\end{sphinxuseclass}
\end{sphinxuseclass}
\end{sphinxuseclass}
\end{sphinxuseclass}
\begin{sphinxuseclass}{sd-col}
\begin{sphinxuseclass}{sd-d-flex-column}
\begin{sphinxuseclass}{sd-col-6}
\begin{sphinxuseclass}{sd-col-xs-6}
\begin{sphinxuseclass}{sd-col-sm-6}
\begin{sphinxuseclass}{sd-col-md-6}
\begin{sphinxuseclass}{sd-col-lg-6}
\sphinxAtStartPar
(c)  
\( \begin{pmatrix} -4 & -4 & - 2\\ 3 & 0 & 4 \\ 1 & 0 & 0 \end{pmatrix} \)

\end{sphinxuseclass}
\end{sphinxuseclass}
\end{sphinxuseclass}
\end{sphinxuseclass}
\end{sphinxuseclass}
\end{sphinxuseclass}
\end{sphinxuseclass}
\begin{sphinxuseclass}{sd-col}
\begin{sphinxuseclass}{sd-d-flex-column}
\begin{sphinxuseclass}{sd-col-6}
\begin{sphinxuseclass}{sd-col-xs-6}
\begin{sphinxuseclass}{sd-col-sm-6}
\begin{sphinxuseclass}{sd-col-md-6}
\begin{sphinxuseclass}{sd-col-lg-6}
\sphinxAtStartPar
(d)  
\( \begin{pmatrix} 4 & 0 & -4 \\ 4 & -1 & 1 \\ 3 & 1 & 2 \end{pmatrix} \)

\end{sphinxuseclass}
\end{sphinxuseclass}
\end{sphinxuseclass}
\end{sphinxuseclass}
\end{sphinxuseclass}
\end{sphinxuseclass}
\end{sphinxuseclass}
\end{sphinxuseclass}
\end{sphinxuseclass}
\end{sphinxuseclass}
\end{sphinxuseclass}\end{sphinxadmonition}
\phantomsection \label{exercise:systems-ex-consistency}

\begin{sphinxadmonition}{note}{Exercise 2.11.7}



\sphinxAtStartPar
For the following linear systems of equations, determine the rank of the coefficient matrix and the augmented matrix and classify the system is consistent, inconsistent or indeterminate and calculate the solution (if possible).

\begin{sphinxuseclass}{sd-container-fluid}
\begin{sphinxuseclass}{sd-sphinx-override}
\begin{sphinxuseclass}{sd-mb-4}
\begin{sphinxuseclass}{sd-row}
\begin{sphinxuseclass}{sd-col}
\begin{sphinxuseclass}{sd-d-flex-column}
\begin{sphinxuseclass}{sd-col-6}
\begin{sphinxuseclass}{sd-col-xs-6}
\begin{sphinxuseclass}{sd-col-sm-6}
\begin{sphinxuseclass}{sd-col-md-6}
\begin{sphinxuseclass}{sd-col-lg-6}
\sphinxAtStartPar
(a)   
\( \begin{align*}
     x_{1} -  x_{2} + 2 x_{3} &= 2, \\
     2 x_{1} +  x_{2} + 4 x_{3} &= 7, \\
     4 x_{1} +  x_{2} +  x_{3} &= 4.
\end{align*} \)

\end{sphinxuseclass}
\end{sphinxuseclass}
\end{sphinxuseclass}
\end{sphinxuseclass}
\end{sphinxuseclass}
\end{sphinxuseclass}
\end{sphinxuseclass}
\begin{sphinxuseclass}{sd-col}
\begin{sphinxuseclass}{sd-d-flex-column}
\begin{sphinxuseclass}{sd-col-6}
\begin{sphinxuseclass}{sd-col-xs-6}
\begin{sphinxuseclass}{sd-col-sm-6}
\begin{sphinxuseclass}{sd-col-md-6}
\begin{sphinxuseclass}{sd-col-lg-6}
\sphinxAtStartPar
(b)  
\( \begin{align*}
     x_{1} -  x_{2} + 2 x_{3} &= 3, \\
     2 x_{1} - 3 x_{2} + 7 x_{3} &= 4, \\
     - x_{1} + 3 x_{2} - 8 x_{3} &= 1.
\end{align*} \)

\end{sphinxuseclass}
\end{sphinxuseclass}
\end{sphinxuseclass}
\end{sphinxuseclass}
\end{sphinxuseclass}
\end{sphinxuseclass}
\end{sphinxuseclass}
\begin{sphinxuseclass}{sd-col}
\begin{sphinxuseclass}{sd-d-flex-column}
\begin{sphinxuseclass}{sd-col-12}
\begin{sphinxuseclass}{sd-col-xs-12}
\begin{sphinxuseclass}{sd-col-sm-12}
\begin{sphinxuseclass}{sd-col-md-12}
\begin{sphinxuseclass}{sd-col-lg-12}
\sphinxAtStartPar
 

\end{sphinxuseclass}
\end{sphinxuseclass}
\end{sphinxuseclass}
\end{sphinxuseclass}
\end{sphinxuseclass}
\end{sphinxuseclass}
\end{sphinxuseclass}
\begin{sphinxuseclass}{sd-col}
\begin{sphinxuseclass}{sd-d-flex-column}
\begin{sphinxuseclass}{sd-col-6}
\begin{sphinxuseclass}{sd-col-xs-6}
\begin{sphinxuseclass}{sd-col-sm-6}
\begin{sphinxuseclass}{sd-col-md-6}
\begin{sphinxuseclass}{sd-col-lg-6}
\sphinxAtStartPar
(c)  
\( \begin{align*}
     x_{1} +  x_{2} - 2 x_{3} &= 1, \\
     2 x_{1} -  x_{2} +  x_{3} &= 9, \\
     x_{1} + 4 x_{2} - 7 x_{3} &= 2.
\end{align*} \)

\end{sphinxuseclass}
\end{sphinxuseclass}
\end{sphinxuseclass}
\end{sphinxuseclass}
\end{sphinxuseclass}
\end{sphinxuseclass}
\end{sphinxuseclass}
\end{sphinxuseclass}
\end{sphinxuseclass}
\end{sphinxuseclass}
\end{sphinxuseclass}\end{sphinxadmonition}

\sphinxstepscope


\chapter{Vectors}
\label{\detokenize{_pages/3.0_Vectors:vectors}}\label{\detokenize{_pages/3.0_Vectors::doc}}
\index{Vector@\spxentry{Vector}}\ignorespaces \phantomsection\label{\detokenize{_pages/3.0_Vectors:vectors-chapter}}
\sphinxAtStartPar
Vectors are way of expressing quantities that cannot be described by a single number. This could  involve a data set with multiple variables, or a geometrical transformation or movement \sphinxhyphen{} vectors are often thought of as being an ‘arrow’ in \(n\)\sphinxhyphen{}dimensional space.

\sphinxAtStartPar
For example, consider the displacement of an object in 2 dimensional space from one location to another: there is a change in the position in one direction, and a change in the position of another, so a single value cannot carry this information. Vectors were originally used in geometry, but have since become important in multivariate calculus, computer graphics, engineering and physics.


\bigskip\hrule\bigskip


\index{Euclidean space@\spxentry{Euclidean space}}\ignorespaces 

\section{Euclidean space}
\label{\detokenize{_pages/3.0_Vectors:euclidean-space}}\label{\detokenize{_pages/3.0_Vectors:index-1}}\label{\detokenize{_pages/3.0_Vectors:euclidean-space-section}}
\sphinxAtStartPar
Before we discuss vectors, it is useful to first define \sphinxstylestrong{Euclidean space} which is the mathematical system in which vectors are most commonly studies. Attributed to the Greek mathematician Euclid, Euclidean space is a representation of physical space where the position of a point in the space can be described by the signed distance along perpendicular real number lines called \sphinxstylestrong{axes} (singular: axis).

\sphinxAtStartPar
An \(n\)\sphinxhyphen{}dimensional Euclidean space is defined by \(n\) perpendicular real axes, and is referred to as \(\mathbb{R}^n\). For example, consider the diagram of \(\mathbb{R}^3\) in \hyperref[\detokenize{_pages/3.0_Vectors:r3-figure}]{Fig.\@ \ref{\detokenize{_pages/3.0_Vectors:r3-figure}}}. Here we can see a portion of 3\sphinxhyphen{}dimensional Euclidean space defined by the 3 axes labelled \(x\), \(y\) and \(z\).

\sphinxAtStartPar
This representation uses the \sphinxstylestrong{right\sphinxhyphen{}hand rule} so\sphinxhyphen{}called because if we use the thumb on our right hand to represent the \(x\) axis, the index finger for the \(y\) axis and the middle finger for the \(z\) axis, we can arrange a right hand into this shape. Doing similar with the left hand gives the \sphinxstylestrong{left\sphinxhyphen{}hand rule}, where the \(x\) axis is pointing in the opposite direction.

\begin{figure}[htbp]
\centering
\capstart

\noindent\sphinxincludegraphics[width=300\sphinxpxdimen]{{3_R3}.svg}
\caption{The position of a point in \(\mathbb{R}^3\) can be defined by its co\sphinxhyphen{}ordinates \((x, y, z)\).}\label{\detokenize{_pages/3.0_Vectors:r3-figure}}\end{figure}

\index{Co\sphinxhyphen{}ordinates@\spxentry{Co\sphinxhyphen{}ordinates}}\ignorespaces 
\index{Tuple@\spxentry{Tuple}}\ignorespaces 
\sphinxAtStartPar
The position of a point in Euclidean space can be defined by its \sphinxstylestrong{co\sphinxhyphen{}ordinates} \sphinxhyphen{} an ordered set of numbers, sometimes called a \sphinxstylestrong{tuple}, where each element contains the signed distances along each axis. For example, \((x, y, z)\) where \(x, y, z \in \mathbb{R}\). (‘Signed’ here means that the value has a number, and also a sign, which can be positive or negative).

\sphinxAtStartPar
This system of defining position in space is sometimes also called \sphinxstylestrong{Cartesian coordinates}, named after the mathematician and philosopher René Descartes.


\bigskip\hrule\bigskip



\section{Definition of a Vector}
\label{\detokenize{_pages/3.0_Vectors:definition-of-a-vector}}\label{\detokenize{_pages/3.0_Vectors:vectors-definition-section}}
\sphinxAtStartPar
In Euclidean space, a \sphinxstylestrong{vector} represents a \sphinxstylestrong{displacement} from a point \(A\) to a point \(B\) (\hyperref[\detokenize{_pages/3.0_Vectors:vector-figure}]{Fig.\@ \ref{\detokenize{_pages/3.0_Vectors:vector-figure}}}). A vector is an object that has a \sphinxstylestrong{magnitude} (length) and a \sphinxstylestrong{direction}. The magnitude of the vector is the distance between the two points, and the direction refers to the direction of displacement from \(A\) to \(B\), with respect to the given Euclidean space. Note that the actual points \(A\) and \(B\) are not important for the definition of the vector \sphinxhyphen{} only the movement required to get from \(A\) to \(B\). In diagrams, vectors are  often represented by arrows. The start of the vector is sometimes called the \sphinxstylestrong{tail}, and the end of the vector (where the arrow is) is called the \sphinxstylestrong{head}.

\begin{figure}[htbp]
\centering
\capstart

\noindent\sphinxincludegraphics[width=250\sphinxpxdimen]{{3_vector}.svg}
\caption{The vector pointing from \(A\) to \(B\).}\label{\detokenize{_pages/3.0_Vectors:vector-figure}}\end{figure}

\sphinxAtStartPar
Vectors in mathematical notation are usually denoted by a single lowercase character. To distinguish vectors from other mathematical objects such as scalars and variables, the character representing a vector is written in boldface when typeset, e.g., \(\mathbf{a}\) or underlined when written by hand, e.g., \(\underline{a}\).

\begin{sphinxadmonition}{important}{Important:}
\sphinxAtStartPar
Get into the habit of always underlining your vectors when writing by hand. This helps to avoid confusion between vectors and other mathematical objects such as variables, matrices etc.
\end{sphinxadmonition}

\sphinxAtStartPar
A vector can be represented by a single column or row matrix:
\begin{equation*}
\begin{split} \begin{align*}
    \mathbf{a} = \begin{pmatrix} a_1 \\ a_2 \\ \vdots \\ a_n\end{pmatrix} = (a_1, a_2, \ldots, a_n),
\end{align*} \end{split}
\end{equation*}
\sphinxAtStartPar
Here, each element represents the signed difference between the tail and head of the vector in that Cartesian coordinate. The choice of whether to use row or column vectors is arbitrary, but here we will express vectors as column matrices in equations, as it makes it easier when dealing with {\hyperref[\detokenize{_pages/3.4_Linear_combinations:linear-combination-of-vectors-section}]{\sphinxcrossref{\DUrole{std,std-ref}{linear combination of vectors}}}} and {\hyperref[\detokenize{_pages/6.0_Linear_transformations:linear-transformations-chapter}]{\sphinxcrossref{\DUrole{std,std-ref}{linear transformations}}}}.

\sphinxAtStartPar
Consider the vector \(\mathbf{a}\) in \(\mathbb{R}^3\) as given in the \hyperref[\detokenize{_pages/3.0_Vectors:r3-vector-figure}]{Fig.\@ \ref{\detokenize{_pages/3.0_Vectors:r3-vector-figure}}}. Here the displacement along the \(x\), \(y\) and \(z\) axes are \(a_1\), \(a_2\) and \(a_3\) respectively, therefore \(\mathbf{a} = (a_1, a_2, a_3)\).

\begin{figure}[htbp]
\centering
\capstart

\noindent\sphinxincludegraphics[width=500\sphinxpxdimen]{{3_R3_vector}.svg}
\caption{The vector \(\mathbf{a} = (a_1, a_2, a_3)\) in \(\mathbb{R}^3\).}\label{\detokenize{_pages/3.0_Vectors:r3-vector-figure}}\end{figure}

\sphinxAtStartPar
We use \(\mathbf{0}\) to denote the \sphinxstylestrong{zero vector} \((0, 0, \ldots, 0)\). This can be thought of as the vector \(A - A\), or the vector from a point \(A\) to itself, for any point \(A\). In this chapter we introduce vectors in Euclidean space, and so the coefficients \(a_i\) will necessarily be real numbers. In the chapter on {\hyperref[\detokenize{_pages/5.0_Vector_spaces:vector-spaces-chapter}]{\sphinxcrossref{\DUrole{std,std-ref}{vector spaces}}}} we will generalise the notion of a vector, and allow its coefficients to be any elements coming from any underlying field.


\bigskip\hrule\bigskip


\index{Vector@\spxentry{Vector}!position vector@\spxentry{position vector}}\ignorespaces 

\section{Position vectors}
\label{\detokenize{_pages/3.0_Vectors:position-vectors}}\label{\detokenize{_pages/3.0_Vectors:index-4}}\label{\detokenize{_pages/3.0_Vectors:position-vector-section}}
\sphinxAtStartPar
A \sphinxstylestrong{position vector} \(\mathbf{p}\) is a vector that points from the origin \((0, 0, 0)\) to a point in the space (\hyperref[\detokenize{_pages/3.0_Vectors:position-vector-figure}]{Fig.\@ \ref{\detokenize{_pages/3.0_Vectors:position-vector-figure}}}).

\begin{figure}[htbp]
\centering
\capstart

\noindent\sphinxincludegraphics[width=350\sphinxpxdimen]{{3_position_vector}.svg}
\caption{The position vector \(\mathbf{p}\) points from the origin \(O\) to a point in the space.}\label{\detokenize{_pages/3.0_Vectors:position-vector-figure}}\end{figure}

\sphinxAtStartPar
So if a point has co\sphinxhyphen{}ordinates \((x, y, z)\) then the position vector that describes that point is \(\mathbf{p} = (x, y, z)\).



\sphinxstepscope

\index{Vector@\spxentry{Vector}!arithmetic@\spxentry{arithmetic}}\ignorespaces 

\section{Arithmetic operations on vectors}
\label{\detokenize{_pages/3.1_Vector_arithmetic:arithmetic-operations-on-vectors}}\label{\detokenize{_pages/3.1_Vector_arithmetic:index-0}}\label{\detokenize{_pages/3.1_Vector_arithmetic:arithmetic-operations-on-vectors-section}}\label{\detokenize{_pages/3.1_Vector_arithmetic::doc}}
\index{Vector@\spxentry{Vector}!equality@\spxentry{equality}}\ignorespaces 

\bigskip\hrule\bigskip



\subsection{Vector equality}
\label{\detokenize{_pages/3.1_Vector_arithmetic:vector-equality}}\label{_pages/3.1_Vector_arithmetic:vector-equality-definition}
\begin{sphinxadmonition}{note}{Definition 3.4.1 (Vector equality)}



\sphinxAtStartPar
Two vectors in \(\mathbb{R}^n\) are said to be equal if their corresponding elements are equal. In other words \(\mathbf{a},\mathbf{b}\in \mathbb{R}^n\) where  \(\mathbf{a} = (a_1, a_2, \ldots, a_n)\) and \(\mathbf{b} = (b_1, b_2, \ldots, b_n)\) respectively, are equal if and only if the elements which correspond co\sphinxhyphen{}ordinate\sphinxhyphen{}wise are equal, i.e.,
\begin{equation*}
\begin{split} \begin{align*}
    a_i = b_i, \qquad i = 1, \ldots, n.
\end{align*} \end{split}
\end{equation*}\end{sphinxadmonition}

\sphinxAtStartPar
Informally, this happens if and only if two vectors have the same magnitude and they point in the same direction.

\sphinxAtStartPar
It follows from the definition of vector equality that a vector can be moved so it remains parallel to itself, and still represent the same vector, i.e., the position of the vector does not matter when considering vector equality.

\sphinxAtStartPar
In general if two vectors \(\mathbf{u}\) and \(\mathbf{v}\) are parallel then we write \(\mathbf{u} \parallel \mathbf{v}\).

\sphinxAtStartPar
Consider \hyperref[\detokenize{_pages/3.1_Vector_arithmetic:vector-equality-figure}]{Fig.\@ \ref{\detokenize{_pages/3.1_Vector_arithmetic:vector-equality-figure}}}, which shows four vectors in \(\mathbb{R}^2\). The vectors \(\mathbf{a}\) and \(\mathbf{b}\) point in the same direction and have the same magnitude, so we can say that \(\mathbf{a}=\mathbf{b}\). The vector \(\mathbf{c}\) has the same magnitude as \(\mathbf{a}\) and \(\mathbf{b}\) ,but points in the opposite direction, so \(\mathbf{c}\neq \mathbf{a}\). The vector \(\mathbf{d}\) points in the same direction as \(\mathbf{a}\), but has a larger magnitude \sphinxhyphen{} so \(\mathbf{d} \neq \mathbf{a}\).

\begin{figure}[htbp]
\centering
\capstart

\noindent\sphinxincludegraphics[width=450\sphinxpxdimen]{{3_vector_equality}.svg}
\caption{Of these four vectors only \(\mathbf{a}\) and \(\mathbf{b}\) are equal.}\label{\detokenize{_pages/3.1_Vector_arithmetic:vector-equality-figure}}\end{figure}




\bigskip\hrule\bigskip


\index{Vector@\spxentry{Vector}!addition@\spxentry{addition}}\ignorespaces 

\subsection{Vector addition}
\label{\detokenize{_pages/3.1_Vector_arithmetic:vector-addition}}\label{\detokenize{_pages/3.1_Vector_arithmetic:index-2}}\label{_pages/3.1_Vector_arithmetic:vector-addition-definition}
\begin{sphinxadmonition}{note}{Definition 3.4.2 (Addition of two vectors)}



\sphinxAtStartPar
The sum of two vectors \(\mathbf{a},\mathbf{b}\in\mathbb{R}^n\) can be calculated as
\begin{equation*}
\begin{split} \begin{align*}
    \mathbf{a} + \mathbf{b} = 
    \begin{pmatrix} a_1 \\ a_2 \\ \vdots \\ a_n \end{pmatrix} + 
    \begin{pmatrix} b_1 \\ b_2 \\ \vdots \\ b_n \end{pmatrix} =
    \begin{pmatrix} a_1 + b_1 \\ a_2 + b_2 \\ \vdots \\ a_n + b_n \end{pmatrix}.
\end{align*} \end{split}
\end{equation*}\end{sphinxadmonition}

\sphinxAtStartPar
A geometric representation of the sum \(\mathbf{a}+\mathbf{b}\) is shown in \hyperref[\detokenize{_pages/3.1_Vector_arithmetic:vector-addition-figure}]{Fig.\@ \ref{\detokenize{_pages/3.1_Vector_arithmetic:vector-addition-figure}}}. We place the vector \(\mathbf{a}\), and then position the vector \(\mathbf{b}\) so its tail is at the head of \(\mathbf{a}\). The endpoint of this instance of \(\mathbf{b}\) is the head of the vector \(\mathbf{a+b}\).

\sphinxAtStartPar
Note also, as the figure demonstrates, that the order in which one adds the two vectors is unimportant \sphinxhyphen{} if we instead placed \(\mathbf{b}\) first, then \(\mathbf{a}\), we would get to the same point. This is known as the \sphinxstylestrong{parallelogram law of addition}.

\begin{figure}[htbp]
\centering
\capstart

\noindent\sphinxincludegraphics[width=500\sphinxpxdimen]{{3_vector_addition}.svg}
\caption{The addition of two vectors.}\label{\detokenize{_pages/3.1_Vector_arithmetic:vector-addition-figure}}\end{figure}


\bigskip\hrule\bigskip


\index{Vector@\spxentry{Vector}!scalar multiplication@\spxentry{scalar multiplication}}\ignorespaces 



\subsection{Scalar multiplication}
\label{\detokenize{_pages/3.1_Vector_arithmetic:scalar-multiplication}}\label{_pages/3.1_Vector_arithmetic:scalar-multiplication-of-a-vector-definition}
\begin{sphinxadmonition}{note}{Definition 3.4.3 (Scalar multiplication of a vector)}



\sphinxAtStartPar
The product of a vector \(\mathbf{a}\in \mathbb{R}^n\) by a scalar \(k \in \mathbb{R}\) is given by
\begin{equation*}
\begin{split} k \mathbf{a} = k\begin{pmatrix} a_1 \\ a_2 \\ \vdots \\ a_n \end{pmatrix} =
\begin{pmatrix} k a_1 \\ k a_2 \\ \vdots \\ k a_n \end{pmatrix}. \end{split}
\end{equation*}\end{sphinxadmonition}

\sphinxAtStartPar
Multiplying a vector by a scalar has the effect of \sphinxstylestrong{scaling} the vector \(\mathbf{a}\) to produce a vector \sphinxstylestrong{parallel} to \(\mathbf{a}\) with a magnitude of \(k\|\mathbf{a}\|\). (We will go into more detail about the magnitude of vectors in the section on {\hyperref[\detokenize{_pages/3.2_Vector_magnitude:vector-magnitude-section}]{\sphinxcrossref{\DUrole{std,std-ref}{vector magnitude}}}}).
\label{_pages/3.1_Vector_arithmetic:properties-of-scalar-multiplication-of-vectors}
\begin{sphinxadmonition}{note}{Theorem 3.4.1 (Properties of scalar multiplication of vectors)}



\sphinxAtStartPar
The product of a vector \(\mathbf{a}\) and a scalar \(k\) has the following properties:
\begin{itemize}
\item {} 
\sphinxAtStartPar
\(k\mathbf{a}\) = \(\mathbf{a}k\) is a vector parallel to \(\mathbf{a}\), i.e. \(\mathbf{a} \parallel k\mathbf{a} \parallel \mathbf{a}k\).

\item {} 
\sphinxAtStartPar
\(0\mathbf{a} = (0, 0, \ldots, 0) = \mathbf{0}\)

\item {} 
\sphinxAtStartPar
\(1\mathbf{a} = \mathbf{a}\)

\item {} 
\sphinxAtStartPar
\(\|k\mathbf{a}\| = |k|\|\mathbf{a}\|\)   (\(\|\mathbf{a}\|\) is the length of the vector \(\mathbf{a}\) \sphinxhyphen{} see {\hyperref[\detokenize{_pages/3.2_Vector_magnitude:magnitude-definition}]{\sphinxcrossref{vector magnitude}}})

\item {} 
\sphinxAtStartPar
If \(k>0\) then \(\mathbf{a}\) and \(k\mathbf{a}\) point in the same direction, whereas if \(k<0\) then \(\mathbf{a}\) and \(k\mathbf{a}\) point in opposite directions

\end{itemize}
\end{sphinxadmonition}

\begin{figure}[htbp]
\centering
\capstart

\noindent\sphinxincludegraphics[width=500\sphinxpxdimen]{{3_scalar_multiplication_of_a_vector}.svg}
\caption{Scaled variations of the vector \(\mathbf{a}\).}\label{\detokenize{_pages/3.1_Vector_arithmetic:scalar-multiplication-of-a-vector-figure}}\end{figure}



\sphinxstepscope

\index{Vector@\spxentry{Vector}!magnitude@\spxentry{magnitude}}\ignorespaces 

\section{Vector magnitude}
\label{\detokenize{_pages/3.2_Vector_magnitude:vector-magnitude}}\label{\detokenize{_pages/3.2_Vector_magnitude:index-0}}\label{\detokenize{_pages/3.2_Vector_magnitude:vector-magnitude-section}}\label{\detokenize{_pages/3.2_Vector_magnitude::doc}}
\sphinxAtStartPar
The \sphinxstylestrong{magnitude} of a vector \(\mathbf{a}\) is the straight\sphinxhyphen{}line distance between the head and tail of \(\mathbf{a}\). In two dimensions, this can be calculated using \sphinxstylestrong{Pythagoras’ Theorem}, and in higher dimensions we can use an extended version Pythagoras’ theorem.
\label{_pages/3.2_Vector_magnitude:magnitude-definition}
\begin{sphinxadmonition}{note}{Definition 3.5.1 (Vector magnitude)}



\sphinxAtStartPar
The \sphinxstylestrong{magnitude} of a vector \(\mathbf{a} = (a_1, a_2, \ldots, a_n)\) denoted by \(\|\mathbf{a}\|\) is calculated using
\begin{equation}\label{equation:_pages/3.2_Vector_magnitude:magnitude-equation}
\begin{split} \|\mathbf{a}\| = \sqrt{\sum_{i=1}^n a_i^2} = \sqrt{a_1^2 + a_2^2 + \cdots + a_n^2}. \end{split}
\end{equation}
\begin{figure}[htbp]
\centering

\noindent\sphinxincludegraphics[width=250\sphinxpxdimen]{{3_magnitude}.svg}
\end{figure}
\end{sphinxadmonition}

\sphinxAtStartPar
This method of calculating vector magnitude is also known as the \sphinxstylestrong{Euclidean norm}.
\label{_pages/3.2_Vector_magnitude:magnitude-example}
\begin{sphinxadmonition}{note}{Example 3.5.1}



\sphinxAtStartPar
Calculate the magnitudes of the following vectors
\begin{equation*}
\begin{split} \begin{align*}
    \mathbf{u} &= (1, 2, 3), &
    \mathbf{v} &= (5, -12, 0) &
    \mathbf{w} &= (1, 0, 1).
\end{align*} \end{split}
\end{equation*}\subsubsection*{Solution}
\begin{equation*}
\begin{split} \begin{align*}
    \|\mathbf{u}\| &= \sqrt{1^2 + 2^2 + 3^2} = \sqrt{1+4+9} = \sqrt{14}, \\
    \|\mathbf{v}\| &= \sqrt{5^2 + (-12)^2 + 0^2} = \sqrt{25+144+0} = \sqrt{169} = 13, \\
    \|\mathbf{w}\| &= \sqrt{1^2 + 0^2 + 1^2} = \sqrt{1+0+1} = \sqrt{2}.
\end{align*} \end{split}
\end{equation*}\end{sphinxadmonition}
\label{_pages/3.2_Vector_magnitude:vector-magnitude-properties}
\begin{sphinxadmonition}{note}{Theorem 3.5.1 (Properties of vector magnitude)}



\sphinxAtStartPar
For any two vectors \(\mathbf{a}, \mathbf{b} \in \mathbb{R}^n\) and any scalar \(k \in \mathbb{R}\), the following properties are satisfied:
\begin{itemize}
\item {} 
\sphinxAtStartPar
\(\| \mathbf{a} \| \geq 0\)   (the magnitude of a vector \(\mathbf{v}\) is always non\sphinxhyphen{}negative)

\item {} 
\sphinxAtStartPar
\(\|\mathbf{a}\| = 0 \iff \mathbf{a} = \mathbf{0}\)   (the magnitude of a vector \(\mathbf{v}\) is zero if and only if the vector is the zero vector)

\item {} 
\sphinxAtStartPar
\(\|k\mathbf{a}\| = |k| \| \mathbf{a} \|\) (the absolute value of \(k\) times the magnitude of \(\mathbf{a}\))

\item {} 
\sphinxAtStartPar
\(\| \mathbf{a} + \mathbf{b} \| \leq \|\mathbf{a}\| + \| \mathbf{b} \|\)   (this property is called the triangle inequality)

\end{itemize}
\end{sphinxadmonition}


\bigskip\hrule\bigskip


\index{Vector@\spxentry{Vector}!unit vector@\spxentry{unit vector}}\ignorespaces 

\subsection{Unit vectors}
\label{\detokenize{_pages/3.2_Vector_magnitude:unit-vectors}}\label{\detokenize{_pages/3.2_Vector_magnitude:index-1}}\label{_pages/3.2_Vector_magnitude:unit-vector-definition}
\begin{sphinxadmonition}{note}{Definition 3.5.2 (Unit vectors)}



\sphinxAtStartPar
A \sphinxstylestrong{unit vector} is a vector with a magnitude of 1.
\end{sphinxadmonition}

\sphinxAtStartPar
For every non\sphinxhyphen{}zero vector \(\mathbf{a}\) there exists a unique unit vector which points in the same direction as \(\mathbf{a}\) but whose magnitude is 1.

\index{Vector@\spxentry{Vector}!normalising@\spxentry{normalising}}\ignorespaces \label{_pages/3.2_Vector_magnitude:normalising-a-vector-proposition}
\begin{sphinxadmonition}{note}{Theorem 3.5.2 (Normalising a vector)}



\sphinxAtStartPar
Any non\sphinxhyphen{}zero vector can be scaled to transform it into a unit vector by dividing each of its entries by the magnitude of the vector:
\begin{equation}\label{equation:_pages/3.2_Vector_magnitude:normalising-a-vector-equation}
\begin{split} \hat{\mathbf{a}} = \frac{\mathbf{a}}{\|\mathbf{a}\|}. \end{split}
\end{equation}
\sphinxAtStartPar
This process is called \sphinxstylestrong{normalising a vector}. Unit vectors are denoted with a caret above the vector name, i.e., \(\hat{\mathbf{a}}\) which is read as \sphinxstyleemphasis{‘a hat’}.
\end{sphinxadmonition}

\begin{sphinxadmonition}{note}
\sphinxAtStartPar
Proof. Let \(\mathbf{a}\) be a non\sphinxhyphen{}zero vector
\begin{equation*}
\begin{split} \begin{align*}
    \left\|\frac{\mathbf{a}}{\|\mathbf{a}\|}\right\| = \left\|\frac{1}{\|\mathbf{a}\|}\mathbf{a}\right\| &= \left\|\frac{1}{\|\mathbf{a}\|}\right\| \|\mathbf{a}\| \\
    &= \frac{1}{\|\mathbf{a}\|} \|\mathbf{a}\| \qquad \text{(since $\|\mathbf{a}\|>0$)}\\
    &= 1.
\end{align*} \end{split}
\end{equation*}\end{sphinxadmonition}


\label{_pages/3.2_Vector_magnitude:normalising-a-vector-example}
\begin{sphinxadmonition}{note}{Example 3.5.2}



\sphinxAtStartPar
Find the unit vectors parallel to the following vectors:

\sphinxAtStartPar
(i)   \(\mathbf{u} = (1, 2, 3)\);  
(ii)   \((5, -12, 0)\);  
(iii)   \(\mathbf{w} = (1, 0, 1)\)
\subsubsection*{Solution}

\sphinxAtStartPar
(i)
\begin{equation*}
\begin{split} \hat{\mathbf{u}} = \dfrac{\mathbf{u}}{\|\mathbf{u}\|} = \dfrac{(1, 2, 3)}{\sqrt{14}} = \left( \frac{1}{\sqrt{14}}, \frac{2}{\sqrt{14}}, \frac{3}{\sqrt{14}}\right)\end{split}
\end{equation*}\begin{equation*}
\begin{split}= \left( \frac{\sqrt{14}}{14}, \frac{\sqrt{14}}{7}, \frac{3\sqrt{14}}{14} \right)\end{split}
\end{equation*}
\sphinxAtStartPar
Check magnitude of \(\hat{\mathbf{u}}\)
\begin{equation*}
\begin{split} \begin{align*}
    \|\hat{\mathbf{u}}\| &= \sqrt{ \left( \frac{\sqrt{14}}{14} \right)^2 + \left( \frac{\sqrt{14}}{7} \right)^2 + \left(\frac{3\sqrt{14}}{14} \right)^2} \end{split}
\end{equation*}\begin{equation*}
\begin{split}= \sqrt{ \frac{14}{196} +\frac{14}{49} + \frac{126}{196}} = \sqrt{1} = 1 \qquad \checkmark
\end{align*} \end{split}
\end{equation*}
\sphinxAtStartPar
(ii)
\begin{equation*}
\begin{split} \hat{\mathbf{v}} = \dfrac{\mathbf{v}}{\|\mathbf{v}\|} = \dfrac{(5, -12, 0)}{13} = \left( \frac{5}{13}, -\frac{12}{13}, 0 \right) \end{split}
\end{equation*}
\sphinxAtStartPar
Check magnitude of \(\hat{\mathbf{v}}\)
\begin{equation*}
\begin{split} \begin{align*}
    \|\hat{\mathbf{v}}\| &= \sqrt{ \left(\frac{5}{13} \right)^2 + \left( -\frac{12}{13} \right)^2 + 0^2} 
    = \sqrt{ \frac{25}{169} + \frac{144}{169} + 0} \end{split}
\end{equation*}
\begin{sphinxVerbatim}[commandchars=\\\{\}]
\PYGZdl{}\PYGZdl{}= \PYGZbs{}sqrt\PYGZob{}1\PYGZcb{} = 1 \PYGZbs{}qquad \PYGZbs{}checkmark
\end{sphinxVerbatim}

\sphinxAtStartPar
\textbackslash{}end\{align*\} \$\$

\sphinxAtStartPar
(iii)
\begin{equation*}
\begin{split} \hat{\mathbf{w}} = \dfrac{\mathbf{w}}{\|\mathbf{w}\|} = \dfrac{(1, 0, 1)}{\sqrt{2}} = \left( \frac{1}{\sqrt{2}}, 0, \frac{1}{\sqrt{2}} \right) = \left( \frac{\sqrt{2}}{2}, 0, \frac{\sqrt{2}}{2} \right)\end{split}
\end{equation*}
\sphinxAtStartPar
Check magnitude of \(\hat{\mathbf{w}}\)
\begin{equation*}
\begin{split} \begin{align*}
    \|\hat{\mathbf{w}}| &= \sqrt{ \left( \frac{\sqrt{2}}{2} \right)^2 + 0^2 + \left( \frac{\sqrt{2}}{2} \right)^2 } 
    = \sqrt{ \frac{2}{4} + 0 + \frac{2}{4} }\end{split}
\end{equation*}
\begin{sphinxVerbatim}[commandchars=\\\{\}]
\PYGZdl{}\PYGZdl{} = \PYGZbs{}sqrt\PYGZob{}1\PYGZcb{} = 1 \PYGZbs{}qquad \PYGZbs{}checkmark
\end{sphinxVerbatim}

\sphinxAtStartPar
\textbackslash{}end\{align*\} \$\$
\end{sphinxadmonition}

\sphinxstepscope


\section{Dot and cross products}
\label{\detokenize{_pages/3.3_Dot_and_cross_products:dot-and-cross-products}}\label{\detokenize{_pages/3.3_Dot_and_cross_products::doc}}
\sphinxAtStartPar
There are two ways in which we calculate the product of two vectors, these are known as the dot product and the cross product.


\bigskip\hrule\bigskip


\index{Vector@\spxentry{Vector}!dot product@\spxentry{dot product}}\ignorespaces 
\index{Dot product@\spxentry{Dot product}}\ignorespaces 

\subsection{Dot product}
\label{\detokenize{_pages/3.3_Dot_and_cross_products:dot-product}}\label{\detokenize{_pages/3.3_Dot_and_cross_products:index-1}}\label{\detokenize{_pages/3.3_Dot_and_cross_products:index-0}}\label{\detokenize{_pages/3.3_Dot_and_cross_products:dot-product-section}}\label{_pages/3.3_Dot_and_cross_products:dot-product-definition}
\begin{sphinxadmonition}{note}{Definition 3.6.1 (The dot product)}



\sphinxAtStartPar
The \sphinxstylestrong{dot product} (also known as the \sphinxstylestrong{scalar product}) of two vectors \(\mathbf{a}\) and \(\mathbf{b}\) in \(\mathbb{R}^n\), is defined as
\begin{equation}\label{equation:_pages/3.3_Dot_and_cross_products:arithmetic-dot-product-equation}
\begin{split} \mathbf{a} \cdot \mathbf{b} = \sum_{i=1}^n a_ib_i = a_1b_1 + a_2b_2 + \cdots + a_nb_n. \end{split}
\end{equation}
\sphinxAtStartPar
The \sphinxstylestrong{dot product} of two vectors \(\mathbf{a}\) and \(\mathbf{b}\) can be defined in geometric terms as
\begin{equation}\label{equation:_pages/3.3_Dot_and_cross_products:geometric-dot-product-equation}
\begin{split} \mathbf{a} \cdot \mathbf{b} = \|\mathbf{a}\| \|\mathbf{b}\| \cos(\theta), \end{split}
\end{equation}
\sphinxAtStartPar
where \(\theta\) is the angle between \(\mathbf{a}\) and \(\mathbf{b}\) (\hyperref[\detokenize{_pages/3.3_Dot_and_cross_products:dot-product-figure}]{Fig.\@ \ref{\detokenize{_pages/3.3_Dot_and_cross_products:dot-product-figure}}}).

\begin{figure}[htbp]
\centering
\capstart

\noindent\sphinxincludegraphics[width=175\sphinxpxdimen]{{3_dot_product}.svg}
\caption{The dot product \(\mathbf{a} \cdot \mathbf{b}\)}\label{\detokenize{_pages/3.3_Dot_and_cross_products:dot-product-figure}}\end{figure}
\end{sphinxadmonition}

\sphinxAtStartPar
Roughly, we can think of the dot product \(\mathbf{a} \cdot \mathbf{b}\) as a scalar that describes the angle between \(\mathbf{a}\) and \(\mathbf{b}\). If two vectors point roughly in the same direction, we would expect their dot product to be larger than the dot product of two vectors pointing in completely different directions.

\sphinxAtStartPar
From the {\hyperref[\detokenize{_pages/3.3_Dot_and_cross_products:dot-product-definition}]{\sphinxcrossref{definition of the dot product}}}, we can say:
\begin{itemize}
\item {} 
\sphinxAtStartPar
The dot product produces a scalar quantity (an element of \(\mathbb{R}\))

\item {} 
\sphinxAtStartPar
The dot product of two perpendicular vectors is zero, and we write \(\mathbf{a}\perp\mathbf{b}\) if \(\mathbf{a}\cdot\mathbf{b}=0\)

\item {} 
\sphinxAtStartPar
The dot product of two co\sphinxhyphen{}directional (parallel) vectors \(\mathbf{a}\) and \(\mathbf{b}\) is equal to \(\|\mathbf{a}\| \|\mathbf{b}\|\)

\end{itemize}
\label{_pages/3.3_Dot_and_cross_products:dot-product-properties-theorem}
\begin{sphinxadmonition}{note}{Theorem 3.6.1 (Properties of the dot product)}



\sphinxAtStartPar
For any three vectors \(\mathbf{a},\mathbf{b},\mathbf{c}\) in \(\mathbb{R}^n\), and any scalar \(k\) in \(\mathbb{R}\), the following properties of the dot product hold:
\begin{itemize}
\item {} 
\sphinxAtStartPar
\((\mathbf{a} + \mathbf{b})\cdot \mathbf{c} = \mathbf{a} \cdot \mathbf{c} + \mathbf{b} \cdot \mathbf{c}\)   (distributive over vector addition)

\item {} 
\sphinxAtStartPar
\((k \mathbf{a})\cdot \mathbf{b} = \mathbf{a} \cdot (k \mathbf{b})= k (\mathbf{a} \cdot \mathbf{b})\)   (associative over scalar multiplication)

\item {} 
\sphinxAtStartPar
\(\mathbf{a} \cdot \mathbf{b} = \mathbf{b} \cdot \mathbf{a}\)   (commutative)

\item {} 
\sphinxAtStartPar
\(\mathbf{a} \cdot \mathbf{a} = \|\mathbf{a}\|^2 \geq 0 \text{ and } \mathbf{a} \cdot \mathbf{a} = 0 \iff \mathbf{a} = \mathbf{0}\)

\end{itemize}
\end{sphinxadmonition}


\label{_pages/3.3_Dot_and_cross_products:dot-product-example}
\begin{sphinxadmonition}{note}{Example 3.6.1}



\sphinxAtStartPar
Given the vectors \(\mathbf{a} = (1, 2, 3)\) and \(\mathbf{b} = (3, -1, 0)\). Calculate:

\sphinxAtStartPar
(i)   \(\mathbf{a} \cdot \mathbf{b}\);  
(ii)   the angle between \(\mathbf{a}\) and \(\mathbf{b}\)
\subsubsection*{Solution}

\sphinxAtStartPar
(i)   Using equation \eqref{equation:_pages/3.3_Dot_and_cross_products:arithmetic-dot-product-equation}
\begin{equation*}
\begin{split} \begin{align*}
    \mathbf{a} \cdot \mathbf{b} = \begin{pmatrix} 1 \\ 2 \\ 3 \end{pmatrix} \cdot \begin{pmatrix} 3 \\ -1 \\ 0 \end{pmatrix} = 1(3) + 2(-1) + 3(0) = 3 - 2 + 0 = 1
\end{align*} \end{split}
\end{equation*}
\sphinxAtStartPar
(ii)   Using equation \eqref{equation:_pages/3.3_Dot_and_cross_products:geometric-dot-product-equation}
\begin{equation*}
\begin{split} \begin{align*}
    \mathbf{a} \cdot \mathbf{b} &= \|\mathbf{a}\| \|\mathbf{b}\| \cos(\theta) \\
    \therefore \theta &= \cos^{-1} \left( \frac{\mathbf{a} \cdot \mathbf{b}}{\|\mathbf{a}\|\|\mathbf{b}\|}\right) \\
    &= \cos^{-1} \left( \frac{1}{\sqrt{14}\sqrt{10}}\right) \approx 1.4862.
\end{align*} \end{split}
\end{equation*}\end{sphinxadmonition}

\sphinxAtStartPar
Note: we will express angles in \sphinxhref{https://en.wikipedia.org/wiki/Radian}{radians} rather than degrees. You may be used to angles given in degrees, but you will see radians used much more in university\sphinxhyphen{}level maths.


\subsubsection{Dot product in matrix multiplication}
\label{\detokenize{_pages/3.3_Dot_and_cross_products:dot-product-in-matrix-multiplication}}
\sphinxAtStartPar
We have already seen that the {\hyperref[\detokenize{_pages/1.2_Matrix_multiplication:matrix-multiplication-definition}]{\sphinxcrossref{multiplication of two matrices}}} is defined by
\begin{equation*}
\begin{split} [AB]_{ij} = \sum_{k = 1}^n a_{ik}b_{kj}. \end{split}
\end{equation*}
\sphinxAtStartPar
You may have noticed that the term in this summation is similar in structure to the dot product. In fact, when we multiply two matrices together, we are calculating the dot product of the rows of the left\sphinxhyphen{}hand matrix and the columns of the right\sphinxhyphen{}hand matrix. We have:
\begin{equation*}
\begin{split} [AB]_{ij} = \mathbf{a}_i \cdot \mathbf{b}_j, \end{split}
\end{equation*}
\sphinxAtStartPar
where \(\mathbf{a}_i\) is row \(i\) of \(A\) and \(\mathbf{b}_j\) is column \(j\) of \(B\).


\bigskip\hrule\bigskip


\index{Vector@\spxentry{Vector}!cross product@\spxentry{cross product}}\ignorespaces 
\index{Cross product@\spxentry{Cross product}}\ignorespaces 

\subsection{The cross product}
\label{\detokenize{_pages/3.3_Dot_and_cross_products:the-cross-product}}\label{\detokenize{_pages/3.3_Dot_and_cross_products:index-3}}\label{\detokenize{_pages/3.3_Dot_and_cross_products:index-2}}\label{\detokenize{_pages/3.3_Dot_and_cross_products:cross-product-section}}\label{_pages/3.3_Dot_and_cross_products:cross-product-definition}
\begin{sphinxadmonition}{note}{Definition 3.6.2 (The cross product)}



\sphinxAtStartPar
The \sphinxstylestrong{cross product} (also known as the \sphinxstylestrong{vector product}) of two vectors in \(\mathbb{R}^3\), \(\mathbf{a}=(a_1,a_2,a_3)\) and \(\mathbf{b}=(b_1,b_2,b_3)\) can be calculated as the {\hyperref[\detokenize{_pages/1.4_Determinants:determinant-section}]{\sphinxcrossref{\DUrole{std,std-ref}{determinant}}}} of a matrix constructed using the elements of \(\mathbf{a}\) and \(\mathbf{b}\), and a set of vectors \(\mathbf{i}\), \(\mathbf{j}\) and \(\mathbf{k}\):
\begin{equation}\label{equation:_pages/3.3_Dot_and_cross_products:cross-product-equation}
\begin{split}\mathbf{a} \times \mathbf{b} = \det 
\begin{pmatrix}
    \mathbf{i} & \mathbf{j} & \mathbf{k} \\
    a_1 & a_2 & a_3 \\
    b_1 & b_2 & b_3
\end{pmatrix}. \end{split}
\end{equation}
\sphinxAtStartPar
and \(\mathbf{i} = (1, 0, 0)\), \(\mathbf{j} = (0, 1, 0)\) and \(\mathbf{k} = (0, 0, 1)\). The cross product between two vectors \(\mathbf{a}\) and \(\mathbf{b}\) returns a vector that is perpendicular to both \(\mathbf{a}\) and \(\mathbf{b}\).

\begin{figure}[htbp]
\centering
\capstart

\noindent\sphinxincludegraphics[width=175\sphinxpxdimen]{{3_cross_product}.svg}
\caption{The cross product \(\mathbf{a} \times \mathbf{b}\) is perpendicular to both \(\mathbf{a}\) and \(\mathbf{b}\).}\label{\detokenize{_pages/3.3_Dot_and_cross_products:cross-product-figure}}\end{figure}
\end{sphinxadmonition}

\sphinxAtStartPar
In order to calculate this product, we can leave the vectors \(\mathbf{i}\), \(\mathbf{j}\) and \(\mathbf{k}\) in vector form, and expand along the top row of the matrix to calculate the determinant. Once we have obtained an excpression for the determinant in terms of \(\mathbf{i}\), \(\mathbf{j}\) and \(\mathbf{k}\), we can use it to calculate the final value.

\sphinxAtStartPar
Roughly, the cross product returns a vector perpendicular to the two vectors being multiplied together, if they were both starting at the same point. The length of this vector depends on the angle between the two vectors you are taking the product of, and can be calculated as \(\|\mathbf{a}\| \|\mathbf{b}\| \sin(\theta)\).

\sphinxAtStartPar
The cross product is only defined for vectors in three\sphinxhyphen{}dimensional space.
\label{_pages/3.3_Dot_and_cross_products:cross-product-properties-theorem}
\begin{sphinxadmonition}{note}{Theorem 3.6.2 (Properties of the cross product)}



\sphinxAtStartPar
Given three vectors \(\mathbf{a}, \mathbf{b}, \mathbf{c} \in \mathbb{R}^3\) the following properties are satisfied:
\begin{itemize}
\item {} 
\sphinxAtStartPar
\(\mathbf{a}\times \mathbf{b}\) produces a vector in the same space as \(\mathbf{a}\) and \(\mathbf{b}\)

\item {} 
\sphinxAtStartPar
\(\mathbf{a}\times \mathbf{b}\) produces a vector that is perpendicular to both \(\mathbf{a}\) and \(\mathbf{b}\)

\item {} 
\sphinxAtStartPar
\(\mathbf{a} \times \mathbf{b} = -(\mathbf{b} \times \mathbf{a})\)  

\item {} 
\sphinxAtStartPar
\(\mathbf{a}\times \mathbf{b} = \mathbf{0} \implies \mathbf{a} \parallel \mathbf{b}\)   (if the cross product of two vectors is the zero vector then the two vectors are parallel)

\item {} 
\sphinxAtStartPar
\(\mathbf{a} \times \mathbf{b} = \mathbf{0}\)

\item {} 
\sphinxAtStartPar
\(\mathbf{a} \times \mathbf{b} \neq \mathbf{b} \times \mathbf{a}\)   (non\sphinxhyphen{}commutative)

\item {} 
\sphinxAtStartPar
\(\mathbf{a} \times (\mathbf{b} + \mathbf{c}) = (\mathbf{a} \times \mathbf{b}) + (\mathbf{a} \times \mathbf{c})\)   (distributive over vector addition)

\item {} 
\sphinxAtStartPar
\(\mathbf{a} \times (\mathbf{b} \times \mathbf{c}) \neq (\mathbf{a} \times \mathbf{b}) \times \mathbf{c}\)   (non\sphinxhyphen{}associative)

\end{itemize}
\end{sphinxadmonition}


\label{_pages/3.3_Dot_and_cross_products:cross-product-example}
\begin{sphinxadmonition}{note}{Example 3.6.2}



\sphinxAtStartPar
Calculate the cross product between the vectors \(\mathbf{a} = (1, 2, 3)\) and \(\mathbf{b} = (4, 5, 6)\) and show that it is perpendicular to both \(\mathbf{a}\) and \(\mathbf{b}\).


\bigskip\hrule\bigskip

\subsubsection*{Solution}
\begin{equation*}
\begin{split} \begin{align*}
    \mathbf{a} \times \mathbf{b} &= \begin{vmatrix} \mathbf{i} & \mathbf{j} & \mathbf{k} \\ 1 & 2 & 3 \\ 4 & 5 & 6 \end{vmatrix}
    = (12 - 15)\mathbf{i} - (6 - 12) \mathbf{j} + (5 - 8) \mathbf{k} \\
    &= -3\mathbf{i} + 6 \mathbf{j} - 3\mathbf{k}
    = (-3, 6, -3).
\end{align*} \end{split}
\end{equation*}
\sphinxAtStartPar
To show that \(\mathbf{a} \times \mathbf{b} \perp \mathbf{a}, \mathbf{b}\) we check whether the dot product between \((3, 6, -3)\) and the vectors \(\mathbf{a}\) and \(\mathbf{b}\) is zero
\begin{equation*}
\begin{split} \begin{align*}
    \mathbf{a} \cdot (\mathbf{a} \times \mathbf{b}) &= 
    \begin{pmatrix} 1 \\ 2 \\ 3 \end{pmatrix} \cdot
    \begin{pmatrix} -3 \\ 6 \\ -3 \end{pmatrix} 
    = -3 + 12 - 9 = 0, \\
    \mathbf{b} \cdot (\mathbf{a} \times \mathbf{b}) &=
    \begin{pmatrix} 4 \\ 5 \\ 6 \end{pmatrix} \cdot
    \begin{pmatrix} -3 \\ 6 \\ -3 \end{pmatrix}
    = -12 + 30 - 18 = 0,
\end{align*} \end{split}
\end{equation*}
\sphinxAtStartPar
therefore \(\mathbf{a} \times \mathbf{b}\) is perpendicular to both \(\mathbf{a}\) and \(\mathbf{b}\).
\end{sphinxadmonition}

\sphinxstepscope


\section{Linear combination of vectors}
\label{\detokenize{_pages/3.4_Linear_combinations:linear-combination-of-vectors}}\label{\detokenize{_pages/3.4_Linear_combinations:linear-combination-of-vectors-section}}\label{\detokenize{_pages/3.4_Linear_combinations::doc}}
\index{Vector@\spxentry{Vector}!linear combination@\spxentry{linear combination}}\ignorespaces \label{_pages/3.4_Linear_combinations:linear-combination-of-vectors-definition}
\begin{sphinxadmonition}{note}{Definition 3.7.1 (Linear combination of vectors)}



\sphinxAtStartPar
Let \(\mathbf{v},\mathbf{u}_1,\dots,\mathbf{u}_m\in\mathbb{R}^n\) be \(n\)\sphinxhyphen{}dimensional vectors such that:
\begin{equation}\label{equation:_pages/3.4_Linear_combinations:linear-combination-equation}
\begin{split} \mathbf{v}=\alpha_1\mathbf{u}_1+\alpha_2\mathbf{u}_2+\dots+\alpha_m\mathbf{u}_m, \end{split}
\end{equation}
\sphinxAtStartPar
for some \(\alpha_1,\alpha_2,\dots,\alpha_m\in \mathbb{R}\) %
\begin{footnote}[1]\sphinxAtStartFootnote
\(\alpha\) is the lowercase Greek character \sphinxstyleemphasis{alpha} and is equivalent to ‘a’ in the Latin alphabet.
%
\end{footnote}. Such a sum is called a \sphinxstylestrong{linear combination of vectors}.
\end{sphinxadmonition}

\sphinxAtStartPar
For example:
\begin{equation*}
\begin{split} \begin{align*}
    \begin{pmatrix} 2 \\ 0 \\ 7 \end{pmatrix} =
    2\begin{pmatrix} 1 \\ 5 \\ -1 \end{pmatrix} +
    \begin{pmatrix} 0 \\ -10 \\ 9 \end{pmatrix},
\end{align*} \end{split}
\end{equation*}
\sphinxAtStartPar
Here we have expressed \((2,0,7)\) as a linear combination of the vectors \((1,5,-1)\) and \((0,-10,9)\). To write a vector \(\mathbf{v}\) as a linear combination of vectors \(\mathbf{u}_1, \mathbf{u}_2, \ldots, \mathbf{u}_m\) we need to calculate the values of the coefficients \(\alpha_1, \alpha_2, \ldots, \alpha_m\) from equation \eqref{equation:_pages/3.4_Linear_combinations:linear-combination-equation}.

\sphinxAtStartPar
To do this we need to solve the system of linear equations:
\begin{equation*}
\begin{split} \mathbf{u}_1 \alpha_1 + \mathbf{u}_2 \alpha_2 + \cdots + \mathbf{u}_m \alpha_m = \mathbf{v}\end{split}
\end{equation*}
\sphinxAtStartPar
which can be written as the augmented matrix made by stacking these vectors together:
\begin{equation*}
\begin{split} \left( \begin{array}{cccc|c}
    \uparrow & \uparrow & & \uparrow & \uparrow \\
    \mathbf{u}_1 & \mathbf{u}_2 & \cdots & \mathbf{u}_m & \mathbf{v} \\
    \downarrow & \downarrow & & \downarrow & \downarrow
\end{array} \right)\end{split}
\end{equation*}
\sphinxAtStartPar
and the values of the coefficients \(\alpha_1, \alpha_2, \ldots, \alpha_m\) can be calculated using {\hyperref[\detokenize{_pages/2.3_Gaussian_elimination:gaussian-elimination-section}]{\sphinxcrossref{\DUrole{std,std-ref}{Gaussian elimination}}}}.
\label{_pages/3.4_Linear_combinations:linear-combination-of-vectors-example}
\begin{sphinxadmonition}{note}{Example 3.7.1}



\sphinxAtStartPar
Express the vector \(\mathbf{v} = (7, -2, -11)\) as a linear combination of the vectors
\begin{equation*}
\begin{split} \begin{align*}
    \mathbf{u}_1 &= \begin{pmatrix} 1 \\ 0 \\ 7 \end{pmatrix}, &
    \mathbf{u}_2 &= \begin{pmatrix} 2 \\ -1 \\ 3 \end{pmatrix}, &
    \mathbf{u}_3 &= \begin{pmatrix} 5 \\ -2 \\ -6 \end{pmatrix}.
\end{align*} \end{split}
\end{equation*}\subsubsection*{Solution}

\sphinxAtStartPar
We need to find the values of the coefficients \(\alpha_1\), \(\alpha_2\) and \(\alpha_3\) in the following linear combination
\begin{equation*}
\begin{split} \begin{align*}
    \alpha_1 \begin{pmatrix} 1 \\ 0 \\ 7 \end{pmatrix}
    + \alpha_2 \begin{pmatrix} 2 \\ -1 \\ 3 \end{pmatrix}
    + \alpha_3 \begin{pmatrix} 5 \\ -2 \\ -6 \end{pmatrix}
    &= \begin{pmatrix} 7 \\ -2 \\ -11 \end{pmatrix},
\end{align*} \end{split}
\end{equation*}
\sphinxAtStartPar
so we need the solution to the linear system
\begin{equation*}
\begin{split} \begin{align*}
    \alpha_1 + 2\alpha_2 + 5\alpha_3 &= 7, \\
    -\alpha_2 - 2\alpha_3 &= -2, \\
    7\alpha_1 + 3\alpha_2 - 6\alpha_3 &= -11.
\end{align*} \end{split}
\end{equation*}
\sphinxAtStartPar
Using Gauss\sphinxhyphen{}Jordan elimination
\begin{equation*}
\begin{split} \begin{align*}
    &\left( \begin{array}{ccc|c}
        1 & 2 & 5 & 7 \\
        0 & -1 & -2 & -2 \\
        7 & 3 & -6 & -11
    \end{array} \right)
    \\ \\
     R_3 - 7R_1 \longrightarrow &
    \left( \begin{array}{ccc|c}
        1 & 2 & 5 & 7 \\
        0 & -1 & -2 & -2 \\
        0 & -11 & -41 & -60
    \end{array} \right)
    \\ \\
    -R_2\longrightarrow &
    \left( \begin{array}{ccc|c}
        1 & 2 & 5 & 7 \\
        0 & 1 & 2 & 2 \\
        0 & -11 & -41 & -60
    \end{array} \right)
 	&
    \begin{array}{l} R_1 - 2R2 \\ R_3 + 11R_2 \end{array}\longrightarrow &
    \left( \begin{array}{ccc|c}
        1 & 0 & 1 & 3 \\
        0 & 1 & 2 & 2 \\
        0 & 0 & -19 & -38
    \end{array} \right)
    \\ \\
    -\frac{1}{19} R_3 \longrightarrow &
    \left( \begin{array}{ccc|c}
        1 & 0 & 1 & 3 \\
        0 & 1 & 2 & 2 \\
        0 & 0 & 1 & 2
    \end{array} \right)
    \\ \\
    \begin{array}{l} R_1 - R_3 \\ R_2 - 2R_3 \end{array} \longrightarrow &
    \left( \begin{array}{ccc|c}
        1 & 0 & 0 & 1 \\
        0 & 1 & 0 & -2 \\
        0 & 0 & 1 & 2
    \end{array} \right),
\end{align*} \end{split}
\end{equation*}
\sphinxAtStartPar
so \(\alpha_1 = 1\), \(\alpha_2 = -2\) and \(\alpha_3 = 2\). Therefore \(\mathbf{v}\) can be expressed as the linear combination of vectors
\begin{equation*}
\begin{split} \begin{align*}
    \mathbf{v} = \mathbf{u}_1 - 2\mathbf{u}_2 + 2\mathbf{u}_3
    =  \begin{pmatrix} 1 \\ 0 \\ 7 \end{pmatrix} 
    - 2 \begin{pmatrix} 2 \\ -1 \\ 3 \end{pmatrix} 
    + 2 \begin{pmatrix} 5 \\ -2 \\ -6 \end{pmatrix}
    &= \begin{pmatrix} 7 \\ -2 \\ -11 \end{pmatrix}
\end{align*} \end{split}
\end{equation*}\end{sphinxadmonition}




\bigskip\hrule\bigskip


\index{Basis vectors@\spxentry{Basis vectors}}\ignorespaces 

\subsection{Basis vectors}
\label{\detokenize{_pages/3.4_Linear_combinations:basis-vectors}}\label{\detokenize{_pages/3.4_Linear_combinations:index-1}}\label{\detokenize{_pages/3.4_Linear_combinations:basis-vectors-section}}
\sphinxAtStartPar
A \sphinxstylestrong{basis vector} is a special type of vector, and a set of basis vectors form a \sphinxstylestrong{basis} for a vector space. All other vectors in the space can be represented as a linear combination of the basis vectors (we will cover basis in more detail {\hyperref[\detokenize{_pages/5.4_Basis:basis-section}]{\sphinxcrossref{\DUrole{std,std-ref}{later}}}}). In Cartesian space, the simplest basis vectors are unit vectors that point in the coordinate directions. In \(\mathbb{R}^3\) we use the basis vectors \(\mathbf{i} = (1, 0, 0)\), \(\mathbf{j} = (0, 1, 0)\) and \(\mathbf{k} = (0, 0, 1)\) (\hyperref[\detokenize{_pages/3.4_Linear_combinations:basis-vectors-figure}]{Fig.\@ \ref{\detokenize{_pages/3.4_Linear_combinations:basis-vectors-figure}}}).

\begin{figure}[htbp]
\centering
\capstart

\noindent\sphinxincludegraphics[width=225\sphinxpxdimen]{{3_basis_vectors}.svg}
\caption{The three basis vectors in \(\mathbb{R}^3\).}\label{\detokenize{_pages/3.4_Linear_combinations:basis-vectors-figure}}\end{figure}

\sphinxAtStartPar
Using basis vectors we can represent any vector, \(\mathbf{a} = (a_1, a_2, a_3)\) as a linear combination of \(\mathbf{i}\), \(\mathbf{j}\) and \(\mathbf{k}\):
\begin{equation*}
\begin{split} \begin{align*}a_1 \mathbf{i} + a_2 \mathbf{j} + a_3 \mathbf{k}
    = a_1 \begin{pmatrix} 1 \\ 0 \\ 0 \end{pmatrix} + a_2 \begin{pmatrix} 0 \\ 1 \\ 0 \end{pmatrix} + a_3
    \begin{pmatrix} 0 \\ 0 \\ 1 \end{pmatrix}
    = \begin{pmatrix} a_1 \\ a_2 \\ a_3 \end{pmatrix} = \mathbf{a}.
\end{align*} \end{split}
\end{equation*}
\sphinxAtStartPar
For example
\begin{equation*}
\begin{split} (2, 5, 3) = 2 \mathbf{i} + 5 \mathbf{j} + 3 \mathbf{k}. \end{split}
\end{equation*}

\bigskip\hrule\bigskip


\sphinxstepscope


\section{Vectors Exercises}
\label{\detokenize{_pages/3.5_Vectors_exercises:vectors-exercises}}\label{\detokenize{_pages/3.5_Vectors_exercises::doc}}
\sphinxAtStartPar
Answer the following exercises based on the content from this chapter. The solutions can be found in the {\hyperref[\detokenize{_pages/A3_Vectors_exercises_solutions:vectors-exercises-solutions-section}]{\sphinxcrossref{\DUrole{std,std-ref}{appendices}}}}.
\phantomsection \label{exercise:vectors-ex-arithmetic}

\begin{sphinxadmonition}{note}{Exercise 3.8.1}



\sphinxAtStartPar
The points \(U\), \(V\) and \(W\) have the position vectors \(\mathbf{u} = (2, 3)\), \(\mathbf{v} = (3, -2)\) and \(\mathbf{w} = (1, 6)\).

\sphinxAtStartPar
Find:

\sphinxAtStartPar
(a)   \(2 \mathbf{u} + \mathbf{w}\)

\sphinxAtStartPar
(b)   \(\mathbf{w} - \mathbf{u}\)

\sphinxAtStartPar
(c)   a unit vector pointing in the same direction of \(\mathbf{u}\)

\sphinxAtStartPar
(d)   a unit vector pointing in the opposite direction of \(\mathbf{v}\)

\sphinxAtStartPar
(e)   a vector pointing in the same direction as \(\mathbf{v}\) but half its length

\sphinxAtStartPar
(f)   the vector pointing from \(U\) to \(V\)

\sphinxAtStartPar
(g)   the vector pointing from \(U\) to \(W\)

\sphinxAtStartPar
(h)   \(\mathbf{u} \cdot \mathbf{w}\)

\sphinxAtStartPar
(i)   the angle \(\angle VUW\)

\sphinxAtStartPar
(j)   show that \(\mathbf{u}\) is at right angles to \(\mathbf{v}\)

\sphinxAtStartPar
(k)   \(\mathbf{v} \times \mathbf{w}\)
\end{sphinxadmonition}
\phantomsection \label{exercise:vectors-ex-linear-combination}

\begin{sphinxadmonition}{note}{Exercise 3.8.2}



\sphinxAtStartPar
Write \(\mathbf{u} = (2,7,1)\) as:

\sphinxAtStartPar
(a)   a linear combination of \(\mathbf{i}\), \(\mathbf{j}\) and \(\mathbf{k}\)

\sphinxAtStartPar
(b)   a linear combination of vectors \(\mathbf{f}_1 = (1, -1, 0), \mathbf{f}_2 = (0, 2, 0)\) and \(\mathbf{f}_3 = (1, 0, -1)\)
\end{sphinxadmonition}
\phantomsection \label{exercise:vectors-ex-perpendicular}

\begin{sphinxadmonition}{note}{Exercise 3.8.3}



\sphinxAtStartPar
Find \(k\) such that the vectors \(\mathbf{u}\) and \(\mathbf{v}\) are perpendicular:

\sphinxAtStartPar
(a)   \(\mathbf{u} =(1, k, -2)\) and \(\mathbf{v} = (2, -5, 4)\) in \(\mathbb{R}^3\)

\sphinxAtStartPar
(b)   \(\mathbf{u} = (1, 0, k+2, -1, 2)\) and \(\mathbf{v} = (1, k, -2, 1, 2)\) in \(\mathbb{R}^5\)
\end{sphinxadmonition}
\phantomsection \label{exercise:vectors-ex-angle}

\begin{sphinxadmonition}{note}{Exercise 3.8.4}



\sphinxAtStartPar
Which pair of the following vectors is perpendicular? For the remaining pairs, what is the angle between them?
\begin{equation*}
\begin{split} \begin{align*}
    \mathbf{u} &= \begin{pmatrix} 1 \\ 2 \\ 3 \end{pmatrix}, &
    \mathbf{v} &= \begin{pmatrix} -1 \\ 2 \\ -1 \end{pmatrix}, &
    \mathbf{w} &= \begin{pmatrix} 2 \\ -3 \\ 1 \end{pmatrix}.
\end{align*} \end{split}
\end{equation*}\end{sphinxadmonition}

\sphinxstepscope


\chapter{Co\sphinxhyphen{}ordinate Geometry}
\label{\detokenize{_pages/4.0_Coordinate_geometry:co-ordinate-geometry}}\label{\detokenize{_pages/4.0_Coordinate_geometry::doc}}
\index{Co\sphinxhyphen{}ordinate geometry@\spxentry{Co\sphinxhyphen{}ordinate geometry}}\ignorespaces 
\begin{figure}[htbp]
\centering
\capstart

\noindent\sphinxincludegraphics[width=200\sphinxpxdimen]{{82882bf2c5dcb23523b22120cf4608119e5e439b}.jpg}
\caption{René Descartes (1596 \sphinxhyphen{} 1650)}\label{\detokenize{_pages/4.0_Coordinate_geometry:id1}}\label{\detokenize{_pages/4.0_Coordinate_geometry:index-0}}\label{\detokenize{_pages/4.0_Coordinate_geometry:co-ordinate-geometry-chapter}}\end{figure}

\sphinxAtStartPar
The area of mathematics dealing with describing geometry in terms of coordinate systems, hence as points and vectors, is called \sphinxstylestrong{coordinate geometry} or \sphinxstylestrong{Cartesian geometry} (named after French mathematician and philosopher \sphinxhref{https://en.wikipedia.org/wiki/Ren\%C3\%A9\_Descartes}{René Descartes}). In this chapter we will lay out the basic terminology, and see ways to describe points, lines and planes in \(n\) dimensional Cartesian coordinates.

\index{Points@\spxentry{Points}}\ignorespaces 

\section{Points}
\label{\detokenize{_pages/4.0_Coordinate_geometry:points}}\label{\detokenize{_pages/4.0_Coordinate_geometry:index-1}}\label{\detokenize{_pages/4.0_Coordinate_geometry:points-section}}
\sphinxAtStartPar
In his book \sphinxhref{https://en.wikipedia.org/wiki/Euclid\%27s\_Elements}{* Elements*}, Euclid described a \sphinxstyleemphasis{point} as \sphinxstyleemphasis{“that which has no part”}. A point has position only \sphinxhyphen{} it has no length, width or thickness, and thus no area or volume. A typical point in \(\mathbb{R}^n\) is described by its co\sphinxhyphen{}ordinates \((a_1,a_2,\dots,a_n)\), where \(a_i \in \mathbb{R}\).

\begin{figure}[htbp]
\centering
\capstart

\noindent\sphinxincludegraphics[width=300\sphinxpxdimen]{{3_R3}.svg}
\caption{The position of a point in \(\mathbb{R}^3\) can be defined by its co\sphinxhyphen{}ordinates \((x, y, z)\).}\label{\detokenize{_pages/4.0_Coordinate_geometry:point-in-r3-figure}}\end{figure}

\sphinxstepscope

\index{Lines@\spxentry{Lines}}\ignorespaces 

\section{Lines}
\label{\detokenize{_pages/4.1_Lines:lines}}\label{\detokenize{_pages/4.1_Lines:index-0}}\label{\detokenize{_pages/4.1_Lines:lines-section}}\label{\detokenize{_pages/4.1_Lines::doc}}
\sphinxAtStartPar
Euclid defined a line as having \sphinxstyleemphasis{“breadth\sphinxhyphen{}less length”}, which is another way of saying that a line is a one\sphinxhyphen{}dimensional object that has a length, but no breadth or volume. Given that we have already taken the time to introduce vectors in \(\mathbb{R}^n\) in the chapter on {\hyperref[\detokenize{_pages/3.0_Vectors:vectors-chapter}]{\sphinxcrossref{\DUrole{std,std-ref}{Vectors}}}}, it makes sense to use them to define a line in \(\mathbb{R}^n\).

\index{Lines@\spxentry{Lines}!vector equation@\spxentry{vector equation}}\ignorespaces \label{_pages/4.1_Lines:vector-equation-of-a-line-definition}
\begin{sphinxadmonition}{note}{Definition 4.2.1 (Vector equation of a line)}



\sphinxAtStartPar
Let \(\mathbf{p}\) be a position vector of a point in \(\mathbb{R}^n\) and \(\mathbf{d}\) a non\sphinxhyphen{}zero vector in \(\mathbb{R}^n\). The line \(\ell\) %
\begin{footnote}[1]\sphinxAtStartFootnote
\(\ell\) is used here instead of \(l\) to avoid confusing with a 1 or uppercase i.
%
\end{footnote} which passes through \(\mathbf{p}\) in the direction of \(\mathbf{d}\) has equation
\begin{equation}\label{equation:_pages/4.1_Lines:line-vector-equation}
\begin{split} \mathbf{r} = \mathbf{p} + t\mathbf{d}, \end{split}
\end{equation}
\sphinxAtStartPar
where \(\mathbf{r}\) is a point on \(\ell\) and \(t \in \mathbb{R}\) is a \sphinxstylestrong{parameter}.
\end{sphinxadmonition}

\sphinxAtStartPar
In other words, every point on \(\ell\) is obtained by adding some scalar multiple of the vector \(\mathbf{d}\) to the point \(\mathbf{p}\) (\hyperref[\detokenize{_pages/4.1_Lines:line-vector-equation-figure}]{Fig.\@ \ref{\detokenize{_pages/4.1_Lines:line-vector-equation-figure}}}). Alternatively, \(\ell\) is the line which passes through \(\mathbf{p}\) in the direction of \(\mathbf{d}\). We call the vector \(\mathbf{d}\) a \sphinxstylestrong{direction vector} of the line.

\begin{figure}[htbp]
\centering
\capstart

\noindent\sphinxincludegraphics[width=500\sphinxpxdimen]{{4_vector_equation_of_a_line}.svg}
\caption{The position of any point on the line \(\ell\) can be obtained by adding a scalar multiple of \(\mathbf{d}\) to \(\mathbf{p}\).}\label{\detokenize{_pages/4.1_Lines:line-vector-equation-figure}}\end{figure}

\sphinxAtStartPar
The equation of the line that passes through the two points with position vectors \(\mathbf{p}_1\) and \(\mathbf{p}_2\) can be determined by using equation \eqref{equation:_pages/4.1_Lines:line-vector-equation} with \(\mathbf{p} = \mathbf{p}_1\) and \(\mathbf{d} = \mathbf{p}_2 - \mathbf{p}_1\). In particular, \(\mathbf{d}\) and \(\mathbf{p}\) are not unique, and we can choose any point on the line for \(\mathbf{p}\), and any non\sphinxhyphen{}zero scalar multiple of \(\mathbf{d}\), to express the same line.

\sphinxAtStartPar
Take for instance the \(x\)\sphinxhyphen{}axis in \(\mathbb{R}^3\). \(\ell\) can be described as:
\begin{equation*}
\begin{split} \mathbf{r} = \begin{pmatrix} 0 \\ 0 \\ 0 \end{pmatrix} + t \begin{pmatrix} 1 \\ 0 \\ 0 \end{pmatrix} = \begin{pmatrix} t \\ 0 \\ 0 \end{pmatrix}, \end{split}
\end{equation*}
\sphinxAtStartPar
which means we are taking \(\mathbf{p} = (0,0,0)\) and \(\mathbf{d}=(1,0,0)\) and for example the point with position vector \(\mathbf{q} = (5,0,0)\) is found when \(t=5\). Alternatively we can describe the \(x\)\sphinxhyphen{}axis as
\begin{equation*}
\begin{split} \mathbf{r} = \begin{pmatrix} 71 \\ 0 \\ 0 \end{pmatrix} + t \begin{pmatrix} 5 \\ 0 \\ 0 \end{pmatrix} = \begin{pmatrix} 71 + 5t \\ 0 \\ 0 \end{pmatrix}, \end{split}
\end{equation*}
\sphinxAtStartPar
which means we are taking \(\mathbf{p} = (71,0,0)\) and \(\mathbf{d} = (5,0,0)\) and the point \(\mathbf{q} = (5,0,0)\) is found when \(71 +5t = 5 \implies t = -66/5\).


\label{_pages/4.1_Lines:line-vector-equation-example}
\begin{sphinxadmonition}{note}{Example 4.2.1}



\sphinxAtStartPar
Given a point with position vector \(\mathbf{p} = (1, 0, 2, 1, 4)\) and a direction vector \(\mathbf{d} = (1, -1, 0, -1, 1)\) in \(\mathbb{R}^5\), find the equation of the line \(\ell\) which passes through \(\mathbf{p}\) in the direction of \(\mathbf{d}\).
\subsubsection*{Solution}
\begin{equation*}
\begin{split} \mathbf{p} + t\mathbf{d}
= \begin{pmatrix} 1 \\ 0 \\ 2 \\ 1 \\ 4 \end{pmatrix} + t
\begin{pmatrix} 1 \\ -1 \\ 0 \\ -1 \\ 1 \end{pmatrix}
= \begin{pmatrix} 1 + t \\ -t \\ 2 \\ 1 - t \\ 4 + t \end{pmatrix}. \end{split}
\end{equation*}\end{sphinxadmonition}


\bigskip\hrule\bigskip


\index{Lines@\spxentry{Lines}!intersecting lines@\spxentry{intersecting lines}}\ignorespaces 

\subsection{Intersecting lines}
\label{\detokenize{_pages/4.1_Lines:intersecting-lines}}\label{\detokenize{_pages/4.1_Lines:index-2}}\label{\detokenize{_pages/4.1_Lines:intersecting-lines-section}}\label{_pages/4.1_Lines:line-line-intersection-definition}
\begin{sphinxadmonition}{note}{Definition 4.2.2 (Intersection between two lines)}



\sphinxAtStartPar
Two lines \(\ell_1\) and \(\ell_2\) with corresponding vector equations \(\mathbf{p}_1 + t_1\mathbf{d}_1\) and \(\mathbf{p}_2 + t_2\mathbf{d}_2\) in \(\mathbb{R}^n\) \sphinxstylestrong{intersect} if there exist values of \(t_1\) and \(t_2\) such that
\begin{equation*}
\begin{split} \mathbf{p}_1 + t_1 \mathbf{d}_1 = \mathbf{p}_2 + t_2 \mathbf{d}_2. \end{split}
\end{equation*}
\begin{figure}[htbp]
\centering
\capstart

\noindent\sphinxincludegraphics[width=500\sphinxpxdimen]{{4_line_line_intersection}.svg}
\caption{The intersection between the two lines \(\ell_1\) and \(\ell_2\).}\label{\detokenize{_pages/4.1_Lines:line-line-intersection-figure}}\end{figure}
\end{sphinxadmonition}

\sphinxAtStartPar
To determine whether two lines intersect, we can equate the vector equations of the two lines and attempt to solve for \(t_1\) and \(t_2\).


\label{_pages/4.1_Lines:line-line-intersection-example}
\begin{sphinxadmonition}{note}{Example 4.2.2}



\sphinxAtStartPar
Three lines \(\ell_1\), \(\ell_2\) and \(\ell_3\) in \(\mathbb{R}^3\) are defined by the vector equations \((1 + t_1, -3 + 2t_1, t_1)\), \((6 - t_2, -5 + 2t_2, t_2 - 1)\) and \((6 + 2t_3, -11 + 2t_3, -2 - t_3)\) respectively. Determine the points of intersection between the three lines (if possible).
\subsubsection*{Solution}

\sphinxAtStartPar
Equating \(\ell_1\) and \(\ell_2\) gives an equation relating \(t_1\) and \(t_2\) for each of the three coordinate directions:
\begin{equation*}
\begin{split} \begin{align*}
    1 + t_1 &= 6 - t_2, \\
    -3 + 2t_1 &= -5 + 2t_2, \\
    t_1 &= -1 + t_2.
\end{align*} \end{split}
\end{equation*}
\sphinxAtStartPar
The third equation can be rearranged to give \(t_1 = t_2 - 1\), and substituting this into the first equation gives \(t_2 = 3\), and so \(t_1 = 2\).

\sphinxAtStartPar
Substituting these values into the second equation gives \(-3 + 2 \cdot 2 = -5 + 2 \cdot 3\), which simplifies to \(1 = 1\). This means that \(\ell_1\) and \(\ell_2\) do intersect.

\sphinxAtStartPar
To find the point of intersection we substitute these values of \(t_1\) and \(t_2\) into the equations for \(\ell_1\) or \(\ell_2\) respectively.
\begin{equation*}
\begin{split} \begin{align*}
    \begin{pmatrix} 1 + 2 \\ -3 + 2 \cdot 2 \\ 2 \end{pmatrix}  &= \begin{pmatrix} 3 \\ 1 \\ 2 \end{pmatrix}, \\
    \begin{pmatrix} 6 - 3 \\ -5 + 2 \cdot 3 \\ 3 - 1 \end{pmatrix} &= \begin{pmatrix} 3 \\ 1 \\ 2 \end{pmatrix}. \\
\end{align*} \end{split}
\end{equation*}
\sphinxAtStartPar
So \(\ell_1\) and \(\ell_2\) intersect at \((3, 1, 2)\).

\sphinxAtStartPar
Equating \(\ell_1\) and \(\ell_3\) gives
\begin{equation*}
\begin{split} \begin{align*}
    1 + t_1 &= 6 + t_3, \\
    -3 + 2t_1 &= -11 + 2t_3, \\
    t_1 &= -2 - t_3.
\end{align*} \end{split}
\end{equation*}
\sphinxAtStartPar
Substituting the third equation into the first equation gives \(1 - 2 - t_3 = 6 + t_3\), giving \(t_3 = -7/2\) and \(t_1 = -21/2\). Substituting these into the second equation gives \(-3 + 2 \cdot (-21/2) = -11 + 2 \cdot (-7/2)\), which simplifies to \(-24 = -18\). This is a contradiction, so no values of \(t_1\) or \(t_3\) satisfy \(\ell_1 = \ell_3\), and hence \(\ell_1\) and \(\ell_3\) do not intersect.

\sphinxAtStartPar
Equating \(\ell_2\) and \(\ell_3\) gives the system
\begin{equation*}
\begin{split} \begin{align*}
    6 - t_2 &= 6 + 2t_3, \\
    -5 + 2t_2 &= -11 + 2t_3, \\
    -1 + t_2 &= -2 - t_3.
\end{align*} \end{split}
\end{equation*}
\sphinxAtStartPar
The third equation gives \(t_2 = -1 - t_3\), and substituting this into the first equation gives \(6 + 1 + t_3 = 6 + 2t_3\), so \(t_3 = 1\) and \(t_2 = -2\). Substituting these into the second equation gives \(-5 + 2 \cdot -2 = -11 + 2 \cdot 1\), which simplifies to \(-9 = -9\), so \(\ell_2\) and \(\ell_3\) do intersect. Substituting \(t_2\) and \(t_3\) into the equations for \(\ell_2\) and \(\ell_3\):
\begin{equation*}
\begin{split} \begin{align*}
    \begin{pmatrix} 6 - (-2) \\ -5 + 2 \cdot (-2) \\ -1 + (-2) \end{pmatrix} &= \begin{pmatrix} 8 \\ -9 \\ -3 \end{pmatrix}, \\
    \begin{pmatrix} 6 + 2 \cdot 1 \\ -11 + 2 \cdot 1 \\ -2 - 1 \end{pmatrix} &= \begin{pmatrix} 8 \\ -9 \\ -3 \end{pmatrix}.
\end{align*} \end{split}
\end{equation*}
\sphinxAtStartPar
So \(\ell_2\) and \(\ell_3\) intersect at \((8, -9, -3)\).
\end{sphinxadmonition}


\bigskip\hrule\bigskip


\index{Lines@\spxentry{Lines}!parallel lines@\spxentry{parallel lines}}\ignorespaces 

\subsection{Parallel lines}
\label{\detokenize{_pages/4.1_Lines:parallel-lines}}\label{\detokenize{_pages/4.1_Lines:index-3}}\label{_pages/4.1_Lines:parallel-lines-definition}
\begin{sphinxadmonition}{note}{Definition 4.2.3 (Parallel lines)}



\sphinxAtStartPar
Two lines \(\ell_1\) and \(\ell_2\) in \(\mathbb{R}^n\) defined by the vector equations \(\mathbf{r}_1 = \mathbf{p}_1 + t_1 \mathbf{d}_1\) and \(\mathbf{r}_2 = \mathbf{p}_2 + t_2 \mathbf{d}_2\) respectively are said to be \sphinxstylestrong{parallel} if their direction vectors are parallel; that is, \(\mathbf{d_1} = k\mathbf{d_2},\) for some non\sphinxhyphen{}zero scalar \(k\) (\hyperref[\detokenize{_pages/4.1_Lines:parallel-lines-figure}]{Fig.\@ \ref{\detokenize{_pages/4.1_Lines:parallel-lines-figure}}}).

\begin{figure}[htbp]
\centering
\capstart

\noindent\sphinxincludegraphics[width=500\sphinxpxdimen]{{4_parallel_lines}.svg}
\caption{The lines \(\ell_1\) and \(\ell_2\) are parallel in \(\mathbb{R}^3\).}\label{\detokenize{_pages/4.1_Lines:parallel-lines-figure}}\end{figure}
\end{sphinxadmonition}

\sphinxAtStartPar
The direction vectors being parallel does not require them to be the same length, hence the condition being merely that one is a scalar multiple of the other (and the value of \(k\) could be \(1\)). The vector only defines a direction, and only in combination with a specified starting point does it define a specific line in space.

\sphinxAtStartPar
In practice the condition from the {\hyperref[\detokenize{_pages/4.1_Lines:parallel-lines-definition}]{\sphinxcrossref{definition of parallel lines}}} can be restated as follows: two lines \(\ell_1\) and \(\ell_2\) in \(\mathbb{R}^n\) are parallel if for any two distinct points \(\mathbf{p}_1, \mathbf{p}_2 \in \ell_1\) and any two distinct points \(\mathbf{q}_1, \mathbf{q}_2 \in \ell_2\):
\begin{equation*}
\begin{split} \mathbf{p}_2 - \mathbf{p}_1 = k(\mathbf{q}_2 - \mathbf{q}_1), \end{split}
\end{equation*}
\sphinxAtStartPar
for some non\sphinxhyphen{}zero \(k \in \mathbb{R}\). This allows us a way to check if two lines are parallel without having to find their direction vectors,  if we know the positions of two points on each line.
\label{_pages/4.1_Lines:parallel-lines-example}
\begin{sphinxadmonition}{note}{Example 4.2.3}



\sphinxAtStartPar
\(\ell_1\) and \(\ell_2\) are two lines in \(\mathbb{R}^3\), defined by \(\{(x,y,z) : z = x+1, y=0\}\) and \(\{(x,y,z) : z = -x-1, y=3\}\).

\sphinxAtStartPar
(i)   Write equations for \(\ell_1\) and \(\ell_2\) in vector form.

\sphinxAtStartPar
(ii)   Show that these lines are not parallel in \(\mathbb{R}^3\).
\subsubsection*{Solution}

\sphinxAtStartPar
(i)   To calculate the direction vector \(\mathbf{d}\) we need the coordinates of two points on the line. Choosing \(x=0\) and \(x=1\), we can use the equations for the two lines to find the two points on the line with those \(x\)\sphinxhyphen{}coordinates, which are \(\mathbf{p}_1 = (0, 0, 1)\) and \(\mathbf{p}_2 = (1, 0, 2)\) on \(\ell_1\). Then we can write
\begin{equation*}
\begin{split} \begin{align*}
    \mathbf{d}_1 = \mathbf{p}_2 - \mathbf{p}_1 = \begin{pmatrix} 1 \\ 0 \\ 2 \end{pmatrix} - 
    \begin{pmatrix} 0 \\ 0 \\ 1 \end{pmatrix} =
    \begin{pmatrix} 1 \\ 0 \\ 1 \end{pmatrix},
\end{align*} \end{split}
\end{equation*}
\sphinxAtStartPar
So an equation for \(\ell_1\) is:
\begin{equation*}
\begin{split} \begin{align*}
    \mathbf{r}_1 = \begin{pmatrix} 0 \\ 0 \\ 1 \end{pmatrix} + t \begin{pmatrix} 1 \\ 0 \\ 1 \end{pmatrix} = \begin{pmatrix} t \\ 0 \\ 1 + t \end{pmatrix}.
\end{align*} \end{split}
\end{equation*}
\sphinxAtStartPar
Similarly, for \(\ell_2\) we can find \(\mathbf{p}_1 = (0, 3, -1)\) and \(\mathbf{p}_2 = (1, 3, -2)\), so:
\begin{equation*}
\begin{split} \begin{align*}
    \mathbf{d}_2 = \mathbf{p}_2 - \mathbf{p}_1 = 
    \begin{pmatrix} 1 \\ 3 \\ -2 \end{pmatrix} - 
    \begin{pmatrix} 0 \\ 3 \\ -1 \end{pmatrix} =
    \begin{pmatrix} 1 \\ 0 \\ -1 \end{pmatrix},
\end{align*} \end{split}
\end{equation*}
\sphinxAtStartPar
so the equation of \(\ell_2\) is
\begin{equation*}
\begin{split} \begin{align*}
    \mathbf{r}_2 = \begin{pmatrix} 0 \\ 3 \\ -1 \end{pmatrix} + t \begin{pmatrix} 1 \\ 0 \\ -1 \end{pmatrix} = \begin{pmatrix} t \\ 3 \\ -1 -t \end{pmatrix}.
\end{align*} \end{split}
\end{equation*}
\sphinxAtStartPar
(ii)   To check if the two lines are parallel, we do not need to know their equations \sphinxhyphen{} just their direction vectors. From {\hyperref[\detokenize{_pages/4.1_Lines:parallel-lines-definition}]{\sphinxcrossref{the definition of parallel lines}}}, \(\ell_1\) and \(\ell_2\) are parallel if \(\mathbf{d}_1 = k\mathbf{d}_2\), therefore
\begin{equation*}
\begin{split} \begin{align*}
    \begin{pmatrix} 1 \\ 0 \\ 1 \end{pmatrix} &= k
    \begin{pmatrix} 1 \\ 0 \\ -1 \end{pmatrix}
\end{align*} \end{split}
\end{equation*}
\sphinxAtStartPar
This gives us a set of three equations which must simultaneously be satisfied: \(1k =1\), \(0k=0\) and \(1k = -1\). Since these imply that both \(k=1\) and \(k=-1\), we have a contradiction \sphinxhyphen{} so no value of \(k\) exists to satisfy \(\mathbf{d}_1 = k\mathbf{d}_2\), and so \(\ell_1\) and \(\ell_2\) are not parallel.
\end{sphinxadmonition}


\bigskip\hrule\bigskip


\index{Lines@\spxentry{Lines}!skew lines@\spxentry{skew lines}}\ignorespaces 

\subsection{Skew lines}
\label{\detokenize{_pages/4.1_Lines:skew-lines}}\label{\detokenize{_pages/4.1_Lines:index-4}}\label{_pages/4.1_Lines:skew-lines-definition}
\begin{sphinxadmonition}{note}{Definition 4.2.4 (Skew lines)}



\sphinxAtStartPar
Two distinct lines in \(\mathbb{R}^3\), which neither intersect nor are parallel, are called \sphinxstylestrong{skew lines} (\hyperref[\detokenize{_pages/4.1_Lines:skew-lines-figure}]{Fig.\@ \ref{\detokenize{_pages/4.1_Lines:skew-lines-figure}}})

\begin{figure}[htbp]
\centering
\capstart

\noindent\sphinxincludegraphics[width=350\sphinxpxdimen]{{4_skew_lines}.svg}
\caption{The lines \(\ell_1\) and \(\ell_2\) are skew lines in \(\mathbb{R}^3\).}\label{\detokenize{_pages/4.1_Lines:skew-lines-figure}}\end{figure}
\end{sphinxadmonition}
\label{_pages/4.1_Lines:skew-lines-example}
\begin{sphinxadmonition}{note}{Example 4.2.4}



\sphinxAtStartPar
Show that the lines \(\ell_1\) and \(\ell_2\) from {\hyperref[\detokenize{_pages/4.1_Lines:parallel-lines-example}]{\sphinxcrossref{Example 4.2.3}}} are skew lines.
\subsubsection*{Solution}

\sphinxAtStartPar
We have shown in {\hyperref[\detokenize{_pages/4.1_Lines:parallel-lines-example}]{\sphinxcrossref{Example 4.2.3}}} that \(\ell_1\) and \(\ell_2\) are not parallel. Therefore to we need to show that they do not intersect. Equating \(\ell_1\) and \(\ell_2\) gives
\begin{equation*}
\begin{split} \begin{align*}
    t_1 &= t_2, \\
    0 &= 3, \\
    -1 &= -1 - t_2.
\end{align*} \end{split}
\end{equation*}
\sphinxAtStartPar
Here the second equation is a contradiction, so this system is inconsistent and the lines \(\ell_1\) and \(\ell_2\) do not intersect. Since \(\ell_1\) and \(\ell_2\) are not parallel and do not intersect, they must be skew.
\end{sphinxadmonition}


\bigskip\hrule\bigskip


\index{Lines@\spxentry{Lines}!perpendicular lines@\spxentry{perpendicular lines}}\ignorespaces 

\subsection{Perpendicular lines}
\label{\detokenize{_pages/4.1_Lines:perpendicular-lines}}\label{\detokenize{_pages/4.1_Lines:index-5}}\label{_pages/4.1_Lines:perpendicular-lines-definition}
\begin{sphinxadmonition}{note}{Definition 4.2.5 (Perpendicular lines)}



\sphinxAtStartPar
Two lines, \(\ell_1\) and \(\ell_2\), are \sphinxstylestrong{perpendicular} if their direction vectors \(\mathbf{d}_1\) and \(\mathbf{d}_2\) are at right angles. We denote this by \(\ell_1 \perp \ell_2\).

\sphinxAtStartPar
In \(\mathbb{R}^n\) two lines are perpendicular if and only if their direction vectors are perpendicular, which happens precisely when the dot product of the direction vectors is zero.
\end{sphinxadmonition}
\label{_pages/4.1_Lines:perpendicular-lines-example}
\begin{sphinxadmonition}{note}{Example 4.2.5}



\sphinxAtStartPar
(i)   Determine whether the two lines from {\hyperref[\detokenize{_pages/4.1_Lines:parallel-lines-example}]{\sphinxcrossref{Example 4.2.3}}} are perpendicular.

\sphinxAtStartPar
(ii)   Find the equation of a line \(\ell_2\) which is perpendicular to \((t,-t,1+t)\) at the point with position vector \(\mathbf{p} = (-1,1,0)\).
\subsubsection*{Solution}

\sphinxAtStartPar
(i)   The direction vectors from Example 4.3 are \(\mathbf{d}_1 = (1, 0, 1)\) and \(\mathbf{d}_2 = (1, 0, -1)\). Find the dot product:
\begin{equation*}
\begin{split} \begin{align*}
    \mathbf{d}_1 \cdot \mathbf{d}_2 = \begin{pmatrix} 1 \\ 0 \\ 1 \end{pmatrix} \cdot
    \begin{pmatrix} 1 \\ 0 \\ -1 \end{pmatrix} = 1 + 0 + (-1) = 0.
\end{align*} \end{split}
\end{equation*}
\sphinxAtStartPar
Since this is zero, we know \(\ell_1 \perp \ell_2\).

\sphinxAtStartPar
(ii)   Writing the line \((t,-t,1+t)\) in the form \(\mathbf{p}_1 + t \mathbf{d}_1\) we have
\begin{equation*}
\begin{split} \begin{align*}
    \mathbf{p}_1 &= \begin{pmatrix} 0 \\ 0 \\ 1 \end{pmatrix}, &
    \mathbf{d}_1 &= \begin{pmatrix} 1 \\ -1 \\ 1 \end{pmatrix}.
\end{align*} \end{split}
\end{equation*}
\sphinxAtStartPar
To find a direction vector perpendicular to \(\mathbf{d}_1\) \sphinxhyphen{} say, \(\mathbf{d}_2\) \sphinxhyphen{} we need to satisfy \(\mathbf{d}_1 \cdot \mathbf{d}_2 = 0\). Using a general vector for \$\textbackslash{}mathbf\{d\}\_2, we have:
\begin{equation*}
\begin{split} \begin{align*}
    \begin{pmatrix} 1 \\ -1 \\ 1 \end{pmatrix}  \cdot \mathbf{d}_2 = 
    \begin{pmatrix} 1 \\ -1 \\ 1 \end{pmatrix} \cdot 
    \begin{pmatrix} d_1 \\ d_2 \\ d_3 \end{pmatrix} = d_1 - d_2 + d_3 = 0.
\end{align*} \end{split}
\end{equation*}
\sphinxAtStartPar
This equation has infinitely many solutions, so to choose one we need to pick values for any two from \(d_1\), \(d_2\) and \(d_3\) (where at least one is non\sphinxhyphen{}zero). Let \(d_1 = 1\) and \(d_2 = 0\); then we can find \(d_3 = -1\). So a possible direction vector for this line is \(\mathbf{d}_2 = (1, 0, -1)\). We know \(\ell_2\) passes through \(\mathbf{p} = (-1, 1, 0)\) and so an equation for \(\ell_2\) is:
\begin{equation*}
\begin{split} \begin{align*}
    \begin{pmatrix} -1 \\ 1 \\ 0 \end{pmatrix} + t
    \begin{pmatrix} 1 \\ 0 \\ -1 \end{pmatrix} = 
    \begin{pmatrix} -1 + t \\ 1 \\ -t \end{pmatrix}.
\end{align*} \end{split}
\end{equation*}\end{sphinxadmonition}

\sphinxAtStartPar
The notion of perpendicularity does not necessarily require that the lines intersect in space. The only requirement is that the direction vectors of the two lines are perpendicular.


\bigskip\hrule\bigskip


\sphinxstepscope

\index{Planes ✈️@\spxentry{Planes ✈️}}\ignorespaces 

\section{Planes}
\label{\detokenize{_pages/4.2_Planes:planes}}\label{\detokenize{_pages/4.2_Planes:index-0}}\label{\detokenize{_pages/4.2_Planes:planes-section}}\label{\detokenize{_pages/4.2_Planes::doc}}
\sphinxAtStartPar
A \sphinxstylestrong{plane} is a flat two\sphinxhyphen{}dimensional surface. The vector equation for a plane in \(\mathbb{R}^n\) is very similar to that of a line \sphinxhyphen{} we just need two direction vectors to define it, instead of the one we needed for a line.
\label{_pages/4.2_Planes:plane-vector-equation}
\begin{sphinxadmonition}{note}{Definition 4.3.1 (Vector equation of a plane)}



\sphinxAtStartPar
Let \(\mathbf{p}\) be a point in \(\mathbb{R}^n\), and let \(\mathbf{d}_1\) and \(\mathbf{d}_2\) be two non\sphinxhyphen{}zero vectors in \(\mathbb{R}^n\) which are not parallel. The plane which passes through \(\mathbf{p}\) in the directions of \(\mathbf{d}_1\) and \(\mathbf{d}_2\) has the equation:
\begin{equation*}
\begin{split} \mathbf{r} = \mathbf{p} + t_1 \mathbf{d}_1 + t_2 \mathbf{d}_2, \end{split}
\end{equation*}
\sphinxAtStartPar
where \(\mathbf{r}\) is a point on the plane, and \(t_1,t_2 \in \mathbb{R}\) are parameters (\hyperref[\detokenize{_pages/4.2_Planes:plane-1-figure}]{Fig.\@ \ref{\detokenize{_pages/4.2_Planes:plane-1-figure}}}).

\begin{figure}[htbp]
\centering
\capstart

\noindent\sphinxincludegraphics[width=400\sphinxpxdimen]{{4_plane}.svg}
\caption{The position of a point \(\mathbf{r}\) on a plane can be obtained by adding scalar multiples of the direction vectors \(\mathbf{d}_1\) and \(\mathbf{d}_2\) to \(\mathbf{p}\).}\label{\detokenize{_pages/4.2_Planes:plane-1-figure}}\end{figure}
\end{sphinxadmonition}

\sphinxAtStartPar
In other words, every point on a plane is obtained by adding some scalar multiples of any two (non\sphinxhyphen{}parallel) vectors on the plane, \(\mathbf{d}_1\) and \(\mathbf{d}_2\), to the coordinate vector of a point on the plane \(\mathbf{p}\). The points will all lie on the plane which passes through \(\mathbf{p}\) in the directions of \(\mathbf{d}_1\) and \(\mathbf{d}_2\).

\sphinxAtStartPar
In the case where \(\mathbf{d}_1\) and \(\mathbf{d}_2\) are parallel, we would  only be describing a single line of points, and not a plane.

\index{Normal vector@\spxentry{Normal vector}}\ignorespaces \label{_pages/4.2_Planes:normal-vector-definition}
\begin{sphinxadmonition}{note}{Definition 4.3.2 (The normal vector)}



\sphinxAtStartPar
The \sphinxstylestrong{normal vector} to a 2D plane is a vector that is perpendicular to that plane. If \(\mathbf{a}\) and \(\mathbf{b}\) are two vectors that lie on a plane then the normal vector is calculated using \(\mathbf{n} = \mathbf{a} \times \mathbf{b}\) (cross product).

\begin{figure}[htbp]
\centering
\capstart

\noindent\sphinxincludegraphics[width=450\sphinxpxdimen]{{4_normal_vector}.svg}
\caption{The normal vector \(\mathbf{n}\) is perpendicular to the plane that the vectors \(\mathbf{a}\) and \(\mathbf{b}\) lie on.}\label{\detokenize{_pages/4.2_Planes:plane-2-figure}}\end{figure}
\end{sphinxadmonition}

\sphinxAtStartPar
Recall that {\hyperref[\detokenize{_pages/3.3_Dot_and_cross_products:dot-product-definition}]{\sphinxcrossref{the dot product}}} between two perpendicular vectors is zero. This allows us an alternative way to define a plane. Consider a normal vector \(\mathbf{n} = (n_x, n_y, n_z)\) and a point with coordinates \(\mathbf{p}=(p_x, p_y, p_z)\). We can define a plane of points which contains the point \(\mathbf{p}\) and is perpendicular to the normal vector.

\sphinxAtStartPar
If \(\mathbb{r} = (x,y,z)\) are the coordinates of a point in the plane perpendicular to the given normal vector, then:
\begin{equation*}
\begin{split} \begin{align*}
    \mathbf{n} \cdot (\mathbf{r} - \mathbf{p}) &= 0 \textrm{ (since the normal vector will be perpendicular to any vector going between two points in the plane it is normal to)}\\
    \textrm{We can also write: }\begin{pmatrix} n_x \\ n_y \\ n_z \end{pmatrix}
    \cdot \left( 
        \begin{pmatrix} x \\ y \\ z \end{pmatrix} -
        \begin{pmatrix} p_x \\ p_y \\ p_z \end{pmatrix}
    \right) &= 0 \\
    n_x(x - p_x) + n_y(y - p_y) + n_z(z - p_z) &= 0.
\end{align*} \end{split}
\end{equation*}
\sphinxAtStartPar
This is known as the point\sphinxhyphen{}normal form of a plane.

\index{Planes@\spxentry{Planes}!point\sphinxhyphen{}normal form@\spxentry{point\sphinxhyphen{}normal form}}\ignorespaces \label{_pages/4.2_Planes:point-normal-definition}
\begin{sphinxadmonition}{note}{Definition 4.3.3 (Point\sphinxhyphen{}normal form of a plane)}



\sphinxAtStartPar
The point\sphinxhyphen{}normal form of the plane in \(\mathbb{R}^3\) which passes through the point with position vector \(\mathbf{p}\) with normal vector \(\mathbf{n}\) is:
\begin{equation}\label{equation:_pages/4.2_Planes:point-normal-equation}
\begin{split} \mathbf{n} \cdot (\mathbf{r} - \mathbf{p}) = 0,\end{split}
\end{equation}
\sphinxAtStartPar
where \(\mathbf{r}\) is the position vector of an arbitrary point on the plane.

\begin{figure}[htbp]
\centering
\capstart

\noindent\sphinxincludegraphics[width=450\sphinxpxdimen]{{4_point_normal}.svg}
\caption{A plane can be defined by a point on the plane and the normal vector to the plane.}\label{\detokenize{_pages/4.2_Planes:point-normal-figure}}\end{figure}
\end{sphinxadmonition}


\label{_pages/4.2_Planes:point-normal-example}
\begin{sphinxadmonition}{note}{Example 4.3.1}



\sphinxAtStartPar
A plane in \(\mathbb{R}^3\) passes through the three points with position vectors \(\mathbf{p}_1 = (2, 3, 1)\), \(\mathbf{p}_2 = (1, 0, -1)\) and \(\mathbf{p}_3 = (2, 1, 3)\).

\sphinxAtStartPar
(i)   Find the point\sphinxhyphen{}normal form of this plane.

\sphinxAtStartPar
(ii)   Do the points with position vectors \(\mathbf{p}_4 = (1, 2, 1)\) and \(\mathbf{p}_5 = (2, 4, 0)\) lie on this plane?
\subsubsection*{Solution}

\sphinxAtStartPar
(i)   First we need to calculate the normal vector to the plane. We can do this by calculating the cross product of any two vectors that line on the plane. Since we know \(\mathbf{p}_1\), \(\mathbf{p}_2\) and \(\mathbf{p}_3\) lie on the plane, then so must the vectors \(\mathbf{p}_2 - \mathbf{p}_1\) and \(\mathbf{p}_3 - \mathbf{p}_2\) (or any other combination of these points).
\begin{equation*}
\begin{split} \begin{align*}
    \mathbf{p}_2 - \mathbf{p}_1 &=
    \begin{pmatrix} 1 \\ 0 \\ -1 \end{pmatrix} -
    \begin{pmatrix} 2 \\ 3 \\ 1 \end{pmatrix} =
    \begin{pmatrix} -1 \\ -3 \\ -2 \end{pmatrix}, \\
    \mathbf{p}_3 - \mathbf{p}_2 &=
    \begin{pmatrix} 2 \\ 1 \\ 3 \end{pmatrix} -
    \begin{pmatrix} 1 \\ 0 \\ -1 \end{pmatrix} =
    \begin{pmatrix} 1 \\ 1 \\ 4 \end{pmatrix}, \\
    \therefore \mathbf{n} &= (\mathbf{p}_2 - \mathbf{p}_1) \times (\mathbf{p}_3 - \mathbf{p}_2) =
    \begin{pmatrix} -1 \\ -3 \\ -2 \end{pmatrix} \times
    \begin{pmatrix} 1 \\ 1 \\ 4 \end{pmatrix} \\
    &=
    \begin{vmatrix}
        \mathbf{i} & \mathbf{j} & \mathbf{k} \\
        -1 & -3 & -2 \\
        1 & 1 & 4
    \end{vmatrix} =
    \begin{pmatrix} -10 \\ 2 \\ 2 \end{pmatrix}.
\end{align*} \end{split}
\end{equation*}
\sphinxAtStartPar
Using equation \eqref{equation:_pages/4.2_Planes:point-normal-equation} with \(\mathbf{p} = \mathbf{p}_1\)
\begin{equation*}
\begin{split} \begin{align*}
    \mathbf{n} \cdot (\mathbf{r} - \mathbf{p}_1) &= 0 \\ \\
    \begin{pmatrix} -10 \\ 2 \\ 2 \end{pmatrix} \cdot \left(
        \begin{pmatrix} x \\ y \\ z \end{pmatrix} -
        \begin{pmatrix} 2 \\ 3 \\ 1 \end{pmatrix}
    \right) &= 0 \\ \\
    -10(x - 2) + 2(y - 3) + 2(z - 1) &= 0 \\ \\
    -10x + 2y + 2z + 12 &= 0.
\end{align*} \end{split}
\end{equation*}
\sphinxAtStartPar
This equation defines the plane \sphinxhyphen{} any point whose coordinates satisfy this equation will lie in the plane.

\sphinxAtStartPar
If we had chosen \(\mathbf{p} = \mathbf{p}_2\) instead, we would have:
\begin{equation*}
\begin{split} \begin{align*}
\mathbf{n} \cdot (\mathbf{r} - \mathbf{p}_2) &= 0 \\ \\
    \begin{pmatrix} -10 \\ 2 \\ 2 \end{pmatrix} \cdot \left(
        \begin{pmatrix} x \\ y \\ z \end{pmatrix} -
        \begin{pmatrix} 1 \\ 0 \\ -1 \end{pmatrix}
    \right) &= 0 \\ \\
    -10(x - 1) + 2(y - 0) + 2(z + 1) &= 0 \\ \\
    -10x + 2y + 2z + 12 &= 0.
\end{align*} \end{split}
\end{equation*}
\sphinxAtStartPar
This is the same as the answer as before. The same applies to any other point on the plane.

\sphinxAtStartPar
(ii)   Checking \(\mathbf{p}_4\)
\begin{equation*}
\begin{split} -10 \cdot 1 + 2 \cdot 2 + 2 \cdot 1 + 12 = 8 \neq 0, \end{split}
\end{equation*}
\sphinxAtStartPar
so \(\mathbf{p}_4\) does not lie on the plane.

\sphinxAtStartPar
Checking \(\mathbf{p}_5\)
\begin{equation*}
\begin{split} -10 \cdot 2 + 2 \cdot 4 + 2 \cdot 0 + 12 = 0,  \end{split}
\end{equation*}
\sphinxAtStartPar
so \(\mathbf{p}_5\) does lie on the plane.
\end{sphinxadmonition}


\bigskip\hrule\bigskip


\index{Planes@\spxentry{Planes}!intersecting planes@\spxentry{intersecting planes}}\ignorespaces 

\subsection{Intersection of planes in \protect\(\mathbb{R}^3\protect\)}
\label{\detokenize{_pages/4.2_Planes:intersection-of-planes-in-mathbb-r-3}}\label{\detokenize{_pages/4.2_Planes:index-3}}\label{\detokenize{_pages/4.2_Planes:intersection-of-planes-section}}
\sphinxAtStartPar
If we define planes in terms of their normal vectors, we can distinguish between various arrangements of planes.

\sphinxAtStartPar
A plane is perpendicular to its normal vector \(\mathbf{n}\). Any other plane that the normal vector \(\mathbf{n}\) lies on will be described as a plane \sphinxstylestrong{perpendicular} to the original plane. This is equivalent to the two normal vectors of the planes being perpendicular to each other. Similarly, if two distinct planes have the same normal vector, then they are \sphinxstylestrong{parallel}.

\sphinxAtStartPar
Often we would like to know when and how two or more objects in \(\mathbb{R}^n\) meet. Similar to the way lines can meet in \(\mathbb{R}^2\), we present a finite list of possible arrangements for three planes meeting in \(\mathbb{R}^3\). If the planes were to be considered as simlutaneous equations, the number of points which satisfy all three equations (solutions to the system) will depend on their arrangement as planes.

\begin{sphinxuseclass}{sd-container-fluid}
\begin{sphinxuseclass}{sd-sphinx-override}
\begin{sphinxuseclass}{sd-mb-4}
\begin{sphinxuseclass}{sd-row}
\begin{sphinxuseclass}{sd-g-2}
\begin{sphinxuseclass}{sd-g-xs-2}
\begin{sphinxuseclass}{sd-g-sm-2}
\begin{sphinxuseclass}{sd-g-md-2}
\begin{sphinxuseclass}{sd-g-lg-2}
\begin{sphinxuseclass}{sd-col}
\begin{sphinxuseclass}{sd-d-flex-row}
\begin{sphinxuseclass}{sd-col-4}
\begin{sphinxuseclass}{sd-col-xs-4}
\begin{sphinxuseclass}{sd-col-sm-4}
\begin{sphinxuseclass}{sd-col-md-4}
\begin{sphinxuseclass}{sd-col-lg-4}
\begin{sphinxuseclass}{sd-card}
\begin{sphinxuseclass}{sd-sphinx-override}
\begin{sphinxuseclass}{sd-w-100}
\begin{sphinxuseclass}{sd-shadow-sm}
\begin{sphinxuseclass}{sd-card-body}
\begin{figure}[htbp]
\centering

\noindent\sphinxincludegraphics{{4_intersecting_planes_1}.svg}
\end{figure}

\sphinxAtStartPar
3 planes intersect at a single point: a unique solution

\end{sphinxuseclass}
\end{sphinxuseclass}
\end{sphinxuseclass}
\end{sphinxuseclass}
\end{sphinxuseclass}
\end{sphinxuseclass}
\end{sphinxuseclass}
\end{sphinxuseclass}
\end{sphinxuseclass}
\end{sphinxuseclass}
\end{sphinxuseclass}
\end{sphinxuseclass}
\begin{sphinxuseclass}{sd-col}
\begin{sphinxuseclass}{sd-d-flex-row}
\begin{sphinxuseclass}{sd-col-4}
\begin{sphinxuseclass}{sd-col-xs-4}
\begin{sphinxuseclass}{sd-col-sm-4}
\begin{sphinxuseclass}{sd-col-md-4}
\begin{sphinxuseclass}{sd-col-lg-4}
\begin{sphinxuseclass}{sd-card}
\begin{sphinxuseclass}{sd-sphinx-override}
\begin{sphinxuseclass}{sd-w-100}
\begin{sphinxuseclass}{sd-shadow-sm}
\begin{sphinxuseclass}{sd-card-body}
\begin{figure}[htbp]
\centering

\noindent\sphinxincludegraphics{{4_intersecting_planes_2}.svg}
\end{figure}

\sphinxAtStartPar
3 planes intersect on a single line: infinite solutions

\end{sphinxuseclass}
\end{sphinxuseclass}
\end{sphinxuseclass}
\end{sphinxuseclass}
\end{sphinxuseclass}
\end{sphinxuseclass}
\end{sphinxuseclass}
\end{sphinxuseclass}
\end{sphinxuseclass}
\end{sphinxuseclass}
\end{sphinxuseclass}
\end{sphinxuseclass}
\begin{sphinxuseclass}{sd-col}
\begin{sphinxuseclass}{sd-d-flex-row}
\begin{sphinxuseclass}{sd-col-4}
\begin{sphinxuseclass}{sd-col-xs-4}
\begin{sphinxuseclass}{sd-col-sm-4}
\begin{sphinxuseclass}{sd-col-md-4}
\begin{sphinxuseclass}{sd-col-lg-4}
\begin{sphinxuseclass}{sd-card}
\begin{sphinxuseclass}{sd-sphinx-override}
\begin{sphinxuseclass}{sd-w-100}
\begin{sphinxuseclass}{sd-shadow-sm}
\begin{sphinxuseclass}{sd-card-body}
\begin{figure}[htbp]
\centering

\noindent\sphinxincludegraphics{{4_intersecting_planes_3}.svg}
\end{figure}

\sphinxAtStartPar
2 planes coincide and intersect a third: infinite solutions

\end{sphinxuseclass}
\end{sphinxuseclass}
\end{sphinxuseclass}
\end{sphinxuseclass}
\end{sphinxuseclass}
\end{sphinxuseclass}
\end{sphinxuseclass}
\end{sphinxuseclass}
\end{sphinxuseclass}
\end{sphinxuseclass}
\end{sphinxuseclass}
\end{sphinxuseclass}
\begin{sphinxuseclass}{sd-col}
\begin{sphinxuseclass}{sd-d-flex-row}
\begin{sphinxuseclass}{sd-col-4}
\begin{sphinxuseclass}{sd-col-xs-4}
\begin{sphinxuseclass}{sd-col-sm-4}
\begin{sphinxuseclass}{sd-col-md-4}
\begin{sphinxuseclass}{sd-col-lg-4}
\begin{sphinxuseclass}{sd-card}
\begin{sphinxuseclass}{sd-sphinx-override}
\begin{sphinxuseclass}{sd-w-100}
\begin{sphinxuseclass}{sd-shadow-sm}
\begin{sphinxuseclass}{sd-card-body}
\begin{figure}[htbp]
\centering

\noindent\sphinxincludegraphics{{4_intersecting_planes_4}.svg}
\end{figure}

\sphinxAtStartPar
2 parallel planes intersect another plane: no solutions

\end{sphinxuseclass}
\end{sphinxuseclass}
\end{sphinxuseclass}
\end{sphinxuseclass}
\end{sphinxuseclass}
\end{sphinxuseclass}
\end{sphinxuseclass}
\end{sphinxuseclass}
\end{sphinxuseclass}
\end{sphinxuseclass}
\end{sphinxuseclass}
\end{sphinxuseclass}
\begin{sphinxuseclass}{sd-col}
\begin{sphinxuseclass}{sd-d-flex-row}
\begin{sphinxuseclass}{sd-col-4}
\begin{sphinxuseclass}{sd-col-xs-4}
\begin{sphinxuseclass}{sd-col-sm-4}
\begin{sphinxuseclass}{sd-col-md-4}
\begin{sphinxuseclass}{sd-col-lg-4}
\begin{sphinxuseclass}{sd-card}
\begin{sphinxuseclass}{sd-sphinx-override}
\begin{sphinxuseclass}{sd-w-100}
\begin{sphinxuseclass}{sd-shadow-sm}
\begin{sphinxuseclass}{sd-card-body}


\begin{figure}[htbp]
\centering

\noindent\sphinxincludegraphics{{4_intersecting_planes_5}.svg}
\end{figure}

\sphinxAtStartPar
Each plane intersects with 2 others: no solutions

\end{sphinxuseclass}
\end{sphinxuseclass}
\end{sphinxuseclass}
\end{sphinxuseclass}
\end{sphinxuseclass}
\end{sphinxuseclass}
\end{sphinxuseclass}
\end{sphinxuseclass}
\end{sphinxuseclass}
\end{sphinxuseclass}
\end{sphinxuseclass}
\end{sphinxuseclass}
\begin{sphinxuseclass}{sd-col}
\begin{sphinxuseclass}{sd-d-flex-row}
\begin{sphinxuseclass}{sd-col-4}
\begin{sphinxuseclass}{sd-col-xs-4}
\begin{sphinxuseclass}{sd-col-sm-4}
\begin{sphinxuseclass}{sd-col-md-4}
\begin{sphinxuseclass}{sd-col-lg-4}
\begin{sphinxuseclass}{sd-card}
\begin{sphinxuseclass}{sd-sphinx-override}
\begin{sphinxuseclass}{sd-w-100}
\begin{sphinxuseclass}{sd-shadow-sm}
\begin{sphinxuseclass}{sd-card-body}
\begin{figure}[htbp]
\centering

\noindent\sphinxincludegraphics{{4_intersecting_planes_6}.svg}
\end{figure}

\sphinxAtStartPar
3 parallel planes: no solutions

\end{sphinxuseclass}
\end{sphinxuseclass}
\end{sphinxuseclass}
\end{sphinxuseclass}
\end{sphinxuseclass}
\end{sphinxuseclass}
\end{sphinxuseclass}
\end{sphinxuseclass}
\end{sphinxuseclass}
\end{sphinxuseclass}
\end{sphinxuseclass}
\end{sphinxuseclass}
\begin{sphinxuseclass}{sd-col}
\begin{sphinxuseclass}{sd-d-flex-row}
\begin{sphinxuseclass}{sd-col-4}
\begin{sphinxuseclass}{sd-col-xs-4}
\begin{sphinxuseclass}{sd-col-sm-4}
\begin{sphinxuseclass}{sd-col-md-4}
\begin{sphinxuseclass}{sd-col-lg-4}
\begin{sphinxuseclass}{sd-card}
\begin{sphinxuseclass}{sd-sphinx-override}
\begin{sphinxuseclass}{sd-w-100}
\begin{sphinxuseclass}{sd-shadow-sm}
\begin{sphinxuseclass}{sd-card-body}
\begin{figure}[htbp]
\centering

\noindent\sphinxincludegraphics{{4_intersecting_planes_7}.svg}
\end{figure}

\sphinxAtStartPar
2 coincident planes parallel to another: no solutions

\end{sphinxuseclass}
\end{sphinxuseclass}
\end{sphinxuseclass}
\end{sphinxuseclass}
\end{sphinxuseclass}
\end{sphinxuseclass}
\end{sphinxuseclass}
\end{sphinxuseclass}
\end{sphinxuseclass}
\end{sphinxuseclass}
\end{sphinxuseclass}
\end{sphinxuseclass}
\begin{sphinxuseclass}{sd-col}
\begin{sphinxuseclass}{sd-d-flex-row}
\begin{sphinxuseclass}{sd-col-4}
\begin{sphinxuseclass}{sd-col-xs-4}
\begin{sphinxuseclass}{sd-col-sm-4}
\begin{sphinxuseclass}{sd-col-md-4}
\begin{sphinxuseclass}{sd-col-lg-4}
\begin{sphinxuseclass}{sd-card}
\begin{sphinxuseclass}{sd-sphinx-override}
\begin{sphinxuseclass}{sd-w-100}
\begin{sphinxuseclass}{sd-shadow-sm}
\begin{sphinxuseclass}{sd-card-body}


\begin{figure}[htbp]
\centering

\noindent\sphinxincludegraphics{{4_intersecting_planes_8}.svg}
\end{figure}



\end{sphinxuseclass}
\end{sphinxuseclass}
\end{sphinxuseclass}
\end{sphinxuseclass}
\end{sphinxuseclass}
\end{sphinxuseclass}
\end{sphinxuseclass}
\end{sphinxuseclass}
\end{sphinxuseclass}
\end{sphinxuseclass}
\end{sphinxuseclass}
\end{sphinxuseclass}
\end{sphinxuseclass}
\end{sphinxuseclass}
\end{sphinxuseclass}
\end{sphinxuseclass}
\end{sphinxuseclass}
\end{sphinxuseclass}
\end{sphinxuseclass}
\end{sphinxuseclass}
\end{sphinxuseclass}
\sphinxAtStartPar
It is possible to classify such arrangements for higher dimensions \sphinxhyphen{} but it makes more sense to do it purely algebraically. That is precisely what linear algebra is useful for.
\label{_pages/4.2_Planes:intersecting-planes-example}
\begin{sphinxadmonition}{note}{Example 4.3.2}



\sphinxAtStartPar
Two non\sphinxhyphen{}parallel planes in \(\mathbb{R}^3\) are defined by the point\sphinxhyphen{}normal forms \(3x - 4y + z - 5 = 0\) and \(x + y - z - 2 = 0\) respectively. Find the intersection of these two planes.
\subsubsection*{Solution}

\sphinxAtStartPar
Since the planes are not parallel, they must intersect in a single line. We need to calculate the equation for this line, by first finding a direction vector, then finding a point on the line.

\sphinxAtStartPar
The normal vectors for these planes are given by the coefficients of \(x\), \(y\) and \(z\), and so they are \(\mathbf{n}_1 = (3, -4, 1)\) and \(\mathbf{n}_2 = (1, 1, -1)\). The direction vector of the line of intersection must be perpendicular to both \(\mathbf{n}_1\) and \(\mathbf{n}_2\) (the intersecting line must lie in both planes), therefore we need to construct a vector perpendicular to both of the normal vectors.

\sphinxAtStartPar
A vector perpendicular to \(\mathbf{n}_1\) and \(\mathbf{n}_2\) can be found by calculating \(\mathbf{n}_1 \times \mathbf{n}_2\) (cross product):
\begin{equation*}
\begin{split} \begin{align*}
    \mathbf{d} = \mathbf{n}_1 \times \mathbf{n}_2 =
    \begin{vmatrix}
        \mathbf{i} & \mathbf{j} & \mathbf{k} \\
        3 & -4 & 1 \\
        1 & 1 & -1
    \end{vmatrix} =
    \begin{pmatrix} 3 \\ 4 \\ 7 \end{pmatrix}.
\end{align*} \end{split}
\end{equation*}
\sphinxAtStartPar
In addition to the direction vector, we need a point on the intersection line which must lie on both planes \sphinxhyphen{} so we need to find a solution that satisfies:
\begin{equation*}
\begin{split} \begin{align*}
    3x - 4y + z &= 5, \\
    x + y - z &= 2
\end{align*} \end{split}
\end{equation*}
\sphinxAtStartPar
Adding these two equations gives:
\begin{equation*}
\begin{split} \begin{align*}
    3x - 4y + z + x + y -z &= 7 \\
    4x - 3y &= 7\\
    \therefore y &= \frac{4x - 7}{3}.
\end{align*} \end{split}
\end{equation*}
\sphinxAtStartPar
Let \(x = 1\); then \(y = -1\), and \(z = -2\) (by substitiution of these values into either plane equation). This point must lie on both planes. Then the line of intersection between the two planes is given by:
\begin{equation*}
\begin{split} \begin{align*}
    \mathbf{r} = \begin{pmatrix} 1 \\ -1 \\ -2 \end{pmatrix} + t
    \begin{pmatrix} 3 \\ 4 \\ 7 \end{pmatrix} =
    \begin{pmatrix} 1 + 3t \\ -1 + 4t \\ -2 + 7t \end{pmatrix}.
\end{align*} \end{split}
\end{equation*}\end{sphinxadmonition}

\sphinxstepscope


\section{Shortest distance problems}
\label{\detokenize{_pages/4.3_Shortest_distance_problems:shortest-distance-problems}}\label{\detokenize{_pages/4.3_Shortest_distance_problems::doc}}
\index{Shortest distance@\spxentry{Shortest distance}}\ignorespaces \phantomsection\label{\detokenize{_pages/4.3_Shortest_distance_problems:id1}}
\sphinxAtStartPar
Finding the shortest distance between two points is a common problem in coordinate geometry, we can make use of vector geometry to solve these kinds of problems.


\bigskip\hrule\bigskip


\index{Shortest distance@\spxentry{Shortest distance}!point to point@\spxentry{point to point}}\ignorespaces 

\subsection{Shortest distance between two points}
\label{\detokenize{_pages/4.3_Shortest_distance_problems:index-1}}\label{\detokenize{_pages/4.3_Shortest_distance_problems:shortest-distance-between-two-points}}\label{\detokenize{_pages/4.3_Shortest_distance_problems:id2}}
\sphinxAtStartPar
In \(\mathbb{R}^n\), the shortest distance \(d\) between two points with coordinates \(\mathbf{p}=(p_1, p_2, \ldots, p_n)\) and \(\mathbf{q} = (q_1, q_2, \ldots, q_n)\) is the length of a straight line segment connecting them. If we consider it as a vector, this line segment has length:
\begin{equation*}
\begin{split} d = \|\mathbf{p} - \mathbf{q}\| = \sqrt{\displaystyle\sum_{i=1}^n (p_i-q_i)^2}. \end{split}
\end{equation*}
\sphinxAtStartPar
Note that \(\|\mathbf{p} - \mathbf{q}\| = \|\mathbf{q} - \mathbf{p}\|\).


\bigskip\hrule\bigskip


\index{Shortest distance@\spxentry{Shortest distance}!point to line@\spxentry{point to line}}\ignorespaces 

\subsection{Shortest distance between a line and a point}
\label{\detokenize{_pages/4.3_Shortest_distance_problems:shortest-distance-between-a-line-and-a-point}}\label{\detokenize{_pages/4.3_Shortest_distance_problems:index-2}}
\sphinxAtStartPar
In \(\mathbb{R}^n\), the shortest distance \(d\) between a line \(\ell\) consisting of points of the form \(\mathbf{r} = \mathbf{p} + t\mathbf{d}\), and a point \(\mathbf{q}\) off the line, is the length of the vector \(\mathbf{q} - \mathbf{r}\), which is perpendicular to \(\mathbf{d}\) (\hyperref[\detokenize{_pages/4.3_Shortest_distance_problems:line-point-distance-figure}]{Fig.\@ \ref{\detokenize{_pages/4.3_Shortest_distance_problems:line-point-distance-figure}}}).

\begin{figure}[htbp]
\centering
\capstart

\noindent\sphinxincludegraphics[width=425\sphinxpxdimen]{{4_line_point_distance}.svg}
\caption{The shortest distance between a point \(\mathbf{q}\) and the line of points \(\mathbf{r} = \mathbf{p} + t \mathbf{d}\) is the length of the vector \(\mathbf{q} - \mathbf{r}\), which is perpendicular to \(\mathbf{d}\).}\label{\detokenize{_pages/4.3_Shortest_distance_problems:line-point-distance-figure}}\end{figure}

\sphinxAtStartPar
To find the value of the parameter \(t\) for which the point \(\mathbf{r}\) gives a perpendicular vector for \(\mathbf{q} - \mathbf{r}\) is perpendicular to \(\mathbf{d}\), we can use the fact that the dot product between two perpendicular vectors is zero. This product is:
\begin{equation*}
\begin{split} \mathbf{d} \cdot (\mathbf{r} - \mathbf{q}) = 0, \end{split}
\end{equation*}
\sphinxAtStartPar
Substituting in the {\hyperref[\detokenize{_pages/4.1_Lines:vector-equation-of-a-line-definition}]{\sphinxcrossref{vector equation of the line}}} \(\mathbf{r} = \mathbf{p} + t\mathbf{d}\), and rearranging to make \(t\) the subject gives:
\begin{equation*}
\begin{split} \begin{align*}
    \mathbf{d} \cdot (\mathbf{r} - \mathbf{q}) &= 0 \\
    \mathbf{d} \cdot (\mathbf{p} + t \mathbf{d} - \mathbf{q}) &= 0 \\
    \mathbf{d} \cdot \mathbf{p} + t \mathbf{d} \cdot \mathbf{d} - \mathbf{d} \cdot \mathbf{q} &= 0 \\
    t \mathbf{d} \cdot \mathbf{d} &= (\mathbf{q} - \mathbf{p}) \cdot \mathbf{d} \\
    \therefore t &= \frac{(\mathbf{q} - \mathbf{p}) \cdot \mathbf{d}}{\mathbf{d} \cdot \mathbf{d}}.
\end{align*} \end{split}
\end{equation*}
\sphinxAtStartPar
This gives the desired value of the parameter \(t\), which can be used to calculate the coordinates of the point \(\mathbf{r}\) and therefore find the shortest distance between the point \(\mathbf{q}\) and the line, which is \(\|\mathbf{q} - \mathbf{r}\|\)%
\begin{footnote}[1]\sphinxAtStartFootnote
The dot product of a vector \(\mathbf{a}\) with itself is sometimes denoted \(\mathbf{a}^2 = \mathbf{a} \cdot \mathbf{a}\).
%
\end{footnote}.
\label{_pages/4.3_Shortest_distance_problems:point-line-distance-theorem}
\begin{sphinxadmonition}{note}{Theorem 4.4.1 (The shortest distance between a point and a line)}



\sphinxAtStartPar
The shortest distance between the point with coordinates given by the vector \(\mathbf{q}\), and a line that passes through the point with coordinates given by the vector \(\mathbf{p}\) in the direction \(\mathbf{d}\), is \(\|\mathbf{q} - \mathbf{r}\|\), where \(\mathbf{r} = \mathbf{p} + t \mathbf{d}\) and:
\begin{equation}\label{equation:_pages/4.3_Shortest_distance_problems:point-line-t-equation}
\begin{split} t = \frac{(\mathbf{q} - \mathbf{p}) \cdot \mathbf{d}}{\mathbf{d} \cdot \mathbf{d}}. \end{split}
\end{equation}\end{sphinxadmonition}


\label{_pages/4.3_Shortest_distance_problems:point-line-distance-example}
\begin{sphinxadmonition}{note}{Example 4.4.1}



\sphinxAtStartPar
Find the shortest distance between the point with position vector \(\mathbf{q} = (2,2,2)\) and the line \((t,t-2, t+1)\).
\subsubsection*{Solution}

\sphinxAtStartPar
Start by writing the line in the form \(\mathbf{r} = \mathbf{p} + t \mathbf{d}\):
\begin{equation*}
\begin{split} \mathbf{r} = \begin{pmatrix} 0 \\ -2 \\ 1 \end{pmatrix} + t
\begin{pmatrix} 1 \\ 1 \\ 1 \end{pmatrix}, \end{split}
\end{equation*}
\sphinxAtStartPar
so the direction vector is \(\mathbf{d} = (1, 1, 1)\). Using equation \eqref{equation:_pages/4.3_Shortest_distance_problems:point-line-t-equation}
\begin{equation*}
\begin{split} \begin{align*}
    t &= \frac{\mathbf{d} \cdot \mathbf{q} - \mathbf{d} \cdot \mathbf{p}}{\mathbf{d} \cdot \mathbf{d}} \\ \\
    &= \frac{
        \begin{pmatrix} 1 \\ 1 \\ 1 \end{pmatrix} \cdot
        \begin{pmatrix} 2 \\ 2 \\ 2 \end{pmatrix} -
        \begin{pmatrix} 1 \\ 1 \\ 1 \end{pmatrix} \cdot
        \begin{pmatrix} 0 \\ -2 \\ 1 \end{pmatrix}}{
        \begin{pmatrix} 1 \\ 1 \\ 1 \end{pmatrix} \cdot
        \begin{pmatrix} 1 \\ 1 \\ 1 \end{pmatrix}} \\ \\
    &= \frac{6 + 1}{3} = \frac{7}{3}.
\end{align*} \end{split}
\end{equation*}
\sphinxAtStartPar
Using this value of \(t\) we can find the point on the line which is closest to \(\mathbf{q}\), which is:
\begin{equation*}
\begin{split} \begin{align*}
    \mathbf{r} =
    \begin{pmatrix} 0 \\ -2 \\ 1 \end{pmatrix} + \frac{7}{3}
    \begin{pmatrix} 1 \\ 1 \\ 1 \end{pmatrix} =
    \begin{pmatrix} 7/3 \\ 1/3 \\ 10/3 \end{pmatrix},
\end{align*} \end{split}
\end{equation*}
\sphinxAtStartPar
and the shortest distance is \(\|\mathbf{q} - \mathbf{r}\|\):
\begin{equation*}
\begin{split} \begin{align*}
    \mathbf{q} - \mathbf{r} &= \begin{pmatrix} 2 \\ 2 \\ 2 \end{pmatrix} -
    \begin{pmatrix} 7/3 \\ 1/3 \\ 10/3 \end{pmatrix} =
    \begin{pmatrix} - 1/3 \\ 5/3 \\ -4/3 \end{pmatrix}, \\ \\
    \therefore d &= \sqrt{\left(-\frac{1}{3}\right)^2 + \left(\frac{5}{3}\right)^2 + \left( -\frac{4}{3}\right)^2} \\ \\
    &= \sqrt{\frac{14}{3}} \approx 2.16.
\end{align*} \end{split}
\end{equation*}\end{sphinxadmonition}


\bigskip\hrule\bigskip


\index{Shortest distance@\spxentry{Shortest distance}!line to line@\spxentry{line to line}}\ignorespaces 

\subsection{Shortest distance between two lines}
\label{\detokenize{_pages/4.3_Shortest_distance_problems:shortest-distance-between-two-lines}}\label{\detokenize{_pages/4.3_Shortest_distance_problems:index-3}}
\sphinxAtStartPar
Given two lines \(\ell_1\) and \(\ell_2\) described by the equations \(\mathbf{r}_1 = \mathbf{p}_1 + t \mathbf{d}_1\) and \(\mathbf{r}_2 =  \mathbf{p}_2 + t \mathbf{d}_2\) in \(\mathbb{R}^n\), we have three situations to consider if we want to calculate the shortest distance between any two points on the two lines.
\begin{itemize}
\item {} 
\sphinxAtStartPar
If the two lines intersect, then the shortest distance is 0.

\item {} 
\sphinxAtStartPar
If the two lines are parallel, then any point on \(\ell_1\) can be used to calculate the shortest distance between \(\ell_1\) and \(\ell_2\). Hence, we choose an arbitrary point on \(\ell_1\), and apply the method for {\hyperref[\detokenize{_pages/4.3_Shortest_distance_problems:point-line-distance-theorem}]{\sphinxcrossref{finding the distance between a point and a line}}}.

\item {} 
\sphinxAtStartPar
If the two lines are skew, then the shortest distance is the length of the chord between them \sphinxhyphen{} this will be a vector that is perpendicular to both \(\ell_1\) and \(\ell_2\) (\hyperref[\detokenize{_pages/4.3_Shortest_distance_problems:line-line-distance-figure}]{Fig.\@ \ref{\detokenize{_pages/4.3_Shortest_distance_problems:line-line-distance-figure}}}).

\end{itemize}

\begin{figure}[htbp]
\centering
\capstart

\noindent\sphinxincludegraphics[width=350\sphinxpxdimen]{{4_line_line_distance}.svg}
\caption{The shortest distance between skew lines is the length of the chord which is perpendicular to both lines.}\label{\detokenize{_pages/4.3_Shortest_distance_problems:line-line-distance-figure}}\end{figure}

\sphinxAtStartPar
If \(\mathbf{r}_1\) and \(\mathbf{r}_2\) are points on the lines \(\ell_1\) and \(\ell_2\) respectively, then the chord \(\mathbf{r}_1 \to  \mathbf{r}_2\) which is perpendicular to both lines has the direction vector \(\mathbf{n} = \mathbf{d}_1 \times \mathbf{d}_2\) (the cross product of the direction vectors of the two lines). Let \(d\) denote the distance between \(\mathbf{r}_1\) and \(\mathbf{r}_2\). Then the vector representing the chord is given by:
\begin{equation}\label{equation:_pages/4.3_Shortest_distance_problems:line-line-distance-equation-1}
\begin{split} \mathbf{r}_2 - \mathbf{r}_1 = d \left( \frac{\mathbf{d}_1 \times \mathbf{d}_2}{\| \mathbf{d}_1 \times \mathbf{d}_2\|} \right), \end{split}
\end{equation}
\sphinxAtStartPar
The term in brackets here is the direction vector of the chord, which will be a normal vector to both planes. We can write \(\hat{\mathbf{n}} = \dfrac{\mathbf{d}_1 \times \mathbf{d}_2}{\|\mathbf{d}_1 \times \mathbf{d}_2\|}\). Now we substitute into  equation \eqref{equation:_pages/4.3_Shortest_distance_problems:line-line-distance-equation-1} the equations of \(\mathbf{r}_1\) and \(\mathbf{r}_2\), to give:
\begin{equation*}
\begin{split} \begin{align*}
    \mathbf{r}_2 - \mathbf{r}_1 &= d \hat{\mathbf{n}}\\ \\
    (\mathbf{p}_2 + t_2 \mathbf{d}_2) - (\mathbf{p}_1 + t_1 \mathbf{d}_1)
    &= d \hat{\mathbf{n}}\\ \\
    (\mathbf{p}_2 + t_2 \mathbf{d}_2) \cdot \hat{\mathbf{n}} - (\mathbf{p}_1 + t_1 \mathbf{d}_1) \cdot \hat{\mathbf{n}}
    &= d \hat{\mathbf{n}} \cdot \hat{\mathbf{n}} \\ \\
    \mathbf{p}_2 \cdot \hat{\mathbf{n}} + t_2 \mathbf{d}_2 \cdot \hat{\mathbf{n}} - \mathbf{p}_1 \cdot \hat{\mathbf{n}} - t_1 \mathbf{d}_1 \cdot \hat{\mathbf{n}}
    &= d \hat{\mathbf{n}} \cdot \hat{\mathbf{n}}.
\end{align*} \end{split}
\end{equation*}
\sphinxAtStartPar
Since \(\hat{\mathbf{n}}\) is perpendicular to both \(\ell_1\) and \(\ell_2\), we know that \(\mathbf{d}_1 \cdot \hat{\mathbf{n}} = \mathbf{d}_2 \cdot \hat{\mathbf{n}} = 0\), and we also have that \(\hat{\mathbf{n}} \cdot \hat{\mathbf{n}} = 1\) (since it is of unit length, and \(\mathbf{a} \cdot \mathbf{a} = \|\mathbf{a}\|^2\) from \sphinxcode{\sphinxupquote{dot\sphinxhyphen{}product\sphinxhyphen{}properties\sphinxhyphen{}theorem}} ).

\sphinxAtStartPar
Then the last line above simplifies to:
\begin{equation*}
\begin{split} \mathbf{p}_2 \cdot \hat{\mathbf{n}} - \mathbf{p}_1 \cdot \hat{\mathbf{n}} = d, \end{split}
\end{equation*}
\sphinxAtStartPar
or
\begin{equation*}
\begin{split} d = (\mathbf{p}_2 - \mathbf{p}_1) \cdot \hat{\mathbf{n}}. \end{split}
\end{equation*}\label{_pages/4.3_Shortest_distance_problems:line-line-distance-theorem}
\begin{sphinxadmonition}{note}{Theorem 4.4.2 (The shortest distance between two skew lines)}



\sphinxAtStartPar
The shortest distance between two skew lines \(\mathbf{r}_1 = \mathbf{p}_1 + t_1 \mathbf{d}_1\) and \(\mathbf{r}_2 = \mathbf{p}_2 + t_2 \mathbf{d}_2\) is
\begin{equation}\label{equation:_pages/4.3_Shortest_distance_problems:line-line-distance-equation}
\begin{split} d = (\mathbf{p}_2 - \mathbf{p}_1) \cdot \hat{\mathbf{n}}.\end{split}
\end{equation}
\sphinxAtStartPar
where \(\hat{\mathbf{n}} = \dfrac{\mathbf{d}_1 \times \mathbf{d}_2}{\|\mathbf{d}_1 \times \mathbf{d}_2\|}\).
\end{sphinxadmonition}


\label{_pages/4.3_Shortest_distance_problems:line-line-distance-example}
\begin{sphinxadmonition}{note}{Example 4.4.2}



\sphinxAtStartPar
Find the shortest distance between the two skew lines \(\ell_1\) and \(\ell_2\) defined by \((t_1, 1 + 4t_1, 3 + 2 t_1)\) and \((1, 1 + 2t_2, 3 + 4t_2)\) respectively.
\subsubsection*{Solution}

\sphinxAtStartPar
First, we need to identify the direction vectors \(\mathbf{d}_1\) and \(\mathbf{d}_2\). Writing \(\ell_1\) and \(\ell_2\) in the form \(\mathbf{r} = \mathbf{p} + t \mathbf{d}\):
\begin{equation*}
\begin{split} \begin{align*}
    \mathbf{r}_1 &= \mathbf{p}_1 + t_1 \mathbf{d}_1 = \begin{pmatrix} 0 \\ 1 \\ 3 \end{pmatrix} + t_1
    \begin{pmatrix} 1 \\ 4 \\ 2 \end{pmatrix} , \\ \\
    \mathbf{r}_2 &= \mathbf{p}_2 + t_2 \mathbf{d}_2 = \begin{pmatrix} 1 \\ 1 \\ 3 \end{pmatrix} + t_2
    \begin{pmatrix} 0 \\ 2 \\ 4 \end{pmatrix},
\end{align*} \end{split}
\end{equation*}
\sphinxAtStartPar
therefore \(\mathbf{d}_1 = (1, 4, 2)\) and \(\mathbf{d}_2 = (0, 2, 4)\). Now we can calculate \(\hat{\mathbf{n}}\):
\begin{equation*}
\begin{split} \begin{align*}
    \mathbf{n} &= \mathbf{d}_1 \times \mathbf{d}_2
    = \begin{vmatrix}
        \mathbf{i} & \mathbf{j} & \mathbf{k} \\
        1 & 4 & 2 \\
        0 & 2 & 4
    \end{vmatrix} =
    \begin{pmatrix} 12 \\ -4 \\ 2 \end{pmatrix}, \\ \\
    \|\mathbf{n}\| &= \sqrt{12^2 + (-4)^2 + 2^2} = \sqrt{164} = 2\sqrt{41}, \\ \\
    \therefore \hat{\mathbf{n}} &= \frac{\mathbf{n}}{\|\mathbf{n}\|}
    = \frac{1}{2\sqrt{41}}
    \begin{pmatrix} 12 \\ -4 \\ 2 \end{pmatrix}
    =
    \begin{pmatrix} 6\sqrt{41}/41 \\ -2\sqrt{41}/41 \\ \sqrt{41}/41 \end{pmatrix}.
\end{align*} \end{split}
\end{equation*}
\sphinxAtStartPar
Note that since \(\mathbf{n}\) is non\sphinxhyphen{}zero, \(\ell_1\) and \(\ell_2\) are skew lines. Using equation \eqref{equation:_pages/4.3_Shortest_distance_problems:line-line-distance-equation}
\begin{equation*}
\begin{split} \begin{align*}
    d &= (\mathbf{p}_2 - \mathbf{p}_1) \cdot \hat{\mathbf{n}} \\
    &= \left( \begin{pmatrix} 1 \\ 1 \\ 3 \end{pmatrix} -
    \begin{pmatrix} 0 \\ 1 \\ 3 \end{pmatrix} \right) \cdot
    \begin{pmatrix} 6\sqrt{41}/41 \\ -2\sqrt{41}/41 \\ \sqrt{41}/41 \end{pmatrix} \\ \\
    &= \begin{pmatrix} 1 \\ 0 \\ 0 \end{pmatrix} \cdot
    \begin{pmatrix} 6\sqrt{41}/41 \\ -2\sqrt{41}/41 \\ \sqrt{41}/41 \end{pmatrix} \\ \\
    &= \frac{6 \sqrt{41}}{41}.
\end{align*} \end{split}
\end{equation*}\end{sphinxadmonition}


\bigskip\hrule\bigskip


\sphinxstepscope


\section{Co\sphinxhyphen{}ordinate Geometry Exercises}
\label{\detokenize{_pages/4.4_Coordinate_geometry_exercises:co-ordinate-geometry-exercises}}\label{\detokenize{_pages/4.4_Coordinate_geometry_exercises::doc}}
\sphinxAtStartPar
Answer the following exercises based on the content from this chapter. The solutions can be found in the {\hyperref[\detokenize{_pages/A4_Coordinate_geometry_exercises_solutions:co-ordinate-geometry-exercises-solutions-section}]{\sphinxcrossref{\DUrole{std,std-ref}{appendices}}}}.
\phantomsection \label{exercise:geometry-ex-line-plane-equations}

\begin{sphinxadmonition}{note}{Exercise 4.5.1}



\sphinxAtStartPar
Given the following position vectors in \(\mathbb{R}^3\)
\begin{equation*}
\begin{split} \begin{align*} 
    \mathbf{a} &= \begin{pmatrix} 2 \\ 1 \\ 0 \end{pmatrix}, &
    \mathbf{b} &= \begin{pmatrix} 1 \\ 1 \\ 0 \end{pmatrix}, &
    \mathbf{c} &= \begin{pmatrix} 3 \\ -1 \\ 4 \end{pmatrix}, &
    \mathbf{d} &= \begin{pmatrix} 5 \\ 2 \\ 6 \end{pmatrix}.
\end{align*} \end{split}
\end{equation*}
\sphinxAtStartPar
find:

\sphinxAtStartPar
(a)   the equation of the line that passes through \(\mathbf{a}\) and \(\mathbf{b}\);

\sphinxAtStartPar
(b)   the equation of the line that passes through \(\mathbf{c}\) and \(\mathbf{d}\);

\sphinxAtStartPar
(c)   the equation of the plane which passes through \(\mathbf{a}\), \(\mathbf{b}\) and \(\mathbf{c}\) lie;

\sphinxAtStartPar
(d)   the equation of the plane upon which passes through \(\mathbf{b}\), \(\mathbf{c}\) and \(\mathbf{d}\).
\end{sphinxadmonition}
\phantomsection \label{exercise:geometry-ex-line-equation-1}

\begin{sphinxadmonition}{note}{Exercise 4.5.2}



\sphinxAtStartPar
Find the equation of the line that passes through the point with position vector \((3, 2, 1)^\mathsf{T}\) which is parallel to \(2 \mathbf{i} + \mathbf{j} + 3 \mathbf{k}\).
\end{sphinxadmonition}
\phantomsection \label{exercise:geometry-ex-plane-equation-1}

\begin{sphinxadmonition}{note}{Exercise 4.5.3}



\sphinxAtStartPar
Find the equation of the plane that passes through the point with position vector \((3, 2, 5)^\mathsf{T}\) which has a normal vector \(\mathbf{n} = (2, 1, 3)^\mathsf{T}\).
\end{sphinxadmonition}
\phantomsection \label{exercise:geometry-ex-plane-1}

\begin{sphinxadmonition}{note}{Exercise 4.5.4}



\sphinxAtStartPar
A plane has the equation \(3x - 2y + z = 10\). Identify the normal to the plane and find the co\sphinxhyphen{}ordinates of 2 points on the plane having \(z = 2\).
\end{sphinxadmonition}
\phantomsection \label{exercise:geometry-ex-line-point-distance}

\begin{sphinxadmonition}{note}{Exercise 4.5.5}



\sphinxAtStartPar
Two lines in \(\mathbb{R}^3\) are defined by \(\ell_1: (1 + 2t, -t, 1 + 3t)^\mathsf{T}\) and \(\ell_2: (1 + 2t, 4, 7 - t)^\mathsf{T}\) respectively.

\sphinxAtStartPar
(a)   find the intersection of the lines or show they are skew;

\sphinxAtStartPar
(b)   find the distance between the point with position vector \(\mathbf{p} = (0, -1, 3)^\mathsf{T}\) and \(\ell_1\);

\sphinxAtStartPar
(c)   find the shortest distance between the lines.
\end{sphinxadmonition}
\phantomsection \label{exercise:geometry-ex-line-plane-intersection}

\begin{sphinxadmonition}{note}{Exercise 4.5.6}



\sphinxAtStartPar
Find the point where the line \(\ell:(1 + 2t, 2 + t, -1 + 4t)^\mathsf{T}\) meets the plane \(6x - y - 4z = 3\).
\end{sphinxadmonition}
\phantomsection \label{exercise:geometry-ex-point-plane-distance}

\begin{sphinxadmonition}{note}{Exercise 4.5.7}



\sphinxAtStartPar
Consider the diagram below that shows a plane that passes through the point \(\mathbf{p}\) and has normal vector \(\mathbf{n}\) and the point with position \(\mathbf{q}\) not on the plane.

\begin{figure}[htbp]
\centering

\noindent\sphinxincludegraphics[width=350\sphinxpxdimen]{{4_point_plane_distance}.svg}
\end{figure}

\sphinxAtStartPar
Using the {\hyperref[\detokenize{_pages/3.3_Dot_and_cross_products:dot-product-definition}]{\sphinxcrossref{geometric definition of a dot product}}} derive an expression for calculating the shortest distance between a point and a plane. Use your expression to find the shortest distance from the point with position vector \((2, 4, -3)^\mathsf{T}\) to the plane \(6x - y - 4z = 3\).
\end{sphinxadmonition}

\sphinxstepscope

\index{Vector spaces@\spxentry{Vector spaces}}\ignorespaces 

\chapter{Vector spaces}
\label{\detokenize{_pages/5.0_Vector_spaces:vector-spaces}}\label{\detokenize{_pages/5.0_Vector_spaces:index-0}}\label{\detokenize{_pages/5.0_Vector_spaces:vector-spaces-chapter}}\label{\detokenize{_pages/5.0_Vector_spaces::doc}}
\sphinxAtStartPar
We have been working with vectors, and have seen that there are ways to combine vectors and scalars, in ways similar to how we can manipulate matrices, or combine numbers and variables to create algebraic expressions. This similarity is not a coincidence \sphinxhyphen{} structures like this exist throughout maths, and we can define them more formally by considering collections of objects within a space that obey particular rules, called a \sphinxstylestrong{vector space}.

\sphinxAtStartPar
A vector space consists of a set of objects that satisfy the axioms of vector addition and scalar multiplication. As the name suggests, vectors in {\hyperref[\detokenize{_pages/3.0_Vectors:euclidean-space-section}]{\sphinxcrossref{\DUrole{std,std-ref}{Euclidean space}}}} satisfy these axioms, and can form a vector space \sphinxhyphen{} but so do lots of other types of mathematical objects. The term \sphinxstyleemphasis{vectors}, when dealing with vector spaces, could apply to matrices, polynomial functions, numbers etc.

\sphinxAtStartPar
Vector spaces are more abstract than much of the maths you’ll be used to. To give a sense of why they’re useful, consider these three example, taken from  Geoffrey Scott’s why we need vector spaces:
\begin{itemize}
\item {} 
\sphinxAtStartPar
Problem 1: Find \(x_1, x_2, x_3 \in \mathbb{R}\) that satisfy

\end{itemize}
\begin{equation*}
\begin{split} x_1 \begin{pmatrix} 2 \\ -1 \\ 1 \end{pmatrix} + x_2 \begin{pmatrix} 1 \\ 2 \\ 0 \end{pmatrix} + x_3 \begin{pmatrix} 3 \\ 1 \\ -1 \end{pmatrix} = \begin{pmatrix} 6 \\ 2 \\ 0 \end{pmatrix}. \end{split}
\end{equation*}\begin{itemize}
\item {} 
\sphinxAtStartPar
Problem 2: Find \(x_1, x_2, x_3 \in \mathbb{R}\) such that

\end{itemize}
\begin{equation*}
\begin{split} x_1 (2t^2 - t + 1) + x_2 (t^2 + 2t -2) + x_3 (0t^2 + t) = 2t^2 + 5t + 1.\end{split}
\end{equation*}\begin{itemize}
\item {} 
\sphinxAtStartPar
Problem 3: Find \(x_1, x_2, \ldots \in \mathbb{C}\) such that

\end{itemize}
\begin{equation*}
\begin{split} \sum_{n=1}^\infty x_n cos(n \pi t) = e^{5nt}. \end{split}
\end{equation*}
\sphinxAtStartPar
These three problems are each described using three different mathematical objects: Problem 1 uses vectors, problem 2 uses polynomials and problem 3 uses trigonometric functions. But these three problems all involve solving the same kind of problem \sphinxhyphen{} finding how many of each of the items to combine to create the desired result, which is fundamentally about solving a system of linear equations.

\sphinxAtStartPar
We know how to solve problem 1 using the methods from the {\hyperref[\detokenize{_pages/2.0_Linear_systems:systems-of-linear-equations-chapter}]{\sphinxcrossref{\DUrole{std,std-ref}{chapter on systems of linear equations}}}} \sphinxhyphen{} so can we use these methods to solve problems 2 and 3 as well, and other problems using different mathematical objects?

\sphinxAtStartPar
In general, when we encounter a situation like this, we could say:
\begin{quote}

\sphinxAtStartPar
“Suppose you have a collection of column vectors \sphinxstyleemphasis{or} polynomials \sphinxstyleemphasis{or} functions \sphinxstyleemphasis{or} any other type of mathematical object that can be multiplied by numbers and added together\(\ldots\)”
\end{quote}

\sphinxAtStartPar
Every time we encounter a similar problem to the ones above, we would need to state a theorem about this particular kind of problem and prove it before we can proceed. Instead of considering each of them separately, we define a \sphinxstyleemphasis{vector space} as being made up of any type of mathematical object that can be multiplied by numbers and added together in this way, and can rephrase the above more concisely as:
\begin{quote}

\sphinxAtStartPar
“Let \(V\) be a vector space \(\ldots\)”
\end{quote}

\sphinxstepscope


\section{Definitions}
\label{\detokenize{_pages/5.1_Vector_spaces_definitions:definitions}}\label{\detokenize{_pages/5.1_Vector_spaces_definitions:vector-spaces-definitions-section}}\label{\detokenize{_pages/5.1_Vector_spaces_definitions::doc}}\label{_pages/5.1_Vector_spaces_definitions:field-definition}
\begin{sphinxadmonition}{note}{Definition 5.1.1 (Field)}



\sphinxAtStartPar
A \sphinxstylestrong{field} is a set \(F\) together with two \sphinxstyleemphasis{binary operations}, which each combine two elements of \(F\), called \sphinxstyleemphasis{addition} and \sphinxstyleemphasis{multiplication}. If \(a,b \in F\) then addition is written as \(a + b\), and multiplication is written as \(a \times b\).
\end{sphinxadmonition}

\sphinxAtStartPar
The classic example of a field is a set of numbers, e.g. the real numbers \(\mathbb{R}\), which will suffice for what we’re doing here. However, a field can be any set of objects for which the above definition is satisfied, and you may meet other examples of fields later in your studies.
\label{_pages/5.1_Vector_spaces_definitions:vector-space-definition}
\begin{sphinxadmonition}{note}{Definition 5.1.2 (Vector space)}



\sphinxAtStartPar
A \sphinxstylestrong{vector space} over a field \(F\) is a non\sphinxhyphen{}empty set \(V\) of objects called vectors, on which the binary operations below are defined
\begin{itemize}
\item {} 
\sphinxAtStartPar
\sphinxstylestrong{addition} \((+)\): \(V \times V \to V\). Given two elements \(u, v \in V\) we can ‘add’ them together to obtain another element \(u + v \in V\).

\item {} 
\sphinxAtStartPar
\sphinxstylestrong{scalar multiplication} \((\cdot)\): \(F \times V \to V\). Given \(\alpha \in F\) and \(u \in V\) we can ‘multiply’ \(u\) by \(\alpha\) to obtain another element \(\alpha \cdot u \in V\).

\end{itemize}
\end{sphinxadmonition}

\sphinxAtStartPar
For example, a field could be the set of all real numbers \(\mathbb{R}\) and the vector set could be the set of all vectors (as we have previously used the word) in \(\mathbb{R}^3\) \sphinxhyphen{} tuples with three entries, each of which is an element of \(\mathbb{R}\). Then we can define a vector space consisting of that set of vectors over that field. We can add two vectors together, and multiply a vector by a scalar.

\sphinxAtStartPar
We use the word ‘over’ to indicate that the set of vectors is combined with the scalars from the given field by scalar multiplication: the set of vectors \(V\), over the field \(F\), is a vector space. It may be that the vectors themselves are made from elements of the field, combined in some way (e.g. entries in a tuple, or coefficients in a polynomial) \sphinxhyphen{} and this is often the case, since then scalar multiplication involves multiplying by other elements of the field, and this is a defined operation within the field.

\sphinxAtStartPar
Not every set of objects can be called a vector space. To be a vector space, we need the set \(V\) to satisfy a particular set of axioms:

\index{Vector spaces@\spxentry{Vector spaces}!axioms@\spxentry{axioms}}\ignorespaces \label{_pages/5.1_Vector_spaces_definitions:vector-space-axioms}
\begin{sphinxadmonition}{note}{Axiom 5.1.1 (Axioms of a vector space)}



\sphinxAtStartPar
\(V\) is a vector space (over \(F\)) if all of the following axioms are satisfied, for all \(u, v, w \in V\) and \(\alpha, \beta \in F\):
\begin{itemize}
\item {} 
\sphinxAtStartPar
\sphinxstylestrong{A1}: Associativity of vector addition: \(u + (v + w) = (u + v) + w\)

\item {} 
\sphinxAtStartPar
\sphinxstylestrong{A2}: Commutativity of vector addition: \(u + v = v + u\)

\item {} 
\sphinxAtStartPar
\sphinxstylestrong{A3}: Identity element of vector addition: there exists an element \(\mathbf{0} \in V\) such that \(u + \mathbf{0} = u\) for all \(v \in V\)

\item {} 
\sphinxAtStartPar
\sphinxstylestrong{A4}: Additive inverse: For every \(u \in V\) there exists an element \(- v \in V\), called the \sphinxstylestrong{additive inverse}, such that \(u + (- u) = \mathbf{0}\)

\item {} 
\sphinxAtStartPar
\sphinxstylestrong{M1}: Associativity of scalar multiplication: \(\alpha(\beta u) = (\alpha \beta) u\)

\item {} 
\sphinxAtStartPar
\sphinxstylestrong{M2}: Identity element of scalar multiplication: there exists an element \(1 \in F\), called the \sphinxstylestrong{multiplicative identity}, such that \(1 u = u\)

\item {} 
\sphinxAtStartPar
\sphinxstylestrong{M3}: Distributivity of scalar multiplication with respect to vector addition: \(\alpha(u + v) = \alpha u + \alpha v\)

\item {} 
\sphinxAtStartPar
\sphinxstylestrong{M4}: Distributivity of scalar multiplication with respect to addition: \((\alpha + \beta)u = \alpha u + \beta u\)

\end{itemize}
\end{sphinxadmonition}

\sphinxAtStartPar
The first four axioms are concerned with addition, and the next four axioms are concerned with scalar multiplication. When the scalars are real numbers, e.g., \(F = \mathbb{R}\), we call \(V\) a \sphinxstyleemphasis{‘real vector space’}, or a \sphinxstyleemphasis{‘vector space over the field of real numbers’}. When the scalars are complex numbers, e.g., \(F=\mathbb{C}\), we call \(V\) a \sphinxstyleemphasis{‘complex vector space’}.

\sphinxAtStartPar
From the axioms, we can derive some basic properties for multiplication by scalars in a vector space.
\label{_pages/5.1_Vector_spaces_definitions:properties-of-vector-spaces-theorem}
\begin{sphinxadmonition}{note}{Theorem 5.1.1 (Properties of vector spaces)}



\sphinxAtStartPar
Let \(V\) be a vector space over \(F\) and \(u \in V,\alpha\in F\). Then the following hold:
\begin{itemize}
\item {} 
\sphinxAtStartPar
\(\alpha \cdot 0 = 0\);

\item {} 
\sphinxAtStartPar
\(0 \cdot u = 0\);

\item {} 
\sphinxAtStartPar
if \(\alpha u = 0\) then either \(\alpha = 0\) or \(u = 0\);

\item {} 
\sphinxAtStartPar
\(-(\alpha u) = (-\alpha)u = \alpha (-u)\).

\end{itemize}
\end{sphinxadmonition}

\sphinxAtStartPar
The name \sphinxstyleemphasis{‘vector space’} comes from the fact that vectors are a natural object we can use to create such spaces. Indeed, Euclidean space \(\mathbb{R}^n\), as introduced in {\hyperref[\detokenize{_pages/3.0_Vectors:vectors-chapter}]{\sphinxcrossref{\DUrole{std,std-ref}{Vectors}}}}, is a vector space over \(F=\mathbb{R}\) consisting of vectors of \(n\) coordinates. Addition and scalar multiplication, in the sense of a vector space, for \(\mathbb{R}^n\) are as given in {\hyperref[\detokenize{_pages/3.1_Vector_arithmetic:vector-addition-definition}]{\sphinxcrossref{vector addition}}} and {\hyperref[\detokenize{_pages/3.1_Vector_arithmetic:scalar-multiplication-of-a-vector-definition}]{\sphinxcrossref{scalar multiplication}}} respectively.

\sphinxAtStartPar
Euclidean space, with vectors of coordinates, is often the first example of a vector space that a student meets. But on its own, it doesn’t quite demonstrate the power behind this construction. The real motivation for the study of vector spaces comes from the fact that many more abstract sets, like the set of differentiable functions or the set of matrices, can be considered as vector spaces. Everything that we learn about vector spaces can be then applied to any set which satisfies the definition of a vector space. This means that the set of \(m\times n\) matrices with matrix addition and scalar multiplication, or the set of all polynomials of degree at most \(n\), in some sense, behave in a very similar fashion to Euclidean space.
\label{_pages/5.1_Vector_spaces_definitions:real-numbers-vector-space-example}
\begin{sphinxadmonition}{note}{Example 5.1.1}



\sphinxAtStartPar
(i)   Show that the set of real numbers \$\textbackslash{}mathbb\{R\}, over itself, is a vector space.

\sphinxAtStartPar
(ii)   Consider the set of all polynomials of degree at most \(n\) with real coefficients, which we denote \(P_n(\mathbb{R})\):
\begin{equation*}
\begin{split} \begin{align*}
    p(x) := a_0x^0 + a_1x^1 + a_2x^2 + \cdots + a_nx^n = \sum_{i=0}^n a_ix^i,
\end{align*} \end{split}
\end{equation*}
\sphinxAtStartPar
where \(a_i \in \mathbb{R}\) and \(x\) is some variable. Show that \(P(\mathbb{R}_n)\), over the field \(\mathbb{R}\), is a vector space.
\subsubsection*{Solution}

\sphinxAtStartPar
(i)   We need to check that all the axioms of vector spaces are satisfied. Here \(V = \mathbb{R}\) and \(F = \mathbb{R}\), so let \(u, v, w \in \mathbb{R}\) and \(\alpha, \beta \in \mathbb{R}\).
\begin{itemize}
\item {} 
\sphinxAtStartPar
A1: \(u + (v + w) = (u + v) + w \quad \checkmark\)

\item {} 
\sphinxAtStartPar
A2: \(u + v = v + u \quad \checkmark\)

\item {} 
\sphinxAtStartPar
A3: \(u + 0 = u \quad \checkmark\) (i.e., 0 is the identity element of addition)

\item {} 
\sphinxAtStartPar
A4: \(u + (-u) = 0 \quad \checkmark\) (i.e., the negative of any number is the additive inverse)

\item {} 
\sphinxAtStartPar
M1: \(\alpha(\beta u) = (\alpha \beta) u \quad \checkmark\)

\item {} 
\sphinxAtStartPar
M2: \(1 u = u \quad \checkmark\) (i.e., 1 is the multiplicative identity of all numbers)

\item {} 
\sphinxAtStartPar
M3: \(\alpha(u + v) = \alpha u + \alpha v \quad \checkmark\)

\item {} 
\sphinxAtStartPar
M4: \((\alpha + \beta) u = \alpha u + \beta u \quad \checkmark\)

\end{itemize}

\sphinxAtStartPar
(ii)   Let \(u(x), v(x), w(x) \in P_n(\mathbb{R})\) where \(u_i, v_i, w_i \in \mathbb{R}\) are the coefficients for \(u(x)\), \(v(x)\) and \(w(x)\) respectively and \(k\in \mathbb{R}\) is some scalar. Then we define:
\begin{itemize}
\item {} 
\sphinxAtStartPar
Vector addition:

\end{itemize}
\begin{equation*}
\begin{split} \begin{align*}
    u(x) + v(x) &= (u_0 + v_0)x^0 + (u_1 + v_1)x^1 + \cdots + (u_n + v_n)x^n \\
    &= \sum_{i=0}^n (u_i + v_i)x^i.
\end{align*} \end{split}
\end{equation*}\begin{itemize}
\item {} 
\sphinxAtStartPar
Scalar multiplication:

\end{itemize}
\begin{equation*}
\begin{split} \begin{align*}
    k \cdot u(x) &= ku_0x^0 + ku_1x^1 + \cdots + ku_nx^n \\
    &= \sum_{i=0}^n ku_ix^i.
\end{align*} \end{split}
\end{equation*}\begin{itemize}
\item {} 
\sphinxAtStartPar
The additive identity element is the zero polynomial

\end{itemize}
\begin{equation*}
\begin{split} \begin{align*}
    0 = 0x^0 + 0x^1 + \cdots + 0x^n.
\end{align*} \end{split}
\end{equation*}\begin{itemize}
\item {} 
\sphinxAtStartPar
The additive inverse to a polynomial \(u(x)\) is

\end{itemize}
\begin{equation*}
\begin{split} \begin{align*}
    -u(x) &= (-u_0)x^0 + (-u_1)x^2 + \cdots + (-u_n)x^n \\
    &= \sum_{i=0}^n (-u_i)x^i \\
    &= -\sum_{i=0}^n u_ix^i.
\end{align*} \end{split}
\end{equation*}
\sphinxAtStartPar
Checking the that axioms of vector spaces are satisfied
\begin{itemize}
\item {} 
\sphinxAtStartPar
A1: Associativity of addition

\end{itemize}
\begin{equation*}
\begin{split} \begin{align*}
    u(x) + (v(x) + w(x)) &= \displaystyle \sum_{i=0}^n (u_i + (v_i + w_i))x^i \\
    &= \displaystyle \sum_{i=0}^n ((u_i + v_i) + w_i)x^i \\
    &= (u(x) + v(x)) + w(x) \quad \checkmark
\end{align*} \end{split}
\end{equation*}\begin{itemize}
\item {} 
\sphinxAtStartPar
A2: Commutativity of addition

\end{itemize}
\begin{equation*}
\begin{split} \begin{align*}
    u(x) + v(x) &= \displaystyle \sum_{i=0}^n (u_i + v_i)x^i \\
    &= \displaystyle \sum_{i=0}^n (v_i + u_i)x^i \\
    &= v(x) + u(x) \quad \checkmark
\end{align*} \end{split}
\end{equation*}\begin{itemize}
\item {} 
\sphinxAtStartPar
A3: Identity element for addition

\end{itemize}
\begin{equation*}
\begin{split} \begin{align*}
    u(x) + 0 &= \displaystyle \sum_{i=0}^n (u_i + 0)x^i \\
    &= \displaystyle \sum_{i=0}^n u_ix^i \\
    &= u(x) \quad \checkmark
\end{align*} \end{split}
\end{equation*}\begin{itemize}
\item {} 
\sphinxAtStartPar
A4: Inverse element for addition

\end{itemize}
\begin{equation*}
\begin{split} \begin{align*}
    u(x) + (-u(x)) &= \displaystyle \sum_{i=0}^n(u_i + (-u_i)) x^i \\
    &= 0 \quad \checkmark
\end{align*} \end{split}
\end{equation*}\begin{itemize}
\item {} 
\sphinxAtStartPar
M1: Associativity of scalar multiplication

\end{itemize}
\begin{equation*}
\begin{split} \begin{align*}
    \alpha(\beta u(x)) &= \alpha \displaystyle \sum_{i=0}^n \beta u_i x^i \\
    &= \alpha \beta \displaystyle \sum_{i=0}^n u_i x^i \\
    &= (\alpha \beta) u(x) \quad \checkmark
\end{align*} \end{split}
\end{equation*}\begin{itemize}
\item {} 
\sphinxAtStartPar
M2: Identity element for scalar multiplication

\end{itemize}
\begin{equation*}
\begin{split} \begin{align*}
    1 \cdot u(x) &= \displaystyle \sum_{i=0}^n 1 u_i x^i \\
    &= \displaystyle \sum_{i=0}^n u_i x^i \\
    &= u(x) \quad \checkmark
\end{align*} \end{split}
\end{equation*}\begin{itemize}
\item {} 
\sphinxAtStartPar
M3: Distributivity of scalar multiplication

\end{itemize}
\begin{equation*}
\begin{split} \begin{align*}
    \alpha (u(x) + v(x)) &= \alpha \sum_{i=0}^n (u_i + v_i) x^i \\
    &= \alpha \sum_{i=0}^n u_i x^i + \alpha \sum_{i=0}^n v_i x^i \\
    &= \alpha u(x) + \alpha v(x) \quad \checkmark
\end{align*} \end{split}
\end{equation*}\begin{itemize}
\item {} 
\sphinxAtStartPar
M4: Distributivity of scalar multiplication over addition

\end{itemize}
\begin{equation*}
\begin{split} \begin{align*}
    (\alpha + \beta) u(x) &= (\alpha + \beta) \displaystyle \sum_{i=0}^n u_i x^i \\
    &= \alpha \displaystyle \sum_{i=0}^n u_i x^i + \beta \displaystyle \sum_{i=0}^n u_i x^i \\
    &= \alpha u(x) + \beta u(x) \quad \checkmark
\end{align*} \end{split}
\end{equation*}\end{sphinxadmonition}


\bigskip\hrule\bigskip



\subsection{Examples of non\sphinxhyphen{}vector spaces}
\label{\detokenize{_pages/5.1_Vector_spaces_definitions:examples-of-non-vector-spaces}}
\sphinxAtStartPar
Not all sets are vector spaces. For example, consider the set of integers \(\mathbb{Z}\) over the field \(F=\mathbb{R}\). It is easy to show that the {\hyperref[\detokenize{_pages/5.1_Vector_spaces_definitions:vector-space-axioms}]{\sphinxcrossref{axioms A1 to A4}}} are satisfied for the set of integers, which can be added together using standard integer addition.

\sphinxAtStartPar
The problem comes when one tries to define scalar multiplication \sphinxhyphen{} from the {\hyperref[\detokenize{_pages/5.1_Vector_spaces_definitions:vector-space-definition}]{\sphinxcrossref{definition of scalar multiplication}}}, we need a binary operation \(\mathbb{R} \times \mathbb{Z} \to \mathbb{Z}\). This would mean that for all \(u \in \mathbb{Z}\) and \(\alpha \in \mathbb{R}\), \(\alpha x\in \mathbb{Z}\).

\sphinxAtStartPar
However, this is not always the case \sphinxhyphen{} for example, when \(u=1\) and \(\alpha = \frac{1}{2}\), we have \(\frac{1}{2} \cdot 1 = \frac{1}{2} \notin \mathbb{Z}\). This is an example of proof by counterexample: we just need to show one instance where the axioms are not satisfied to prove that we do not have a vector space.
\label{_pages/5.1_Vector_spaces_definitions:non-vector-space-example}
\begin{sphinxadmonition}{note}{Example 5.1.2}



\sphinxAtStartPar
Show that the following are not vector spaces:

\sphinxAtStartPar
(i)   The set of all positive real numbers, \(\mathbb{R}_+\), over itself;

\sphinxAtStartPar
(ii)   \(V\) is defined to be the set of points on the parabola \(y=x^2\) in \(\mathbb{R}^2\), i.e., all the points in \(\mathbb{R}^2\) that can be written as a pair \((x, x^2)\), over the field \(\mathbb{R}\). Addition of vectors here is defined pointwise, so \((x, x^2) + (y, y^2) = (x+y, x^2 + y^2)\).
\subsubsection*{Solution}

\sphinxAtStartPar
(i)   We do not have an identity element for addition, since \(0 \notin \mathbb{R}_+\) so axiom A3 is not satisfied. Also, by axiom A4, if \(u, v \in \mathbb{R}_+\) are two positive real numbers then \(x + y > 0\) so no additive inverse exists.

\sphinxAtStartPar
(ii)   If we consider the vector \((1, 1)\), its additive inverse under vector addition as defined would be \((-1, -1) \notin V\), so \(V\) is not a vector space.
\end{sphinxadmonition}



\sphinxstepscope

\index{Vector spaces@\spxentry{Vector spaces}!subspaces@\spxentry{subspaces}}\ignorespaces 

\section{Subspaces}
\label{\detokenize{_pages/5.2_Subspaces:subspaces}}\label{\detokenize{_pages/5.2_Subspaces:index-0}}\label{\detokenize{_pages/5.2_Subspaces:subspaces-section}}\label{\detokenize{_pages/5.2_Subspaces::doc}}
\sphinxAtStartPar
A {\hyperref[\detokenize{_pages/5.1_Vector_spaces_definitions:vector-space-definition}]{\sphinxcrossref{vector space}}} is a set, and a natural next step is to study its subsets. Subsets are smaller sets contained within larger sets, and a subset of a vector space can be itself a vector space.

\sphinxAtStartPar
Consider the Euclidean space \(\mathbb{R}^3\) (denoted using \(x,y,z\)\sphinxhyphen{}coordinates). This set of points, considered over \(\mathbb{R}\), is a vector space. Now consider the plane \(p = \{(x,y,z) : x + y + z = 0\} \subseteq \mathbb{R}^3\). We can view \(p\) itself as a vector space over \(\mathbb{R}\), as it satisfies the {\hyperref[\detokenize{_pages/5.1_Vector_spaces_definitions:vector-space-axioms}]{\sphinxcrossref{vector space axioms}}} A1 – A4 and M1 – M4, if we define addition and scalar multiplication as coming from \(\mathbb{R}^3\). We saw in the chapter on {\hyperref[\detokenize{_pages/4.0_Coordinate_geometry:co-ordinate-geometry-chapter}]{\sphinxcrossref{\DUrole{std,std-ref}{co\sphinxhyphen{}ordinate geometry}}}} that this vector space \(p\) actually closely resembles \(\mathbb{R}^2\).
\label{_pages/5.2_Subspaces:subspace-definition}
\begin{sphinxadmonition}{note}{Definition 5.2.1 (Subspace)}



\sphinxAtStartPar
Let \(V\) be a vector space over a field \(F\). A non\sphinxhyphen{}empty subset \(W\) of \(V\) is called a \sphinxstylestrong{subspace} (or a vector subspace) if it is also a vector space over \(F\) \sphinxhyphen{} that is, it satisfies all the {\hyperref[\detokenize{_pages/5.1_Vector_spaces_definitions:vector-space-axioms}]{\sphinxcrossref{vector space axioms}}} A1 – A4 and M1 – M4.
\end{sphinxadmonition}

\sphinxAtStartPar
Most of the {\hyperref[\detokenize{_pages/5.1_Vector_spaces_definitions:vector-space-axioms}]{\sphinxcrossref{axioms}}} A1 – A4 and M1 – M4 are automatically satisfied, simply because they are satisfied in \(V\). The only thing we really need to check is that addition and scalar multiplication are well\sphinxhyphen{}defined \sphinxhyphen{} in other words, we need to check that when we ‘add’ or ‘multiply’ in \(W\) we are not taken outside \(W\) (\hyperref[\detokenize{_pages/5.2_Subspaces:subspaces-figure}]{Fig.\@ \ref{\detokenize{_pages/5.2_Subspaces:subspaces-figure}}}). This idea is formalised in {\hyperref[\detokenize{_pages/5.2_Subspaces:subspace-condition-theorem}]{\sphinxcrossref{the subspace condition}}}.

\begin{figure}[htbp]
\centering
\capstart

\noindent\sphinxincludegraphics[width=350\sphinxpxdimen]{{5_subspaces}.svg}
\caption{Illustration of a subspace \(W\) of a vector space \(V\).}\label{\detokenize{_pages/5.2_Subspaces:subspaces-figure}}\end{figure}

\index{Subspace condition@\spxentry{Subspace condition}}\ignorespaces \label{_pages/5.2_Subspaces:subspace-condition-theorem}
\begin{sphinxadmonition}{note}{Theorem 5.2.1 (Subspace condition)}



\sphinxAtStartPar
Let \(V\) be a vector space over \(F\) and \(W\) be a non\sphinxhyphen{}empty subset of \(V\). Then \(W\) is a subspace if and only if the following condition is satisfied for all \(\mathbf{u}, \mathbf{w} \in W\) and \(\alpha \in F\):
\begin{equation}\label{equation:_pages/5.2_Subspaces:subspace-condition-equation}
\begin{split} \mathbf{u} + \alpha \mathbf{w} \in W\end{split}
\end{equation}\end{sphinxadmonition}

\sphinxAtStartPar
It makes sense that to verify that vector addition and scalar multiplication work, we need to do some of each \sphinxhyphen{} and this is the simplest expression that involves doing both. To see {\hyperref[\detokenize{_pages/5.2_Subspaces:subspace-condition-theorem}]{\sphinxcrossref{the subspace condition}}} in action, let’s check whether \(W = \{(x, y, z) : x + 2y - 3z = 0\}\) is a subspace of \(\mathbb{R}^3\).

\sphinxAtStartPar
First we need to verify that \(W\) is non\sphinxhyphen{}empty. To do this we simply need to find an element of \(W\), e.g., \(\mathbf{0} \in W\) \sphinxhyphen{} so \(W\) is non\sphinxhyphen{}empty.

\sphinxAtStartPar
Next we need to consider vector addition and scalar multiplication in \(W\). Let \(\mathbf{u} = (u_1, u_2, u_3), \mathbf{v} = (v_1, v_2, v_3) \in W\) be two vectors, and let \(\alpha \in \mathbb{R}\) be a scalar. Then:
\begin{equation*}
\begin{split} \mathbf{u} + \alpha \mathbf{v} = \begin{pmatrix} u_1 \\ u_2 \\ u_3 \end{pmatrix} + \alpha \begin{pmatrix} v_1 \\ v_2 \\ v_3 \end{pmatrix} = \begin{pmatrix} u_1 + \alpha v_1 \\ u_2 + \alpha v_2 \\ u_3 + \alpha v_3 \end{pmatrix}. \end{split}
\end{equation*}
\sphinxAtStartPar
We now need to show that \((u_1 + \alpha v_1, u_2 + \alpha v_2, u_3 + \alpha v_3)  \in W\), i.e., that \(x + 2y - 3z = 0\) for this combination. Since \(x = u_1 + \alpha v_1\), \(y = u_2 + \alpha v_2\) and \(z = u_3 + \alpha v_3\), we have”
\begin{equation*}
\begin{split} \begin{align*}
    x + 2y - 3z &= (u_1 + \alpha v_1) + 2(u_2 + \alpha v_2) - 3(u_3 + \alpha v_3) \\
    &= (u_1 + 2u_2 - 3u_3) + \alpha (v_1 + 2v_2 - 3v_3) \\
    &= 0 + \alpha \cdot 0 = 0,
\end{align*} \end{split}
\end{equation*}
\sphinxAtStartPar
therefore \(\mathbf{u} + \alpha \mathbf{v} \in W\) and \(W\) is a subspace of \(\mathbb{R}^3\).


\label{_pages/5.2_Subspaces:subspace-example}
\begin{sphinxadmonition}{note}{Example 5.2.1}



\sphinxAtStartPar
(i)   \(W\) is a subset of\(M_{2\times 2}\) (the set of all \(2\times 2\) matrices) given by
\begin{equation*}
\begin{split} \begin{align*}
    W = \left\{ \begin{pmatrix} a & b \\ c & d \end{pmatrix} : a + b + c + d = 0 \right\}.
\end{align*} \end{split}
\end{equation*}
\sphinxAtStartPar
Show that \(W\) is a subspace of \(M_{2\times 2}\), considered as a vector space over the field \(\mathbb{R}\).

\sphinxAtStartPar
(ii)   \(X\) is a subset of \(P(\mathbb{R}_n)\) (the set of all polynomials of degree up to and including \(n\)) given by
\begin{equation*}
\begin{split}W = \{ p(x) \in P_n(\mathbb{R}) : p(x) = p(-x)\} \subseteq P(\mathbb{R}_n)\end{split}
\end{equation*}
\sphinxAtStartPar
Show that \(X\) is a subspace of \(P_n(\mathbb{R})\), considered as a vector space over the field \(\mathbb{R}\).
\subsubsection*{Solution}

\sphinxAtStartPar
(i)   \(\mathbf{0}_{2\times 2} \in W\) so \(W\) is non\sphinxhyphen{}empty. Let \(U, V \in W\) and \(\alpha \in \mathbb{R}\). Then
\begin{equation*}
\begin{split} \begin{align*}
    U + \alpha V =
    \begin{pmatrix} u_{11} & u_{12} \\ u_{21} & u_{22} \end{pmatrix} + \alpha
    \begin{pmatrix} v_{11} & v_{12} \\ v_{21} & v_{22} \end{pmatrix} \\ \\
    = \begin{pmatrix}
        u_{11} + \alpha v_{11} & u_{12} + \alpha v_{12} \\
        u_{21} + \alpha v_{21} & u_{22} + \alpha v_{22}
    \end{pmatrix}.
\end{align*} \end{split}
\end{equation*}
\sphinxAtStartPar
Checking that \(U + \alpha V \in W\)
\begin{equation*}
\begin{split} \begin{align*}
    a + b + c + d \\
    & = (u_{11} + \alpha v_{11}) + (u_{12} + \alpha v_{12}) + (u_{21} + \alpha v_{21}) + (u_{22} + \alpha v_{22}) \\
    & = (u_{11} + u_{12} + u_{21} + u_{22}) + \alpha (v_{11} + v_{12} + v_{21} + v_{22}) \\
    & = 0 + \alpha \cdot 0 = 0,
\end{align*} \end{split}
\end{equation*}
\sphinxAtStartPar
so \(U + \alpha V \in W\), and therefore \(W\) is a subspace of \(M_{2\times 2}\).

\sphinxAtStartPar
(ii)   The zero polynomial \(0 \in X\) so \(X\) is non\sphinxhyphen{}empty. Let \(u(x), v(x) \in X\) and \(\alpha \in \mathbb{R}\). We need to show that \(u(x) + \alpha v(x) \in X\). Now we know that \(u(-x) = u(x)\) and \(v(-x) = v(x)\) so
\begin{equation*}
\begin{split} \begin{align*}
    (u + \alpha v)(x) = u(x) + \alpha v(x) = u(-x) + \alpha v(-x) = (u + \alpha v)(-x),
\end{align*} \end{split}
\end{equation*}
\sphinxAtStartPar
so \((u + \alpha v)(x) \in X\) and therefore \(X\) is a subspace.

\sphinxAtStartPar
Note: We call polynomials with this property \sphinxhref{https://en.wikipedia.org/wiki/Even\_and\_odd\_functions}{even polynomials}.
\end{sphinxadmonition}


\bigskip\hrule\bigskip



\subsection{Non\sphinxhyphen{}examples of Subspaces}
\label{\detokenize{_pages/5.2_Subspaces:non-examples-of-subspaces}}
\sphinxAtStartPar
Not all subsets of a vector space will be a subspace. For example, define \(p\) as the polynomial
\begin{equation*}
\begin{split} p=a_0 + a_1 x + a_2 x^2 + \cdots + a_n x^n = \sum_{i = 0}^n a_ix^i, \end{split}
\end{equation*}
\sphinxAtStartPar
with \(a_n\neq 0\). We say that \(p\) has degree \(n\), and write \(\operatorname{deg}(p)=n\). Let
\begin{equation*}
\begin{split} W = \{p \in P_n(\mathbb{R}) : \operatorname{deg}(p) \in \{0,2,4\}\} \end{split}
\end{equation*}
\sphinxAtStartPar
i.e, the set of all polynomials of even degree at most 4, together with the zero polynomial. Is \(W\) a subspace of \(P_4(\mathbb{R})\)?

\sphinxAtStartPar
Let \(u = x^2\) and \(v = -x ^ 2 + x\). Then \(u, v \in W\) but \(u + v = x^2 - x^2 + x = x \notin W\) since \(\operatorname{deg}(x) = 1\) is odd. Therefore, \(W\) is not a subspace of \(P_4(\mathbb{R})\). Here, we came up with a single counterexample to show that the linear combination \(u + v \notin W\).

\sphinxAtStartPar
This counterexample works, but how can you come up with such a counterexample if you can’t immediately think of one? If you do not see an obvious counterexample, try to construct one: we are looking for two even\sphinxhyphen{}degree polynomials, whose sum is not an even polynomial. Let’s start at the lowest degree: \(0\). The sum of two polynomials of degree \(0\) is just the sum of two real numbers, so there is no counterexample here. Let’s try degree \(2\). The degree of a polynomial is just the highest power present \sphinxhyphen{} so degree \(2\) polynomials can include \(x\) terms as well as \(x^2\) terms.

\sphinxAtStartPar
For a formal approach to finding such a polynomial, we could let \(u = ax^2 + bx + c\) and \(v = dx^2 + ex + f\), and try to find values for a\sphinxhyphen{}f such that \(u + v\) has an odd degree. We can write:
\begin{equation*}
\begin{split} u + v = (a + d)x^2 + (b + e)x + (c + f) \end{split}
\end{equation*}
\sphinxAtStartPar
This polynomial will have degree one when \((a + d) = 0\) and \((b + e) \neq 0\). So let’s take \(a = 1\), \(d = -1\) and \(b = 0\), \(e = 1\) (we don’t care about \(c\) or \(f\), so let’s just set both to be 0). This yields \(u = x^2\) and \(v = -x^2 + x\), which was our counterexample.

\sphinxAtStartPar
If we try constructing a counterexample and but fail, then it might mean that either a counterexample is not easy to find, or that the statement is actually true. With a lot of pure maths research, mathematicians are often in this situation, trying to find a counterexample or a proof, not knowing which one it should be.


\label{_pages/5.2_Subspaces:non-subspace-example}
\begin{sphinxadmonition}{note}{Example 5.2.2}



\sphinxAtStartPar
Show that the following are not subspaces

\sphinxAtStartPar
(i)   \(W = \{z = (a+bi) : a + b = 5\} \subseteq \mathbb{C}\), over the field \(\mathbb{C}\);

\sphinxAtStartPar
(ii)   \(X = \{z = (a + bi) : a^2 + b^2 < 0\} \subseteq \mathbb{C}\), over the field \(\mathbb{C}\);

\sphinxAtStartPar
(iii)   \(Y = \{(x,y,z) : x + y + z = 5\} \subseteq{\mathbb{R}^3}\), over the field \(\mathbb{R}\).
\subsubsection*{Solution}

\sphinxAtStartPar
(i)   Here \(0 (= 0+0i) \notin W\) (since \(0+0 \neq 5\)) so \(W\) is not a subspace.

\sphinxAtStartPar
(ii)   \(X = \emptyset\) (the empty set), since \(a^2 + b^2 \geq 0\) for all \(a, b \in \mathbb{R}\). Hence \(X\) is not a subspace.

\sphinxAtStartPar
(iii)   Let \(\mathbf{u}, \mathbf{v} \in Y\) such that \(\mathbf{u} = (1, 2, 2)\) and \(\mathbf{v} = (2, 1, 2)\). Then \(\mathbf{u} + \mathbf{v} = (3, 3, 4) \notin Y\). We could have also just noted that \(\mathbf{0} \notin Y\).
\end{sphinxadmonition}

\sphinxstepscope

\index{Linear dependence@\spxentry{Linear dependence}}\ignorespaces 

\section{Linear dependence}
\label{\detokenize{_pages/5.3_Linear_dependence:linear-dependence}}\label{\detokenize{_pages/5.3_Linear_dependence:index-0}}\label{\detokenize{_pages/5.3_Linear_dependence:linear-dependence-section}}\label{\detokenize{_pages/5.3_Linear_dependence::doc}}
\sphinxAtStartPar
When working in a vector space, we can add together scalar multiples of vectors to form a \sphinxstylestrong{linear combination}. An important concept in linear algebra is that of whether a specific vector from a set of vectors can be expressed as a linear combination of the other vectors in the set.

\sphinxAtStartPar
If it can, we say that the vector is \sphinxstylestrong{linearly dependent} upon the other vectors. Linear dependence can help identify redundant or superfluous vectors within a set, and provides insight into the wider structure of a vector space.
\label{_pages/5.3_Linear_dependence:linear-dependence-definition}
\begin{sphinxadmonition}{note}{Definition 5.3.1 (Linear dependence)}



\sphinxAtStartPar
Let \(v_1, v_2, \ldots, v_n \in V\) and consider the equation
\begin{equation}\label{equation:_pages/5.3_Linear_dependence:linear-dependence-equation}
\begin{split} \alpha_1 v_1 + \alpha_2 v_2 + \cdots + \alpha_n v_n = 0, \end{split}
\end{equation}
\sphinxAtStartPar
where \(\alpha \in F\). The objects \(v_1, v_2, \ldots, v_n \in V\) are said to be \sphinxstylestrong{linearly independent over \(F\)} if the only solution to the above equation is when all of the \(\alpha_i\) values are zero (called the trivial solution).

\sphinxAtStartPar
If the above equation is satisfied where \(\alpha_i \neq 0\) for \(1 \leq i \leq n\), then \(v_1, v_2, \ldots, v_n \in V\) are said to be \sphinxstylestrong{linearly dependent over \(F\)}.
\end{sphinxadmonition}

\sphinxAtStartPar
Another way to think about linear independence is that a set of vectors is \sphinxstylestrong{linearly independent} if none of the vectors in the set can be represented as a linear combination of the other vectors in the same set. For example, consider the set of matrices
\begin{equation*}
\begin{split} \begin{align*}
    A &= \begin{pmatrix} 1 & 1 \\ 0 & 2 \end{pmatrix}, &
    B &= \begin{pmatrix} -1 & -1 \\ 0 & -2 \end{pmatrix}, &
    C &= \begin{pmatrix} -4 & 0 \\ 0 & -8 \end{pmatrix},
\end{align*} \end{split}
\end{equation*}
\sphinxAtStartPar
Is this set linearly independent over \(\mathbb{R}\)? We can see by inspection that \(B = -A\), so therefore \(A\), \(B\) and \(C\) are linearly dependent, since we can write:
\begin{equation*}
\begin{split} 1A + 1B + 0C = \mathbf{0}_{2\times 2}. \end{split}
\end{equation*}
\sphinxAtStartPar
So if any two members of a set are scalar multiples of each other, then the set is linearly dependent, because we can choose \(\alpha_i\) values to satisfy equation \eqref{equation:_pages/5.3_Linear_dependence:linear-dependence-equation}.

\sphinxAtStartPar
Note that linear dependence and linear independence are properties of a set, not of a particular vector: one vector cannot be linearly independent, as this property is defined relative to the other vectors in the set.


\label{_pages/5.3_Linear_dependence:linear-dependence-equation}
\begin{sphinxadmonition}{note}{Example 5.3.1}



\sphinxAtStartPar
Determine whether the following sets are linearly dependent:

\sphinxAtStartPar
(i)   \((1, 0, 2), (2, 1, 3),(-3, -4, -2) \in \mathbb{R}^3\) over \(\mathbb{R}\)

\sphinxAtStartPar
(ii)   \(u = x^2 + x + 1\), \(v = x - 1\) and \(w = x^2 - 1 \in P(\mathbb{R})\) over \(\mathbb{R}\)
\subsubsection*{Solution}

\sphinxAtStartPar
(i)   Let \(\alpha_1, \alpha_2, \alpha_3 \in \mathbb{R}\). Then equation \eqref{equation:_pages/5.3_Linear_dependence:linear-dependence-equation} becomes
\begin{equation*}
\begin{split} \begin{align*}
    \alpha_1 \begin{pmatrix} 1 \\ 0 \\ 2 \end{pmatrix} +
    \alpha_2 \begin{pmatrix} 2 \\ 1 \\ 3 \end{pmatrix} +
    \alpha_3 \begin{pmatrix} -3 \\ -4 \\ -2 \end{pmatrix} &=
    \begin{pmatrix} 0 \\ 0 \\ 0 \end{pmatrix}.
\end{align*} \end{split}
\end{equation*}
\sphinxAtStartPar
This holds if and only if
\begin{equation*}
\begin{split} \begin{align*}
    \alpha_1 + 2\alpha_2 - 3\alpha_3 &= 0, \\
    \alpha_2 - 4\alpha_3 &= 0, \\
    2\alpha_3 + 3\alpha_2 - 2\alpha_3 &= 0.
\end{align*} \end{split}
\end{equation*}
\sphinxAtStartPar
Solving this homogeneous system using Gauss\sphinxhyphen{}Jordan elimination
\begin{equation*}
\begin{split} \begin{align*}
    & \left( \begin{array}{ccc}
        1 & 2 & -3\\
        0 & 1 & -4\\
        2 & 3 & -2
    \end{array} \right)
    \\ \\
    R_3 - 2R_1 \longrightarrow
    &\left( \begin{array}{ccc}
        1 & 2 & -3\\
        0 & 1 & -4\\
        0 & -1 & 4
    \end{array} \right)
	\\ \\
    \begin{matrix} R_1 - 2R_2 \\ \\ R_3 + R_2 \end{matrix} \longrightarrow
    &\left( \begin{array}{ccc}
        1 & 0 & 5\\
        0 & 1 & -4\\
        0 & 0 & 0
    \end{array} \right)
\end{align*} \end{split}
\end{equation*}
\sphinxAtStartPar
Here \(\alpha_3\) is a free variable, so let \(\alpha_3 = r\). Then \(\alpha_1 = -5r\) and \(\alpha_2 = 4r\). These vectors are therefore linearly dependent, i.e., if \(r = 1\), then \(\alpha_1 = -5\) and \(\alpha_2 = 4\), meaning we have:
\begin{equation*}
\begin{split} \begin{align*}
    - 5 \begin{pmatrix} 1 \\ 0 \\ 2 \end{pmatrix}
    + 4 \begin{pmatrix} 2 \\ 1 \\ 3 \end{pmatrix}
    + \begin{pmatrix} -3 \\ -4 \\ -2 \end{pmatrix} =
    \begin{pmatrix} 0 \\ 0 \\ 0 \end{pmatrix}.
\end{align*} \end{split}
\end{equation*}
\sphinxAtStartPar
(ii)   Let \(\alpha_1, \alpha_2, \alpha_3 \in \mathbb{R}\). Then we need to ascertain when
\begin{equation*}
\begin{split} \begin{align*}
    \alpha_1 u + \alpha_2 v + \alpha_3 w = 0.
\end{align*} \end{split}
\end{equation*}
\sphinxAtStartPar
Substituting in the expressions for \(u\), \(v\) and \(w\):
\begin{equation*}
\begin{split} \begin{align*}
    \alpha_1 (x^2 + x + 1) + \alpha_2 (x - 1) + \alpha_3 (x^2 - 1) &= 0 \\ \\
    (\alpha_1 + \alpha_3)x^2 + (\alpha_1 + \alpha_2)x + (\alpha_1 - \alpha_2 - \alpha_3)x^0 &= 0.
\end{align*} \end{split}
\end{equation*}
\sphinxAtStartPar
For a polynomial to be equal to zero, the coefficients of \(x^i\) must all be equal to zero. This means we need:
\begin{equation*}
\begin{split} \begin{align*}
    \alpha_1 + \alpha_3 &= 0, \\
    \alpha_1 + \alpha_2 &= 0, \\
    \alpha_1 - \alpha_2 - \alpha_3 &= 0.
\end{align*} \end{split}
\end{equation*}
\sphinxAtStartPar
Solving this system using Gauss\sphinxhyphen{}Jordan elimination:
\begin{equation*}
\begin{split} \begin{align*}
    & \left( \begin{array}{ccc}
        1 & 0 & 1\\
        1 & 1 & 0\\
        1 & -1 & -1
    \end{array} \right)
    \\ \\
    \begin{matrix} R_2 - R_1 \\ R_3 - R_1 \end{matrix} \longrightarrow  &
    \left( \begin{array}{ccc}
        1 & 0 & 1\\
        0 & 1 & -1\\
        0 & -1 & -2
    \end{array} \right)
    \\ \\
    R_3 + R_2\longrightarrow  &
    \left( \begin{array}{ccc}
        1 & 0 & 1\\
        0 & 1 & -1\\
        0 & 0 & -3
    \end{array} \right)
	\\ \\
    -\frac{1}{3}R_3\longrightarrow &
    \left( \begin{array}{ccc}
        1 & 0 & 1\\
        0 & 1 & -1\\
        0 & 0 & 1
    \end{array} \right)
    \\ \\
    \begin{matrix} R_1 - R_3 \\ R_2 + R_3 \end{matrix}\longrightarrow  &
    \left( \begin{array}{ccc}
        1 & 0 & 0\\
        0 & 1 & 0\\
        0 & 0 & 1
    \end{array} \right)
\end{align*} \end{split}
\end{equation*}
\sphinxAtStartPar
Therefore the only solution is \(\alpha_1 = \alpha_2 = \alpha_3 = 0\), which means the polynomials \(u\), \(v\) and \(w\) are linearly independent.
\end{sphinxadmonition}

\sphinxstepscope

\index{Basis@\spxentry{Basis}}\ignorespaces 

\section{Basis}
\label{\detokenize{_pages/5.4_Basis:basis}}\label{\detokenize{_pages/5.4_Basis:index-0}}\label{\detokenize{_pages/5.4_Basis:basis-section}}\label{\detokenize{_pages/5.4_Basis::doc}}
\sphinxAtStartPar
We have already met the concept of a basis in the chapter on vectors, where we saw that {\hyperref[\detokenize{_pages/3.4_Linear_combinations:basis-vectors-section}]{\sphinxcrossref{\DUrole{std,std-ref}{the basis vectors for \(\mathbb{R}^3\)}}}} are the vectors \(\mathbf{i} = (1, 0, 0)\), \(\mathbf{j} = (0, 1, 0)\) and \(\mathbf{k} = (0, 0, 1)\) and any vector in \(\mathbb{R}^3\) can be represented as a linear combination of these basis vectors.

\sphinxAtStartPar
The concept of a basis is not limited to Euclidean geometry, and we can define a basis for any vector space, so that every element in the vector space can be expressed as a {\hyperref[\detokenize{_pages/3.4_Linear_combinations:linear-combination-of-vectors-definition}]{\sphinxcrossref{linear combination}}} of the basis elements.


\bigskip\hrule\bigskip


\index{Spanning set@\spxentry{Spanning set}}\ignorespaces 

\subsection{Spanning sets}
\label{\detokenize{_pages/5.4_Basis:spanning-sets}}\label{\detokenize{_pages/5.4_Basis:index-1}}\label{_pages/5.4_Basis:spanning-set-definition}
\begin{sphinxadmonition}{note}{Definition 5.4.1 (Spanning set)}



\sphinxAtStartPar
Let \(V\) be a vector space over the field \(F\), and let \(S\) be a subset of \(V\). We define \(W\) to be the subset of \(V\) containing all vectors that are expressible as a linear combination of vectors in \(S\) \sphinxhyphen{} that is, for any \(u \in W\), given the vectors \(v_1, \ldots, v_n \in S\) we can find \(\alpha_1, \ldots, \alpha_n \in F\) so that:
\begin{equation*}
\begin{split} u = \alpha_1 v_1 + \alpha_2 v_2 + \cdots + \alpha_n v_n, \end{split}
\end{equation*}
\sphinxAtStartPar
Then \(W\) is a subspace of \(V\), and \(S\) is a \sphinxstylestrong{spanning set} for \(W\). We call \(W\) the \sphinxstylestrong{span} of the set \(S\), and write this as \(W = \operatorname{span}(S)\).
\end{sphinxadmonition}

\sphinxAtStartPar
For example, to show that \(\{ \mathbf{v}_1, \mathbf{v}_2 \}\) is a spanning set for \(\mathbb{R}^2\), where \(\mathbf{v}_1 = (2, 1)\) and \(\mathbf{v}_2 = (4, 3)\), we need to show that
\begin{equation*}
\begin{split}\alpha_1 \begin{pmatrix} 2\\ 1 \end{pmatrix} + \alpha_2 \begin{pmatrix} 4 \\ 3 \end{pmatrix} = \begin{pmatrix} a\\ b \end{pmatrix},\end{split}
\end{equation*}
\sphinxAtStartPar
for any \(a, b \in \mathbb{R}\), i.e., that the following system has a non\sphinxhyphen{}trivial solution:
\begin{equation*}
\begin{split} \begin{align*}
    2 \alpha_1 + 4 \alpha_2 &= a, \\
    \alpha_1 + 3 \alpha_2 &= b.
\end{align*} \end{split}
\end{equation*}
\sphinxAtStartPar
A system of linear equations has a solution if it is {\hyperref[\detokenize{_pages/1.5_Inverse_matrix:inverse-matrix-definition}]{\sphinxcrossref{non\sphinxhyphen{}singular}}} (the determinant of the coefficient matrix is non\sphinxhyphen{}zero), therefore
\begin{equation*}
\begin{split} \det \begin{pmatrix} 2 & 4 \\ 1 & 3 \end{pmatrix} = 2 \neq 0, \end{split}
\end{equation*}
\sphinxAtStartPar
so there is a solution to the system and \(\{\mathbf{v}_1, \mathbf{v}_2\}\) is a spanning set for \(\mathbb{R}^2\).


\bigskip\hrule\bigskip



\subsection{Basis of a vector space}
\label{\detokenize{_pages/5.4_Basis:basis-of-a-vector-space}}\label{_pages/5.4_Basis:basis-definition}
\begin{sphinxadmonition}{note}{Definition 5.4.2 (Basis of a vector space)}



\sphinxAtStartPar
A \sphinxstylestrong{basis} of a vector space \(V\) over a field \(F\) is a linearly independent subset of \(V\) that spans \(V\). A subset \(B\) is a basis if it satisfies the following:
\begin{itemize}
\item {} 
\sphinxAtStartPar
{\hyperref[\detokenize{_pages/5.3_Linear_dependence:linear-dependence-definition}]{\sphinxcrossref{linear independence property}}}: if \(B = \{b_1, \ldots , b_n\}\), the following equation only has the trivial solution \(\alpha_i = 0\):

\end{itemize}
\begin{equation*}
\begin{split} \alpha_1 b_1 + \alpha_2 b_2 + \cdots + \alpha_n b_n = 0; \end{split}
\end{equation*}\begin{itemize}
\item {} 
\sphinxAtStartPar
{\hyperref[\detokenize{_pages/5.4_Basis:spanning-set-definition}]{\sphinxcrossref{spanning property}}}: for all \(u \in V\) we can write \(u\) as a linear combination of \(v \in B\), i.e.,

\end{itemize}
\begin{equation*}
\begin{split} u = \alpha_1 b_1 + \alpha_2 b_2 + \cdots + \alpha_n b_n. \end{split}
\end{equation*}\end{sphinxadmonition}

\index{Basis@\spxentry{Basis}!orthogonal basis@\spxentry{orthogonal basis}}\ignorespaces \label{_pages/5.4_Basis:orthogonal-basis-definition}
\begin{sphinxadmonition}{note}{Definition 5.4.3 (Orthogonal basis)}



\sphinxAtStartPar
An \sphinxstylestrong{orthogonal basis} of a vector space is one in which each of the vectors are orthogonal (perpendicular) to one another.
\end{sphinxadmonition}

\index{Basis@\spxentry{Basis}!orthonormal basis@\spxentry{orthonormal basis}}\ignorespaces \label{_pages/5.4_Basis:orthonormal-basis-definition}
\begin{sphinxadmonition}{note}{Definition 5.4.4 (Orthonormal basis)}



\sphinxAtStartPar
An \sphinxstylestrong{orthonormal basis} of a vector space is one in which each of the vectors are orthogonal to one another and each vector is a unit vector.
\end{sphinxadmonition}

\index{Vector space@\spxentry{Vector space}!dimension@\spxentry{dimension}}\ignorespaces \label{_pages/5.4_Basis:vector-space-dimension-definition}
\begin{sphinxadmonition}{note}{Definition 5.4.5 (Dimension of a vector space)}



\sphinxAtStartPar
The \sphinxstylestrong{dimension} of a vector space \(V\) is denoted by \(\dim(V)\) and is the number of elements in a basis for the vector space.
\end{sphinxadmonition}


\label{_pages/5.4_Basis:basis-example}
\begin{sphinxadmonition}{note}{Example 5.4.1}



\sphinxAtStartPar
Show that \(\{ \mathbf{v}_1, \mathbf{v}_2, \mathbf{v}_3\}\), where \(\mathbf{v}_1 = (1, 1, 0)\), \(\mathbf{v}_2 = (1, -1, 1)\) and \(\mathbf{v}_3 = (1, -1, -2)\), is a basis for \(\mathbb{R}^3\).
\subsubsection*{Solution}

\sphinxAtStartPar
We need to show that the three vectors are linearly independent, i.e., show that the only solution to the following system is \(\alpha_1 = \alpha_2 = \alpha_3 = 0\)
\begin{equation*}
\begin{split} \begin{align*}
    \alpha_1 \begin{pmatrix} 1 \\ 1 \\ 0 \end{pmatrix} +
    \alpha_2 \begin{pmatrix} 1 \\ -1 \\ 1 \end{pmatrix} +
    \alpha_3 \begin{pmatrix} 1 \\ -1 \\ -2 \end{pmatrix}  =
    \begin{pmatrix} 0 \\ 0 \\ 0 \end{pmatrix}.
\end{align*} \end{split}
\end{equation*}
\sphinxAtStartPar
Using Gauss\sphinxhyphen{}Jordan elimination
\begin{equation*}
\begin{split} \begin{align*}
    & \begin{pmatrix}
        1 & 1 & 1 \\
        1 & -1 & 1 \\
        1 & -1 & -2
    \end{pmatrix}
     \\ \\
    \begin{matrix} R_2 - R_1 \\ R_3 - R_1 \end{matrix}\longrightarrow &
    \begin{pmatrix}
        1 & 1 & 1 \\
        0 & -2 & 0 \\
        0 & -2 & -3
    \end{pmatrix}
	\\ \\
    -\frac{1}{2} R_2\longrightarrow &
    \begin{pmatrix}
        1 & 1 & 1 \\
        0 & 1 & 0 \\
        0 & -2 & -3
    \end{pmatrix}
     \\\\
    \begin{matrix} R_1 - R_2 \\ R_3 + 2R_2 \end{matrix} \longrightarrow &
    \begin{pmatrix}
        1 & 0 & 1 \\
        0 & 1 & 0 \\
        0 & 0 & -3
    \end{pmatrix}
     \\ \\
    -\frac{1}{3}R_3 \longrightarrow &
    \begin{pmatrix}
        1 & 0 & 1 \\
        0 & 1 & 0 \\
        0 & 0 & 1
    \end{pmatrix}
    \\ \\
    R_1 - R_3\longrightarrow &
    \begin{pmatrix}
        1 & 0 & 0 \\
        0 & 1 & 0 \\
        0 & 0 & 1
    \end{pmatrix}
\end{align*} \end{split}
\end{equation*}
\sphinxAtStartPar
So \(\mathbf{v}_1\), \(\mathbf{v}_2\) and \(\mathbf{v}_3\) are linearly independent. We also need to show that \(\{ \mathbf{v}_1, \mathbf{v}_2, \mathbf{v}_3 \}\) spans \(\mathbb{R}^3\).
\begin{equation*}
\begin{split} \det \begin{pmatrix} 1 & 1 & 1 \\ 1 & -1 & -1 \\ 0 & 1 & -2 \end{pmatrix} = 3 + 3 = 6 \neq 0, \end{split}
\end{equation*}
\sphinxAtStartPar
so \(\{ \mathbf{v}_1, \mathbf{v}_2, \mathbf{v}_3 \}\) is a spanning set for \(\mathbb{R}^3\). Note that in showing that \(\mathbf{v}_1\), \(\mathbf{v}_2\) and \(\mathbf{v}_3\) are linearly independent we have also shown that \(\alpha_1 \mathbf{v}_1 + \alpha_2 \mathbf{v}_2 + \alpha_3 \mathbf{v}_3 = (a, b, c)\) has a solution and \(\{ \mathbf{v}_1, \mathbf{v}_2, \mathbf{v}_3 \}\) is a spanning set. We can also check if this is an orthogonal basis:
\begin{equation*}
\begin{split} \begin{align*}
    \mathbf{v}_1 \cdot \mathbf{v}_2 &= \begin{pmatrix} 1 \\ 1 \\ 0 \end{pmatrix} \cdot
    \begin{pmatrix} 1 \\ -1 \\ 1 \end{pmatrix} = 0, \\
    \mathbf{v}_1 \cdot \mathbf{v}_3 &= \begin{pmatrix} 1 \\ 1 \\ 0 \end{pmatrix} \cdot
    \begin{pmatrix} 1 \\ -1 \\ 2 \end{pmatrix} = 0, \\
    \mathbf{v}_2 \cdot \mathbf{v}_3 &= \begin{pmatrix} 1 \\ -1 \\ 1 \end{pmatrix} \cdot
    \begin{pmatrix} 1 \\ -1 \\ -2 \end{pmatrix} = 0,
\end{align*} \end{split}
\end{equation*}
\sphinxAtStartPar
so \(\{ \mathbf{v}_1, \mathbf{v}_2, \mathbf{v}_3 \}\) is an orthogonal basis, since each pair of vectors in it is perpendicular. It is not an orthonormal basis (all vectors of unit length) since \(\|\mathbf{v}_1\| = \sqrt{2} \neq 1\).
\end{sphinxadmonition}


\bigskip\hrule\bigskip


\index{Basis@\spxentry{Basis}!change of basis@\spxentry{change of basis}}\ignorespaces 
\index{Basis@\spxentry{Basis}!standard basis@\spxentry{standard basis}}\ignorespaces 

\subsection{Change of basis}
\label{\detokenize{_pages/5.4_Basis:change-of-basis}}\label{\detokenize{_pages/5.4_Basis:change-of-basis-section}}\label{\detokenize{_pages/5.4_Basis:index-6}}\label{\detokenize{_pages/5.4_Basis:index-5}}
\sphinxAtStartPar
\(\mathbb{R}^n\) has a particularly nice basis that is easy to write down:
\begin{equation*}
\begin{split} E = \left\{
        \mathbf{e}_1 = \begin{pmatrix} 1 \\ 0 \\ 0 \\ \vdots \\ 0 \end{pmatrix},
        \mathbf{e}_2 = \begin{pmatrix} 0 \\ 1 \\ 0 \\ \vdots \\ 0 \end{pmatrix}, \ldots,
        \mathbf{e}_n = \begin{pmatrix} 0 \\ 0 \\ 0 \\ \vdots \\ 1 \end{pmatrix}
        \right\}, \end{split}
\end{equation*}
\sphinxAtStartPar
which is called the \sphinxstylestrong{standard basis}. Note that \(\mathbf{e}_i\) is column \(i\) of the identity matrix, and for \(\mathbb{R}^3\) we have \(\mathbf{i} = \mathbf{e}_1\), \(\mathbf{j} = \mathbf{e}_2\) and \(\mathbf{k} = \mathbf{e}_3\).

\sphinxAtStartPar
We can represent a vector \(\mathbf{u} = (u_1, u_2, \ldots, u_n) \in \mathbb{R}^n\) as a sum of elements of the standard basis, since it will just be a sum of the basis vectors using the entries of \(\mathbf{u}\) as coefficients:
\begin{equation*}
\begin{split}\mathbf{u} = u_1 \mathbf{e}_1 + u_2 \mathbf{e}_2 + \cdots + u_n \mathbf{e}_n \end{split}
\end{equation*}
\sphinxAtStartPar
We can also express the same vector \(\mathbf{u}\) as a sum of the elements of another basis \(W = \{\mathbf{v}_1, \mathbf{v}_2, \ldots, \mathbf{v}_n\}\). We denote this \([\mathbf{u}]_W\), and in order to express \(\mathbf{u}\) with respect to the basis \(W\) we need to solve:
\begin{equation*}
\begin{split} \mathbf{u} = w_1 \mathbf{v}_1 + w_2 \mathbf{v}_2 + \cdots + w_n \mathbf{v}_n\end{split}
\end{equation*}
\sphinxAtStartPar
for \(w_1, w_2, \ldots, w_n\), using methods we have already seen.

\sphinxAtStartPar
A visual representation of a single vector being expressed in terms of two different bases is shown in \hyperref[\detokenize{_pages/5.4_Basis:change-of-basis-figure}]{Fig.\@ \ref{\detokenize{_pages/5.4_Basis:change-of-basis-figure}}}.

\begin{figure}[htbp]
\centering
\capstart

\noindent\sphinxincludegraphics[width=400\sphinxpxdimen]{{5_change_of_basis}.svg}
\caption{Representing the same point with respect to two different basis.}\label{\detokenize{_pages/5.4_Basis:change-of-basis-figure}}\end{figure}

\sphinxAtStartPar
We often think of the coordinates of a point as being equivalent to a vector (from the origin to that point). Since we have different sets of basis vectors, we can use them to give the coordinates to a point but with respect to a different basis.

\sphinxAtStartPar
We would say, the point whose co\sphinxhyphen{}ordinates are \((u_1, u_2)\) with respect to the standard basis \(\{ \mathbf{e}_1, \mathbf{e}_2\}\) can also be defined using the co\sphinxhyphen{}ordinates \((w_1, w_2)\), with respect to the basis \(\{ \mathbf{w}_1, \mathbf{w}_2 \}\).


\label{_pages/5.4_Basis:change-of-basis-example}
\begin{sphinxadmonition}{note}{Example 5.4.2}



\sphinxAtStartPar
Represent the vector \(\mathbf{u} = (4, 0, 5)\) with respect to the basis \(W = \{ (1, 1, 0), (1, -1, 1), (1, -1, -2)\}\).
\subsubsection*{Solution}

\sphinxAtStartPar
We need to solve the system
\begin{equation*}
\begin{split} w_1 \begin{pmatrix} 1 \\ 1 \\ 0 \end{pmatrix} +
    w_2 \begin{pmatrix} 1 \\ -1 \\ 1 \end{pmatrix} +
    w_3 \begin{pmatrix} 1 \\ -1 \\ -2 \end{pmatrix} =
    \begin{pmatrix} 4 \\ 0 \\ 5 \end{pmatrix}, \end{split}
\end{equation*}
\sphinxAtStartPar
which can be written as the matrix equation
\begin{equation*}
\begin{split} \begin{pmatrix}
        1 & 1 & 1 \\
        1 & -1 & -1 \\
        0 & 1 & -2
    \end{pmatrix}
    \begin{pmatrix} w_1 \\ w_2 \\ w_3 \end{pmatrix} =
    \begin{pmatrix} 4 \\ 0 \\ 5 \end{pmatrix}, \end{split}
\end{equation*}
\sphinxAtStartPar
Calculating the inverse of the coefficient matrix
\begin{equation*}
\begin{split} \begin{align*}
    A^{-1} &=
    \begin{pmatrix}
        1 & 1 & 1 \\
        1 & -1 & -1 \\
        0 & 1 & -2
    \end{pmatrix}^{-1}
    =
    \frac{1}{6}
    \begin{pmatrix}
        3 & 2 & 1 \\
        3 & -2 & -1 \\
        0 & 2 & -2
    \end{pmatrix}^\mathsf{T}
    =
    \begin{pmatrix}
        \frac{1}{2} & \frac{1}{2} & 0 \\
        \frac{1}{3} & -\frac{1}{3} & \frac{1}{3} \\
        \frac{1}{6} & -\frac{1}{6} & -\frac{1}{3}
    \end{pmatrix},
\end{align*} \end{split}
\end{equation*}
\sphinxAtStartPar
so
\begin{equation*}
\begin{split} \begin{align*}
    [\mathbf{u}]_W =
    \begin{pmatrix}
        \frac{1}{2} & \frac{1}{2} & 0 \\
        \frac{1}{3} & -\frac{1}{3} & \frac{1}{3} \\
        \frac{1}{6} & -\frac{1}{6} & -\frac{1}{3}
    \end{pmatrix}
    \begin{pmatrix} 4 \\ 0 \\ 5 \end{pmatrix} =
    \begin{pmatrix} 2 \\ 3 \\ -1 \end{pmatrix}.
\end{align*} \end{split}
\end{equation*}\end{sphinxadmonition}

\sphinxAtStartPar
In {\hyperref[\detokenize{_pages/5.4_Basis:change-of-basis-example}]{\sphinxcrossref{Example 5.4.2}}} we calculated an inverse matrix (shown in the final line of the solution), which we used to find the vector \(\mathbf{u}\) expressed in terms of the new basis.

\sphinxAtStartPar
In general, once we have calculated this matrix, we can use this to find any vector \(\mathbf{u}\) with respect to the basis \(W\). This matrix is known as the \sphinxstylestrong{change of basis matrix}.

\index{Basis@\spxentry{Basis}!change of basis matrix@\spxentry{change of basis matrix}}\ignorespaces \label{_pages/5.4_Basis:change-of-basis-matrix-definition}
\begin{sphinxadmonition}{note}{Definition 5.4.6 (Change of basis matrix)}



\sphinxAtStartPar
Let \(V\) be a vector space over the field \(F\) and \(\mathbf{u} \in V\). If \(E\) and \(W\) are two bases for \(V\), then the change of basis matrix is the matrix \(A_{E \to W}\) such that \([u]_{W} = A_{E \to W} [u]_E\).
\end{sphinxadmonition}

\sphinxAtStartPar
To express the vector \(\mathbf{u} \in \mathbb{R}^3\) with respect to the basis \(W\), we multiply \(\mathbf{u}\) by the change of basis matrix \(A_{E \to W}\).

\sphinxAtStartPar
Going from the standard basis to a non\sphinxhyphen{}standard basis, the calculations to find this matrix are relatively straightforward; changing between two non\sphinxhyphen{}standard bases is a slightly more complicated procedure, and will be covered in the more advanced units on linear algebra.

\sphinxstepscope


\section{Vector spaces exercises}
\label{\detokenize{_pages/5.5_Vector_spaces_exercises:vector-spaces-exercises}}\label{\detokenize{_pages/5.5_Vector_spaces_exercises:vector-spaces-exercises-section}}\label{\detokenize{_pages/5.5_Vector_spaces_exercises::doc}}
\sphinxAtStartPar
Answer the following exercises based on the content from this chapter. The solutions can be found in the {\hyperref[\detokenize{_pages/A5_Vector_spaces_exercises_solutions:vector-spaces-exercises-solutions-section}]{\sphinxcrossref{\DUrole{std,std-ref}{appendices}}}}.
\phantomsection \label{exercise:vector-spaces-ex-R3-axioms}

\begin{sphinxadmonition}{note}{Exercise 5.5.1}



\sphinxAtStartPar
Prove that the {\hyperref[\detokenize{_pages/5.1_Vector_spaces_definitions:vector-space-axioms}]{\sphinxcrossref{axioms}}} of the vector space \(\mathbb{R}^3\) hold.
\end{sphinxadmonition}
\phantomsection \label{exercise:vector-spaces-ex-R3-subspaces}

\begin{sphinxadmonition}{note}{Exercise 5.5.2}



\sphinxAtStartPar
Using the {\hyperref[\detokenize{_pages/5.2_Subspaces:subspace-condition-theorem}]{\sphinxcrossref{subspace condition}}}, determine whether the following subsets of \(\mathbb{R}^3\) are subspaces:

\begin{sphinxuseclass}{sd-container-fluid}
\begin{sphinxuseclass}{sd-sphinx-override}
\begin{sphinxuseclass}{sd-mb-4}
\begin{sphinxuseclass}{sd-row}
\begin{sphinxuseclass}{sd-col}
\begin{sphinxuseclass}{sd-d-flex-column}
\begin{sphinxuseclass}{sd-col-6}
\begin{sphinxuseclass}{sd-col-xs-6}
\begin{sphinxuseclass}{sd-col-sm-6}
\begin{sphinxuseclass}{sd-col-md-6}
\begin{sphinxuseclass}{sd-col-lg-6}
\sphinxAtStartPar
(a)   \(U = \{ (x, y, 0) : x, y \in \mathbb{R} \}\)

\end{sphinxuseclass}
\end{sphinxuseclass}
\end{sphinxuseclass}
\end{sphinxuseclass}
\end{sphinxuseclass}
\end{sphinxuseclass}
\end{sphinxuseclass}
\begin{sphinxuseclass}{sd-col}
\begin{sphinxuseclass}{sd-d-flex-column}
\begin{sphinxuseclass}{sd-col-6}
\begin{sphinxuseclass}{sd-col-xs-6}
\begin{sphinxuseclass}{sd-col-sm-6}
\begin{sphinxuseclass}{sd-col-md-6}
\begin{sphinxuseclass}{sd-col-lg-6}
\sphinxAtStartPar
(b)   \(V = \{ (1, 2, 0) \}\)

\end{sphinxuseclass}
\end{sphinxuseclass}
\end{sphinxuseclass}
\end{sphinxuseclass}
\end{sphinxuseclass}
\end{sphinxuseclass}
\end{sphinxuseclass}
\begin{sphinxuseclass}{sd-col}
\begin{sphinxuseclass}{sd-d-flex-column}
\begin{sphinxuseclass}{sd-col-6}
\begin{sphinxuseclass}{sd-col-xs-6}
\begin{sphinxuseclass}{sd-col-sm-6}
\begin{sphinxuseclass}{sd-col-md-6}
\begin{sphinxuseclass}{sd-col-lg-6}
\sphinxAtStartPar
(c)   \(W = \{ (0, y, 0) : y \in \mathbb{R} \}\)

\end{sphinxuseclass}
\end{sphinxuseclass}
\end{sphinxuseclass}
\end{sphinxuseclass}
\end{sphinxuseclass}
\end{sphinxuseclass}
\end{sphinxuseclass}
\begin{sphinxuseclass}{sd-col}
\begin{sphinxuseclass}{sd-d-flex-column}
\begin{sphinxuseclass}{sd-col-6}
\begin{sphinxuseclass}{sd-col-xs-6}
\begin{sphinxuseclass}{sd-col-sm-6}
\begin{sphinxuseclass}{sd-col-md-6}
\begin{sphinxuseclass}{sd-col-lg-6}
\sphinxAtStartPar
(d)   \(X = \{ (x, y, z) : y = |x|, x, y, z \in \mathbb{R}\}\)

\end{sphinxuseclass}
\end{sphinxuseclass}
\end{sphinxuseclass}
\end{sphinxuseclass}
\end{sphinxuseclass}
\end{sphinxuseclass}
\end{sphinxuseclass}
\end{sphinxuseclass}
\end{sphinxuseclass}
\end{sphinxuseclass}
\end{sphinxuseclass}\end{sphinxadmonition}
\phantomsection \label{exercise:vector-spaces-ex-M2-subspaces}

\begin{sphinxadmonition}{note}{Exercise 5.5.3}



\sphinxAtStartPar
Which of the following sets are subspaces of \(M_2(\mathbb{R})\)?

\begin{sphinxuseclass}{sd-container-fluid}
\begin{sphinxuseclass}{sd-sphinx-override}
\begin{sphinxuseclass}{sd-mb-4}
\begin{sphinxuseclass}{sd-row}
\begin{sphinxuseclass}{sd-col}
\begin{sphinxuseclass}{sd-d-flex-column}
\begin{sphinxuseclass}{sd-col-6}
\begin{sphinxuseclass}{sd-col-xs-6}
\begin{sphinxuseclass}{sd-col-sm-6}
\begin{sphinxuseclass}{sd-col-md-6}
\begin{sphinxuseclass}{sd-col-lg-6}
\sphinxAtStartPar
(a)   \(A = \left\{ \begin{pmatrix} a & b \\ c & d \end{pmatrix} : a, b, c, d \in \mathbb{R}, a + b = 1,  \right\}\)

\end{sphinxuseclass}
\end{sphinxuseclass}
\end{sphinxuseclass}
\end{sphinxuseclass}
\end{sphinxuseclass}
\end{sphinxuseclass}
\end{sphinxuseclass}
\begin{sphinxuseclass}{sd-col}
\begin{sphinxuseclass}{sd-d-flex-column}
\begin{sphinxuseclass}{sd-col-6}
\begin{sphinxuseclass}{sd-col-xs-6}
\begin{sphinxuseclass}{sd-col-sm-6}
\begin{sphinxuseclass}{sd-col-md-6}
\begin{sphinxuseclass}{sd-col-lg-6}
\sphinxAtStartPar
(b)   \(B = \left\{ \begin{pmatrix} a & b \\ c & d \end{pmatrix}: a, b, c, d \in \mathbb{R}, a = c = d \right\}\)

\end{sphinxuseclass}
\end{sphinxuseclass}
\end{sphinxuseclass}
\end{sphinxuseclass}
\end{sphinxuseclass}
\end{sphinxuseclass}
\end{sphinxuseclass}
\begin{sphinxuseclass}{sd-col}
\begin{sphinxuseclass}{sd-d-flex-column}
\begin{sphinxuseclass}{sd-col-12}
\begin{sphinxuseclass}{sd-col-xs-12}
\begin{sphinxuseclass}{sd-col-sm-12}
\begin{sphinxuseclass}{sd-col-md-12}
\begin{sphinxuseclass}{sd-col-lg-12}
\sphinxAtStartPar
 

\end{sphinxuseclass}
\end{sphinxuseclass}
\end{sphinxuseclass}
\end{sphinxuseclass}
\end{sphinxuseclass}
\end{sphinxuseclass}
\end{sphinxuseclass}
\begin{sphinxuseclass}{sd-col}
\begin{sphinxuseclass}{sd-d-flex-column}
\begin{sphinxuseclass}{sd-col-6}
\begin{sphinxuseclass}{sd-col-xs-6}
\begin{sphinxuseclass}{sd-col-sm-6}
\begin{sphinxuseclass}{sd-col-md-6}
\begin{sphinxuseclass}{sd-col-lg-6}
\sphinxAtStartPar
(c)   \(C = \{ A \in M_{2\times 2} : A^2 = A \}\)

\end{sphinxuseclass}
\end{sphinxuseclass}
\end{sphinxuseclass}
\end{sphinxuseclass}
\end{sphinxuseclass}
\end{sphinxuseclass}
\end{sphinxuseclass}
\end{sphinxuseclass}
\end{sphinxuseclass}
\end{sphinxuseclass}
\end{sphinxuseclass}\end{sphinxadmonition}
\phantomsection \label{exercise:vector-spaces-R3-basis}

\begin{sphinxadmonition}{note}{Exercise 5.5.4}



\sphinxAtStartPar
Prove that \(\left\{ (1, 2, 0), (0, 5, 7), (-1, 1, 3) \right\}\) is a basis for \(\mathbb{R}^3\) and represent the vectors \((0, 13, 17)\) and \((2, 3, 1)\) with respect to this basis.
\end{sphinxadmonition}
\phantomsection \label{exercise:vector-spaces-ex-R4-basis}

\begin{sphinxadmonition}{note}{Exercise 5.5.5}



\sphinxAtStartPar
Extend \(\{ (1, 1, 2, 4), (2, -1, -5, 2)\}\) to a basis of \(\mathbb{R}^4\).
\end{sphinxadmonition}
\phantomsection \label{exercise:vector-spaces-ex-R4-basis-2}

\begin{sphinxadmonition}{note}{Exercise 5.5.6}



\sphinxAtStartPar
Suppose that \(W = \{\mathbf{u}, \mathbf{v}, \mathbf{w}, \mathbf{x}, \mathbf{y}\}\) is a subspace of \(\mathbb{R}^n\) where
\begin{equation*}
\begin{split} \begin{align*}
    \mathbf{u} &= \begin{pmatrix} 1 \\ 2 \\ 3 \\ 4 \end{pmatrix}, &
    \mathbf{v} &= \begin{pmatrix} 1 \\ -1 \\ 2 \\ 0 \end{pmatrix}, &
    \mathbf{w} &= \begin{pmatrix} 2 \\ 1 \\ 5 \\ 3 \end{pmatrix}, &
    \mathbf{x} &= \begin{pmatrix} 1 \\ -1 \\ 0 \\ 3 \end{pmatrix}, &
    \mathbf{y} &= \begin{pmatrix} 1 \\ -1 \\ 0 \\ 4 \end{pmatrix}.
\end{align*} \end{split}
\end{equation*}
\sphinxAtStartPar
Find a basis for \(W\) and determine \(\dim(W)\).
\end{sphinxadmonition}

\sphinxstepscope

\index{Linear transformations@\spxentry{Linear transformations}}\ignorespaces 

\chapter{Linear Transformations}
\label{\detokenize{_pages/6.0_Linear_transformations:linear-transformations}}\label{\detokenize{_pages/6.0_Linear_transformations:index-0}}\label{\detokenize{_pages/6.0_Linear_transformations:linear-transformations-chapter}}\label{\detokenize{_pages/6.0_Linear_transformations::doc}}
\sphinxAtStartPar
Functions take inputs from one set and produce an output in another set. To describe these kinds of mappings, we use notation of the form \(f: X \to Y\), where \(f\) is the function, \(X\) is the set of inputs, or \sphinxstylestrong{domain}, and \(Y\) is the set of outputs, or \sphinxstylestrong{co\sphinxhyphen{}domain}.

\sphinxAtStartPar
The function defines the way elements from the domain are mapped to elements of the codomain, and is denoted \(y = f(x)\) where \(x\) is an element of the domain, and \(y\) is an element of the co\sphinxhyphen{}domain.

\sphinxAtStartPar
In linear algebra, we study functions between set of vectors, and we write \(T: V \to W\), where \(V\) and \(W\) are vector spaces and the function mapping an input vector \(\mathbf{v} \in V\) to an output vector \(\mathbf{w} \in W\) is given by \(\mathbf{w} = T(\mathbf{v})\). The result of applying a function to an input is known as the \sphinxstylestrong{image}.

\sphinxAtStartPar
\sphinxstylestrong{Linear transformations}, also called linear mappings, are a specific type of function on vectors. They have many uses in mathematics and computing. A good example is in the field of computer graphics and computer games, where they are fundamental to the manipulation and visualisation of three\sphinxhyphen{}dimensional objects.

\sphinxAtStartPar
We begin with the formal definition of a linear transformation.
\label{_pages/6.0_Linear_transformations:linear-transformation-definition}
\begin{sphinxadmonition}{note}{Definition 6.1 (Linear transformation)}



\sphinxAtStartPar
If \(V\) and \(W\) are two vector spaces over the same field \(F\), then by a \sphinxstylestrong{linear transformation} (or \sphinxstylestrong{linear mapping}) is a function \(T: V \to W\) such that for any two vectors \(\mathbf{u}, \mathbf{v} \in V\) and any scalar \(\alpha \in F\) the following conditions hold:
\begin{itemize}
\item {} 
\sphinxAtStartPar
addition operation: \(T(\mathbf{u} + \mathbf{v}) = T(\mathbf{u}) + T(\mathbf{v})\);

\item {} 
\sphinxAtStartPar
scalar multiplication:  \(T(\alpha \mathbf{u}) = \alpha T(\mathbf{u})\).

\end{itemize}
\end{sphinxadmonition}

\sphinxAtStartPar
For example, let \(V = \mathbb{R}^2\) and \(W = \mathbb{R}^3\), both defined as vector spaces over \(\mathbb{R}\). Then \(T : V \to W\) defined by \(T : (x, y) \mapsto (x, y, 0)\) is a linear transformation.

\sphinxAtStartPar
Indeed, let \(\mathbf{u} = (u_1, u_2), \mathbf{v} = (v_1, v_2) \in \mathbb{R}^2\) and \(\alpha \in \mathbb{R}\). Then we can check:
\begin{equation*}
\begin{split} \begin{align*}
    T (\mathbf{u} + \mathbf{v})
    &= T \begin{pmatrix} u_1 + v_1 \\ u_2 + v_2 \end{pmatrix}
    = \begin{pmatrix} u_1 + v_1 \\ u_2 + v_2 \\ 0 \end{pmatrix}, \\
    T (\mathbf{u}) + T(\mathbf{v}) &=
    \begin{pmatrix} u_1 \\ u_2 \\ 0 \end{pmatrix} + \begin{pmatrix} v_1 \\ v_2 \\ 0 \end{pmatrix}
    = \begin{pmatrix} u_1 + v_1 \\ u_2 + v_2 \\ 0 \end{pmatrix},
\end{align*} \end{split}
\end{equation*}
\sphinxAtStartPar
so \(T(\mathbf{u} + \mathbf{v}) = T(\mathbf{u}) + T(\mathbf{v})\) and the addition condition is satisfied. Similarly:
\begin{equation*}
\begin{split} \begin{align*}
    T(\alpha \mathbf{u})
    &= T \begin{pmatrix} \alpha u_1 \\ \alpha u_2 \end{pmatrix}
    = \begin{pmatrix} \alpha u_1 \\ \alpha u_2 \\ 0 \end{pmatrix}, \\
    \alpha T(\mathbf{u}) &= \alpha \begin{pmatrix} u_1 \\ u_2 \\ 0 \end{pmatrix}
    = \begin{pmatrix} \alpha u_1 \\ \alpha u_2 \\ 0 \end{pmatrix},
\end{align*} \end{split}
\end{equation*}
\sphinxAtStartPar
so \(T(\alpha \mathbf{u}) = \alpha T(\mathbf{u})\), so the scalar multiplication condition is satisfied. Combined with the addition condition being satisfied, we have shown that \(T\) is a linear transformation. We can combine the addition and scalar multiplication conditions to give a single condition.
\label{_pages/6.0_Linear_transformations:linear-transformation-condition -definition}
\begin{sphinxadmonition}{note}{Definition 6.2 (Linear transformation condition)}



\sphinxAtStartPar
A transformation \(T : V \to W\) is a linear transformation if the following condition is satisfied for any \(\mathbf{u}, \mathbf{v} \in V\) and \(\alpha \in F\):
\begin{equation}\label{equation:_pages/6.0_Linear_transformations:linear-transformation-condition-equation}
\begin{split} T(\mathbf{u} + \alpha \mathbf{v}) = T(\mathbf{u}) + \alpha T(\mathbf{v}). \end{split}
\end{equation}\end{sphinxadmonition}
\label{_pages/6.0_Linear_transformations:linear-transformation-example}
\begin{sphinxadmonition}{note}{Example 6.1}



\sphinxAtStartPar
Determine which of the following transformations are linear transformations

\sphinxAtStartPar
(i)   \(T: \mathbb{R}^3 \to \mathbb{R}^2\) defined by \(T: (x, y, z) \mapsto (x, y)\)

\sphinxAtStartPar
(ii)   \(T: \mathbb{R}^3 \to \mathbb{R}^2\) defined by \(T: (x, y, z) \mapsto (x + 3, y)\)

\sphinxAtStartPar
(iii)   \(T: P(\mathbb{R}) \to P(\mathbb{R})\) (the space of polynomials with coefficients in \(\mathbb{R}\)) defined by \(T: p \mapsto p \dfrac{\mathrm{d}p}{\mathrm{d}x}\)
\subsubsection*{Solution}

\sphinxAtStartPar
(i)   Let \(\mathbf{u} = (u_1, u_2, u_3), \mathbf{v} = (v_1, v_2, v_3) \in \mathbb{R}^3\) and \(\alpha \in \mathbb{R}\). Then:
\begin{equation*}
\begin{split} \begin{align*}
    T(\mathbf{u} + \alpha \mathbf{v}) &= T\left(
        \begin{pmatrix} u_1 \\ u_2 \\ u_3 \end{pmatrix} + \alpha
        \begin{pmatrix} v_1 \\ v_2 \\ v_3 \end{pmatrix}
    \right)
    = T \begin{pmatrix} u_1 + \alpha v_1 \\ u_2 + \alpha v_2 \\ u_3 + \alpha v_3 \end{pmatrix}
    = \begin{pmatrix} u_1 + \alpha v_1 \\  u_2 + \alpha v_2 \end{pmatrix}, \\
    T(\mathbf{u}) + \alpha T(\mathbf{v}) &= T
    \begin{pmatrix} u_1 \\ u_2 \\ u_3 \end{pmatrix} + \alpha T
    \begin{pmatrix} v_1 \\ v_2 \\ v_3 \end{pmatrix} =
    \begin{pmatrix} u_1 \\ u_2 \end{pmatrix} + \alpha
    \begin{pmatrix} v_1 \\ v_2 \end{pmatrix} =
    \begin{pmatrix} u_1 + \alpha v_1 \\ u_2 + \alpha v_2 \end{pmatrix}.
\end{align*} \end{split}
\end{equation*}
\sphinxAtStartPar
Since \(T(\mathbf{u} + \alpha \mathbf{v}) = T(\mathbf{u}) + \alpha T(\mathbf{v})\) then \(T: (x, y, z) \mapsto (x, y)\) is a linear transformation.

\sphinxAtStartPar
(ii)   Let \(\mathbf{u} = (u_1, u_2, u_3), \mathbf{v} = (v_1, v_2, v_3) \in \mathbb{R}^3\) and \(\alpha \in \mathbb{R}\). Then:
\begin{equation*}
\begin{split} \begin{align*}
    T(\mathbf{u} + \alpha \mathbf{v}) &= T
    \begin{pmatrix} u_1 + \alpha v_1 \\ u_2 + \alpha v_2 \\ u_3 + \alpha v_3 \end{pmatrix}
    = \begin{pmatrix} u_1 + \alpha v_1 + 3 \\ u_2 + \alpha v_2 \end{pmatrix}, \\
    T(\mathbf{u}) + \alpha T(\mathbf{v}) &= T
    \begin{pmatrix} u_1 \\ u_2 \\ u_3 \end{pmatrix} + \alpha T
    \begin{pmatrix} v_1 \\ v_2 \\ v_3 \end{pmatrix} =
    \begin{pmatrix} u_1 + 3 \\ u_2 \end{pmatrix} + \alpha
    \begin{pmatrix} v_1 + 3 \\ v_2 \end{pmatrix} \\
    =
    \begin{pmatrix} u_1 + v_1 + 3 + 3\alpha \\ u_2 + \alpha v_2 \end{pmatrix}
\end{align*} \end{split}
\end{equation*}
\sphinxAtStartPar
Since \(T(\mathbf{u} + \alpha \mathbf{v}) \neq T(\mathbf{u}) + \alpha T(\mathbf{v})\) then \(T: (x, y, z) \mapsto (x + 3, y)\) is not a linear transformation.

\sphinxAtStartPar
Note that we could have shown this by a counterexample. For example, let \(\mathbf{u} = ( 1, 0 , 0 )\) and \(\mathbf{v} = (2, 0, 0) \in \mathbb{R}^3\). Then
\begin{equation*}
\begin{split} \begin{align*}
    T(\mathbf{u} + \mathbf{v}) &= T
    \begin{pmatrix} 3 \\ 0 \\ 0 \end{pmatrix}
    = \begin{pmatrix} 6 \\ 0 \end{pmatrix}, \\
    T(\mathbf{u}) + T(\mathbf{v}) &=
    \begin{pmatrix} 4 \\ 0 \\ 0 \end{pmatrix} + \begin{pmatrix} 5 \\ 0 \\ 0 \end{pmatrix}
    = \begin{pmatrix} 9 \\ 0 \end{pmatrix}.
\end{align*} \end{split}
\end{equation*}
\sphinxAtStartPar
(iii)   Again, we can show this is not a linear transformation by choosing a suitable counterexample. Let \(u = x \in P(\mathbb{R})\), and choose \(\alpha = 2 \in \mathbb{R}\). Then:
\begin{equation*}
\begin{split} \begin{align*}
    T(2u) &= T(2x) = 2x \cdot 2 = 4x, \\
    2T(u) &= 2T(x) = 2 \cdot x \cdot 1 = 2x
\end{align*}  \end{split}
\end{equation*}
\sphinxAtStartPar
Therefore \(T(2u) \neq 2T(u)\) and \(T: p \mapsto p \dfrac{\mathrm{d}p}{\mathrm{d}x}\) is not a linear transformation.
\end{sphinxadmonition}

\index{Linear transformations@\spxentry{Linear transformations}!transformation matrices@\spxentry{transformation matrices}}\ignorespaces \phantomsection\label{\detokenize{_pages/6.0_Linear_transformations:transformation-matrix-section}}
\sphinxstepscope


\section{Transformation matrices}
\label{\detokenize{_pages/6.1_Transformation_matrices:transformation-matrices}}\label{\detokenize{_pages/6.1_Transformation_matrices:transformation-matrices-section}}\label{\detokenize{_pages/6.1_Transformation_matrices::doc}}
\sphinxAtStartPar
One of the benefits of studying linear transformations is that since they are linear, they can be represented using matrices. Let \(T: V \to W\) be a linear transformation between the vector spaces \(V\) and \(W\) where \(V, W \subset \mathbb{R}^n\). If \(\{\mathbf{v}_1, \mathbf{v}_2, \ldots, \mathbf{v}_n\}\) is a basis for \(V\), then for a vector \(\mathbf{u} \in V\), we can write this in terms of the basis:
\begin{equation*}
\begin{split} \mathbf{u} = u_1 \mathbf{v}_1 + u_2 \mathbf{v}_2 + \cdots + u_n \mathbf{v}_n, \end{split}
\end{equation*}
\sphinxAtStartPar
By the {\hyperref[\detokenize{_pages/6.0_Linear_transformations:linear-transformation-definition}]{\sphinxcrossref{definition of a linear transformation}}}, we can apply a linear transformation \(T\) to the vector \(\mathbf{u}\) by applying it separately to each of the \(\mathbf{v}_1, \mathbf{v}_2, \ldots, \mathbf{v}_n\):
\begin{equation*}
\begin{split} T(\mathbf{u}) = u_1 T(\mathbf{v}_1) + u_2 T(\mathbf{v}_2) + \cdots + u_n T(\mathbf{v}_n)\end{split}
\end{equation*}
\sphinxAtStartPar
This means that \(T(\mathbf{u})\) depends on the vectors \(T(\mathbf{v}_1), T(\mathbf{v}_2), \ldots, T(\mathbf{v}_n)\). We can write this as the matrix equation:
\begin{equation*}
\begin{split} \begin{align*}
    T(\mathbf{u}) &= \begin{pmatrix}
        \uparrow & \uparrow & & \uparrow \\
        T(\mathbf{v}_1) & T(\mathbf{v}_2) & \cdots & T(\mathbf{v}_n) \\
        \downarrow & \downarrow & & \downarrow
    \end{pmatrix}
    \begin{pmatrix} u_1 \\ u_2 \\ \vdots \\ u_n \end{pmatrix}
    = A \mathbf{u}.
\end{align*} \end{split}
\end{equation*}
\sphinxAtStartPar
In other words, we can apply the linear transformation by multiplying \(\mathbf{u}\) by the matrix \(A\), which is created by applying \(T\) to each of the basis vectors and placing the resulting vectors into the columns of the matrix.
\label{_pages/6.1_Transformation_matrices:transformation-matrix-definition}
\begin{sphinxadmonition}{note}{Definition 6.1.1 (Transformation matrix)}



\sphinxAtStartPar
Let \(T : V \to W\) be a linear transformation, and let \(A\) be a matrix such that
\begin{equation*}
\begin{split} A = \begin{pmatrix}
    \uparrow & \uparrow & & \uparrow \\
    T(\mathbf{v}_1) & T(\mathbf{v}_2) & \cdots & T(\mathbf{v}_n) \\
    \downarrow & \downarrow & & \downarrow
\end{pmatrix} \end{split}
\end{equation*}
\sphinxAtStartPar
then
\begin{equation*}
\begin{split} T(\mathbf{u}) = A\mathbf{u}. \end{split}
\end{equation*}
\sphinxAtStartPar
\(A\) is said to be the \sphinxstylestrong{matrix representation of the linear transformation} \(T\) (also known as the \sphinxstylestrong{transformation matrix}).
\end{sphinxadmonition}
\label{_pages/6.1_Transformation_matrices:transformation-matrix-example}
\begin{sphinxadmonition}{note}{Example 6.1.1}



\sphinxAtStartPar
A linear transformation \(T:\mathbb{R}^2 \to \mathbb{R}^2\) is defined by \(T: (x, y) \mapsto (3x + y, x + 2y)\). Calculate the transformation matrix and use it to calculate \(T(1,1)\).
\subsubsection*{Solution}

\sphinxAtStartPar
Since we are mapping from \(\mathbb{R}^2\) the transformation matrix is
\begin{equation*}
\begin{split} A = \begin{pmatrix} T(\mathbf{e}_1) & T(\mathbf{e}_2) \end{pmatrix} \end{split}
\end{equation*}
\sphinxAtStartPar
Applying the transformation to the standard basis vectors
\begin{equation*}
\begin{split} \begin{align*}
    T(\mathbf{e}_1) = T\begin{pmatrix} 1 \\ 0 \end{pmatrix}
    = \begin{pmatrix} 3(1) + 0 \\ 1 + 2(0) \end{pmatrix}
    = \begin{pmatrix} 3 \\ 1 \end{pmatrix}, \\
    T(\mathbf{e}_2) = T\begin{pmatrix} 0 \\ 1 \end{pmatrix}
    = \begin{pmatrix} 3(0) + 1 \\ 0 + 2(1) \end{pmatrix}
    = \begin{pmatrix} 1 \\ 2 \end{pmatrix},
\end{align*} \end{split}
\end{equation*}
\sphinxAtStartPar
so the transformation matrix is
\begin{equation*}
\begin{split} A = \begin{pmatrix} 3 & 1 \\ 1 & 2 \end{pmatrix}. \end{split}
\end{equation*}
\sphinxAtStartPar
Applying the transformation matrix to \((1, 1)\):
\begin{equation*}
\begin{split} T\begin{pmatrix} 1 \\ 1 \end{pmatrix} = A \cdot \begin{pmatrix} 1 \\ 1 \end{pmatrix} =  
    \begin{pmatrix} 3 & 1 \\ 1 & 2 \end{pmatrix}
    \begin{pmatrix} 1 \\ 1 \end{pmatrix} =
    \begin{pmatrix} 4 \\ 3 \end{pmatrix}. \end{split}
\end{equation*}\end{sphinxadmonition}

\sphinxAtStartPar
The effect of applying the linear transformation from {\hyperref[\detokenize{_pages/6.0_Linear_transformations:linear-transformation-example}]{\sphinxcrossref{Example 6.1}}} is illustrated in \hyperref[\detokenize{_pages/6.1_Transformation_matrices:linear-transformation-figure}]{Fig.\@ \ref{\detokenize{_pages/6.1_Transformation_matrices:linear-transformation-figure}}}. Note that the transformation \(T\) can be thought of as changing the basis of the vector space. The unit square with respect to the basis \(\{\mathbf{e}_1, \mathbf{e}_1\}\) has been transformed into a unit parallelogram with respect to the basis \(\{ T(\mathbf{e}_1), T(\mathbf{e}_2)\}\).

\begin{figure}[htbp]
\centering
\capstart

\noindent\sphinxincludegraphics[width=400\sphinxpxdimen]{{6_linear_transformation}.svg}
\caption{The effect of applying a linear transformation \(T: (x,y) \mapsto (3x + y, x + 2y)\) to the vector \((1,1)\).}\label{\detokenize{_pages/6.1_Transformation_matrices:linear-transformation-figure}}\end{figure}

\sphinxAtStartPar
We can also apply a transformation matrix to a set of vectors at once, by putting them into a matrix, and multiplying the matrix by \(A\) on the left. Let \(\mathbf{u}_1, \mathbf{u}_2, \ldots, \mathbf{u}_n\) be a set of vectors in \(V\). Then:
\begin{equation*}
\begin{split} \begin{align*}
    A
    \begin{pmatrix}
        \uparrow & \uparrow & & \uparrow \\
        \mathbf{u}_1 & \mathbf{u}_2 & \cdots & \mathbf{u}_n \\
        \downarrow & \downarrow & & \downarrow
    \end{pmatrix} =
    \begin{pmatrix}
        \uparrow & \uparrow & & \uparrow \\
        T(\mathbf{u}_1) & T(\mathbf{u}_2) & \cdots & T(\mathbf{u}_n) \\
        \downarrow & \downarrow & & \downarrow
    \end{pmatrix}
\end{align*} \end{split}
\end{equation*}

\bigskip\hrule\bigskip



\subsection{Finding the transformation matrix from the image of a transformation}
\label{\detokenize{_pages/6.1_Transformation_matrices:finding-the-transformation-matrix-from-the-image-of-a-transformation}}
\sphinxAtStartPar
The calculation of the transformation matrix in {\hyperref[\detokenize{_pages/6.1_Transformation_matrices:transformation-matrix-example}]{\sphinxcrossref{Example 6.1.1}}} was straightforward, as we knew what the transformation was. This will not always be a the case, and we may know what the output of the transformation (known as the image) is \sphinxhyphen{} but not the transformation itself.

\sphinxAtStartPar
Consider a linear transformation \(T: \mathbb{R}^n \to \mathbb{R}^m\) applied to vectors \(\mathbf{u}_1, \mathbf{u}_2, \ldots, \mathbf{u}_n\). If \(A\) is the transformation matrix for \(T\), we have:
\begin{equation*}
\begin{split} \begin{align*}
    A
    \begin{pmatrix}
        \uparrow & \uparrow & & \uparrow \\
        \mathbf{u}_1 & \mathbf{u}_2 & \cdots & \mathbf{u}_n \\
        \downarrow & \downarrow & & \downarrow
    \end{pmatrix} =
    \begin{pmatrix}
        \uparrow & \uparrow & & \uparrow \\
        T(\mathbf{u}_1) & T(\mathbf{u}_2) & \cdots & T(\mathbf{u}_n) \\
        \downarrow & \downarrow & & \downarrow
    \end{pmatrix}
\end{align*} \end{split}
\end{equation*}
\sphinxAtStartPar
Therefore we can multoply by the inverse of the matrix of vectors \(\mathbb{u}_i\) on the right to get an expression we can use to work out \(A\):
\begin{equation*}
\begin{split} \begin{align*}
    A &=  
    \begin{pmatrix}
        \uparrow & \uparrow & & \uparrow \\
        T(\mathbf{u}_1) & T(\mathbf{u}_2) & \cdots & T(\mathbf{u}_n) \\
        \downarrow & \downarrow & & \downarrow
    \end{pmatrix}
    \begin{pmatrix}
        \uparrow & \uparrow & & \uparrow \\
        \mathbf{u}_1 & \mathbf{u}_2 & \cdots & \mathbf{u}_n \\
        \downarrow & \downarrow & & \downarrow
    \end{pmatrix}^{-1}
\end{align*} \end{split}
\end{equation*}\label{_pages/6.1_Transformation_matrices:finding-transformation-matrix-theorem}
\begin{sphinxadmonition}{note}{Theorem 6.1.1 (Determining the linear transformation given the inputs and image vectors)}



\sphinxAtStartPar
Given a linear transformation \(T: \mathbb{R}^n \to \mathbb{R}^m\) applied to a set of \(n\) vectors \(\mathbf{u}_1, \mathbf{u}_2, \ldots, \mathbf{u}_n\) with known image vectors \(T(\mathbf{u}_1), T(\mathbf{u}_2), \ldots, T(\mathbf{u}_n)\) then the transformation matrix \(A\) for \(T\) is:
\begin{equation}\label{equation:_pages/6.1_Transformation_matrices:determining-the-transformation-matrix}
\begin{split} A = \begin{pmatrix} T(\mathbf{u}_1) & T(\mathbf{u}_2) & \cdots & T(\mathbf{u}_n) \end{pmatrix}  
\begin{pmatrix} \mathbf{u}_1 & \mathbf{u}_2 & \cdots & \mathbf{u}_n \end{pmatrix}^{-1}. \end{split}
\end{equation}\end{sphinxadmonition}
\label{_pages/6.1_Transformation_matrices:transformation-matrix-example-2}
\begin{sphinxadmonition}{note}{Example 6.1.2}



\sphinxAtStartPar
Determine the transformation matrix \(A\) for the linear transformation \(T\) such that
\begin{equation*}
\begin{split} \begin{align*}
    T\begin{pmatrix} 1 \\ 1 \end{pmatrix} &= \begin{pmatrix} 4 \\ 3 \end{pmatrix}, &
    T\begin{pmatrix} 1 \\ 2 \end{pmatrix} &= \begin{pmatrix} 5 \\ 5 \end{pmatrix}.
\end{align*} \end{split}
\end{equation*}\subsubsection*{Solution}

\sphinxAtStartPar
The inverse of \((\mathbf{u}_1, \mathbf{u}_2)\) is:
\begin{equation*}
\begin{split} \begin{align*}
    \begin{pmatrix} 1 & 1 \\ 1 & 2 \end{pmatrix}^{-1} &= \begin{pmatrix} 2 & -1 \\ -1 & 1 \end{pmatrix}
\end{align*} \end{split}
\end{equation*}
\sphinxAtStartPar
If we multiply the image matrix by this, we get:
\begin{equation*}
\begin{split} \begin{align*}
    A &= \begin{pmatrix} 4 & 5 \\ 3 & 5 \end{pmatrix} \begin{pmatrix} 2 & -1 \\ -1 & 1 \end{pmatrix}
    = \begin{pmatrix} 3 & 1 \\ 1 & 2 \end{pmatrix}.
\end{align*} \end{split}
\end{equation*}
\sphinxAtStartPar
This is the transformation matrix from {\hyperref[\detokenize{_pages/6.1_Transformation_matrices:transformation-matrix-example}]{\sphinxcrossref{Example 6.1.1}}}.
\end{sphinxadmonition}


\bigskip\hrule\bigskip


\index{Linear transformations@\spxentry{Linear transformations}!inverse transformation@\spxentry{inverse transformation}}\ignorespaces 

\subsection{Inverse linear transformation}
\label{\detokenize{_pages/6.1_Transformation_matrices:inverse-linear-transformation}}\label{\detokenize{_pages/6.1_Transformation_matrices:index-0}}\label{_pages/6.1_Transformation_matrices:inverse-transformation-definition}
\begin{sphinxadmonition}{note}{Definition 6.1.2 (Inverse linear transformation)}



\sphinxAtStartPar
Let \(T: V \to W\) be a linear transformation with the transformation matrix \(A\). Then \(T\) has an \sphinxstylestrong{inverse transformation} denoted by \(T^{-1}: W \to V\) which reverses the effects of \(T\). If \(\mathbf{v} \in V\), and \(\mathbf{w}\) is the image of \(v\) in \(W\), then:
\begin{equation*}
\begin{split} \begin{align*}
    \mathbf{w} &= A \mathbf{v} \\
    \therefore \mathbf{v} & = A^{-1}\mathbf{w},
\end{align*} \end{split}
\end{equation*}
\sphinxAtStartPar
where \(A^{-1}\) is the transformation matrix for \(T^{-1}\).
\end{sphinxadmonition}
\label{_pages/6.1_Transformation_matrices:inverse-transformation-example}
\begin{sphinxadmonition}{note}{Example 6.1.3}



\sphinxAtStartPar
Determine the inverse of the transformation \(T: \mathbb{R}^2 \to \mathbb{R}^2\) defined by \(T(x, y) \mapsto (3 x + y, x + 2 y)\) and calculate \(T^{-1}(4,3)\).
\subsubsection*{Solution}

\sphinxAtStartPar
We saw in {\hyperref[\detokenize{_pages/6.1_Transformation_matrices:transformation-matrix-example}]{\sphinxcrossref{Example 6.1.1}}} that the transformation matrix for \(T\) is:
\begin{equation*}
\begin{split} A = \begin{pmatrix} 3 & 1 \\ 1 & 2 \end{pmatrix}, \end{split}
\end{equation*}
\sphinxAtStartPar
which has the inverse:
\begin{equation*}
\begin{split} A^{-1} = \frac{1}{5} \begin{pmatrix} 2 & -1 \\ -1 & 3 \end{pmatrix}. \end{split}
\end{equation*}
\sphinxAtStartPar
Determining the inverse transformation:
\begin{equation*}
\begin{split} \begin{align*}
    T^{-1}\begin{pmatrix} x \\ y \end{pmatrix} &= A^{-1} \cdot \begin{pmatrix} x \\ y \end{pmatrix} =
    \frac{1}{5} \begin{pmatrix} 2 & -1 \\ -1 & 3 \end{pmatrix}
    \begin{pmatrix} x \\ y \end{pmatrix} \\
    &= \begin{pmatrix} \frac{2}{5}x - \frac{1}{5}y \\ -\frac{1}{5}x + \frac{3}{5}y \end{pmatrix}.
\end{align*} \end{split}
\end{equation*}
\sphinxAtStartPar
Calculating \(T^{-1}\begin{pmatrix} 4 \\ 3 \end{pmatrix}\):
\begin{equation*}
\begin{split} \begin{align*}
    A^{-1} \begin{pmatrix} 4 \\ 3 \end{pmatrix} &=
   \frac{1}{5} \begin{pmatrix} 2 & -1 \\ -1 & 3 \end{pmatrix}
    \begin{pmatrix} 4 \\ 3 \end{pmatrix}
    = \begin{pmatrix} 1 \\ 1 \end{pmatrix}.
\end{align*} \end{split}
\end{equation*}\end{sphinxadmonition}

\sphinxstepscope

\index{Linear transformations@\spxentry{Linear transformations}!composite transformations@\spxentry{composite transformations}}\ignorespaces 

\section{Composite linear transformations}
\label{\detokenize{_pages/6.2_Composite_transformations:composite-linear-transformations}}\label{\detokenize{_pages/6.2_Composite_transformations:index-0}}\label{\detokenize{_pages/6.2_Composite_transformations:composite-linear-transformations-section}}\label{\detokenize{_pages/6.2_Composite_transformations::doc}}\label{_pages/6.2_Composite_transformations:composite-transformation-definition}
\begin{sphinxadmonition}{note}{Definition 6.2.1 (Composite transformations)}



\sphinxAtStartPar
Let \(T : V \to W\) and \(S: W \to X\) be two linear transformations over the vector spaces \(V, W\) and \(X\). The \sphinxstylestrong{composition} of \(S\) and \(T\) is the transformation \(S \circ T: V \to X\) defined by
\begin{equation*}
\begin{split} (S \circ T)(\mathbf{u}) = S(T(\mathbf{u})), \end{split}
\end{equation*}
\sphinxAtStartPar
for all vectors \(\mathbf{u} \in V\).
\end{sphinxadmonition}

\sphinxAtStartPar
For example, let \(T\) and \(S\) be two linear transformations on \(\mathbb{R}^3\), defined by \(T:(x, y, z) \mapsto (2 x + 4 y, -x + 3 y, x + 2 y)\) and \(S:(x, y, z) \mapsto (2x + y - z, 3x + z, y - 2z)\).

\sphinxAtStartPar
The composite linear transformation \(S \circ T(x, y, z)\) is
\begin{equation*}
\begin{split} \begin{align*}
    S \circ T\begin{pmatrix} x \\ y \\ z \end{pmatrix} &=
    S \left( T\begin{pmatrix} x \\ y \\ z \end{pmatrix} \right)
    = S \begin{pmatrix} 2 x + 4 \\ - x + 3y \\ x + 2 y \end{pmatrix} \\
    &= \begin{pmatrix}
        2(2 x + 4 y) + (-x + 3 y) - (x + 2 y) \\
        3(2 x + 4 y) + (x + 2 y) \\
        (-x + 3 y) - 2(x + 2 y)
    \end{pmatrix} \\
    &= \begin{pmatrix} 2 x + 9 y \\ 7 x + 14 y \\ -3 x - y \end{pmatrix},
\end{align*} \end{split}
\end{equation*}
\sphinxAtStartPar
i.e., \(S \circ T(x,y , z) \mapsto (2x + 9y, 7x + 14y, -3x - y)\).


\bigskip\hrule\bigskip



\subsection{Composite transformation matrices}
\label{\detokenize{_pages/6.2_Composite_transformations:composite-transformation-matrices}}
\sphinxAtStartPar
We have seen that a linear transformation \(T: V \to W\) can be represented by a {\hyperref[\detokenize{_pages/6.1_Transformation_matrices:transformation-matrix-definition}]{\sphinxcrossref{transformation matrix}}}, so that given a vector \(\mathbf{u} \in V\) the image is calculated as:
\begin{equation}\label{equation:_pages/6.2_Composite_transformations:composite-transformation-matrix-equation-1}
\begin{split} T(\mathbf{u}) = A \mathbf{u}. \end{split}
\end{equation}
\sphinxAtStartPar
Consider the composition of \(T\) with another linear transformation \(S: W \to X\), where \(S\) has a transformation matrix \(B\):
\begin{equation}\label{equation:_pages/6.2_Composite_transformations:composite-transformation-matrix-equation-2}
\begin{split} S \circ T(\mathbf{u}) = B \cdot T(\mathbf{u}) \end{split}
\end{equation}
\sphinxAtStartPar
Substituting equation \eqref{equation:_pages/6.2_Composite_transformations:composite-transformation-matrix-equation-1} into equation \eqref{equation:_pages/6.2_Composite_transformations:composite-transformation-matrix-equation-2} gives
\begin{equation*}
\begin{split} S \circ T (\mathbf{u}) = B \cdot A \mathbf{u}. \end{split}
\end{equation*}\label{_pages/6.2_Composite_transformations:composite-transformation-matrices-theorem}
\begin{sphinxadmonition}{note}{Theorem 6.2.1 (Composite transformation matrices)}



\sphinxAtStartPar
Given two linear transformations \(T:V \to W\) and \(S:W \to X\) with transformation matrices \(A\) and \(B\) respectively, the image of the vector \(\mathbf{u} under the composition \)S \textbackslash{}circ T\(  \in V\) is:
\begin{equation}\label{equation:_pages/6.2_Composite_transformations:composite-transformation-matrices-theorem-equation}
\begin{split} S \circ T (\mathbf{u}) = B \cdot A \cdot \mathbf{u}. \end{split}
\end{equation}\end{sphinxadmonition}

\sphinxAtStartPar
Note that the matrices here occur in the same order as the functions are listed in the composition, which is the reverse of the order in which they are applied to the vector. Here, we apply the function \(T\) first, then the function \(S\), but they are written as \(S\) first then \(T\).

\sphinxAtStartPar
To apply the transformation, we can multiply the matrices together and then multiply this by the vector on the left \sphinxhyphen{} but note that we need to calculate \(B \cdot A\), since the order of matrix multiplication matters.
\label{_pages/6.2_Composite_transformations:composite-transformation-matrix-example}
\begin{sphinxadmonition}{note}{Example 6.2.1}



\sphinxAtStartPar
Write down the transformation matrices \(A\) and \(B\) for the transformations \(T:(x, y, z) \mapsto (2 x + 4 y, -x + 3 y, x + 2 y)\) and \(S:(x, y, z) \mapsto (2x + y - z, 3x + z, y - 2z)\) respectively. Use them to calculate the transformation matrix \(C\) for \(S\circ T\).
\subsubsection*{Solution}

\sphinxAtStartPar
The transformation matrices for \(T\) and \(S\) are:
\begin{equation*}
\begin{split} \begin{align*}
    A &= \begin{pmatrix} 2 & 4 \\ -1 & 3 \\ 1 & 2 \end{pmatrix}, &
    B &= \begin{pmatrix} 2 & 1 & - 1 \\ 3 & 0 & 1 \\ 0 & 1 & -2 \end{pmatrix}.
\end{align*} \end{split}
\end{equation*}
\sphinxAtStartPar
The transformation matrix for the composite transformation \(S \circ T\) is \(C = BA\):
\begin{equation*}
\begin{split} \begin{align*}
    C &= BA = \begin{pmatrix} 2 & 1 & - 1 \\ 3 & 0 & 1 \\ 0 & 1 & -2 \end{pmatrix}
    \begin{pmatrix} 2 & 4 \\ -1 & 3 \\ 1 & 2 \end{pmatrix}
    = \begin{pmatrix} 2 & 9 \\ 7 & 14 \\ -3 & -1 \end{pmatrix}.
\end{align*} \end{split}
\end{equation*}
\sphinxAtStartPar
So \(S \circ T(x,y , z) \mapsto (2x + 9y, 7x + 14y, -3x - y)\), as we saw in the example above.
\end{sphinxadmonition}

\sphinxstepscope


\section{Rotation, reflection and scaling}
\label{\detokenize{_pages/6.3_Rotation_reflection_and_translation:rotation-reflection-and-scaling}}\label{\detokenize{_pages/6.3_Rotation_reflection_and_translation:rotation-reflection-and-scaling-section}}\label{\detokenize{_pages/6.3_Rotation_reflection_and_translation::doc}}
\sphinxAtStartPar
Linear transformations are very useful for geometrical transformations. Three of the most commonly used transformations are rotation of vectors about the origin, reflection of vectors in a line, and translation of vectors from one position to another.


\bigskip\hrule\bigskip


\index{Linear transformations@\spxentry{Linear transformations}!rotation@\spxentry{rotation}}\ignorespaces 

\subsection{Rotation}
\label{\detokenize{_pages/6.3_Rotation_reflection_and_translation:rotation}}\label{\detokenize{_pages/6.3_Rotation_reflection_and_translation:index-0}}\label{\detokenize{_pages/6.3_Rotation_reflection_and_translation:rotation-section}}\label{_pages/6.3_Rotation_reflection_and_translation:rotation-definition}
\begin{sphinxadmonition}{note}{Definition 6.3.1 (Rotation transformation)}



\sphinxAtStartPar
The linear transformation \(Rot(\theta): \mathbb{R}^2 \to \mathbb{R}^2\) applied to the vector \(\mathbf{u}\) rotates \(\mathbf{u}\) by an angle \(\theta\) \sphinxstyleemphasis{anti\sphinxhyphen{}clockwise} about the origin.
\end{sphinxadmonition}

\begin{figure}[htbp]
\centering
\capstart

\noindent\sphinxincludegraphics[width=300\sphinxpxdimen]{{6_rotation}.svg}
\caption{Rotation of the vector \(\mathbf{u}\) anti\sphinxhyphen{}clockwise about the origin.}\label{\detokenize{_pages/6.3_Rotation_reflection_and_translation:rotation-figure}}\end{figure}

\sphinxAtStartPar
Consider the diagram in \hyperref[\detokenize{_pages/6.3_Rotation_reflection_and_translation:rotation-figure}]{Fig.\@ \ref{\detokenize{_pages/6.3_Rotation_reflection_and_translation:rotation-figure}}}. The vector \(\mathbf{u}\) is rotated through the angle \(\theta\) anti\sphinxhyphen{}clockwise about the origin, to give the vector \(\mathbf{v}\).

\sphinxAtStartPar
To determine the linear mapping \(Rot(\theta): \mathbf{u} \mapsto \mathbf{v}\) we must first consider the rotation from the vector \(\mathbf{e}_1\|\mathbf{u}\|\), which points along the \(x\)\sphinxhyphen{}axis and has the same magnitude as \(\mathbf{u}\). If we rotate this by an angle \(\phi\) anti\sphinxhyphen{}clockwise about the origin, we get to \(\mathbf{u}=(u_x, u_y)\).

\sphinxAtStartPar
We can find an expression for the coordinates of \(\mathbf{u}\) and \(\mathbf{v}\) in terms of \(\|\mathbf{u}\|\) and \(\phi\). Consider the right\sphinxhyphen{}angled triangle (shown in \(\textcolor{blue}{\textrm{blue}}\)) with the angle \(\phi\) and hypotenuse of length \(\|\mathbf{u}\|\). Then
\begin{equation}\label{equation:_pages/6.3_Rotation_reflection_and_translation:rotation-equation-1}
\begin{split} \begin{align*}
    \cos(\phi) &= \frac{u_x}{\|\mathbf{u}\|} & \therefore u_x &= \|\mathbf{u}\| \cos(\phi), \\
    \sin(\phi) &= \frac{u_y}{\|\mathbf{u}\|} & \therefore u_y &= \|\mathbf{u}\| \sin(\phi).
\end{align*} \end{split}
\end{equation}
\sphinxAtStartPar
Similarly, for \(\mathbf{v} = (v_x, v_y)\) (the \(\textcolor{red}{\textrm{red}}\) triangle) we have
\begin{equation*}
\begin{split} \begin{align*}
    v_x &= \|\mathbf{u}\| \cos(\phi + \theta), \\
    v_y &= \|\mathbf{u}\| \sin(\phi + \theta),
\end{align*} \end{split}
\end{equation*}
\sphinxAtStartPar
Now, the angle sum identities tell us that:
\begin{equation*}
\begin{split} \begin{align*}
    \cos(\phi + \theta) &= \cos(\phi) \cos(\theta) - \sin(\phi) \sin(\theta), \\
    \sin(\phi + \theta) &= \sin(\phi) \cos(\theta) + \cos(\phi) \sin(\theta),
\end{align*} \end{split}
\end{equation*}
\sphinxAtStartPar
Then:
\begin{equation}\label{equation:_pages/6.3_Rotation_reflection_and_translation:rotation-equation-2}
\begin{split} \begin{align*}
    v_x &= \|\mathbf{u}\| \cos(\phi) \cos(\theta) - \|\mathbf{u}\|\sin(\phi) \sin(\theta) \\
    v_y &= \|\mathbf{u}\| \sin(\phi) \cos(\theta) + \|\mathbf{u}\|\cos(\phi) \sin(\theta).
\end{align*} \end{split}
\end{equation}
\sphinxAtStartPar
Substituting equations \eqref{equation:_pages/6.3_Rotation_reflection_and_translation:rotation-equation-1} into equations \eqref{equation:_pages/6.3_Rotation_reflection_and_translation:rotation-equation-2}:
\begin{equation*}
\begin{split} \begin{align*}
    v_x &= u_x \cos(\theta) - u_y \sin(\theta), \\
    v_y &= u_y \cos(\theta) + u_x \sin(\theta).
\end{align*} \end{split}
\end{equation*}
\sphinxAtStartPar
Therefore the vector \(\mathbf{u}\), rotated through the angle \(\theta\) anti\sphinxhyphen{}clockwise about the origin, is given by:
\begin{equation*}
\begin{split} Rot(\theta) \cdot \mathbf{u} =
    \begin{pmatrix} \cos(\theta) & -\sin(\theta) \\ \sin(\theta) & \cos(\theta) \end{pmatrix}
    \begin{pmatrix} u_x \\ u_y \end{pmatrix} =
    \begin{pmatrix}
        u_x \cos(\theta) - u_y \sin(\theta) \\
        u_x \sin(\theta) + u_y \cos(\theta)
    \end{pmatrix} \end{split}
\end{equation*}\label{_pages/6.3_Rotation_reflection_and_translation:rotation-in-R2-theorem}
\begin{sphinxadmonition}{note}{Theorem 6.3.1 (Rotation in \protect\(\mathbb{R}^2\protect\))}



\sphinxAtStartPar
The rotation of a vector \(\mathbf{u} = (u_x, u_y) \in \mathbb{R}^2\) \sphinxstylestrong{anti\sphinxhyphen{}clockwise} by an angle \(\theta\) is the linear transformation \(Rot(\theta): \mathbb{R}^2 \to \mathbb{R}^2\) defined by:
\begin{equation*}
\begin{split} Rot(\theta) :
    \begin{pmatrix} u_x \\ u_y \end{pmatrix} \mapsto
    \begin{pmatrix} u_x \cos(\theta) - u_y \sin(\theta) \\ u_x \sin(\theta) + u_y \cos(\theta) \end{pmatrix}. \end{split}
\end{equation*}
\sphinxAtStartPar
The transformation matrix for rotation in \(\mathbb{R}^2\) is:
\begin{equation}\label{equation:_pages/6.3_Rotation_reflection_and_translation:rotation-matrix-equation}
\begin{split} Rot(\theta) = \begin{pmatrix}
        \cos(\theta) & -\sin(\theta) \\
        \sin(\theta) & \cos(\theta)
    \end{pmatrix}. \end{split}
\end{equation}\end{sphinxadmonition}

\sphinxAtStartPar
The inverse transformation of rotating a vector anti\sphinxhyphen{}clockwise by an angle \(\theta\) is rotating \sphinxstylestrong{clockwise} by the same angle. This can be achieved by negating \(\theta\). Since \(\cos(-\theta) = \cos(\theta)\) and \(\sin(-\theta) = -\sin(\theta)\), we have:
\begin{equation*}
\begin{split} Rot(\theta)^{-1} = \begin{pmatrix}
        \cos(-\theta) & -\sin(-\theta) \\
        \sin(-\theta) & \cos(-\theta)
   \end{pmatrix}
   = \begin{pmatrix}
       \cos(\theta) & \sin(\theta) \\
       -\sin(\theta) & \cos(\theta)
   \end{pmatrix}. \end{split}
\end{equation*}
\sphinxAtStartPar
If we instead calculate the inverse of the rotation matrix, we get the same result:
\begin{equation*}
\begin{split} \begin{align*}
    Rot(\theta)^{-1} &= \frac{1}{\cos^2(\theta) + \sin^2(\theta)}
    \begin{pmatrix}
        \cos(\theta) & -\sin(\theta) \\
        \sin(\theta) & \cos(\theta)
    \end{pmatrix}^\mathsf{T}
    = \begin{pmatrix}
        \cos(\theta) & \sin(\theta) \\
        -\sin(\theta) & \cos(\theta)
    \end{pmatrix}.
\end{align*} \end{split}
\end{equation*}\label{_pages/6.3_Rotation_reflection_and_translation:rotation-example}
\begin{sphinxadmonition}{note}{Example 6.3.1}



\sphinxAtStartPar
Rotate the vector \(\mathbf{u} = (2, 1)\) through an angle \(\theta = \frac{\pi}{2}\) anti\sphinxhyphen{}clockwise about the origin.
\subsubsection*{Solution}

\sphinxAtStartPar
The transformation matrix for this rotation is
\begin{equation*}
\begin{split} Rot \left( \frac{\pi}{2} \right) =
    \begin{pmatrix}
        \cos({\pi}/{2}) & -\sin({\pi}/{2}) \\
        \sin({\pi}/{2}) & \cos({\pi}/{2})
    \end{pmatrix}
    = \begin{pmatrix}
        0 & -1 \\
        1 & 0
    \end{pmatrix}. \end{split}
\end{equation*}
\sphinxAtStartPar
Applying the transformation to \(\mathbf{u} = (2, 1)\):
\begin{equation*}
\begin{split} Rot\left( \frac{\pi}{2} \right) \cdot \mathbf{u} =
    \begin{pmatrix}
        0 & -1 \\
        1 & 0
    \end{pmatrix}
    \begin{pmatrix} 2 \\ 1 \end{pmatrix}
    = \begin{pmatrix} -1 \\ 2 \end{pmatrix}. \end{split}
\end{equation*}
\sphinxAtStartPar
This rotation is illustrated in the diagram below:

\begin{figure}[htbp]
\centering
\capstart

\noindent\sphinxincludegraphics[width=400\sphinxpxdimen]{{6_rotation_example}.svg}
\caption{Rotation of the vector \(\mathbf{u} = (2,1)\) anti\sphinxhyphen{}clockwise by angle \(\theta = \frac{\pi}{2}\).}\label{\detokenize{_pages/6.3_Rotation_reflection_and_translation:id1}}\end{figure}
\end{sphinxadmonition}


\bigskip\hrule\bigskip


\index{Linear transformations@\spxentry{Linear transformations}!reflection@\spxentry{reflection}}\ignorespaces 

\subsection{Reflection}
\label{\detokenize{_pages/6.3_Rotation_reflection_and_translation:reflection}}\label{\detokenize{_pages/6.3_Rotation_reflection_and_translation:index-1}}\label{\detokenize{_pages/6.3_Rotation_reflection_and_translation:reflection-section}}\label{_pages/6.3_Rotation_reflection_and_translation:reflection-definition}
\begin{sphinxadmonition}{note}{Definition 6.3.2 (Reflection about a line)}



\sphinxAtStartPar
The linear transformation \(Re\!f(\theta): \mathbb{R}^2 \to \mathbb{R^2}\) is the reflection of a vector \(\mathbf{u} \in \mathbb{R}^2\) in the line which passes through the origin and makes an angle of \(\theta\) with the \(x\)\sphinxhyphen{}axis.

\sphinxAtStartPar
After reflection, the distance from the head of the image vector \(\mathbf{v}\) to this line will be the same as the distance from the head of \(\mathbf{u}\) to the line.
\end{sphinxadmonition}

\begin{figure}[htbp]
\centering
\capstart

\noindent\sphinxincludegraphics[width=300\sphinxpxdimen]{{6_reflection}.svg}
\caption{The reflection of the vector \(\mathbf{u}\) about a line that passes through the origin.}\label{\detokenize{_pages/6.3_Rotation_reflection_and_translation:reflection-figure}}\end{figure}

\sphinxAtStartPar
The simplest reflection we can perform is to reflect a vector \(\mathbf{u} =(u_x, u_y)\) in the \(x\)\sphinxhyphen{}axis (\hyperref[\detokenize{_pages/6.3_Rotation_reflection_and_translation:reflection-about-x-figure}]{Fig.\@ \ref{\detokenize{_pages/6.3_Rotation_reflection_and_translation:reflection-about-x-figure}}}).

\begin{figure}[htbp]
\centering
\capstart

\noindent\sphinxincludegraphics[width=300\sphinxpxdimen]{{6_reflection_about_x}.svg}
\caption{Reflection of the vector \(\mathbf{u}\) in the \(x\) axis.}\label{\detokenize{_pages/6.3_Rotation_reflection_and_translation:reflection-about-x-figure}}\end{figure}

\sphinxAtStartPar
To reflect in the \(x\)\sphinxhyphen{}axis, we simply change the sign of the \(y\) co\sphinxhyphen{}ordinate:
\begin{equation*}
\begin{split} Re\!f(0) \cdot \mathbf{u} =
\begin{pmatrix} u_x \\ -u_y \end{pmatrix}, \end{split}
\end{equation*}
\sphinxAtStartPar
and the transformation matrix is
\begin{equation}\label{equation:_pages/6.3_Rotation_reflection_and_translation:reflection-about-x-equation}
\begin{split} Re\!f(0) = \begin{pmatrix} 1 & 0 \\ 0 & -1 \end{pmatrix}. \end{split}
\end{equation}
\sphinxAtStartPar
Note that here the line of reflection makes an angle of \(\theta=0\) with the \(x\)\sphinxhyphen{}axis, so the transformation is denoted \(Re\!f(0)\).

\sphinxAtStartPar
To determine the transformation matrix for a reflection in a line that makes an arbitrary angle \(\theta\) with the \(x\)\sphinxhyphen{}axis, we first perform a rotation by \(-\theta\) so that the line of reflection is on the \(x\)\sphinxhyphen{}axis (\hyperref[\detokenize{_pages/6.3_Rotation_reflection_and_translation:reflection-about-line-1-figure}]{Fig.\@ \ref{\detokenize{_pages/6.3_Rotation_reflection_and_translation:reflection-about-line-1-figure}}}).

\sphinxAtStartPar
This means we can then use equation \eqref{equation:_pages/6.3_Rotation_reflection_and_translation:reflection-about-x-equation} to reflect the rotated \(\mathbf{u}\) vector to give the vector \(\mathbf{v}\) (\hyperref[\detokenize{_pages/6.3_Rotation_reflection_and_translation:reflection-about-line-2-figure}]{Fig.\@ \ref{\detokenize{_pages/6.3_Rotation_reflection_and_translation:reflection-about-line-2-figure}}}), before rotating by \(\theta\) so that the line of reflection is back to its original position (\hyperref[\detokenize{_pages/6.3_Rotation_reflection_and_translation:reflection-about-line-3-figure}]{Fig.\@ \ref{\detokenize{_pages/6.3_Rotation_reflection_and_translation:reflection-about-line-3-figure}}}).

\begin{figure}[htbp]
\centering
\capstart

\noindent\sphinxincludegraphics[width=300\sphinxpxdimen]{{6_reflection_about_line_1}.svg}
\caption{Rotate by \(-\theta\).}\label{\detokenize{_pages/6.3_Rotation_reflection_and_translation:reflection-about-line-1-figure}}\end{figure}

\begin{figure}[htbp]
\centering
\capstart

\noindent\sphinxincludegraphics[width=300\sphinxpxdimen]{{6_reflection_about_line_2}.svg}
\caption{Reflect about the \(x\)\sphinxhyphen{}axis.}\label{\detokenize{_pages/6.3_Rotation_reflection_and_translation:reflection-about-line-2-figure}}\end{figure}

\begin{figure}[htbp]
\centering
\capstart

\noindent\sphinxincludegraphics[width=300\sphinxpxdimen]{{6_reflection_about_line_3}.svg}
\caption{Rotate by \(\theta\).}\label{\detokenize{_pages/6.3_Rotation_reflection_and_translation:reflection-about-line-3-figure}}\end{figure}

\sphinxAtStartPar
Since these are all linear transformations, we can compose them as we have seen previously. First we rotate through \(-\theta\), then reflect, then rotate through \(\theta\) (but remembering these will be applied in reverse order, from right to left as we write them).

\sphinxAtStartPar
The transformation matrix for reflection in a line with angle \(\theta\) from the \(x\)\sphinxhyphen{}axis is hence given by:
\begin{equation*}
\begin{split} Re\!f(\theta) = Rot(\theta) \cdot Re\!f(0) \cdot Rot(-\theta), \end{split}
\end{equation*}
\sphinxAtStartPar
therefore
\begin{equation*}
\begin{split} \begin{align*}
    Re\!f(\theta) &=
    \begin{pmatrix} \cos(\theta) & -\sin(\theta) \\ \sin(\theta) & \cos(\theta) \end{pmatrix}
    \begin{pmatrix} 1 & 0 \\ 0 & -1 \end{pmatrix}
    \begin{pmatrix} \cos(\theta) & \sin(\theta) \\ -\sin(\theta) & \cos(\theta) \end{pmatrix} \\
    &= \begin{pmatrix}
        \cos^2(\theta) - \sin^2(\theta) & 2\cos(\theta)\sin(\theta) \\
        2\cos(\theta)\sin(\theta) & \sin^2(\theta) - \cos^2(\theta)
    \end{pmatrix}.
\end{align*} \end{split}
\end{equation*}
\sphinxAtStartPar
Using \sphinxhref{https://en.wikipedia.org/wiki/List\_of\_trigonometric\_identities\#Double-angle\_formulae}{double angle formulae}, we can simplify this \sphinxhyphen{} the formulae are:
\begin{equation*}
\begin{split} \begin{align*}
    \cos(2\theta) &= \cos^2(\theta) - \sin^2(\theta), \\
    \sin(2\theta) &= 2\cos(\theta) \sin(\theta),
\end{align*} \end{split}
\end{equation*}
\sphinxAtStartPar
This means we can simplify to:
\begin{equation*}
\begin{split} \begin{align*}
    Re\!f(\theta) &=
    \begin{pmatrix}
        \cos(2\theta) & \sin(2\theta) \\
        \sin(2\theta) & -\cos(2\theta)
    \end{pmatrix}.
\end{align*} \end{split}
\end{equation*}\label{_pages/6.3_Rotation_reflection_and_translation:reflection-theorem}
\begin{sphinxadmonition}{note}{Theorem 6.3.2 (Reflection in \protect\(\mathbb{R}^2\protect\))}



\sphinxAtStartPar
The reflection of a vector \(\mathbf{u} = (u_x, u_y) \in \mathbb{R}^2\) in a line which passes through the origin and makes an angle \(\theta\) with the \(x\)\sphinxhyphen{}axis (i.e., \(y = \tan(\theta)x\)) is the linear transformation \(Re\!f(\theta) : \mathbb{R}^2 \to \mathbb{R}^2\), defined by:
\begin{equation*}
\begin{split} \begin{align*}
    Re\!f(\theta) :
    \begin{pmatrix} u_x \\ u_y \end{pmatrix} \mapsto
    \begin{pmatrix} u_x \cos (2\theta) + u_y \sin(2 \theta) \\ u_x \sin(2\theta) - u_y \cos (2\theta) \end{pmatrix}.
\end{align*} \end{split}
\end{equation*}
\sphinxAtStartPar
The transformation matrix for reflection in \(\mathbb{R}^2\) is
\begin{equation}\label{equation:_pages/6.3_Rotation_reflection_and_translation:reflection-matrix-equation}
\begin{split} Re\!f(\theta) =
\begin{pmatrix}
    \cos(2\theta) & \sin(2 \theta) \\
    \sin(2\theta) & -\cos(2\theta)
\end{pmatrix}. \end{split}
\end{equation}\end{sphinxadmonition}

\sphinxAtStartPar
The inverse of reflecting in a given line is to simply perform the reflection again. That is,
\begin{equation*}
\begin{split}Re\!f^{-1}(\theta) = Re\!f(\theta).\end{split}
\end{equation*}\label{_pages/6.3_Rotation_reflection_and_translation:reflection-example}
\begin{sphinxadmonition}{note}{Example 6.3.2}



\sphinxAtStartPar
Reflect the vector \(\mathbf{u} = (3, -1)\) in the line that passes through the origin and has gradient of 1.
\subsubsection*{Solution}

\sphinxAtStartPar
Since the line of reflection has gradient 1, we know \(\theta = \tan^{-1}(1) = \pi/4\) and the reflection matrix is:
\begin{equation*}
\begin{split} \begin{align*}
    Re\!f\left( \frac{\pi}{4} \right) = \begin{pmatrix}
        \cos(\pi/2) & \sin(\pi/2) \\
        \sin(\pi/2) & -\cos(\pi/2)
    \end{pmatrix} =
    \begin{pmatrix}
        0 & 1 \\
        1 & 0
    \end{pmatrix}.
\end{align*} \end{split}
\end{equation*}
\sphinxAtStartPar
Applying the transformation to \(\mathbf{u} = \begin{pmatrix} 3 \\ -1 \end{pmatrix}\):
\begin{equation*}
\begin{split} \begin{align*}
    Re\!f\left(\frac{\pi}{4}\right) = \begin{pmatrix} 0 & 1 \\ 1 & 0 \end{pmatrix}
    \begin{pmatrix} 3 \\ -1 \end{pmatrix} =
    \begin{pmatrix} -1 \\ 3 \end{pmatrix}.
\end{align*} \end{split}
\end{equation*}
\begin{figure}[htbp]
\centering
\capstart

\noindent\sphinxincludegraphics[width=400\sphinxpxdimen]{{6_reflection_example}.svg}
\caption{Reflection of the vector \(\mathbf{u} = (3, -1)\) about the line that passes through the origin and makes an angle of \(\theta = \frac{\pi}{4}\) with the \(x\)\sphinxhyphen{}axis (having a gradient of \(1\)).}\label{\detokenize{_pages/6.3_Rotation_reflection_and_translation:id2}}\end{figure}
\end{sphinxadmonition}


\bigskip\hrule\bigskip


\index{Linear transformations@\spxentry{Linear transformations}!scaling@\spxentry{scaling}}\ignorespaces 

\subsection{Scaling}
\label{\detokenize{_pages/6.3_Rotation_reflection_and_translation:scaling}}\label{\detokenize{_pages/6.3_Rotation_reflection_and_translation:index-2}}\label{\detokenize{_pages/6.3_Rotation_reflection_and_translation:scaling-section}}
\sphinxAtStartPar
We have seen that vectors can be scaled by multiplying them by scalars \sphinxhyphen{} this will result in a vector pointing in the same direction, with a length given by multiplying its original length by the scalar.

\sphinxAtStartPar
It’s also possible to scale an object by different amounts in different directions. This can be achieved using a \sphinxstylestrong{scaling vector}.
\label{_pages/6.3_Rotation_reflection_and_translation:scaling-definition}
\begin{sphinxadmonition}{note}{Definition 6.3.3 (Scaling transformation)}



\sphinxAtStartPar
The linear transformation \(S(\mathbf{s}) : \mathbb{R}^n \to \mathbb{R}^n\) (where \(\mathbf{s}\) is a scaling vector in \(\mathbb{R}^n\)), applied to the position vector \(\mathbf{u} \in \mathbb{R}^n\), scales \(\mathbf{u}\) by the corresponding amounts in each coordinate.
\end{sphinxadmonition}

\begin{figure}[htbp]
\centering
\capstart

\noindent\sphinxincludegraphics[width=400\sphinxpxdimen]{{6_scaling}.svg}
\caption{The scaling of the vector \(\mathbf{u}\) by the scaling vector \(\mathbf{s}\), which changes both the direction and length of the vector.}\label{\detokenize{_pages/6.3_Rotation_reflection_and_translation:scaling-figure}}\end{figure}

\sphinxAtStartPar
The scaling of a vector \(\mathbf{u} = (u_1, u_2, \ldots, u_n) \in \mathbb{R}^n\) by a scaling vector \(\mathbf{s} = (s_1, s_2, \ldots, s_n)\) is achieved by multiplying the corresponding elements in \(\mathbf{u}\) and \(\mathbf{s}\).
\label{_pages/6.3_Rotation_reflection_and_translation:scaling-theorem}
\begin{sphinxadmonition}{note}{Theorem 6.3.3 (Scaling transformation)}



\sphinxAtStartPar
The scaling of a vector \(\mathbf{u} \in \mathbb{R}^n\) by the scaling vector \(\mathbf{s}\) is the linear transformation \(S(\mathbf{s}): \mathbf{u} \mapsto \mathbf{v}\) calculated using
\begin{equation*}
\begin{split} \begin{align*}
    S(\mathbf{s}): \begin{pmatrix} u_1 \\ u_2 \\ \vdots \\ u_n \end{pmatrix} \mapsto
    \begin{pmatrix} s_1 u_1 \\ s_2 u_2 \\ \vdots \\ s_n u_n \end{pmatrix}.
\end{align*} \end{split}
\end{equation*}
\sphinxAtStartPar
The transformation matrix for scaling is:
\begin{equation}\label{equation:_pages/6.3_Rotation_reflection_and_translation:scaling-matrix-equation}
\begin{split} \begin{align*}
    S(\mathbf{s}) =
    \begin{pmatrix}
        s_1 & 0 & \cdots & 0 \\
        0 & s_2 & \ddots & \vdots \\
        \vdots & \ddots & \ddots & 0 \\
        0 & \cdots & 0 & s_n
        \end{pmatrix}.
\end{align*} \end{split}
\end{equation}\end{sphinxadmonition}

\sphinxAtStartPar
Since non\sphinxhyphen{}zero the entries in this matrix are all in separate rows and columns, the scalar multiplication will be applied to each element of the vector being scaled separately.

\sphinxAtStartPar
If the entries on the diagonal of this matrix are all the same, this is just the same as standard scalar multiplication, since each entry in the vector is multiplied by the same scalar.

\sphinxAtStartPar
To invert the operation of scaling by a vector, the inverse scaling matrix is:
\begin{equation*}
\begin{split} \begin{align*}
    S^{-1}(\mathbf{s}) =
    \begin{pmatrix}
        \frac{1}{s_1} & 0 & \cdots & 0 \\
        0 & \frac{1}{s_2} & \ddots & \vdots \\
        \vdots & \ddots & \ddots & 0 \\
        0 & \cdots & 0 & \frac{1}{s_n}
        \end{pmatrix}.
\end{align*} \end{split}
\end{equation*}\label{_pages/6.3_Rotation_reflection_and_translation:scaling-example}
\begin{sphinxadmonition}{note}{Example 6.3.3}



\sphinxAtStartPar
Scale the point with position vector \(\mathbf{u} = (2, 1)\) by scaling scaling vector \(\mathbf{s} = (2, 3)\).
\subsubsection*{Solution}

\sphinxAtStartPar
The transformation matrix is
\begin{equation*}
\begin{split} S\begin{pmatrix} 2 \\ 3 \end{pmatrix} = \begin{pmatrix} 2 & 0 \\ 0 & 3 \end{pmatrix}. \end{split}
\end{equation*}
\sphinxAtStartPar
Applying the transformation to \(\mathbf{u} = \begin{pmatrix} 2 \\ 1 \end{pmatrix}\)
\begin{equation*}
\begin{split} S\begin{pmatrix} 2 \\ 3 \end{pmatrix} \cdot \mathbf{u} = \begin{pmatrix}
    2 & 0 \\ 0 & 3
\end{pmatrix}
\begin{pmatrix} 2 \\ 1 \end{pmatrix}
= \begin{pmatrix} 4 \\ 3 \end{pmatrix}. \end{split}
\end{equation*}
\begin{figure}[htbp]
\centering
\capstart

\noindent\sphinxincludegraphics[width=400\sphinxpxdimen]{{6_scaling_example}.svg}
\caption{Scaling of the vector \(\mathbf{u} = (2, 1)\) by the scaling vector \(\mathbf{s} = (2, 3)\).}\label{\detokenize{_pages/6.3_Rotation_reflection_and_translation:id3}}\end{figure}
\end{sphinxadmonition}

\sphinxstepscope

\index{Linear transformations@\spxentry{Linear transformations}!translation@\spxentry{translation}}\ignorespaces 

\section{Translation}
\label{\detokenize{_pages/6.4_Translation:translation}}\label{\detokenize{_pages/6.4_Translation:index-0}}\label{\detokenize{_pages/6.4_Translation:translation-section}}\label{\detokenize{_pages/6.4_Translation::doc}}
\sphinxAtStartPar
In the {\hyperref[\detokenize{_pages/6.3_Rotation_reflection_and_translation:rotation-reflection-and-scaling-section}]{\sphinxcrossref{\DUrole{std,std-ref}{previous section}}}} we looked at how to rotate, reflect and scale a vector. To complete the set of vector transformations, we need to be able to \sphinxstylestrong{translate} vectors.
\label{_pages/6.4_Translation:translation-definition}
\begin{sphinxadmonition}{note}{Definition 6.4.1 (Translation)}



\sphinxAtStartPar
The translation of a point \(\mathbf{u} \in \mathbb{R}^n\) by a translation vector \(\mathbf{t} \in \mathbb{R}^n\) is the linear transformation \(T(\mathbf{t}): \mathbf{u} \mapsto \mathbf{u} + \mathbf{t}\).
\end{sphinxadmonition}

\begin{figure}[htbp]
\centering
\capstart

\noindent\sphinxincludegraphics[width=400\sphinxpxdimen]{{6_translation}.svg}
\caption{The translation of the point \(\mathbf{u}\) by the vector \(\mathbf{t}\) in \(\mathbb{R}^2\).}\label{\detokenize{_pages/6.4_Translation:translation-figure}}\end{figure}

\sphinxAtStartPar
It might not immediately be obvious how to find a matrix representation of a translation corresponding to the vector \sphinxhyphen{} for translations in \(\mathbb{R}^2\), with translation vector \(\mathbb{t} = (t_x, t_y)\), we need to find a matrix \(T\) that satisfies:
\begin{equation*}
\begin{split} T(\mathbf{t}) \cdot \begin{pmatrix} u_x \\ u_y \end{pmatrix}  = \begin{pmatrix} u_x + t_x \\ u_y + t_y \end{pmatrix}. \end{split}
\end{equation*}
\index{Homogeneous co\sphinxhyphen{}ordinates@\spxentry{Homogeneous co\sphinxhyphen{}ordinates}}\ignorespaces 
\sphinxAtStartPar
In order to construct such a matrix, we can use a trick which makes use of \sphinxstylestrong{homogeneous co\sphinxhyphen{}ordinates}.

\sphinxAtStartPar
A point in \(\mathbb{R}^2\) is represented in Cartesian co\sphinxhyphen{}ordinates by \((x, y)\). Homogeneous co\sphinxhyphen{}ordinates extend this representation, by introducing an additional value \(w\), and giving the point in homogeneous co\sphinxhyphen{}ordinates as \((wx, wy, w)\). To convert homogeneous co\sphinxhyphen{}ordinates to Cartesian co\sphinxhyphen{}ordinates, we must divide through by \(w\), i.e.,
\begin{equation*}
\begin{split}\underbrace{(wx, wy, w)}_{\textsf{homogeneous}} \mapsto \underbrace{\left( \frac{x}{w}, \frac{y}{w} \right)}_{\textsf{Cartesian}}.\end{split}
\end{equation*}\label{_pages/6.4_Translation:homogeneous-co-ordinates-definition}
\begin{sphinxadmonition}{note}{Definition 6.4.2 (Homogeneous co\sphinxhyphen{}ordinates)}



\sphinxAtStartPar
The homogeneous co\sphinxhyphen{}ordinates of a point \(\mathbf{u}\) in \(\mathbb{R}^n\) expressed using the Cartesian co\sphinxhyphen{}ordinates are the \((n+1)\)\sphinxhyphen{}tuple \((w \cdot u_1, w u_2, \ldots, w \cdot u_n, w)\) where \(w \in \mathbb{R}\backslash \{0\}\).
\end{sphinxadmonition}

\sphinxAtStartPar
We often set \(w=1\) when using homogeneous co\sphinxhyphen{}ordinates, since this makes mapping from Cartesian co\sphinxhyphen{}ordinates easy:
\begin{equation*}
\begin{split}\underbrace{(x, y, 1)}_{\textsf{homogeneous}} \mapsto \underbrace{(x, y)_{\textsf{Cartesian}}\end{split}
\end{equation*}
\sphinxAtStartPar
If we express a point in \(\mathbb{R}^2\) with Cartesian co\sphinxhyphen{}ordinates \((u_x, u_y)\) using the homogeneous co\sphinxhyphen{}ordinates \((u_x, u_y, 1)\), then we can find a \(3 \times 3\) translation matrix \(T(\mathbf{t})\) that satisfies:
\begin{equation*}
\begin{split} T(\mathbf{t}) \cdot
\begin{pmatrix} u_x \\ u_y \\ 1 \end{pmatrix} =
\begin{pmatrix} u_x + t_x \\ u_y + t_y \\ 1 \end{pmatrix}.\end{split}
\end{equation*}
\sphinxAtStartPar
Indeed,
\begin{equation*}
\begin{split}
\begin{pmatrix}
    1 & 0 & t_x \\
    0 & 1 & t_y \\
    0 & 0 & 1
\end{pmatrix}
\begin{pmatrix} u_x \\ u_y \\ 1 \end{pmatrix}
= \begin{pmatrix} u_x + t_x \\ u_y + t_y \\ 1 \end{pmatrix} \end{split}
\end{equation*}\label{_pages/6.4_Translation:translation-theorem}
\begin{sphinxadmonition}{note}{Theorem 6.4.1 (Translation in \protect\(\mathbb{R}^2\protect\))}



\sphinxAtStartPar
The translation of the vector \(\mathbf{u} \in \mathbb{R}^2\) by the translation vector \(\mathbf{t} \in \mathbb{R}^2\) is given by:
\begin{equation*}
\begin{split} T(\mathbf{t}) :
\begin{pmatrix} u_x \\ u_y  \end{pmatrix} \mapsto
\begin{pmatrix} u_x + t_x \\ u_y + t_y  \end{pmatrix}. \end{split}
\end{equation*}
\sphinxAtStartPar
If \(\mathbf{u}\) is expressed using homogeneous co\sphinxhyphen{}ordinates with \(w=1\), the transformation matrix for this translation is:
\begin{equation}\label{equation:_pages/6.4_Translation:translation-matrix-equation}
\begin{split} T(\mathbf{t}) =
\begin{pmatrix}
    1 & 0 & t_x \\
    0 & 1 & t_y \\
    0 & 0 & 1
\end{pmatrix}. \end{split}
\end{equation}\end{sphinxadmonition}

\sphinxAtStartPar
The inverse translation is to translate by the vector \(-\mathbf{t}\), and so the transformation matrix is given by:
\begin{equation*}
\begin{split} T^{-1}(-\mathbf{t}) =
\begin{pmatrix}
    1 & 0 & -t_x \\
    0 & 1 & -t_y \\
    0 & 0 & 1
\end{pmatrix}. \end{split}
\end{equation*}\label{_pages/6.4_Translation:translation-example}
\begin{sphinxadmonition}{note}{Example 6.4.1}



\sphinxAtStartPar
Translate the point with position vector \(\mathbf{u} = (2, 3)\) by the vector \(\mathbf{t} = (3, -1)\).
\subsubsection*{Solution}

\sphinxAtStartPar
Expressing the position vector \(\mathbf{u} = \begin{pmatrix} 2 \\ 3 \end{pmatrix}\) using homogeneous co\sphinxhyphen{}ordinates we have \(\mathbf{u} = \begin{pmatrix} 2 \\ 3 \\ 1 \end{pmatrix}\).

\sphinxAtStartPar
The translation matrix is
\begin{equation*}
\begin{split} \begin{align*}
    T\begin{pmatrix} 3 \\ -1 \end{pmatrix} =
    \begin{pmatrix}
        1 & 0 & 3 \\
        0 & 1 & -1 \\
        0 & 0 & 1
    \end{pmatrix}.
\end{align*} \end{split}
\end{equation*}
\sphinxAtStartPar
Applying the transformation to \(\mathbf{u}\)
\begin{equation*}
\begin{split} \begin{align*}
    T\begin{pmatrix} 3 \\ -1 \end{pmatrix} \cdot \begin{pmatrix} 2 \\ 3 \\ 1 \end{pmatrix} =
    \begin{pmatrix}
        1 & 0 & 3 \\
        0 & 1 & -1 \\
        0 & 0 & 1
    \end{pmatrix}
    \begin{pmatrix} 2 \\ 3 \\ 1 \end{pmatrix} =
    \begin{pmatrix} 5 \\ 2 \\ 1 \end{pmatrix}.
\end{align*} \end{split}
\end{equation*}
\begin{figure}[htbp]
\centering
\capstart

\noindent\sphinxincludegraphics[width=400\sphinxpxdimen]{{6_translation_example}.svg}
\caption{Translation of \(\mathbf{u} = (2, 3)\) (blue) by the translation vector \(\mathbf{t} = (3, -1)\) (green).}\label{\detokenize{_pages/6.4_Translation:id1}}\end{figure}
\end{sphinxadmonition}


\bigskip\hrule\bigskip



\subsection{Combining rotation, reflection, scaling and translation transformation}
\label{\detokenize{_pages/6.4_Translation:combining-rotation-reflection-scaling-and-translation-transformation}}
\sphinxAtStartPar
If we are working with vectors in \(\mathbb{R}^2\), it would be nice to be able to combine translation with rotation, reflection and scaling transformations \sphinxhyphen{} but to multiply matrices together, they must be the right size.

\sphinxAtStartPar
Since our matrix for translation uses homeogenous coordinates, we would not be able to combine it by multiplying it by one of our rotation, reflection or scaling matrices \sphinxhyphen{} we need all of the transformation matrices to be \(3 \times 3\) matrices.

\sphinxAtStartPar
Luckily, this can be achieved by appending the third row and column of the identity matrix to \(Rot(\theta)\), \(Re\!f(\theta)\) and \(S(\mathbf{s})\) from equations \eqref{equation:_pages/6.3_Rotation_reflection_and_translation:rotation-matrix-equation}, \eqref{equation:_pages/6.3_Rotation_reflection_and_translation:reflection-matrix-equation} and \eqref{equation:_pages/6.3_Rotation_reflection_and_translation:scaling-matrix-equation} , i.e.,
\begin{equation*}
\begin{split} \begin{align*}
    Rot(\theta) &= \begin{pmatrix}
        \cos(\theta) & \sin(\theta) & 0 \\
        -\sin(\theta) & \cos(\theta) & 0 \\
        0 & 0 & 1
    \end{pmatrix}, \\
    Re\!f(\theta) &= \begin{pmatrix}
        \cos(2\theta) & \sin(2\theta) & 0 \\
        \sin(2\theta) & -\cos(2\theta) & 0 \\
        0 & 0 & 1
    \end{pmatrix}, \\
    S(\mathbf{s}) &= \begin{pmatrix}
        s_1 & 0 & 0 \\
        0 & s_2 & 0 \\
        0 & 0 & 1
    \end{pmatrix}.
\end{align*} \end{split}
\end{equation*}

\bigskip\hrule\bigskip



\subsection{Applying multiple transformations to an object}
\label{\detokenize{_pages/6.4_Translation:applying-multiple-transformations-to-an-object}}
\sphinxAtStartPar
We have seen how to apply transformations to single vectors \sphinxhyphen{} and we can extend this to objects with vectors defining multiple vertices, transforming the whole object at once.

\sphinxAtStartPar
Recall that to apply transformation to multiple vectors in \(\mathbf{u}_1, \mathbf{u}_2, \ldots, \mathbf{u}_m \in \mathbb{R}^2\) at once, we can combine them into a matrix. First we should express them using homogeneous co\sphinxhyphen{}ordinates, by appending an entry of \(1\) to each, and then we can form a \(3 \times m\) matrix \(U\) where each of these vectors are the columns of \(U\), i.e.,
\begin{equation*}
\begin{split} U =
\begin{pmatrix}
    \small{\uparrow} & \small{\uparrow} & & \small{\uparrow} \\
    \mathbf{u}_1 & \mathbf{u}_2 & \cdots & \mathbf{u}_m \\
    \small{\downarrow} & \small{\downarrow} & & \small{\downarrow} \\
    1 & 1 & & 1 
\end{pmatrix}. \end{split}
\end{equation*}
\sphinxAtStartPar
Given an \(3 \times 3\) transformation matrix \(A\), we can apply this transformation to all of the vectors \(\mathbf{u}_1, \mathbf{u}_2, \ldots, \mathbf{u}_m\) together using \(V = A \cdot U.\)
\label{_pages/6.4_Translation:rotation-scaling-and-translating-example}
\begin{sphinxadmonition}{note}{Example 6.4.2}



\sphinxAtStartPar
An isosceles triangle has the vertices with co\sphinxhyphen{}ordinates \((-1,-1)\), \((1,-1)\) and \((0,2)\). The triangle is transformed using the following transformations:
\begin{itemize}
\item {} 
\sphinxAtStartPar
Scaled by a factor of 2 in both directions

\item {} 
\sphinxAtStartPar
Rotated by \(\pi/4\) anti\sphinxhyphen{}clockwise about the origin

\item {} 
\sphinxAtStartPar
Translated by the vector \(\mathbf{t} = (6,4)\)

\end{itemize}

\sphinxAtStartPar
Calculate the co\sphinxhyphen{}ordinates of the triangle after each of these transformations has been applied, and determine a transformation matrix that performs all three transformations at the same time.
\subsubsection*{Solution (Scaling)}

\sphinxAtStartPar
Using homogeneous co\sphinxhyphen{}ordinates, the vertices have position vectors \(\begin{pmatrix} -1 \\ -1 \\ 1  \end{pmatrix}\), \(\begin{pmatrix} 1 \\ -1 \\ 1  \end{pmatrix}\) and \(\begin{pmatrix} 0 \\ 2 \\ 1  \end{pmatrix}\). We can combine into a single matrix”
\begin{equation*}
\begin{split} U =
\begin{pmatrix}
    -1 & 1 & 0 \\
    -1 & -1 & 2 \\
    1 & 1 & 1
\end{pmatrix}. \end{split}
\end{equation*}
\sphinxAtStartPar
Since we are scaling by a factor of 2 in both directions, the scaling vector is \(\mathbf{s} = \begin{pmatrix} 2 \\ 2  \end{pmatrix}\), and so the scaling matrix is:
\begin{equation*}
\begin{split} \begin{align*}
    S\begin{pmatrix} 2 \\ 2 \end{pmatrix} = \begin{pmatrix} 2 & 0 & 0 \\ 0 & 2 & 0 \\ 0 & 0 & 1 \end{pmatrix}.
\end{align*} \end{split}
\end{equation*}
\sphinxAtStartPar
Applying the scaling matrix to the homogeneous co\sphinxhyphen{}ordinates:
\begin{equation*}
\begin{split} \begin{align*}
    U_1 &= S\begin{pmatrix} 2 \\ 2 \end{pmatrix} \cdot U \\
    &= \begin{pmatrix} 2 & 0 & 0 \\ 0 & 2 & 0 \\ 0 & 0 & 1 \end{pmatrix}
    \begin{pmatrix} -1 & 1 & 0 \\ -1 & -1 & 2 \\ 1 & 1 & 1 \end{pmatrix}\\
    &= \begin{pmatrix} -2 & 2 & 0 \\ -2 & -2 & 4 \\ 1 & 1 & 1 \end{pmatrix}.
\end{align*} \end{split}
\end{equation*}
\begin{figure}[htbp]
\centering
\capstart

\noindent\sphinxincludegraphics[width=400\sphinxpxdimen]{{6_rotation_scaling_and_translation_example_1}.svg}
\caption{Triangle scaled up by \(\mathbf{s} = (2, 2)\)}\label{\detokenize{_pages/6.4_Translation:id2}}\end{figure}
\subsubsection*{Solution (Rotation)}

\sphinxAtStartPar
We then want to rotate by an angle of \(\frac{\pi}{4}\), so the rotation matrix is
\begin{equation*}
\begin{split} \begin{align*}
    Rot\left( \frac{\pi}{4} \right) &=
    \begin{pmatrix}
        \cos(\pi/4) & -\sin(\pi/4) & 0 \\
        \sin(\pi/4) & \cos(\pi/4) & 0 \\
        0 & 0 & 1
    \end{pmatrix}
    = \begin{pmatrix}
        \sqrt{2}/2 & -\sqrt{2}/2 & 0 \\
        \sqrt{2}/2 & \sqrt{2}/2 & 0 \\
        0 & 0 & 1
    \end{pmatrix}.
\end{align*} \end{split}
\end{equation*}
\sphinxAtStartPar
Applying the rotating matrix to \(U_1\)
\begin{equation*}
\begin{split} \begin{align*}
    U_2 &= Rot\left(\frac{\pi}{4}\right) \cdot U_1  \\
    &= \begin{pmatrix}
        \sqrt{2}/2 & -\sqrt{2}/2 & 0 \\
        \sqrt{2}/2 & \sqrt{2}/2 & 0 \\
        0 & 0 & 1
    \end{pmatrix}
    \begin{pmatrix} -2 & 2 & 0 \\ -2 & -2 & 4 \\ 1 & 1 & 1 \end{pmatrix} \\
    &= \begin{pmatrix}
        0 & 2 \sqrt{2} & -2\sqrt{2} \\
        -2\sqrt{2} & 0 & 2\sqrt{2} \\
        1 & 1 & 1
    \end{pmatrix} \\
    &\approx \begin{pmatrix}
        0 & 2.8482 & -2.8482 \\
        -2.8482 & 0 & 2.8482 \\
        1 & 1 & 1
    \end{pmatrix}.
\end{align*} \end{split}
\end{equation*}
\begin{figure}[htbp]
\centering
\capstart

\noindent\sphinxincludegraphics[width=400\sphinxpxdimen]{{6_rotation_scaling_and_translation_example_2}.svg}
\caption{The scaled triangle rotated by angle \(\theta=\frac{\pi}{4}\) anti\sphinxhyphen{}clockwise.}\label{\detokenize{_pages/6.4_Translation:id3}}\end{figure}
\subsubsection*{Solution (Translation)}

\sphinxAtStartPar
Finally, we need to translate by the translation vector \(\mathbf{t} = \begin{pmatrix} 6 \\ 4 \end{pmatrix}\), so the translation matrix is:
\begin{equation*}
\begin{split} T\begin{pmatrix} 6 \\ 4 \end{pmatrix} =
\begin{pmatrix}
    1 & 0 & 6 \\
    0 & 1 & 4 \\
    0 & 0 & 1
\end{pmatrix}. \end{split}
\end{equation*}
\sphinxAtStartPar
Applying the translation matrix to \(U_2\):
\begin{equation*}
\begin{split} \begin{align*}
    U_3 &= T\begin{pmatrix} 6 \\ 4 \end{pmatrix} \cdot U_2 \\
    &= \begin{pmatrix} 1 & 0 & 6 \\ 0 & 1 & 4 \\ 0 & 0 & 1 \end{pmatrix}
    \begin{pmatrix}
        0 & 2 \sqrt{2} & -2\sqrt{2} \\
        -2\sqrt{2} & 0 & 2\sqrt{2} \\
        1 & 1 & 1
    \end{pmatrix} \\
    &= \begin{pmatrix}
        6 & 6 + 2\sqrt{2} & 6 - 2\sqrt{2} \\
        4 - 2\sqrt{2} & 4 & 4 + 2\sqrt{2} \\
        1 & 1 & 1
    \end{pmatrix} \\
    &\approx \begin{pmatrix}
        6 & 8.8284 & 3.1716 \\
        1.1716 & 4 & 6.8284 \\
        1 & 1 & 1
    \end{pmatrix}.
\end{align*} \end{split}
\end{equation*}
\begin{figure}[htbp]
\centering
\capstart

\noindent\sphinxincludegraphics[width=500\sphinxpxdimen]{{6_rotation_scaling_and_translation_example_3}.svg}
\caption{Rotated triangle translated by \(\mathbf{t} = (6, 3)\)}\label{\detokenize{_pages/6.4_Translation:id4}}\end{figure}
\subsubsection*{Solution (Combined)}

\sphinxAtStartPar
The single transformation matrix that performs the three transformations at the same time (remembering we put the matrix for the transformation we want to apply first on the right) is:
\begin{equation*}
\begin{split} \begin{align*}
    A &= T\begin{pmatrix} 6 \\ 4 \end{pmatrix} \cdot Rot(\pi/4) \cdot S\begin{pmatrix} 2 \\ 2 \end{pmatrix} \\
    &= \begin{pmatrix}
        1 & 0 & 6 \\
        0 & 1 & 4 \\
        0 & 0 & 1
    \end{pmatrix}
    \begin{pmatrix}
        \sqrt{2}/2 & -\sqrt{2}/2 & 0 \\
        \sqrt{2}/2 & \sqrt{2}/2 & 0 \\
        0 & 0 & 1
    \end{pmatrix}
    \begin{pmatrix}
        2 & 0 & 0 \\
        0 & 2 & 0 \\
        0 & 0 & 1
    \end{pmatrix} \\
    &= \begin{pmatrix}
        \sqrt{2} & - \sqrt{2} & 6 \\
        \sqrt{2} & \sqrt{2} & 4 \\
        0 & 0 & 1
    \end{pmatrix}
    \approx
    \begin{pmatrix}
        1.4142 & -1.4142 & 6 \\
        1.4142 & 1.4142 & 4 \\
        0 & 0 & 1
    \end{pmatrix}.
\end{align*} \end{split}
\end{equation*}
\sphinxAtStartPar
Checking that \(A \cdot U = U_3\) as we calculated it earlier:
\begin{equation*}
\begin{split} \begin{align*}
    A \cdot V &= \begin{pmatrix}
        \sqrt{2} & -\sqrt{2} & 6 \\
        \sqrt{2} & \sqrt{2} & 4 \\
        0 & 0 & 1
    \end{pmatrix}
    \begin{pmatrix}
        -1 & 1 & 0 \\
        -1 & -1 & 2 \\
        1 & 1 & 1
    \end{pmatrix} \\
    &= \begin{pmatrix}
        6 & 6 + 2\sqrt{2} & 6 - 2\sqrt{2} \\
        4 - 2\sqrt{2} & 4 & 4 + 2\sqrt{2} \\
        1 & 1 & 1
    \end{pmatrix} = U_3 \quad \checkmark
\end{align*} \end{split}
\end{equation*}\end{sphinxadmonition}

\sphinxstepscope


\section{Transformations exercises}
\label{\detokenize{_pages/6.5_Linear_transforation_exercises:transformations-exercises}}\label{\detokenize{_pages/6.5_Linear_transforation_exercises:transformations-exercises-section}}\label{\detokenize{_pages/6.5_Linear_transforation_exercises::doc}}
\sphinxAtStartPar
Answer the following exercises based on the content from this chapter. The solutions can be found in the {\hyperref[\detokenize{_pages/A5_Vector_spaces_exercises_solutions:vector-spaces-exercises-solutions-section}]{\sphinxcrossref{\DUrole{std,std-ref}{appendices}}}}.
\phantomsection \label{exercise:transformations-ex-linear-transformations}

\begin{sphinxadmonition}{note}{Exercise 6.5.1}



\sphinxAtStartPar
Which of the following transformations are linear transformations?

\begin{sphinxuseclass}{sd-container-fluid}
\begin{sphinxuseclass}{sd-sphinx-override}
\begin{sphinxuseclass}{sd-mb-4}
\begin{sphinxuseclass}{sd-row}
\begin{sphinxuseclass}{sd-col}
\begin{sphinxuseclass}{sd-d-flex-column}
\begin{sphinxuseclass}{sd-col-6}
\begin{sphinxuseclass}{sd-col-xs-6}
\begin{sphinxuseclass}{sd-col-sm-6}
\begin{sphinxuseclass}{sd-col-md-6}
\begin{sphinxuseclass}{sd-col-lg-6}
\sphinxAtStartPar
(a)   \(T: (x, y) \mapsto (0, y)\)

\end{sphinxuseclass}
\end{sphinxuseclass}
\end{sphinxuseclass}
\end{sphinxuseclass}
\end{sphinxuseclass}
\end{sphinxuseclass}
\end{sphinxuseclass}
\begin{sphinxuseclass}{sd-col}
\begin{sphinxuseclass}{sd-d-flex-column}
\begin{sphinxuseclass}{sd-col-6}
\begin{sphinxuseclass}{sd-col-xs-6}
\begin{sphinxuseclass}{sd-col-sm-6}
\begin{sphinxuseclass}{sd-col-md-6}
\begin{sphinxuseclass}{sd-col-lg-6}
\sphinxAtStartPar
(b)   \(T: (x, y) \mapsto (x, 5)\)

\end{sphinxuseclass}
\end{sphinxuseclass}
\end{sphinxuseclass}
\end{sphinxuseclass}
\end{sphinxuseclass}
\end{sphinxuseclass}
\end{sphinxuseclass}
\begin{sphinxuseclass}{sd-col}
\begin{sphinxuseclass}{sd-d-flex-column}
\begin{sphinxuseclass}{sd-col-12}
\begin{sphinxuseclass}{sd-col-xs-12}
\begin{sphinxuseclass}{sd-col-sm-12}
\begin{sphinxuseclass}{sd-col-md-12}
\begin{sphinxuseclass}{sd-col-lg-12}
\sphinxAtStartPar
 

\end{sphinxuseclass}
\end{sphinxuseclass}
\end{sphinxuseclass}
\end{sphinxuseclass}
\end{sphinxuseclass}
\end{sphinxuseclass}
\end{sphinxuseclass}
\begin{sphinxuseclass}{sd-col}
\begin{sphinxuseclass}{sd-d-flex-column}
\begin{sphinxuseclass}{sd-col-6}
\begin{sphinxuseclass}{sd-col-xs-6}
\begin{sphinxuseclass}{sd-col-sm-6}
\begin{sphinxuseclass}{sd-col-md-6}
\begin{sphinxuseclass}{sd-col-lg-6}
\sphinxAtStartPar
(c)   \(T: (x, y, z) \mapsto (x, x - y)\)

\end{sphinxuseclass}
\end{sphinxuseclass}
\end{sphinxuseclass}
\end{sphinxuseclass}
\end{sphinxuseclass}
\end{sphinxuseclass}
\end{sphinxuseclass}
\begin{sphinxuseclass}{sd-col}
\begin{sphinxuseclass}{sd-d-flex-column}
\begin{sphinxuseclass}{sd-col-6}
\begin{sphinxuseclass}{sd-col-xs-6}
\begin{sphinxuseclass}{sd-col-sm-6}
\begin{sphinxuseclass}{sd-col-md-6}
\begin{sphinxuseclass}{sd-col-lg-6}
\sphinxAtStartPar
(d)   \(T: (x, y, z) \mapsto \begin{pmatrix} x + y \\ z \end{pmatrix}\)

\end{sphinxuseclass}
\end{sphinxuseclass}
\end{sphinxuseclass}
\end{sphinxuseclass}
\end{sphinxuseclass}
\end{sphinxuseclass}
\end{sphinxuseclass}
\begin{sphinxuseclass}{sd-col}
\begin{sphinxuseclass}{sd-d-flex-column}
\begin{sphinxuseclass}{sd-col-12}
\begin{sphinxuseclass}{sd-col-xs-12}
\begin{sphinxuseclass}{sd-col-sm-12}
\begin{sphinxuseclass}{sd-col-md-12}
\begin{sphinxuseclass}{sd-col-lg-12}
\sphinxAtStartPar
 

\end{sphinxuseclass}
\end{sphinxuseclass}
\end{sphinxuseclass}
\end{sphinxuseclass}
\end{sphinxuseclass}
\end{sphinxuseclass}
\end{sphinxuseclass}
\begin{sphinxuseclass}{sd-col}
\begin{sphinxuseclass}{sd-d-flex-column}
\begin{sphinxuseclass}{sd-col-6}
\begin{sphinxuseclass}{sd-col-xs-6}
\begin{sphinxuseclass}{sd-col-sm-6}
\begin{sphinxuseclass}{sd-col-md-6}
\begin{sphinxuseclass}{sd-col-lg-6}
\sphinxAtStartPar
(e)   \(T: (x, y) \mapsto (3x + 1, y)\)

\end{sphinxuseclass}
\end{sphinxuseclass}
\end{sphinxuseclass}
\end{sphinxuseclass}
\end{sphinxuseclass}
\end{sphinxuseclass}
\end{sphinxuseclass}
\begin{sphinxuseclass}{sd-col}
\begin{sphinxuseclass}{sd-d-flex-column}
\begin{sphinxuseclass}{sd-col-6}
\begin{sphinxuseclass}{sd-col-xs-6}
\begin{sphinxuseclass}{sd-col-sm-6}
\begin{sphinxuseclass}{sd-col-md-6}
\begin{sphinxuseclass}{sd-col-lg-6}
\sphinxAtStartPar
(f)   \(T: f(x) \mapsto \dfrac{\mathrm{d}}{\mathrm{d}x} f(x)\)

\end{sphinxuseclass}
\end{sphinxuseclass}
\end{sphinxuseclass}
\end{sphinxuseclass}
\end{sphinxuseclass}
\end{sphinxuseclass}
\end{sphinxuseclass}
\begin{sphinxuseclass}{sd-col}
\begin{sphinxuseclass}{sd-d-flex-column}
\begin{sphinxuseclass}{sd-col-12}
\begin{sphinxuseclass}{sd-col-xs-12}
\begin{sphinxuseclass}{sd-col-sm-12}
\begin{sphinxuseclass}{sd-col-md-12}
\begin{sphinxuseclass}{sd-col-lg-12}
\sphinxAtStartPar
 

\end{sphinxuseclass}
\end{sphinxuseclass}
\end{sphinxuseclass}
\end{sphinxuseclass}
\end{sphinxuseclass}
\end{sphinxuseclass}
\end{sphinxuseclass}
\begin{sphinxuseclass}{sd-col}
\begin{sphinxuseclass}{sd-d-flex-column}
\begin{sphinxuseclass}{sd-col-6}
\begin{sphinxuseclass}{sd-col-xs-6}
\begin{sphinxuseclass}{sd-col-sm-6}
\begin{sphinxuseclass}{sd-col-md-6}
\begin{sphinxuseclass}{sd-col-lg-6}
\sphinxAtStartPar
(g)   \(T: f(x) \mapsto xf(x)\)

\end{sphinxuseclass}
\end{sphinxuseclass}
\end{sphinxuseclass}
\end{sphinxuseclass}
\end{sphinxuseclass}
\end{sphinxuseclass}
\end{sphinxuseclass}
\begin{sphinxuseclass}{sd-col}
\begin{sphinxuseclass}{sd-d-flex-column}
\begin{sphinxuseclass}{sd-col-6}
\begin{sphinxuseclass}{sd-col-xs-6}
\begin{sphinxuseclass}{sd-col-sm-6}
\begin{sphinxuseclass}{sd-col-md-6}
\begin{sphinxuseclass}{sd-col-lg-6}
\sphinxAtStartPar
(h)   \(T: \mathbb{C}^2 \to \mathbb{C}\) where \(T: (x, y) \mapsto x + y\)

\end{sphinxuseclass}
\end{sphinxuseclass}
\end{sphinxuseclass}
\end{sphinxuseclass}
\end{sphinxuseclass}
\end{sphinxuseclass}
\end{sphinxuseclass}
\begin{sphinxuseclass}{sd-col}
\begin{sphinxuseclass}{sd-d-flex-column}
\begin{sphinxuseclass}{sd-col-12}
\begin{sphinxuseclass}{sd-col-xs-12}
\begin{sphinxuseclass}{sd-col-sm-12}
\begin{sphinxuseclass}{sd-col-md-12}
\begin{sphinxuseclass}{sd-col-lg-12}
\sphinxAtStartPar
 

\end{sphinxuseclass}
\end{sphinxuseclass}
\end{sphinxuseclass}
\end{sphinxuseclass}
\end{sphinxuseclass}
\end{sphinxuseclass}
\end{sphinxuseclass}
\begin{sphinxuseclass}{sd-col}
\begin{sphinxuseclass}{sd-d-flex-column}
\begin{sphinxuseclass}{sd-col-6}
\begin{sphinxuseclass}{sd-col-xs-6}
\begin{sphinxuseclass}{sd-col-sm-6}
\begin{sphinxuseclass}{sd-col-md-6}
\begin{sphinxuseclass}{sd-col-lg-6}
\sphinxAtStartPar
(i)   \(T: \mathbb{C}^2 \to \mathbb{C}\) where \(T: (x, y) \mapsto x y\)

\end{sphinxuseclass}
\end{sphinxuseclass}
\end{sphinxuseclass}
\end{sphinxuseclass}
\end{sphinxuseclass}
\end{sphinxuseclass}
\end{sphinxuseclass}
\begin{sphinxuseclass}{sd-col}
\begin{sphinxuseclass}{sd-d-flex-column}
\begin{sphinxuseclass}{sd-col-6}
\begin{sphinxuseclass}{sd-col-xs-6}
\begin{sphinxuseclass}{sd-col-sm-6}
\begin{sphinxuseclass}{sd-col-md-6}
\begin{sphinxuseclass}{sd-col-lg-6}
\sphinxAtStartPar
(j)   \(T: \mathbb{C}^2 \to \mathbb{C}\) where \(T: (x, y) \mapsto \bar{y}\)

\end{sphinxuseclass}
\end{sphinxuseclass}
\end{sphinxuseclass}
\end{sphinxuseclass}
\end{sphinxuseclass}
\end{sphinxuseclass}
\end{sphinxuseclass}
\end{sphinxuseclass}
\end{sphinxuseclass}
\end{sphinxuseclass}
\end{sphinxuseclass}
\sphinxAtStartPar
(\(\bar{x}\) is the complex conjugate of \(x = a + bi\), defined by \(\bar{x} = a - bi\).)
\end{sphinxadmonition}
\phantomsection \label{exercise:transformations-ex-transformation-matrix}

\begin{sphinxadmonition}{note}{Exercise 6.5.2}



\sphinxAtStartPar
A linear transformation \(T: \mathbb{R}^2 \to \mathbb{R}^2\) is defined by \(T: (x, y) \mapsto (-x + 3y, x - 4y)\). Determine the transformation matrix for \(T\) and hence calculate \(T (2, 5)\).
\end{sphinxadmonition}
\phantomsection \label{exercise:transformations-ex-R2}

\begin{sphinxadmonition}{note}{Exercise 6.5.3}



\sphinxAtStartPar
A linear transformation \(T: \mathbb{R}^2 \to \mathbb{R}^2\) is defined by \(T: (x, y) \mapsto (x - 2y, 2x + 3y)\). Given \(T(\mathbf{u}) = (-1, 5)\) determine \(\mathbf{u}\).
\end{sphinxadmonition}
\phantomsection \label{exercise:transformations-ex-R3}

\begin{sphinxadmonition}{note}{Exercise 6.5.4}



\sphinxAtStartPar
\(T: \mathbb{R}^3 \to \mathbb{R}^3\) is a linear transformation such that
\begin{equation*}
\begin{split} \begin{align*}
    T\begin{pmatrix} 1 \\ -1 \\ 0 \end{pmatrix} &= \begin{pmatrix} 1 \\ -2 \\ -4 \end{pmatrix}, &
    T\begin{pmatrix} 0 \\ 1 \\ 2 \end{pmatrix} &= \begin{pmatrix} 6 \\ 5 \\ 10 \end{pmatrix}, &
    T\begin{pmatrix} -1 \\ 1 \\ 1 \end{pmatrix} &= \begin{pmatrix} 2 \\ 4 \\ 7 \end{pmatrix}.
\end{align*} \end{split}
\end{equation*}
\sphinxAtStartPar
Find the transformation matrix for \(T\).
\end{sphinxadmonition}
\phantomsection \label{exercise:transformations-ex-rotation}

\begin{sphinxadmonition}{note}{Exercise 6.5.5}



\sphinxAtStartPar
Rotate the position vector \((2, 1) \in \mathbb{R}^2\) by angle \(\pi/6\) anti\sphinxhyphen{}clockwise about the origin.
\end{sphinxadmonition}
\phantomsection \label{exercise:transformations-ex-reflection}

\begin{sphinxadmonition}{note}{Exercise 6.5.6}



\sphinxAtStartPar
Reflect the position vector \((5, 3) \in \mathbb{R}^2\) about the line that passes through \((0, 0)\) and makes an angle \(\pi/3\) with the \(x\)\sphinxhyphen{}axis.
\end{sphinxadmonition}
\phantomsection \label{exercise:transformations-ex-transform-square}

\begin{sphinxadmonition}{note}{Exercise 6.5.7}



\sphinxAtStartPar
A square with side length 2 is centred at the co\sphinxhyphen{}ordinates \((3, 2)\). It is to be translated so the centre is at the origin, rotated by an angle \(\pi/3\) clockwise about the origin and then translated back to its initial position.

\sphinxAtStartPar
(a)   Write down a matrix containing the homogeneous co\sphinxhyphen{}ordinates for the vertices of the square.

\sphinxAtStartPar
(b)   Determine the transformation matrices that perform the three transformations.

\sphinxAtStartPar
(c)   Calculate the composite transformation matrix and apply it to the co\sphinxhyphen{}ordinate matrix from part (a).
\end{sphinxadmonition}

\sphinxstepscope


\part{Exercise Solutions}

\sphinxstepscope


\chapter{Matrices Exercise Solutions}
\label{\detokenize{_pages/A1_Matrices_exercises_solutions:matrices-exercise-solutions}}\label{\detokenize{_pages/A1_Matrices_exercises_solutions:matrices-exercises-solutions}}\label{\detokenize{_pages/A1_Matrices_exercises_solutions::doc}}\phantomsection \label{_pages/A1_Matrices_exercises_solutions:_pages/A1_Matrices_exercises_solutions-solution-0}

\begin{sphinxadmonition}{note}{Solution to Exercise 1.8.1}



\sphinxAtStartPar
(a) \( A = \begin{pmatrix} 
    2 & 3 & 4 \\
    3 & 4 & 5 \\
    4 & 5 & 6 
    \end{pmatrix} \)

\sphinxAtStartPar
(b) \( B = \begin{pmatrix}
        1 & -1 & 1 & -1 \\
        -1 & 1 & -1 & 1 \\
        1 & -1 & 1 & -1 \\
        -1 & 1 & -1 & 1
        \end{pmatrix} \)

\sphinxAtStartPar
(c) \( C = \begin{pmatrix}
        0 & 1 & 1 & 1 \\
        -1 & 0 & 1 & 1 \\
        -1 & -1 & 0 & 1 \\
        -1 & -1 & -1 & 0
    \end{pmatrix} \)
\end{sphinxadmonition}
\phantomsection \label{_pages/A1_Matrices_exercises_solutions:_pages/A1_Matrices_exercises_solutions-solution-1}

\begin{sphinxadmonition}{note}{Solution to Exercise 1.8.2}



\sphinxAtStartPar
(a)   The \(4 \times 4\) Hilbert matrix is
\begin{equation*}
\begin{split} \begin{align*}
    H = 
    \begin{pmatrix}
        1   & 1/2 & 1/3 & 1/4 \\
        1/2 & 1/3 & 1/4 & 1/5 \\
        1/3 & 1/4 & 1/5 & 1/6 \\
        1/4 & 1/5 & 1/6 & 1/7
    \end{pmatrix}.
\end{align*} \end{split}
\end{equation*}
\sphinxAtStartPar
(b)   Since addition of two numbers is commutative, i.e., \(i + j = j + i\) then \(h_{ij} = h_{ji}\) for all \(i, j = 1, 2, \ldots, n\) then the \(n \times n\) Hilbert matrix is symmetric.
\end{sphinxadmonition}
\phantomsection \label{_pages/A1_Matrices_exercises_solutions:_pages/A1_Matrices_exercises_solutions-solution-2}

\begin{sphinxadmonition}{note}{Solution to Exercise 1.8.3}



\sphinxAtStartPar
(a)   \(A + B = \begin{pmatrix} 1 + 3 & -3 + 0 \\ 4 + (-1) & 2 + 5 \end{pmatrix} = \begin{pmatrix} 4 & -3 \\ 3 & 7 \end{pmatrix}\)

\sphinxAtStartPar
(b)   \(B + C\) is undefined since \(B\) is \(2\times 2\) and \(C\) is \(2\times 1\)

\sphinxAtStartPar
(c)   \(A^\mathsf{T} = \begin{pmatrix} 1 & 4 \\ -3 & 2 \end{pmatrix}\)

\sphinxAtStartPar
(d)   \(C^\mathsf{T} = \begin{pmatrix} 5 & 9 \end{pmatrix}\)

\sphinxAtStartPar
(e)   \(3 B - A = \begin{pmatrix} 9 & 0 \\ -3 & 15 \end{pmatrix} - \begin{pmatrix} 1 & -3 \\ 4 & 2 \end{pmatrix} = \begin{pmatrix} 8 & 3 \\ -7 & 13 \end{pmatrix}\)

\sphinxAtStartPar
(f)    \((F^\mathsf{T})^\mathsf{T} = \begin{pmatrix} 1 \\ 2 \\ 4 \end{pmatrix}^\mathsf{T} = \begin{pmatrix} 1 & -2 & 4 \end{pmatrix} = F\)

\sphinxAtStartPar
(g)   \(A^\mathsf{T} + B^\mathsf{T} = \begin{pmatrix} 1 & 4 \\ -3 & 2 \end{pmatrix} + \begin{pmatrix} 3 & -1 \\ 0 & 5 \end{pmatrix} = \begin{pmatrix} 4 & 3 \\ -3 & 7 \end{pmatrix}\)

\sphinxAtStartPar
(h)   \((A + B)^\mathsf{T} = \begin{pmatrix} 4 & -3 \\ 3 & 7 \end{pmatrix}^\mathsf{T} = \begin{pmatrix} 4 & 3 \\ -3 & 7 \end{pmatrix}\)
\end{sphinxadmonition}
\phantomsection \label{_pages/A1_Matrices_exercises_solutions:_pages/A1_Matrices_exercises_solutions-solution-3}

\begin{sphinxadmonition}{note}{Solution to Exercise 1.8.4}



\sphinxAtStartPar
(a)   \(AB = \begin{pmatrix} 1 & -3 \\ 4 & 2 \end{pmatrix} \begin{pmatrix} 3 & 0 \\ -1 & 5 \end{pmatrix} = \begin{pmatrix} 3 + 3 & 0 - 15 \\ 7 - 2 & 0 + 10 \end{pmatrix} = \begin{pmatrix}6 & -15 \\ 10 & 10 \end{pmatrix}\)

\sphinxAtStartPar
(b)   \(BA = \begin{pmatrix} 3 & 0 \\ -1 & 5 \end{pmatrix}\begin{pmatrix} 1 & -3 \\ 4 & 2 \end{pmatrix} = \begin{pmatrix} 3 + 0 & -9 + 0 \\ -1 + 20 & 3 + 10 \end{pmatrix} = \begin{pmatrix} 3 & -9 \\ 19 & 13 \end{pmatrix}\)

\sphinxAtStartPar
(c)   \(AC = \begin{pmatrix} 1 & -3 \\ 4 & 2 \end{pmatrix}\begin{pmatrix} 5 \\ 9 \end{pmatrix} = \begin{pmatrix} 15 - 27 \\ 20 + 18 \end{pmatrix} = \begin{pmatrix} -22 \\ 38 \end{pmatrix}\)

\sphinxAtStartPar
(d)   \(CA\) is undefined since \(C\) has 1 column and \(A\) has 2 rows

\sphinxAtStartPar
(e)   \(C^\mathsf{T}C = \begin{pmatrix} 5 & 9 \end{pmatrix} \begin{pmatrix} 5 \\ 9 \end{pmatrix} = 25 + 81 = 106\)

\sphinxAtStartPar
(f)   \(CC^\mathsf{T} = \begin{pmatrix} 5 \\ 9 \end{pmatrix}\begin{pmatrix} 5 & 9 \end{pmatrix} = \begin{pmatrix} 25 & 45 \\ 45 & 81 \end{pmatrix}\)

\sphinxAtStartPar
(g)  
\begin{equation*}
\begin{split} \begin{align*}
    DE &= \begin{pmatrix} 1 & 1 & 3 \\ 4 & -2 & 3 \end{pmatrix} \begin{pmatrix} 1 & 2 \\ 0 & 6 \\ -2 & 3 \end{pmatrix} = \begin{pmatrix} 1 + 0 - 6 & 2 + 6 + 9 \\ 4 + 0 - 6 & 8 - 12 + 9 \end{pmatrix} \\
    &= \begin{pmatrix} -5 & 17 \\ -2 & 5 \end{pmatrix}
\end{align*} \end{split}
\end{equation*}
\sphinxAtStartPar
(h)  
\begin{equation*}
\begin{split} \begin{align*}
    GH &= \begin{pmatrix} 4 & 2 & 3 \\ -2 & 6 & 0 \\ 0 & 7 & 1 \end{pmatrix}
    \begin{pmatrix} 1 & 0 & 1 \\ 5 & 2 & -2 \\ 2 & -3 & 4 \end{pmatrix} \\
    &= \begin{pmatrix} 
        4 + 10 + 6 & 0 + 4 - 9 & 4 - 4 + 12 \\
        -2 + 30 + 0 & 0 + 12 + 0 & -2 - 12 + 0 \\
        0 + 35 + 2 & 0 + 14 - 3 & 0 - 14 + 4
    \end{pmatrix} \\
    &= \begin{pmatrix} 20 & -5 & 12 \\ 28 & 12 & -14 \\ 37 & 11 & -10 \end{pmatrix}
\end{align*} \end{split}
\end{equation*}
\sphinxAtStartPar
(i)  
\begin{equation*}
\begin{split} \begin{align*}
    A(DE) &= \begin{pmatrix} 1 & -3 \\ 4 & 2 \end{pmatrix} \left(
    \begin{pmatrix} 1 & 1 & 3 \\ 4 & -2 & 3 \end{pmatrix} 
    \begin{pmatrix} 1 & 2 \\ 0 & 6 \\ -2 & 3 \end{pmatrix} \right) \\
    &= \begin{pmatrix} 1 & -3 \\ 4 & 2 \end{pmatrix}
    \begin{pmatrix} -5 & 17 \\ -2 & 5 \end{pmatrix} \\
    &= \begin{pmatrix} 1 & 2 \\ -24 & 78 \end{pmatrix}
\end{align*} \end{split}
\end{equation*}
\sphinxAtStartPar
(j)  
\begin{equation*}
\begin{split} \begin{align*}
    (AD)E &= \left(
    \begin{pmatrix} 1 & -3 \\ 4 & 2 \end{pmatrix}
    \begin{pmatrix} 1 & 1 & 3 \\ 4 & -2 & 3 \end{pmatrix}
    \right)
    \begin{pmatrix} 1 & 2 \\ 0 & 6 \\ -2 & 3 \end{pmatrix} \\
    &= \begin{pmatrix} -11 & 7 & -6 \\ 12 & 0 & 18 \end{pmatrix}
    \begin{pmatrix} 1 & 2 \\ 0 & 6 \\ -2 & 3 \end{pmatrix} \\
    &= \begin{pmatrix} 1 & 2 \\ -24 & 78 \end{pmatrix}
\end{align*} \end{split}
\end{equation*}
\sphinxAtStartPar
(k)  
\begin{equation*}
\begin{split} \begin{align*}
    A^3 &= \begin{pmatrix} 1 & -3 \\ 4 & 2 \end{pmatrix}
    \begin{pmatrix} 1 & -3 \\ 4 & 2 \end{pmatrix}
    \begin{pmatrix} 1 & -3 \\ 4 & 2 \end{pmatrix} \\
    &= \begin{pmatrix} -11 & -9 \\ 12 & -8 \end{pmatrix}
    \begin{pmatrix} 1 & -3 \\ 4 & 2 \end{pmatrix} \\
    &= \begin{pmatrix} -47 & 15 \\ -20 & -52 \end{pmatrix}
\end{align*} \end{split}
\end{equation*}
\sphinxAtStartPar
(l)  
\begin{equation*}
\begin{split} \begin{align*}
    G^2 &= \begin{pmatrix} 4 & 2 & 3 \\ -2 & 6 & 0 \\ 0 & 7 & 1 \end{pmatrix}
    \begin{pmatrix} 4 & 2 & 3 \\ -2 & 6 & 0 \\ 0 & 7 & 1 \end{pmatrix} \\
    &= \begin{pmatrix} 12 & 41 & 15 \\ -20 & 32 & -6 \\ -14 & 49 & 1 \end{pmatrix} \\
    \therefore G^4 &= G^2G^2 = 
    \begin{pmatrix} 12 & 41 & 15 \\ -20 & 32 & -6 \\ -14 & 49 & 1 \end{pmatrix}
    \begin{pmatrix} 12 & 41 & 15 \\ -20 & 32 & -6 \\ -14 & 49 & 1 \end{pmatrix} \\
    &=
    \begin{pmatrix} -886 & 2539 & -51 \\ -796 & -90 & -498 \\ -1162 & 1043 & -503 \end{pmatrix}
\end{align*} \end{split}
\end{equation*}\end{sphinxadmonition}
\phantomsection \label{_pages/A1_Matrices_exercises_solutions:_pages/A1_Matrices_exercises_solutions-solution-4}

\begin{sphinxadmonition}{note}{Solution to Exercise 1.8.5}



\sphinxAtStartPar
(a)   \(\det(A) = \begin{vmatrix} 1 & -3 \\ 4 & 2 \end{vmatrix} = 1(2) - (-3)(4) = 14\)

\sphinxAtStartPar
(b)   \(\det(B) = \begin{vmatrix} 3 & 0 \\ -1 & 5 \end{vmatrix} = 3(5) - 0 (-1) = 15\)

\sphinxAtStartPar
(c)  
\(\begin{align*}
    \det(G) &= 
    \begin{vmatrix}
         4 & 2 & 3 \\
        -2 & 6 & 0 \\
         0 & 7 & 1
    \end{vmatrix}
    = 4
    \begin{vmatrix} 6 & 0 \\ 7 & 1 \end{vmatrix} 
    - (-2)
    \begin{vmatrix} 2 & 3 \\ 7 & 1 \end{vmatrix}
    = 4(6 - 0) + 2(2 - 21) = -14
\end{align*} \)

\sphinxAtStartPar
(d)  
\(\begin{align*}
    \det(H) &=
    \begin{vmatrix}
        1 &  0 &  1 \\
        5 &  2 & -2 \\ 
        2 & -3 &  4
    \end{vmatrix}
    =
    \begin{vmatrix} 2 & -2 \\ -3 &  4 \end{vmatrix} +
    \begin{vmatrix} 5 &  2 \\  2 & -3 \end{vmatrix}
    = (8 - 6) + (-15 - 4) = -17
\end{align*} \)
\end{sphinxadmonition}
\phantomsection \label{_pages/A1_Matrices_exercises_solutions:_pages/A1_Matrices_exercises_solutions-solution-5}

\begin{sphinxadmonition}{note}{Solution to Exercise 1.8.6}



\sphinxAtStartPar
\(A\):
\begin{equation*}
\begin{split} \begin{align*}
    \adj(A) &=
    \begin{pmatrix} 2 & -4 \\ 3 & 1 \end{pmatrix}^\mathsf{T}
    = 
    \begin{pmatrix} 2 & 3 \\ -4 & 1 \end{pmatrix},  \\
    \therefore A^{-1} &= \frac{1}{14} \begin{pmatrix} 2 & 3 \\ -4 & 1 \end{pmatrix}
    = \begin{pmatrix} 1/7 & 3/14 \\ -2/7 & 1/14 \end{pmatrix}.
\end{align*} \end{split}
\end{equation*}
\sphinxAtStartPar
Check:
\begin{equation*}
\begin{split} \begin{align*}
    AA^{-1} &=
    \begin{pmatrix} 1 & -3 \\ 4 & 2 \end{pmatrix}
    \begin{pmatrix} 1/7 & 3/14 \\ -2/7 & 1/14 \end{pmatrix}
    =
    \begin{pmatrix} 1/7 + 6/7 & 3/14 - 3/14 \\ 4/7 - 4/7 & 12/14 + 2/14 \end{pmatrix}
    = \begin{pmatrix} 1 & 0 \\ 0 & 1 \end{pmatrix}.
\end{align*} \end{split}
\end{equation*}
\sphinxAtStartPar
\(B\):
\begin{equation*}
\begin{split} \begin{align*}
    \adj{B} &=
    \begin{pmatrix} 5 & 1 \\ 0 & 3 \end{pmatrix}^\mathsf{T}
    =
    \begin{pmatrix} 5 & 0 \\ 1 & 3 \end{pmatrix}, \\
    \therefore B^{-1} &= \frac{1}{15} \begin{pmatrix} 5 & 0 \\ 1 & 3 \end{pmatrix} 
    =
    \begin{pmatrix} 1/3 & 0 \\ 1/15 & 1/5 \end{pmatrix}.
\end{align*} \end{split}
\end{equation*}
\sphinxAtStartPar
Check:
\begin{equation*}
\begin{split} \begin{align*}
    BB^{-1} &=
    \begin{pmatrix} 3 & 0 \\ -1 & 5 \end{pmatrix}
    \begin{pmatrix} 1/3 & 0 \\ 1/15 & 1/5 \end{pmatrix}
    =
    \begin{pmatrix} 1 + 0 & 0 + 0 \\ -1/3 + 5/15 & 0 + 1 \end{pmatrix}
    =
    \begin{pmatrix} 1 & 0 \\ 0 & 1 \end{pmatrix}.
\end{align*} \end{split}
\end{equation*}
\sphinxAtStartPar
\(G\):
\begin{equation*}
\begin{split} \begin{align*}
    \adj{G} &= 
    \begin{pmatrix}
          \begin{vmatrix} 6 & 0 \\ 7 & 1 \end{vmatrix} 
        &
        - \begin{vmatrix} -2 & 0 \\ 0 & 1 \end{vmatrix}
        &
          \begin{vmatrix} -2 & 6 \\ 0 & 7 \end{vmatrix}
        \\
        - \begin{vmatrix} 2 & 3 \\ 7 & 1 \end{vmatrix}
        &
          \begin{vmatrix} 4 & 3 \\ 0 & 1 \end{vmatrix}
        &
        - \begin{vmatrix} 4 & 2 \\ 0 & 7 \end{vmatrix}
        \\
          \begin{vmatrix} 2 & 3 \\ 6 & 0 \end{vmatrix}
        &
        - \begin{vmatrix} 4 & 3 \\ -2 & 0 \end{vmatrix}
        &
          \begin{vmatrix} 4 & 2 \\ -2 & 6 \end{vmatrix}
    \end{pmatrix}^\mathsf{T} \\   
    &=
    \begin{pmatrix}
          6 &  2 & -14 \\
         19 &  4 & -28 \\
        -18 & -6 &  28 
    \end{pmatrix}^\mathsf{T}
    =
    \begin{pmatrix}
          6 &  19 & -18 \\
          2 &   4 &  -6 \\
        -14 & -28 &  28
    \end{pmatrix}, \\
    \therefore G^{-1} 
    &= -\frac{1}{14}
    \begin{pmatrix}
          6 &  19 & -18 \\
          2 &   4 &  -6 \\
        -14 & -28 &  28
    \end{pmatrix}
    =
    \begin{pmatrix}
        -3/7 & -19/14 &  9/7 \\
        -1/7 &  -2/7  &  3/7 \\
         1   &   2    & -2
    \end{pmatrix}.
\end{align*} \end{split}
\end{equation*}
\sphinxAtStartPar
Check:
\begin{equation*}
\begin{split} \begin{align*}
    GG^{-1} &=
    \begin{pmatrix}
         4 & 2 & 3 \\
        -2 & 6 & 0 \\
         0 & 7 & 1 
    \end{pmatrix}
    \begin{pmatrix}
        -3/7 & -19/14 &  9/7 \\
        -1/7 &  -2/7  &  3/7 \\
         1   &   2    & -2
    \end{pmatrix} \\
    &=
    \begin{pmatrix}
        -12/7 - 2/7 + 3 & -76/14 -  4/7 + 6 &  36/7 +  6/7 - 6 \\
          6/7 - 6/7 + 0 &  38/14 - 12/7 + 0 & -18/7 + 18/7 + 0 \\
          0   - 7/7 + 1 &   0    - 14/7 + 2 &   0   + 21/7 - 2
    \end{pmatrix} \\
    &=
    \begin{pmatrix}
        1 & 0 & 0 \\
        0 & 1 & 0 \\
        0 & 0 & 1
    \end{pmatrix}
\end{align*} \end{split}
\end{equation*}
\sphinxAtStartPar
\(H\):
\begin{equation*}
\begin{split} \begin{align*}
    \adj(H) &=
    \begin{pmatrix}
          \begin{vmatrix} 2 & -2 \\ -3 & 4 \end{vmatrix}
        &
        - \begin{vmatrix} 5 & -2 \\ 2 & 4 \end{vmatrix}
        &
          \begin{vmatrix} 5 & 2 \\ 2 & -3 \end{vmatrix}
        \\
        - \begin{vmatrix} 0 & 1 \\ -3 & 4 \end{vmatrix}
        &
          \begin{vmatrix} 1 & 1 \\ 2 & 4 \end{vmatrix}
        &
        - \begin{vmatrix} 1 & 0 \\ 2 & -3 \end{vmatrix}
        \\
          \begin{vmatrix} 0 & 1 \\ 2 & -2 \end{vmatrix}
        &
        - \begin{vmatrix} 1 & 1 \\ 5 & -2 \end{vmatrix}
        &
          \begin{vmatrix} 1 & 0 \\ 5 & 2 \end{vmatrix}
    \end{pmatrix}^\mathsf{T} \\
    &=
    \begin{pmatrix}
        2 & -24 & -19 \\
        -3 & 2 & 3 \\
        -2 & 7 & 2 
    \end{pmatrix}^\mathsf{T}
    =
    \begin{pmatrix}
         2  & -3 & -2 \\
        -24 &  2 &  7 \\
        -19 &  3 &  2 
    \end{pmatrix}, \\
    \therefore H^{-1}
    &= -\frac{1}{17}
    \begin{pmatrix}
         2  & -3 & -2 \\
        -24 &  2 &  7 \\
        -19 &  3 &  2 
    \end{pmatrix}
    =
    \begin{pmatrix}
        -2/17 &  3/17 &  2/17 \\
        24/17 & -2/17 & -7/17 \\
        19/17 & -3/17 & -2/17
    \end{pmatrix} .
\end{align*} \end{split}
\end{equation*}
\sphinxAtStartPar
Check:
\begin{equation*}
\begin{split} \begin{align*}
    HH^{-1} &=
    \begin{pmatrix}
        1 &  0 &  1 \\
        5 &  2 & -2 \\
        2 & -3 &  4 
    \end{pmatrix}
    \begin{pmatrix}
        -2/17 &  3/17 &  2/17 \\
        24/17 & -2/17 & -7/17 \\
        19/17 & -3/17 & -2/17
    \end{pmatrix}
    \\
    &=
    \begin{pmatrix}
        -2/17 +  0    + 19/17 &  3/17 - 3/17         &  2/17 - 2/17 \\
        -5/17 + 48/17 - 38/17 & 15/17 - 4/17 +  6/17 & 10/17 - 14/17 + 4/17 \\ 
        -4/17 - 72/17 + 76/17 &  6/17 + 6/17 - 12/17 &  4/17 + 21/17 - 8/17
    \end{pmatrix}
    \\
    &=
    \begin{pmatrix}
        1 & 0 & 0 \\
        0 & 1 & 0 \\
        0 & 0 & 1
    \end{pmatrix}
\end{align*} \end{split}
\end{equation*}\end{sphinxadmonition}
\phantomsection \label{_pages/A1_Matrices_exercises_solutions:_pages/A1_Matrices_exercises_solutions-solution-6}

\begin{sphinxadmonition}{note}{Solution to Exercise 1.8.7}



\sphinxAtStartPar
For \(AA^\mathsf{T}\) to be symmetric we need to show that \(AA^\mathsf{T} = (AA^\mathsf{T})^\mathsf{T}\)
\begin{equation*}
\begin{split} \begin{align*}
    (AA^\mathsf{T})^\mathsf{T} &= 
    (A^\mathsf{T})^\mathsf{T}A^\mathsf{T} && \textsf{since $(AB)^\mathsf{T} = B^\mathsf{T}A^\mathsf{T}$} \\
    &= AA^\mathsf{T} && \textsf{since $(A^\mathsf{T})^\mathsf{T} = A$.}
\end{align*} \end{split}
\end{equation*}
\sphinxAtStartPar
So \(AA^\mathsf{T}\) is symmetric
\end{sphinxadmonition}
\phantomsection \label{_pages/A1_Matrices_exercises_solutions:_pages/A1_Matrices_exercises_solutions-solution-7}

\begin{sphinxadmonition}{note}{Solution to Exercise 1.8.8}



\sphinxAtStartPar
To prove that \((AB)^{-1} = B^{-1}A^{-1}\) we need to show that \((AB)(B^{-1}A^{-1}) = I\)
\begin{equation*}
\begin{split} \begin{align*}
    (AB)(B^{-1}A^{-1}) 
    &= A(BB^{-1}A^{-1}) && \textsf{associativity law} \\
    &= A(IA^{-1}) && \textsf{since $BB^{-1} = I$} \\
    &= AA^{-1} && \textsf{since $AI = IA = A$}\\
    &= I    && \textsf{definition of $A^{-1}.$} 
\end{align*} \end{split}
\end{equation*}
\sphinxAtStartPar
Therefore \((AB)^{-1} = B^{-1}A^{-1}\).
\end{sphinxadmonition}
\phantomsection \label{_pages/A1_Matrices_exercises_solutions:_pages/A1_Matrices_exercises_solutions-solution-8}

\begin{sphinxadmonition}{note}{Solution to Exercise 1.8.9}



\sphinxAtStartPar
Expanding out \((A + B)^2\)
\begin{equation*}
\begin{split} (A + B)^2 = (A + B)(A + B) = A^2 + AB + BA + B^2, \end{split}
\end{equation*}
\sphinxAtStartPar
therefore \((A + B)^2 \neq A^2 + 2AB + B^2\).

\sphinxAtStartPar
This would be true if \(AB = BA\), i.e., \(A = B\) or \(A = I\) or \(B = I\).
\end{sphinxadmonition}
\phantomsection \label{_pages/A1_Matrices_exercises_solutions:_pages/A1_Matrices_exercises_solutions-solution-9}

\begin{sphinxadmonition}{note}{Solution to Exercise 1.8.10}



\sphinxAtStartPar
We need to show that \(A^2 = I\)
\begin{equation*}
\begin{split} \begin{align*}
    A^2 =
    \begin{pmatrix} a & b \\ c & -a \end{pmatrix} 
    \begin{pmatrix} a & b \\ c & -a \end{pmatrix} 
    = 
    \begin{pmatrix} a^2 + bc & 0 \\ 0 & a^2 + bc \end{pmatrix},
\end{align*} \end{split}
\end{equation*}
\sphinxAtStartPar
so \(a^2 + bc = 1\) for \(A\) to be an \sphinxhref{https://en.wikipedia.org/wiki/Involutory\_matrix}{involutory matrix}.
\end{sphinxadmonition}
\phantomsection \label{_pages/A1_Matrices_exercises_solutions:_pages/A1_Matrices_exercises_solutions-solution-10}

\begin{sphinxadmonition}{note}{Solution to Exercise 1.8.11}



\sphinxAtStartPar
(a)   True

\sphinxAtStartPar
Proof: Right multiply both sides of \(A = B\) by \(C\).

\sphinxAtStartPar
(b)   False, e.g., if \(C = \begin{pmatrix} 0 & 0 \\ 0 & 0 \end{pmatrix}\) then this would hold for any two matrices.

\sphinxAtStartPar
(c)   False, e.g., if \(A = \begin{pmatrix} 1 & 0 \\ 0 & 0 \end{pmatrix}\) and \(B = \begin{pmatrix} 0 & 0 \\ 1 & 0 \end{pmatrix}\) then \(AB = O\).

\sphinxAtStartPar
(d)   True

\sphinxAtStartPar
Proof: Add \(C\) to both sides of \(A = B\).

\sphinxAtStartPar
(e)   False, e.g., if \(A = \begin{pmatrix} 1 & 0 \\ 0 & -1 \end{pmatrix}\) then \(A^2 = I\).

\sphinxAtStartPar
(f)   True

\sphinxAtStartPar
Proof: If \(B = A^2\) then \(b_{ii} = \sum_{k=1}^n a_{ik}a_{ki}\) and since \(A\) is symmetric then \(a_{ik} = a_{ki}\) for \(i = 1, 2, \ldots, n\) so \(b_{ii} = \sum_{k=1}^n = a_{ij}^2\) which is always greater than or equal to 0.

\sphinxAtStartPar
(g)   True

\sphinxAtStartPar
Proof:
\begin{itemize}
\item {} 
\sphinxAtStartPar
Let \(A\) and \(B\) be two \(n \times n\) matrices then \(C\) is also an \(n \times n\) matrix.

\item {} 
\sphinxAtStartPar
Let \(A\) be an \(n \times n\) matrix and \(C\) be an \(m \times m\) matrix and \(B\) is a \(p \times q\) matrix. \(p = n\) and \(q = m\) but \(m = n\) so \(p = q = m\) and \(B\) is a square matrix.

\item {} 
\sphinxAtStartPar
Ley \(B\) be an \(m \times m\) matrix and \(C\) be an \(n \times n\) matrix and \(A\) is a \(p \times q\) matrix. \(p = n\) and \(q = m\) but \(m = n\) so \(p = q = m\) and \(A\) is a square matrix.

\end{itemize}

\sphinxAtStartPar
(h)   True

\sphinxAtStartPar
Proof: Let \(B\) be a \(p \times q\) matrix then since \(C\) is an \(m \times 1\) matrix then \(q = 1\).

\sphinxAtStartPar
(i)   False, e.g., if \(A = \begin{pmatrix} 0 & 1 \\ 1 & 0 \end{pmatrix}\) then \(A^3 = A\).
\end{sphinxadmonition}
\phantomsection \label{_pages/A1_Matrices_exercises_solutions:_pages/A1_Matrices_exercises_solutions-solution-11}

\begin{sphinxadmonition}{note}{Solution to Exercise 1.8.12}



\sphinxAtStartPar
(a)
\begin{equation*}
\begin{split} \begin{align*}
    5X &= A \\
    X &= \frac{1}{5} A \\
    &= \dfrac{1}{5}\begin{pmatrix} 1 & -3 \\ 4 & 2 \end{pmatrix} 
    = \begin{pmatrix} \frac{1}{5} & -\frac{3}{5} \\ \frac{4}{5} & \frac{2}{5} \end{pmatrix}
\end{align*} \end{split}
\end{equation*}
\sphinxAtStartPar
(b)
\begin{equation*}
\begin{split} \begin{align*}
    X + A &= I \\
    X &= I - A \\
    &= \begin{pmatrix} 1 & 0 \\ 0 & 1 \end{pmatrix} - \begin{pmatrix} 1 & -3 \\ 4 & 2 \end{pmatrix} = \begin{pmatrix} 0 & 3 \\ -4 & -1 \end{pmatrix}
\end{align*} \end{split}
\end{equation*}
\sphinxAtStartPar
(c)
\begin{equation*}
\begin{split} \begin{align*}
    2X - B &= A \\
    2X &= A + B \\
    X &= \dfrac{1}{2}(A + B) = \dfrac{1}{2}\left( \begin{pmatrix} 1 & -3 \\ 4 & 2 \end{pmatrix} + \begin{pmatrix} 3 & 0 \\ -1 & 5 \end{pmatrix} \right) \\
    &= \dfrac{1}{2} \begin{pmatrix} 4 & -3 \\ 3 & 7 \end{pmatrix} = \begin{pmatrix} 2 & -\frac{3}{2} \\ \frac{3}{2} & \frac{7}{2} \end{pmatrix}
\end{align*} \end{split}
\end{equation*}
\sphinxAtStartPar
(d)
\begin{equation*}
\begin{split} \begin{align*}
    XA &= I \\
    X &= IA^{-1} = A^{-1}I \\
    &= \dfrac{1}{14} \begin{pmatrix} 2 & 3 \\ -4 & 1 \end{pmatrix} = \begin{pmatrix} \frac{1}{7} & \frac{3}{14} \\ -\frac{2}{7} & \frac{1}{14} \end{pmatrix}
\end{align*} \end{split}
\end{equation*}
\sphinxAtStartPar
(e)
\begin{equation*}
\begin{split} \begin{align*}
    BX &= A \\
    X &= B^{-1}A \\
    &= \dfrac{1}{15}\begin{pmatrix} 5 & 0 \\ 1 & 3 \end{pmatrix} \begin{pmatrix} 1 & -3 \\ 4 & 2 \end{pmatrix} \\
    &= \dfrac{1}{15} \begin{pmatrix} 5 & -15 \\ 13 & 3\end{pmatrix} = \begin{pmatrix} \frac{1}{3} & -1 \\ \frac{13}{15} & \frac{1}{5} \end{pmatrix}
\end{align*} \end{split}
\end{equation*}
\sphinxAtStartPar
(f)
\begin{equation*}
\begin{split} \begin{align*}
    A^2 &= X \\
    X &= \begin{pmatrix} 1 & -3 \\ 4 & 2 \end{pmatrix}\begin{pmatrix} 1 & -3 \\ 4 & 2 \end{pmatrix} = \begin{pmatrix} -11 & -9 \\ 12 & -8 \end{pmatrix}
\end{align*} \end{split}
\end{equation*}
\sphinxAtStartPar
(g)   We begin by setting \(X = \begin{pmatrix}a & b \\ c & d\end{pmatrix}\), and then \(X^2 = B\) gives
\begin{equation*}
\begin{split} \begin{align*}
    \begin{pmatrix}a^2 + bc & ab + bd \\ ca+dc & cb+d^2\end{pmatrix} =
    \begin{pmatrix} a^2 + bc & b(a + d) \\ c(a + d) & cb + d^2 \end{pmatrix} =
    \begin{pmatrix} 3 & 0 \\ -1 & 5 \end{pmatrix}.
\end{align*} \end{split}
\end{equation*}
\sphinxAtStartPar
Now we can solve the quadratic equations over \(\mathbb{R}\), since \([X^2]_{12} = 0\) then
\begin{equation*}
\begin{split} \begin{align*}
    b(a + d) &= 0 \\
    \therefore b &= 0 \text{ or } a = -d.
\end{align*} \end{split}
\end{equation*}
\sphinxAtStartPar
For the case when \(b = 0\)
\begin{equation*}
\begin{split} \begin{align*}
    a^2 + bc &= 3 \implies a = \pm \sqrt{3}, \\
    cb + d^2 &= 5 \implies d = \pm \sqrt{5}.
\end{align*} \end{split}
\end{equation*}
\sphinxAtStartPar
So we have four possible solutions
\begin{equation*}
\begin{split} \begin{align*}
    X &= \begin{pmatrix} \pm \sqrt{3} & 0 \\ - 1 / (\pm \sqrt{3} \pm \sqrt{5}) & \pm\sqrt{5} \end{pmatrix}, \text{ or }\\
    X &= \begin{pmatrix} \pm \sqrt{3} & 0 \\ - 1 / (\pm \sqrt{3} \mp \sqrt{5}) & \mp\sqrt{5} \end{pmatrix}.
\end{align*} \end{split}
\end{equation*}
\sphinxAtStartPar
For the case when \(a = -d\)
\begin{equation*}
\begin{split} \begin{align*}
    c(a + d) = 0c = -1,
\end{align*} \end{split}
\end{equation*}
\sphinxAtStartPar
which is a contradiction so \(a = -d\) yields no solutions.

\sphinxAtStartPar
(h)
\begin{equation*}
\begin{split} \begin{align*}
    (X + A)B &= I \\
    X + A &= I B^{-1} \\
    X &= B^{-1} - A \\
    &= \frac{1}{15} \begin{pmatrix} 5 & 0 \\ 1 & 3 \end{pmatrix} - 
    \begin{pmatrix} 1 & -3 \\ 4 & 2 \end{pmatrix} \\
    &= \begin{pmatrix} -2/3 & 3 \\ -59/15 & -9/5 \end{pmatrix} \\
    &= \frac{1}{15} \begin{pmatrix} -10 & 45 \\ -59 & -27 \end{pmatrix}
\end{align*} \end{split}
\end{equation*}\end{sphinxadmonition}

\sphinxstepscope


\chapter{Systems of Linear Equations Exercise Solutions}
\label{\detokenize{_pages/A2_Linear_systems_exercises_solutions:systems-of-linear-equations-exercise-solutions}}\label{\detokenize{_pages/A2_Linear_systems_exercises_solutions:systems-exercises-solutions}}\label{\detokenize{_pages/A2_Linear_systems_exercises_solutions::doc}}\phantomsection \label{_pages/A2_Linear_systems_exercises_solutions:_pages/A2_Linear_systems_exercises_solutions-solution-0}

\begin{sphinxadmonition}{note}{Solution to Exercise 2.11.1}



\sphinxAtStartPar
(a)
\begin{equation*}
\begin{split} \begin{align*}
    \det\left(\begin{matrix}-4 & 2\\3 & 4\end{matrix}\right) &= -22, \\ 
    \operatorname{adj}\left(\begin{matrix}-4 & 2\\3 & 4\end{matrix}\right) &= \left(\begin{matrix}4 & -2\\-3 & -4\end{matrix}\right), \\ 
    \therefore A^{-1} &= - \frac{1}{22}\left(\begin{matrix}4 & -2\\-3 & -4\end{matrix}\right) = \left(\begin{matrix}- \frac{2}{11} & \frac{1}{11}\\\frac{3}{22} & \frac{2}{11}\end{matrix}\right) 
\end{align*} \end{split}
\end{equation*}
\sphinxAtStartPar
therefore
\begin{equation*}
\begin{split}\begin{align*}
    \mathbf{x} &= \left(\begin{matrix}- \frac{2}{11} & \frac{1}{11}\\\frac{3}{22} & \frac{2}{11}\end{matrix}\right)
    \left(\begin{matrix}-22\\11\end{matrix}\right) = 
    \left(\begin{matrix}5\\-1\end{matrix}\right)
\end{align*} \end{split}
\end{equation*}
\sphinxAtStartPar
so the solution is \(x_1 = 5\) and \(x_2 = -1\).

\sphinxAtStartPar
(b)
\begin{equation*}
\begin{split} \begin{align*}
    \det\left(\begin{matrix}-4 & 2\\-1 & -3\end{matrix}\right) &= 14, \\ 
    \operatorname{adj}\left(\begin{matrix}-4 & 2\\-1 & -3\end{matrix}\right) &= \left(\begin{matrix}-3 & -2\\1 & -4\end{matrix}\right), \\ 
    \therefore A^{-1} &= \frac{1}{14}\left(\begin{matrix}-3 & -2\\1 & -4\end{matrix}\right) = \left(\begin{matrix}- \frac{3}{14} & - \frac{1}{7}\\\frac{1}{14} & - \frac{2}{7}\end{matrix}\right) 
\end{align*} \end{split}
\end{equation*}
\sphinxAtStartPar
therefore
\begin{equation*}
\begin{split}\begin{align*}
    \mathbf{x} &= \left(\begin{matrix}- \frac{3}{14} & - \frac{1}{7}\\\frac{1}{14} & - \frac{2}{7}\end{matrix}\right)
    \left(\begin{matrix}6\\-2\end{matrix}\right) = 
    \left(\begin{matrix}-1\\1\end{matrix}\right)
\end{align*} \end{split}
\end{equation*}
\sphinxAtStartPar
so the solution is \(x_1 = -1\) and \(x_2 = 1\).

\sphinxAtStartPar
(c)
\begin{equation*}
\begin{split} \begin{align*}
    \det\left(\begin{matrix}-4 & -4 & -2\\3 & 0 & 4\\1 & 0 & 0\end{matrix}\right) &= -16, \\ 
    \operatorname{adj}\left(\begin{matrix}-4 & -4 & -2\\3 & 0 & 4\\1 & 0 & 0\end{matrix}\right) &= \left(\begin{matrix}0 & 0 & -16\\4 & 2 & 10\\0 & -4 & 12\end{matrix}\right), \\ 
    \therefore A^{-1} &= - \frac{1}{16}\left(\begin{matrix}0 & 0 & -16\\4 & 2 & 10\\0 & -4 & 12\end{matrix}\right) = \left(\begin{matrix}0 & 0 & 1\\- \frac{1}{4} & - \frac{1}{8} & - \frac{5}{8}\\0 & \frac{1}{4} & - \frac{3}{4}\end{matrix}\right) 
\end{align*} \end{split}
\end{equation*}
\sphinxAtStartPar
therefore
\begin{equation*}
\begin{split}\begin{align*}
    \mathbf{x} &= \left(\begin{matrix}0 & 0 & 1\\- \frac{1}{4} & - \frac{1}{8} & - \frac{5}{8}\\0 & \frac{1}{4} & - \frac{3}{4}\end{matrix}\right)
    \left(\begin{matrix}16\\-8\\0\end{matrix}\right) = 
    \left(\begin{matrix}0\\-3\\-2\end{matrix}\right)
\end{align*} \end{split}
\end{equation*}
\sphinxAtStartPar
so the solution is \(x_1 = 0\), \(x_2 = -3\) and \(x_3 = -2\).

\sphinxAtStartPar
(d)
\begin{equation*}
\begin{split} \begin{align*}
    \det\left(\begin{matrix}4 & 0 & -4\\4 & -1 & 1\\3 & 1 & 2\end{matrix}\right) &= -40, \\ 
    \operatorname{adj}\left(\begin{matrix}4 & 0 & -4\\4 & -1 & 1\\3 & 1 & 2\end{matrix}\right) &= \left(\begin{matrix}-3 & -4 & -4\\-5 & 20 & -20\\7 & -4 & -4\end{matrix}\right), \\ 
    \therefore A^{-1} &= - \frac{1}{40}\left(\begin{matrix}-3 & -4 & -4\\-5 & 20 & -20\\7 & -4 & -4\end{matrix}\right) = \left(\begin{matrix}\frac{3}{40} & \frac{1}{10} & \frac{1}{10}\\\frac{1}{8} & - \frac{1}{2} & \frac{1}{2}\\- \frac{7}{40} & \frac{1}{10} & \frac{1}{10}\end{matrix}\right) 
\end{align*} \end{split}
\end{equation*}
\sphinxAtStartPar
therefore
\begin{equation*}
\begin{split}\begin{align*}
    \mathbf{x} &= \left(\begin{matrix}\frac{3}{40} & \frac{1}{10} & \frac{1}{10}\\\frac{1}{8} & - \frac{1}{2} & \frac{1}{2}\\- \frac{7}{40} & \frac{1}{10} & \frac{1}{10}\end{matrix}\right)
    \left(\begin{matrix}8\\-4\\-12\end{matrix}\right) = 
    \left(\begin{matrix}-1\\-3\\-3\end{matrix}\right)
\end{align*} \end{split}
\end{equation*}
\sphinxAtStartPar
so the solution is \(x_1 = -1\), \(x_2 = -3\) and \(x_3 = -3\).
\end{sphinxadmonition}
\phantomsection \label{_pages/A2_Linear_systems_exercises_solutions:_pages/A2_Linear_systems_exercises_solutions-solution-1}

\begin{sphinxadmonition}{note}{Solution to Exercise 2.11.2}



\sphinxAtStartPar
(a)
\begin{equation*}
\begin{split} \begin{align*} 
    x_{1} &= \frac{\det\left(\begin{matrix}-20 & 4\\-5 & 1\end{matrix}\right)}{\det\left(\begin{matrix}1 & 4\\-4 & 1\end{matrix}\right)} 
    = \frac{0}{17} = 0, \\ 
    x_{2} &= \frac{\det\left(\begin{matrix}1 & -20\\-4 & -5\end{matrix}\right)}{\det\left(\begin{matrix}1 & 4\\-4 & 1\end{matrix}\right)} 
    = \frac{-85}{17} = -5. 
\end{align*} \end{split}
\end{equation*}
\sphinxAtStartPar
(b)
\begin{equation*}
\begin{split} \begin{align*} 
    x_{1} &= \frac{\det\left(\begin{matrix}4 & 1\\12 & 4\end{matrix}\right)}{\det\left(\begin{matrix}1 & 1\\0 & 4\end{matrix}\right)} 
    = \frac{4}{4} = 1, \\ 
    x_{2} &= \frac{\det\left(\begin{matrix}1 & 4\\0 & 12\end{matrix}\right)}{\det\left(\begin{matrix}1 & 1\\0 & 4\end{matrix}\right)} 
    = \frac{12}{4} = 3. 
\end{align*} \end{split}
\end{equation*}
\sphinxAtStartPar
(c)
\begin{equation*}
\begin{split} \begin{align*} 
    x_{1} &= \frac{\det\left(\begin{matrix}21 & -4 & -4\\8 & -1 & -1\\-14 & -1 & 3\end{matrix}\right)}{\det\left(\begin{matrix}3 & -4 & -4\\-2 & -1 & -1\\4 & -1 & 3\end{matrix}\right)} 
    = \frac{44}{-44} = -1, \\ 
    x_{2} &= \frac{\det\left(\begin{matrix}3 & 21 & -4\\-2 & 8 & -1\\4 & -14 & 3\end{matrix}\right)}{\det\left(\begin{matrix}3 & -4 & -4\\-2 & -1 & -1\\4 & -1 & 3\end{matrix}\right)} 
    = \frac{88}{-44} = -2, \\ 
    x_{3} &= \frac{\det\left(\begin{matrix}3 & -4 & 21\\-2 & -1 & 8\\4 & -1 & -14\end{matrix}\right)}{\det\left(\begin{matrix}3 & -4 & -4\\-2 & -1 & -1\\4 & -1 & 3\end{matrix}\right)} 
    = \frac{176}{-44} = -4. 
\end{align*} \end{split}
\end{equation*}
\sphinxAtStartPar
(d)
\begin{equation*}
\begin{split} \begin{align*} 
    x_{1} &= \frac{\det\left(\begin{matrix}5 & 4 & 1\\-1 & 1 & 1\\-14 & 4 & 2\end{matrix}\right)}{\det\left(\begin{matrix}4 & 4 & 1\\-2 & 1 & 1\\-5 & 4 & 2\end{matrix}\right)} 
    = \frac{-48}{-15} = \frac{16}{5}, \\ 
    x_{2} &= \frac{\det\left(\begin{matrix}4 & 5 & 1\\-2 & -1 & 1\\-5 & -14 & 2\end{matrix}\right)}{\det\left(\begin{matrix}4 & 4 & 1\\-2 & 1 & 1\\-5 & 4 & 2\end{matrix}\right)} 
    = \frac{66}{-15} = - \frac{22}{5}, \\ 
    x_{3} &= \frac{\det\left(\begin{matrix}4 & 4 & 5\\-2 & 1 & -1\\-5 & 4 & -14\end{matrix}\right)}{\det\left(\begin{matrix}4 & 4 & 1\\-2 & 1 & 1\\-5 & 4 & 2\end{matrix}\right)} 
    = \frac{-147}{-15} = \frac{49}{5}. 
\end{align*} \end{split}
\end{equation*}\end{sphinxadmonition}
\phantomsection \label{_pages/A2_Linear_systems_exercises_solutions:_pages/A2_Linear_systems_exercises_solutions-solution-2}

\begin{sphinxadmonition}{note}{Solution to Exercise 2.11.3}



\sphinxAtStartPar
(a)
\begin{equation*}
\begin{split} \begin{align*} 
    & \left( \begin{array}{cc|c} 
         -1 & 3 & -2 \\ 
         -2 & 1 & 1 \\ 
    \end{array} \right) 
    \begin{array}{l} \phantom{x} \\ R_{2} - 2 R_{1} \end{array} & 
    \longrightarrow 
    & \left( \begin{array}{cc|c} 
         -1 & 3 & -2 \\ 
         0 & -5 & 5 \\ 
    \end{array} \right) 
\end{align*} \end{split}
\end{equation*}
\sphinxAtStartPar
Solving by back substitution gives
\begin{equation*}
\begin{split} \begin{align*} 
    x_{2} &=  - \frac{1}{5} \left( 5 \right) = -1, \\ 
    x_{1} &=  - \left( -2 - 3 \left( -1 \right) \right) = -1. 
\end{align*} \end{split}
\end{equation*}
\sphinxAtStartPar
(b)
\begin{equation*}
\begin{split} \begin{align*} 
    & \left( \begin{array}{ccc|c} 
         3 & 1 & 2 & 11 \\ 
         4 & 0 & -4 & -4 \\ 
         4 & -2 & 1 & 13 \\ 
    \end{array} \right) 
    \begin{array}{l} \phantom{x} \\ R_{2} - \frac{4}{3} R_{1} \\ R_{3} - \frac{4}{3} R_{1} \end{array} & 
    \longrightarrow 
    & \left( \begin{array}{ccc|c} 
         3 & 1 & 2 & 11 \\ 
         0 & - \frac{4}{3} & - \frac{20}{3} & - \frac{56}{3} \\ 
         0 & - \frac{10}{3} & - \frac{5}{3} & - \frac{5}{3} \\ 
    \end{array} \right) 
    \begin{array}{l} \phantom{x} \\ \phantom{x} \\ R_{3} - \frac{5}{2} R_{2} \end{array} \\ \\ 
    \longrightarrow 
    & \left( \begin{array}{ccc|c} 
         3 & 1 & 2 & 11 \\ 
         0 & - \frac{4}{3} & - \frac{20}{3} & - \frac{56}{3} \\ 
         0 & 0 & 15 & 45 \\ 
    \end{array} \right) 
\end{align*} \end{split}
\end{equation*}
\sphinxAtStartPar
Solving by back substitution gives
\begin{equation*}
\begin{split} \begin{align*} 
    x_{3} &=  \frac{1}{15} \left( 45 \right) = 3, \\ 
    x_{2} &=  - \frac{1}{\frac{4}{3}} \left( - \frac{56}{3} - \left( - \frac{20}{3} \right) \left( 3 \right) \right) = -1, \\ 
    x_{1} &=  \frac{1}{3} \left( 11 - 1 \left( -1 \right) - 2 \left( 3 \right) \right) = 2. 
\end{align*} \end{split}
\end{equation*}
\sphinxAtStartPar
(c)
\begin{equation*}
\begin{split} \begin{align*} 
    & \left( \begin{array}{ccc|c} 
         -1 & -5 & -2 & -17 \\ 
         2 & -2 & -3 & -14 \\ 
         3 & -1 & 4 & -13 \\ 
    \end{array} \right) 
    \begin{array}{l} \phantom{x} \\ R_{2} + 2 R_{1}\\ R_{3} + 3 R_{1}\end{array} & 
    \longrightarrow 
    & \left( \begin{array}{ccc|c} 
         -1 & -5 & -2 & -17 \\ 
         0 & -12 & -7 & -48 \\ 
         0 & -16 & -2 & -64 \\ 
    \end{array} \right) 
    \begin{array}{l} \phantom{x} \\ \phantom{x} \\ R_{3} - \frac{4}{3} R_{2} \end{array} \\ \\ 
    \longrightarrow 
    & \left( \begin{array}{ccc|c} 
         -1 & -5 & -2 & -17 \\ 
         0 & -12 & -7 & -48 \\ 
         0 & 0 & \frac{22}{3} & 0 \\ 
    \end{array} \right) 
\end{align*} \end{split}
\end{equation*}
\sphinxAtStartPar
Solving by back substitution gives
\begin{equation*}
\begin{split} \begin{align*} 
    x_{3} &=  \frac{1}{\frac{22}{3}} \left( 0 \right) = 0, \\ 
    x_{2} &=  - \frac{1}{12} \left( -48 - \left( -7 \right) \left( 0 \right) \right) = 4, \\ 
    x_{1} &=  - \left( -17 - \left( -5 \right) \left( 4 \right) - \left( -2 \right) \left( 0 \right) \right) = -3. 
\end{align*} \end{split}
\end{equation*}
\sphinxAtStartPar
(d)
\begin{equation*}
\begin{split} \begin{align*} 
    & \left( \begin{array}{ccc|c} 
         -1 & -5 & -2 & -26 \\ 
         2 & -2 & -3 & -19 \\ 
         3 & -1 & -4 & -20 \\ 
    \end{array} \right) 
    \begin{array}{l} \phantom{x} \\ R_{2} + 2 R_{1}\\ R_{3} + 3 R_{1}\end{array} & 
    \longrightarrow 
    & \left( \begin{array}{ccc|c} 
         -1 & -5 & -2 & -26 \\ 
         0 & -12 & -7 & -71 \\ 
         0 & -16 & -10 & -98 \\ 
    \end{array} \right) 
    \begin{array}{l} \phantom{x} \\ \phantom{x} \\ R_{3} - \frac{4}{3} R_{2} \end{array} \\ \\ 
    \longrightarrow 
    & \left( \begin{array}{ccc|c} 
         -1 & -5 & -2 & -26 \\ 
         0 & -12 & -7 & -71 \\ 
         0 & 0 & - \frac{2}{3} & - \frac{10}{3} \\ 
    \end{array} \right) 
\end{align*} \end{split}
\end{equation*}
\sphinxAtStartPar
Solving by back substitution gives
\begin{equation*}
\begin{split} \begin{align*} 
    x_{3} &=  - \frac{1}{\frac{2}{3}} \left( - \frac{10}{3} \right) = 5, \\ 
    x_{2} &=  - \frac{1}{12} \left( -71 - \left( -7 \right) \left( 5 \right) \right) = 3, \\ 
    x_{1} &=  - \left( -26 - \left( -5 \right) \left( 3 \right) - \left( -2 \right) \left( 5 \right) \right) = 1. 
\end{align*} \end{split}
\end{equation*}
\sphinxAtStartPar
(e)
\begin{equation*}
\begin{split} \begin{align*} 
    & \left( \begin{array}{cccc|c} 
         3 & -5 & -4 & -1 & 28 \\ 
         0 & -4 & 3 & -4 & 41 \\ 
         2 & 3 & 3 & -3 & 11 \\ 
         -2 & 2 & -5 & -4 & -21 \\ 
    \end{array} \right) 
    \begin{array}{l} \phantom{x} \\ \phantom{x} \\ R_{3} - \frac{2}{3} R_{1} \\ R_{4} + \frac{2}{3} R_{1}\end{array} \\ \\ 
    \longrightarrow 
    & \left( \begin{array}{cccc|c} 
         3 & -5 & -4 & -1 & 28 \\ 
         0 & -4 & 3 & -4 & 41 \\ 
         0 & \frac{19}{3} & \frac{17}{3} & - \frac{7}{3} & - \frac{23}{3} \\ 
         0 & - \frac{4}{3} & - \frac{23}{3} & - \frac{14}{3} & - \frac{7}{3} \\ 
    \end{array} \right) 
    \begin{array}{l} \phantom{x} \\ \phantom{x} \\ R_{3} + \frac{19}{12} R_{2}\\ R_{4} - \frac{1}{3} R_{2} \end{array} \\ \\ 
    \longrightarrow 
    & \left( \begin{array}{cccc|c} 
         3 & -5 & -4 & -1 & 28 \\ 
         0 & -4 & 3 & -4 & 41 \\ 
         0 & 0 & \frac{125}{12} & - \frac{26}{3} & \frac{229}{4} \\ 
         0 & 0 & - \frac{26}{3} & - \frac{10}{3} & -16 \\ 
    \end{array} \right) 
    \begin{array}{l} \phantom{x} \\ \phantom{x} \\ \phantom{x} \\ R_{4} + \frac{104}{125} R_{3}\end{array} \\ \\ 
    \longrightarrow 
    & \left( \begin{array}{cccc|c} 
         3 & -5 & -4 & -1 & 28 \\ 
         0 & -4 & 3 & -4 & 41 \\ 
         0 & 0 & \frac{125}{12} & - \frac{26}{3} & \frac{229}{4} \\ 
         0 & 0 & 0 & - \frac{1318}{125} & \frac{3954}{125} \\ 
    \end{array} \right) 
\end{align*} \end{split}
\end{equation*}
\sphinxAtStartPar
Solving by back substitution gives
\begin{equation*}
\begin{split} \begin{align*} 
    x_{4} &=  - \frac{1}{\frac{1318}{125}} \left( \frac{3954}{125} \right) = -3, \\ 
    x_{3} &=  \frac{1}{\frac{125}{12}} \left( \frac{229}{4} - \left( - \frac{26}{3} \right) \left( -3 \right) \right) = 3, \\ 
    x_{2} &=  - \frac{1}{4} \left( 41 - 3 \left( 3 \right) - \left( -4 \right) \left( -3 \right) \right) = -5, \\ 
    x_{1} &=  \frac{1}{3} \left( 28 - \left( -5 \right) \left( -5 \right) - \left( -4 \right) \left( 3 \right) - \left( -1 \right) \left( -3 \right) \right) = 4. 
\end{align*} \end{split}
\end{equation*}
\sphinxAtStartPar
(f)
\begin{equation*}
\begin{split} \begin{align*} 
    & \left( \begin{array}{cccc|c} 
         2 & -3 & -3 & 4 & -1 \\ 
         4 & -5 & 1 & -5 & 42 \\ 
         3 & 3 & -1 & -5 & 20 \\ 
         1 & 0 & 1 & 3 & -4 \\ 
    \end{array} \right) 
    \begin{array}{l} \phantom{x} \\ R_{2} - 2 R_{1} \\ R_{3} - \frac{3}{2} R_{1} \\ R_{4} - \frac{1}{2} R_{1} \end{array} \\ \\ 
    \longrightarrow 
    & \left( \begin{array}{cccc|c} 
         2 & -3 & -3 & 4 & -1 \\ 
         0 & 1 & 7 & -13 & 44 \\ 
         0 & \frac{15}{2} & \frac{7}{2} & -11 & \frac{43}{2} \\ 
         0 & \frac{3}{2} & \frac{5}{2} & 1 & - \frac{7}{2} \\ 
    \end{array} \right) 
    \begin{array}{l} \phantom{x} \\ \phantom{x} \\ R_{3} - \frac{15}{2} R_{2} \\ R_{4} - \frac{3}{2} R_{2} \end{array} \\ \\ 
    \longrightarrow 
    & \left( \begin{array}{cccc|c} 
         2 & -3 & -3 & 4 & -1 \\ 
         0 & 1 & 7 & -13 & 44 \\ 
         0 & 0 & -49 & \frac{173}{2} & - \frac{617}{2} \\ 
         0 & 0 & -8 & \frac{41}{2} & - \frac{139}{2} \\ 
    \end{array} \right) 
    \begin{array}{l} \phantom{x} \\ \phantom{x} \\ \phantom{x} \\ R_{4} - \frac{8}{49} R_{3} \end{array} \\ \\ 
    \longrightarrow 
    & \left( \begin{array}{cccc|c} 
         2 & -3 & -3 & 4 & -1 \\ 
         0 & 1 & 7 & -13 & 44 \\ 
         0 & 0 & -49 & \frac{173}{2} & - \frac{617}{2} \\ 
         0 & 0 & 0 & \frac{625}{98} & - \frac{1875}{98} \\ 
    \end{array} \right) 
\end{align*} \end{split}
\end{equation*}
\sphinxAtStartPar
Solving by back substitution gives
\begin{equation*}
\begin{split} \begin{align*} 
    x_{4} &=  \frac{1}{\frac{625}{98}} \left( - \frac{1875}{98} \right) = -3, \\ 
    x_{3} &=  - \frac{1}{49} \left( - \frac{617}{2} - \frac{173}{2} \left( -3 \right) \right) = 1, \\ 
    x_{2} &= 44 - 7 \left( 1 \right) - \left( -13 \right) \left( -3 \right) = -2, \\ 
    x_{1} &=  \frac{1}{2} \left( -1 - \left( -3 \right) \left( -2 \right) - \left( -3 \right) \left( 1 \right) - 4 \left( -3 \right) \right) = 4. 
\end{align*} \end{split}
\end{equation*}\end{sphinxadmonition}
\phantomsection \label{_pages/A2_Linear_systems_exercises_solutions:_pages/A2_Linear_systems_exercises_solutions-solution-3}

\begin{sphinxadmonition}{note}{Solution to Exercise 2.11.4}



\sphinxAtStartPar
(a)
\begin{equation*}
\begin{split} \begin{align*} 
    & \left( \begin{array}{cc|c} 
         -1 & 3 & -2 \\ 
         -2 & 1 & 1 \\ 
    \end{array} \right) 
    \begin{array}{l} \phantom{x} \\ R_{1} \leftrightarrow R_{2} \\ \end{array} & 
    \longrightarrow 
    & \left( \begin{array}{cc|c} 
         -2 & 1 & 1 \\ 
         -1 & 3 & -2 \\ 
    \end{array} \right) 
    \begin{array}{l} \phantom{x} \\ R_{2} - \frac{1}{2} R_{1} \end{array} \\ \\ 
    \longrightarrow 
    & \left( \begin{array}{cc|c} 
         -2 & 1 & 1 \\ 
         0 & \frac{5}{2} & - \frac{5}{2} \\ 
    \end{array} \right) 
\end{align*} \end{split}
\end{equation*}
\sphinxAtStartPar
Solving by back substitution gives
\begin{equation*}
\begin{split} \begin{align*} 
    x_{2} &=  \frac{1}{\frac{5}{2}} \left( - \frac{5}{2} \right) = -1, \\ 
    x_{1} &=  - \frac{1}{2} \left( 1 - 1 \left( -1 \right) \right) = -1. 
\end{align*} \end{split}
\end{equation*}
\sphinxAtStartPar
(b)
\begin{equation*}
\begin{split} \begin{align*} 
    & \left( \begin{array}{ccc|c} 
         3 & 1 & 2 & 11 \\ 
         4 & 0 & -4 & -4 \\ 
         4 & -2 & 1 & 13 \\ 
    \end{array} \right) 
    \begin{array}{l} \phantom{x} \\ R_{1} \leftrightarrow R_{2} \\ \\ \phantom{x} \end{array} & 
    \longrightarrow 
    & \left( \begin{array}{ccc|c} 
         4 & 0 & -4 & -4 \\ 
         3 & 1 & 2 & 11 \\ 
         4 & -2 & 1 & 13 \\ 
    \end{array} \right) 
    \begin{array}{l} \phantom{x} \\ R_{2} - \frac{3}{4} R_{1} \\ R_{3} - R_{1} \end{array} \\ \\ 
    \longrightarrow 
    & \left( \begin{array}{ccc|c} 
         4 & 0 & -4 & -4 \\ 
         0 & 1 & 5 & 14 \\ 
         0 & -2 & 5 & 17 \\ 
    \end{array} \right) 
    \begin{array}{l} \phantom{x} \\ \phantom{x} \\ R_{2} \leftrightarrow R_{3} \\ \end{array} & 
    \longrightarrow 
    & \left( \begin{array}{ccc|c} 
         4 & 0 & -4 & -4 \\ 
         0 & -2 & 5 & 17 \\ 
         0 & 1 & 5 & 14 \\ 
    \end{array} \right) 
    \begin{array}{l} \phantom{x} \\ \phantom{x} \\ R_{3} + \frac{1}{2} R_{2}\end{array} \\ \\ 
    \longrightarrow 
    & \left( \begin{array}{ccc|c} 
         4 & 0 & -4 & -4 \\ 
         0 & -2 & 5 & 17 \\ 
         0 & 0 & \frac{15}{2} & \frac{45}{2} \\ 
    \end{array} \right) 
\end{align*} \end{split}
\end{equation*}
\sphinxAtStartPar
Solving by back substitution gives
\begin{equation*}
\begin{split} \begin{align*} 
    x_{3} &=  \frac{1}{\frac{15}{2}} \left( \frac{45}{2} \right) = 3, \\ 
    x_{2} &=  - \frac{1}{2} \left( 17 - 5 \left( 3 \right) \right) = -1, \\ 
    x_{1} &=  \frac{1}{4} \left( -4 - 0 \left( -1 \right) - \left( -4 \right) \left( 3 \right) \right) = 2. 
\end{align*} \end{split}
\end{equation*}
\sphinxAtStartPar
(c)
\begin{equation*}
\begin{split} \begin{align*} 
    & \left( \begin{array}{ccc|c} 
         -1 & -5 & -2 & -17 \\ 
         2 & -2 & -3 & -14 \\ 
         3 & -1 & 4 & -13 \\ 
    \end{array} \right) 
    \begin{array}{l} \phantom{x} \\ \phantom{x} \\ R_{1} \leftrightarrow R_{3} \\ \end{array} & 
    \longrightarrow 
    & \left( \begin{array}{ccc|c} 
         3 & -1 & 4 & -13 \\ 
         2 & -2 & -3 & -14 \\ 
         -1 & -5 & -2 & -17 \\ 
    \end{array} \right) 
    \begin{array}{l} \phantom{x} \\ R_{2} - \frac{2}{3} R_{1} \\ R_{3} + \frac{1}{3} R_{1}\end{array} \\ \\ 
    \longrightarrow 
    & \left( \begin{array}{ccc|c} 
         3 & -1 & 4 & -13 \\ 
         0 & - \frac{4}{3} & - \frac{17}{3} & - \frac{16}{3} \\ 
         0 & - \frac{16}{3} & - \frac{2}{3} & - \frac{64}{3} \\ 
    \end{array} \right) 
    \begin{array}{l} \phantom{x} \\ \phantom{x} \\ R_{2} \leftrightarrow R_{3} \\ \end{array} & 
    \longrightarrow 
    & \left( \begin{array}{ccc|c} 
         3 & -1 & 4 & -13 \\ 
         0 & - \frac{16}{3} & - \frac{2}{3} & - \frac{64}{3} \\ 
         0 & - \frac{4}{3} & - \frac{17}{3} & - \frac{16}{3} \\ 
    \end{array} \right) 
    \begin{array}{l} \phantom{x} \\ \phantom{x} \\ R_{3} - \frac{1}{4} R_{2} \end{array} \\ \\ 
    \longrightarrow 
    & \left( \begin{array}{ccc|c} 
         3 & -1 & 4 & -13 \\ 
         0 & - \frac{16}{3} & - \frac{2}{3} & - \frac{64}{3} \\ 
         0 & 0 & - \frac{11}{2} & 0 \\ 
    \end{array} \right) 
\end{align*} \end{split}
\end{equation*}
\sphinxAtStartPar
Solving by back substitution gives
\begin{equation*}
\begin{split} \begin{align*} 
    x_{3} &=  - \frac{1}{\frac{11}{2}} \left( 0 \right) = 0, \\ 
    x_{2} &=  - \frac{1}{\frac{16}{3}} \left( - \frac{64}{3} - \left( - \frac{2}{3} \right) \left( 0 \right) \right) = 4, \\ 
    x_{1} &=  \frac{1}{3} \left( -13 - \left( -1 \right) \left( 4 \right) - 4 \left( 0 \right) \right) = -3. 
\end{align*} \end{split}
\end{equation*}
\sphinxAtStartPar
(d)
\begin{equation*}
\begin{split} \begin{align*} 
    & \left( \begin{array}{ccc|c} 
         -1 & -5 & -2 & -26 \\ 
         2 & -2 & -3 & -19 \\ 
         3 & -1 & -4 & -20 \\ 
    \end{array} \right) 
    \begin{array}{l} \phantom{x} \\ \phantom{x} \\ R_{1} \leftrightarrow R_{3} \\ \end{array} & 
    \longrightarrow 
    & \left( \begin{array}{ccc|c} 
         3 & -1 & -4 & -20 \\ 
         2 & -2 & -3 & -19 \\ 
         -1 & -5 & -2 & -26 \\ 
    \end{array} \right) 
    \begin{array}{l} \phantom{x} \\ R_{2} - \frac{2}{3} R_{1} \\ R_{3} + \frac{1}{3} R_{1}\end{array} \\ \\ 
    \longrightarrow 
    & \left( \begin{array}{ccc|c} 
         3 & -1 & -4 & -20 \\ 
         0 & - \frac{4}{3} & - \frac{1}{3} & - \frac{17}{3} \\ 
         0 & - \frac{16}{3} & - \frac{10}{3} & - \frac{98}{3} \\ 
    \end{array} \right) 
    \begin{array}{l} \phantom{x} \\ \phantom{x} \\ R_{2} \leftrightarrow R_{3} \\ \end{array} & 
    \longrightarrow 
    & \left( \begin{array}{ccc|c} 
         3 & -1 & -4 & -20 \\ 
         0 & - \frac{16}{3} & - \frac{10}{3} & - \frac{98}{3} \\ 
         0 & - \frac{4}{3} & - \frac{1}{3} & - \frac{17}{3} \\ 
    \end{array} \right) 
    \begin{array}{l} \phantom{x} \\ \phantom{x} \\ R_{3} - \frac{1}{4} R_{2} \end{array} \\ \\ 
    \longrightarrow 
    & \left( \begin{array}{ccc|c} 
         3 & -1 & -4 & -20 \\ 
         0 & - \frac{16}{3} & - \frac{10}{3} & - \frac{98}{3} \\ 
         0 & 0 & \frac{1}{2} & \frac{5}{2} \\ 
    \end{array} \right) 
\end{align*} \end{split}
\end{equation*}
\sphinxAtStartPar
Solving by back substitution gives
\begin{equation*}
\begin{split} \begin{align*} 
    x_{3} &=  \frac{1}{\frac{1}{2}} \left( \frac{5}{2} \right) = 5, \\ 
    x_{2} &=  - \frac{1}{\frac{16}{3}} \left( - \frac{98}{3} - \left( - \frac{10}{3} \right) \left( 5 \right) \right) = 3, \\ 
    x_{1} &=  \frac{1}{3} \left( -20 - \left( -1 \right) \left( 3 \right) - \left( -4 \right) \left( 5 \right) \right) = 1. 
\end{align*} \end{split}
\end{equation*}
\sphinxAtStartPar
(e)
\begin{equation*}
\begin{split} \begin{align*} 
    & \left( \begin{array}{cccc|c} 
         3 & -5 & -4 & -1 & 28 \\ 
         0 & -4 & 3 & -4 & 41 \\ 
         2 & 3 & 3 & -3 & 11 \\ 
         -2 & 2 & -5 & -4 & -21 \\ 
    \end{array} \right) 
    \begin{array}{l} \phantom{x} \\ \phantom{x} \\ R_{3} - \frac{2}{3} R_{1} \\ R_{4} + \frac{2}{3} R_{1}\end{array} \\ \\ 
    \longrightarrow 
    & \left( \begin{array}{cccc|c} 
         3 & -5 & -4 & -1 & 28 \\ 
         0 & -4 & 3 & -4 & 41 \\ 
         0 & \frac{19}{3} & \frac{17}{3} & - \frac{7}{3} & - \frac{23}{3} \\ 
         0 & - \frac{4}{3} & - \frac{23}{3} & - \frac{14}{3} & - \frac{7}{3} \\ 
    \end{array} \right) 
    \begin{array}{l} \phantom{x} \\ \phantom{x} \\ R_{2} \leftrightarrow R_{3} \\ \\ \phantom{x} \end{array} \\ \\ 
    \longrightarrow 
    & \left( \begin{array}{cccc|c} 
         3 & -5 & -4 & -1 & 28 \\ 
         0 & \frac{19}{3} & \frac{17}{3} & - \frac{7}{3} & - \frac{23}{3} \\ 
         0 & -4 & 3 & -4 & 41 \\ 
         0 & - \frac{4}{3} & - \frac{23}{3} & - \frac{14}{3} & - \frac{7}{3} \\ 
    \end{array} \right) 
    \begin{array}{l} \phantom{x} \\ \phantom{x} \\ R_{3} + \frac{12}{19} R_{2}\\ R_{4} + \frac{4}{19} R_{2}\end{array} \\ \\ 
    \longrightarrow 
    & \left( \begin{array}{cccc|c} 
         3 & -5 & -4 & -1 & 28 \\ 
         0 & \frac{19}{3} & \frac{17}{3} & - \frac{7}{3} & - \frac{23}{3} \\ 
         0 & 0 & \frac{125}{19} & - \frac{104}{19} & \frac{687}{19} \\ 
         0 & 0 & - \frac{123}{19} & - \frac{98}{19} & - \frac{75}{19} \\ 
    \end{array} \right) 
    \begin{array}{l} \phantom{x} \\ \phantom{x} \\ \phantom{x} \\ R_{4} + \frac{123}{125} R_{3}\end{array} \\ \\ 
    \longrightarrow 
    & \left( \begin{array}{cccc|c} 
         3 & -5 & -4 & -1 & 28 \\ 
         0 & \frac{19}{3} & \frac{17}{3} & - \frac{7}{3} & - \frac{23}{3} \\ 
         0 & 0 & \frac{125}{19} & - \frac{104}{19} & \frac{687}{19} \\ 
         0 & 0 & 0 & - \frac{1318}{125} & \frac{3954}{125} \\ 
    \end{array} \right) 
\end{align*} \end{split}
\end{equation*}
\sphinxAtStartPar
Solving by back substitution gives
\begin{equation*}
\begin{split} \begin{align*} 
    x_{4} &=  - \frac{1}{\frac{1318}{125}} \left( \frac{3954}{125} \right) = -3, \\ 
    x_{3} &=  \frac{1}{\frac{125}{19}} \left( \frac{687}{19} - \left( - \frac{104}{19} \right) \left( -3 \right) \right) = 3, \\ 
    x_{2} &=  \frac{1}{\frac{19}{3}} \left( - \frac{23}{3} - \frac{17}{3} \left( 3 \right) - \left( - \frac{7}{3} \right) \left( -3 \right) \right) = -5, \\ 
    x_{1} &=  \frac{1}{3} \left( 28 - \left( -5 \right) \left( -5 \right) - \left( -4 \right) \left( 3 \right) - \left( -1 \right) \left( -3 \right) \right) = 4. 
\end{align*} \end{split}
\end{equation*}
\sphinxAtStartPar
(f)
\begin{equation*}
\begin{split} \begin{align*} 
    & \left( \begin{array}{cccc|c} 
         2 & -3 & -3 & 4 & -1 \\ 
         4 & -5 & 1 & -5 & 42 \\ 
         3 & 3 & -1 & -5 & 20 \\ 
         1 & 0 & 1 & 3 & -4 \\ 
    \end{array} \right) 
    \begin{array}{l} \phantom{x} \\ R_{1} \leftrightarrow R_{2} \\ \\ \phantom{x} \\ \phantom{x} \end{array} \\ \\ 
    \longrightarrow 
    & \left( \begin{array}{cccc|c} 
         4 & -5 & 1 & -5 & 42 \\ 
         2 & -3 & -3 & 4 & -1 \\ 
         3 & 3 & -1 & -5 & 20 \\ 
         1 & 0 & 1 & 3 & -4 \\ 
    \end{array} \right) 
    \begin{array}{l} \phantom{x} \\ R_{2} - \frac{1}{2} R_{1} \\ R_{3} - \frac{3}{4} R_{1} \\ R_{4} - \frac{1}{4} R_{1} \end{array} \\ \\ 
    \longrightarrow 
    & \left( \begin{array}{cccc|c} 
         4 & -5 & 1 & -5 & 42 \\ 
         0 & - \frac{1}{2} & - \frac{7}{2} & \frac{13}{2} & -22 \\ 
         0 & \frac{27}{4} & - \frac{7}{4} & - \frac{5}{4} & - \frac{23}{2} \\ 
         0 & \frac{5}{4} & \frac{3}{4} & \frac{17}{4} & - \frac{29}{2} \\ 
    \end{array} \right) 
    \begin{array}{l} \phantom{x} \\ \phantom{x} \\ R_{2} \leftrightarrow R_{3} \\ \\ \phantom{x} \end{array} \\ \\ 
    \longrightarrow 
    & \left( \begin{array}{cccc|c} 
         4 & -5 & 1 & -5 & 42 \\ 
         0 & \frac{27}{4} & - \frac{7}{4} & - \frac{5}{4} & - \frac{23}{2} \\ 
         0 & - \frac{1}{2} & - \frac{7}{2} & \frac{13}{2} & -22 \\ 
         0 & \frac{5}{4} & \frac{3}{4} & \frac{17}{4} & - \frac{29}{2} \\ 
    \end{array} \right) 
    \begin{array}{l} \phantom{x} \\ \phantom{x} \\ R_{3} + \frac{2}{27} R_{2}\\ R_{4} - \frac{5}{27} R_{2} \end{array} \\ \\ 
    \longrightarrow 
    & \left( \begin{array}{cccc|c} 
         4 & -5 & 1 & -5 & 42 \\ 
         0 & \frac{27}{4} & - \frac{7}{4} & - \frac{5}{4} & - \frac{23}{2} \\ 
         0 & 0 & - \frac{98}{27} & \frac{173}{27} & - \frac{617}{27} \\ 
         0 & 0 & \frac{29}{27} & \frac{121}{27} & - \frac{334}{27} \\ 
    \end{array} \right) 
    \begin{array}{l} \phantom{x} \\ \phantom{x} \\ \phantom{x} \\ R_{4} + \frac{29}{98} R_{3}\end{array} \\ \\ 
    \longrightarrow 
    & \left( \begin{array}{cccc|c} 
         4 & -5 & 1 & -5 & 42 \\ 
         0 & \frac{27}{4} & - \frac{7}{4} & - \frac{5}{4} & - \frac{23}{2} \\ 
         0 & 0 & - \frac{98}{27} & \frac{173}{27} & - \frac{617}{27} \\ 
         0 & 0 & 0 & \frac{625}{98} & - \frac{1875}{98} \\ 
    \end{array} \right) 
\end{align*} \end{split}
\end{equation*}
\sphinxAtStartPar
Solving by back substitution gives
\begin{equation*}
\begin{split} \begin{align*} 
    x_{4} &=  \frac{1}{\frac{625}{98}} \left( - \frac{1875}{98} \right) = -3, \\ 
    x_{3} &=  - \frac{1}{\frac{98}{27}} \left( - \frac{617}{27} - \frac{173}{27} \left( -3 \right) \right) = 1, \\ 
    x_{2} &=  \frac{1}{\frac{27}{4}} \left( - \frac{23}{2} - \left( - \frac{7}{4} \right) \left( 1 \right) - \left( - \frac{5}{4} \right) \left( -3 \right) \right) = -2, \\ 
    x_{1} &=  \frac{1}{4} \left( 42 - \left( -5 \right) \left( -2 \right) - 1 \left( 1 \right) - \left( -5 \right) \left( -3 \right) \right) = 4. 
\end{align*} \end{split}
\end{equation*}\end{sphinxadmonition}
\phantomsection \label{_pages/A2_Linear_systems_exercises_solutions:_pages/A2_Linear_systems_exercises_solutions-solution-4}

\begin{sphinxadmonition}{note}{Solution to Exercise 2.11.5}



\sphinxAtStartPar
(a)
\begin{equation*}
\begin{split} \begin{align*} 
    & \left( \begin{array}{cc|c} 
         -1 & 3 & -2 \\ 
         -2 & 1 & 1 \\ 
    \end{array} \right) 
    \begin{array}{l} -1 R_{1}\\ \phantom{x} \end{array} & 
    \longrightarrow 
    & \left( \begin{array}{cc|c} 
         1 & -3 & 2 \\ 
         -2 & 1 & 1 \\ 
    \end{array} \right) 
    \begin{array}{l} \phantom{x} \\ R_{2} + 2 R_{1}\end{array} \\ \\ 
    \longrightarrow 
    & \left( \begin{array}{cc|c} 
         1 & -3 & 2 \\ 
         0 & -5 & 5 \\ 
    \end{array} \right) 
    \begin{array}{l} \phantom{x} \\ - \frac{1}{5} R_{2}\end{array} & 
    \longrightarrow 
    & \left( \begin{array}{cc|c} 
         1 & -3 & 2 \\ 
         0 & 1 & -1 \\ 
    \end{array} \right) 
    \begin{array}{l} R_{1} + 3 R_{2}\\ \phantom{x} \end{array} \\ \\ 
    \longrightarrow 
    & \left( \begin{array}{cc|c} 
         1 & 0 & -1 \\ 
         0 & 1 & -1 \\ 
    \end{array} \right) 
\end{align*} \end{split}
\end{equation*}
\sphinxAtStartPar
So the solution is \(x_{1} = -1\) and \(x_{2} = -1\).

\sphinxAtStartPar
(b)
\begin{equation*}
\begin{split} \begin{align*} 
    & \left( \begin{array}{ccc|c} 
         3 & 1 & 2 & 11 \\ 
         4 & 0 & -4 & -4 \\ 
         4 & -2 & 1 & 13 \\ 
    \end{array} \right) 
    \begin{array}{l} \frac{1}{3} R_{1}\\ \phantom{x} \\ \phantom{x} \end{array} & 
    \longrightarrow 
    & \left( \begin{array}{ccc|c} 
         1 & \frac{1}{3} & \frac{2}{3} & \frac{11}{3} \\ 
         4 & 0 & -4 & -4 \\ 
         4 & -2 & 1 & 13 \\ 
    \end{array} \right) 
    \begin{array}{l} \phantom{x} \\ R_{2} - 4 R_{1} \\ R_{3} - 4 R_{1} \end{array} \\ \\ 
    \longrightarrow 
    & \left( \begin{array}{ccc|c} 
         1 & \frac{1}{3} & \frac{2}{3} & \frac{11}{3} \\ 
         0 & - \frac{4}{3} & - \frac{20}{3} & - \frac{56}{3} \\ 
         0 & - \frac{10}{3} & - \frac{5}{3} & - \frac{5}{3} \\ 
    \end{array} \right) 
    \begin{array}{l} \phantom{x} \\ - \frac{3}{4} R_{2}\\ \phantom{x} \end{array} & 
    \longrightarrow 
    & \left( \begin{array}{ccc|c} 
         1 & \frac{1}{3} & \frac{2}{3} & \frac{11}{3} \\ 
         0 & 1 & 5 & 14 \\ 
         0 & - \frac{10}{3} & - \frac{5}{3} & - \frac{5}{3} \\ 
    \end{array} \right) 
    \begin{array}{l} R_{1} - \frac{1}{3} R_{2} \\ \phantom{x} \\ R_{3} + \frac{10}{3} R_{2}\end{array} \\ \\ 
    \longrightarrow 
    & \left( \begin{array}{ccc|c} 
         1 & 0 & -1 & -1 \\ 
         0 & 1 & 5 & 14 \\ 
         0 & 0 & 15 & 45 \\ 
    \end{array} \right) 
    \begin{array}{l} \phantom{x} \\ \phantom{x} \\ \frac{1}{15} R_{3}\end{array} & 
    \longrightarrow 
    & \left( \begin{array}{ccc|c} 
         1 & 0 & -1 & -1 \\ 
         0 & 1 & 5 & 14 \\ 
         0 & 0 & 1 & 3 \\ 
    \end{array} \right) 
    \begin{array}{l} R_{1} + R_{3} \\ R_{2} - 5 R_{3} \\ \phantom{x} \end{array} \\ \\ 
    \longrightarrow 
    & \left( \begin{array}{ccc|c} 
         1 & 0 & 0 & 2 \\ 
         0 & 1 & 0 & -1 \\ 
         0 & 0 & 1 & 3 \\ 
    \end{array} \right) 
\end{align*} \end{split}
\end{equation*}
\sphinxAtStartPar
So the solution is \(x_{1} = 2\), \(x_{2} = -1\) and \(x_{3} = 3\).

\sphinxAtStartPar
(c)
\begin{equation*}
\begin{split} \begin{align*} 
    & \left( \begin{array}{ccc|c} 
         -1 & -5 & -2 & -17 \\ 
         2 & -2 & -3 & -14 \\ 
         3 & -1 & 4 & -13 \\ 
    \end{array} \right) 
    \begin{array}{l} -1 R_{1}\\ \phantom{x} \\ \phantom{x} \end{array} & 
    \longrightarrow 
    & \left( \begin{array}{ccc|c} 
         1 & 5 & 2 & 17 \\ 
         2 & -2 & -3 & -14 \\ 
         3 & -1 & 4 & -13 \\ 
    \end{array} \right) 
    \begin{array}{l} \phantom{x} \\ R_{2} - 2 R_{1} \\ R_{3} - 3 R_{1} \end{array} \\ \\ 
    \longrightarrow 
    & \left( \begin{array}{ccc|c} 
         1 & 5 & 2 & 17 \\ 
         0 & -12 & -7 & -48 \\ 
         0 & -16 & -2 & -64 \\ 
    \end{array} \right) 
    \begin{array}{l} \phantom{x} \\ - \frac{1}{12} R_{2}\\ \phantom{x} \end{array} & 
    \longrightarrow 
    & \left( \begin{array}{ccc|c} 
         1 & 5 & 2 & 17 \\ 
         0 & 1 & \frac{7}{12} & 4 \\ 
         0 & -16 & -2 & -64 \\ 
    \end{array} \right) 
    \begin{array}{l} R_{1} - 5 R_{2} \\ \phantom{x} \\ R_{3} + 16 R_{2}\end{array} \\ \\ 
    \longrightarrow 
    & \left( \begin{array}{ccc|c} 
         1 & 0 & - \frac{11}{12} & -3 \\ 
         0 & 1 & \frac{7}{12} & 4 \\ 
         0 & 0 & \frac{22}{3} & 0 \\ 
    \end{array} \right) 
    \begin{array}{l} \phantom{x} \\ \phantom{x} \\ \frac{3}{22} R_{3}\end{array} & 
    \longrightarrow 
    & \left( \begin{array}{ccc|c} 
         1 & 0 & - \frac{11}{12} & -3 \\ 
         0 & 1 & \frac{7}{12} & 4 \\ 
         0 & 0 & 1 & 0 \\ 
    \end{array} \right) 
    \begin{array}{l} R_{1} + \frac{11}{12} R_{3}\\ R_{2} - \frac{7}{12} R_{3} \\ \phantom{x} \end{array} \\ \\ 
    \longrightarrow 
    & \left( \begin{array}{ccc|c} 
         1 & 0 & 0 & -3 \\ 
         0 & 1 & 0 & 4 \\ 
         0 & 0 & 1 & 0 \\ 
    \end{array} \right) 
\end{align*} \end{split}
\end{equation*}
\sphinxAtStartPar
So the solution is \(x_{1} = -3\), \(x_{2} = 4\) and \(x_{3} = 0\).

\sphinxAtStartPar
(d)
\begin{equation*}
\begin{split} \begin{align*} 
    & \left( \begin{array}{ccc|c} 
         -1 & -5 & -2 & -26 \\ 
         2 & -2 & -3 & -19 \\ 
         3 & -1 & -4 & -20 \\ 
    \end{array} \right) 
    \begin{array}{l} -1 R_{1}\\ \phantom{x} \\ \phantom{x} \end{array} & 
    \longrightarrow 
    & \left( \begin{array}{ccc|c} 
         1 & 5 & 2 & 26 \\ 
         2 & -2 & -3 & -19 \\ 
         3 & -1 & -4 & -20 \\ 
    \end{array} \right) 
    \begin{array}{l} \phantom{x} \\ R_{2} - 2 R_{1} \\ R_{3} - 3 R_{1} \end{array} \\ \\ 
    \longrightarrow 
    & \left( \begin{array}{ccc|c} 
         1 & 5 & 2 & 26 \\ 
         0 & -12 & -7 & -71 \\ 
         0 & -16 & -10 & -98 \\ 
    \end{array} \right) 
    \begin{array}{l} \phantom{x} \\ - \frac{1}{12} R_{2}\\ \phantom{x} \end{array} & 
    \longrightarrow 
    & \left( \begin{array}{ccc|c} 
         1 & 5 & 2 & 26 \\ 
         0 & 1 & \frac{7}{12} & \frac{71}{12} \\ 
         0 & -16 & -10 & -98 \\ 
    \end{array} \right) 
    \begin{array}{l} R_{1} - 5 R_{2} \\ \phantom{x} \\ R_{3} + 16 R_{2}\end{array} \\ \\ 
    \longrightarrow 
    & \left( \begin{array}{ccc|c} 
         1 & 0 & - \frac{11}{12} & - \frac{43}{12} \\ 
         0 & 1 & \frac{7}{12} & \frac{71}{12} \\ 
         0 & 0 & - \frac{2}{3} & - \frac{10}{3} \\ 
    \end{array} \right) 
    \begin{array}{l} \phantom{x} \\ \phantom{x} \\ - \frac{3}{2} R_{3}\end{array} & 
    \longrightarrow 
    & \left( \begin{array}{ccc|c} 
         1 & 0 & - \frac{11}{12} & - \frac{43}{12} \\ 
         0 & 1 & \frac{7}{12} & \frac{71}{12} \\ 
         0 & 0 & 1 & 5 \\ 
    \end{array} \right) 
    \begin{array}{l} R_{1} + \frac{11}{12} R_{3}\\ R_{2} - \frac{7}{12} R_{3} \\ \phantom{x} \end{array} \\ \\ 
    \longrightarrow 
    & \left( \begin{array}{ccc|c} 
         1 & 0 & 0 & 1 \\ 
         0 & 1 & 0 & 3 \\ 
         0 & 0 & 1 & 5 \\ 
    \end{array} \right) 
\end{align*} \end{split}
\end{equation*}
\sphinxAtStartPar
So the solution is \(x_{1} = 1\), \(x_{2} = 3\) and \(x_{3} = 5\).

\sphinxAtStartPar
(e)
\begin{equation*}
\begin{split} \begin{align*} 
    & \left( \begin{array}{cccc|c} 
         3 & -5 & -4 & -1 & 28 \\ 
         0 & -4 & 3 & -4 & 41 \\ 
         2 & 3 & 3 & -3 & 11 \\ 
         -2 & 2 & -5 & -4 & -21 \\ 
    \end{array} \right) 
    \begin{array}{l} \frac{1}{3} R_{1}\\ \phantom{x} \\ \phantom{x} \\ \phantom{x} \end{array} \\ \\ 
    \longrightarrow 
    & \left( \begin{array}{cccc|c} 
         1 & - \frac{5}{3} & - \frac{4}{3} & - \frac{1}{3} & \frac{28}{3} \\ 
         0 & -4 & 3 & -4 & 41 \\ 
         2 & 3 & 3 & -3 & 11 \\ 
         -2 & 2 & -5 & -4 & -21 \\ 
    \end{array} \right) 
    \begin{array}{l} \phantom{x} \\ \phantom{x} \\ R_{3} - 2 R_{1} \\ R_{4} + 2 R_{1}\end{array} \\ \\ 
    \longrightarrow 
    & \left( \begin{array}{cccc|c} 
         1 & - \frac{5}{3} & - \frac{4}{3} & - \frac{1}{3} & \frac{28}{3} \\ 
         0 & -4 & 3 & -4 & 41 \\ 
         0 & \frac{19}{3} & \frac{17}{3} & - \frac{7}{3} & - \frac{23}{3} \\ 
         0 & - \frac{4}{3} & - \frac{23}{3} & - \frac{14}{3} & - \frac{7}{3} \\ 
    \end{array} \right) 
    \begin{array}{l} \phantom{x} \\ - \frac{1}{4} R_{2}\\ \phantom{x} \\ \phantom{x} \end{array} \\ \\ 
    \longrightarrow 
    & \left( \begin{array}{cccc|c} 
         1 & - \frac{5}{3} & - \frac{4}{3} & - \frac{1}{3} & \frac{28}{3} \\ 
         0 & 1 & - \frac{3}{4} & 1 & - \frac{41}{4} \\ 
         0 & \frac{19}{3} & \frac{17}{3} & - \frac{7}{3} & - \frac{23}{3} \\ 
         0 & - \frac{4}{3} & - \frac{23}{3} & - \frac{14}{3} & - \frac{7}{3} \\ 
    \end{array} \right) 
    \begin{array}{l} R_{1} + \frac{5}{3} R_{2}\\ \phantom{x} \\ R_{3} - \frac{19}{3} R_{2} \\ R_{4} + \frac{4}{3} R_{2}\end{array} \\ \\ 
    \longrightarrow 
    & \left( \begin{array}{cccc|c} 
         1 & 0 & - \frac{31}{12} & \frac{4}{3} & - \frac{31}{4} \\ 
         0 & 1 & - \frac{3}{4} & 1 & - \frac{41}{4} \\ 
         0 & 0 & \frac{125}{12} & - \frac{26}{3} & \frac{229}{4} \\ 
         0 & 0 & - \frac{26}{3} & - \frac{10}{3} & -16 \\ 
    \end{array} \right) 
    \begin{array}{l} \phantom{x} \\ \phantom{x} \\ \frac{12}{125} R_{3}\\ \phantom{x} \end{array} \\ \\ 
    \longrightarrow 
    & \left( \begin{array}{cccc|c} 
         1 & 0 & - \frac{31}{12} & \frac{4}{3} & - \frac{31}{4} \\ 
         0 & 1 & - \frac{3}{4} & 1 & - \frac{41}{4} \\ 
         0 & 0 & 1 & - \frac{104}{125} & \frac{687}{125} \\ 
         0 & 0 & - \frac{26}{3} & - \frac{10}{3} & -16 \\ 
    \end{array} \right) 
    \begin{array}{l} R_{1} + \frac{31}{12} R_{3}\\ R_{2} + \frac{3}{4} R_{3}\\ \phantom{x} \\ R_{4} + \frac{26}{3} R_{3}\end{array} \\ \\ 
    \longrightarrow 
    & \left( \begin{array}{cccc|c} 
         1 & 0 & 0 & - \frac{102}{125} & \frac{806}{125} \\ 
         0 & 1 & 0 & \frac{47}{125} & - \frac{766}{125} \\ 
         0 & 0 & 1 & - \frac{104}{125} & \frac{687}{125} \\ 
         0 & 0 & 0 & - \frac{1318}{125} & \frac{3954}{125} \\ 
    \end{array} \right) 
    \begin{array}{l} \phantom{x} \\ \phantom{x} \\ \phantom{x} \\ - \frac{125}{1318} R_{4}\end{array} \\ \\ 
    \longrightarrow 
    & \left( \begin{array}{cccc|c} 
         1 & 0 & 0 & - \frac{102}{125} & \frac{806}{125} \\ 
         0 & 1 & 0 & \frac{47}{125} & - \frac{766}{125} \\ 
         0 & 0 & 1 & - \frac{104}{125} & \frac{687}{125} \\ 
         0 & 0 & 0 & 1 & -3 \\ 
    \end{array} \right) 
    \begin{array}{l} R_{1} + \frac{102}{125} R_{4}\\ R_{2} - \frac{47}{125} R_{4} \\ R_{3} + \frac{104}{125} R_{4}\\ \phantom{x} \end{array} \\ \\ 
    \longrightarrow 
    & \left( \begin{array}{cccc|c} 
         1 & 0 & 0 & 0 & 4 \\ 
         0 & 1 & 0 & 0 & -5 \\ 
         0 & 0 & 1 & 0 & 3 \\ 
         0 & 0 & 0 & 1 & -3 \\ 
    \end{array} \right) 
\end{align*} \end{split}
\end{equation*}
\sphinxAtStartPar
So the solution is \(x_{1} = 4\), \(x_{2} = -5\), \(x_{3} = 3\) and \(x_{4} = -3\).

\sphinxAtStartPar
(f)
\begin{equation*}
\begin{split} \begin{align*} 
    & \left( \begin{array}{cccc|c} 
         2 & -3 & -3 & 4 & -1 \\ 
         4 & -5 & 1 & -5 & 42 \\ 
         3 & 3 & -1 & -5 & 20 \\ 
         1 & 0 & 1 & 3 & -4 \\ 
    \end{array} \right) 
    \begin{array}{l} \frac{1}{2} R_{1}\\ \phantom{x} \\ \phantom{x} \\ \phantom{x} \end{array} \\ \\ 
    \longrightarrow 
    & \left( \begin{array}{cccc|c} 
         1 & - \frac{3}{2} & - \frac{3}{2} & 2 & - \frac{1}{2} \\ 
         4 & -5 & 1 & -5 & 42 \\ 
         3 & 3 & -1 & -5 & 20 \\ 
         1 & 0 & 1 & 3 & -4 \\ 
    \end{array} \right) 
    \begin{array}{l} \phantom{x} \\ R_{2} - 4 R_{1} \\ R_{3} - 3 R_{1} \\ R_{4} - R_{1} \end{array} \\ \\ 
    \longrightarrow 
    & \left( \begin{array}{cccc|c} 
         1 & - \frac{3}{2} & - \frac{3}{2} & 2 & - \frac{1}{2} \\ 
         0 & 1 & 7 & -13 & 44 \\ 
         0 & \frac{15}{2} & \frac{7}{2} & -11 & \frac{43}{2} \\ 
         0 & \frac{3}{2} & \frac{5}{2} & 1 & - \frac{7}{2} \\ 
    \end{array} \right) 
    \begin{array}{l} R_{1} + \frac{3}{2} R_{2}\\ \phantom{x} \\ R_{3} - \frac{15}{2} R_{2} \\ R_{4} - \frac{3}{2} R_{2} \end{array} \\ \\ 
    \longrightarrow 
    & \left( \begin{array}{cccc|c} 
         1 & 0 & 9 & - \frac{35}{2} & \frac{131}{2} \\ 
         0 & 1 & 7 & -13 & 44 \\ 
         0 & 0 & -49 & \frac{173}{2} & - \frac{617}{2} \\ 
         0 & 0 & -8 & \frac{41}{2} & - \frac{139}{2} \\ 
    \end{array} \right) 
    \begin{array}{l} \phantom{x} \\ \phantom{x} \\ - \frac{1}{49} R_{3}\\ \phantom{x} \end{array} \\ \\ 
    \longrightarrow 
    & \left( \begin{array}{cccc|c} 
         1 & 0 & 9 & - \frac{35}{2} & \frac{131}{2} \\ 
         0 & 1 & 7 & -13 & 44 \\ 
         0 & 0 & 1 & - \frac{173}{98} & \frac{617}{98} \\ 
         0 & 0 & -8 & \frac{41}{2} & - \frac{139}{2} \\ 
    \end{array} \right) 
    \begin{array}{l} R_{1} - 9 R_{3} \\ R_{2} - 7 R_{3} \\ \phantom{x} \\ R_{4} + 8 R_{3}\end{array} \\ \\ 
    \longrightarrow 
    & \left( \begin{array}{cccc|c} 
         1 & 0 & 0 & - \frac{79}{49} & \frac{433}{49} \\ 
         0 & 1 & 0 & - \frac{9}{14} & - \frac{1}{14} \\ 
         0 & 0 & 1 & - \frac{173}{98} & \frac{617}{98} \\ 
         0 & 0 & 0 & \frac{625}{98} & - \frac{1875}{98} \\ 
    \end{array} \right) 
    \begin{array}{l} \phantom{x} \\ \phantom{x} \\ \phantom{x} \\ \frac{98}{625} R_{4}\end{array} \\ \\ 
    \longrightarrow 
    & \left( \begin{array}{cccc|c} 
         1 & 0 & 0 & - \frac{79}{49} & \frac{433}{49} \\ 
         0 & 1 & 0 & - \frac{9}{14} & - \frac{1}{14} \\ 
         0 & 0 & 1 & - \frac{173}{98} & \frac{617}{98} \\ 
         0 & 0 & 0 & 1 & -3 \\ 
    \end{array} \right) 
    \begin{array}{l} R_{1} + \frac{79}{49} R_{4}\\ R_{2} + \frac{9}{14} R_{4}\\ R_{3} + \frac{173}{98} R_{4}\\ \phantom{x} \end{array} \\ \\ 
    \longrightarrow 
    & \left( \begin{array}{cccc|c} 
         1 & 0 & 0 & 0 & 4 \\ 
         0 & 1 & 0 & 0 & -2 \\ 
         0 & 0 & 1 & 0 & 1 \\ 
         0 & 0 & 0 & 1 & -3 \\ 
    \end{array} \right) 
\end{align*} \end{split}
\end{equation*}
\sphinxAtStartPar
So the solution is \(x_{1} = 4\), \(x_{2} = -2\), \(x_{3} = 1\) and \(x_{4} = -3\).
\end{sphinxadmonition}
\phantomsection \label{_pages/A2_Linear_systems_exercises_solutions:_pages/A2_Linear_systems_exercises_solutions-solution-5}

\begin{sphinxadmonition}{note}{Solution to Exercise 2.11.6}



\sphinxAtStartPar
(a)
\begin{equation*}
\begin{split} \begin{align*} 
    & \left( \begin{array}{cc|cc} 
         -4 & 2 & 1 & 0 \\ 
         3 & 4 & 0 & 1 \\ 
    \end{array} \right) 
    \begin{array}{l} - \frac{1}{4} R_{1}\\ \phantom{x} \end{array} & 
    \longrightarrow 
    & \left( \begin{array}{cc|cc} 
         1 & - \frac{1}{2} & - \frac{1}{4} & 0 \\ 
         3 & 4 & 0 & 1 \\ 
    \end{array} \right) 
    \begin{array}{l} \phantom{x} \\ R_{2} - 3 R_{1} \end{array} \\ \\ 
    \longrightarrow 
    & \left( \begin{array}{cc|cc} 
         1 & - \frac{1}{2} & - \frac{1}{4} & 0 \\ 
         0 & \frac{11}{2} & \frac{3}{4} & 1 \\ 
    \end{array} \right) 
    \begin{array}{l} \phantom{x} \\ \frac{2}{11} R_{2}\end{array} & 
    \longrightarrow 
    & \left( \begin{array}{cc|cc} 
         1 & - \frac{1}{2} & - \frac{1}{4} & 0 \\ 
         0 & 1 & \frac{3}{22} & \frac{2}{11} \\ 
    \end{array} \right) 
    \begin{array}{l} R_{1} + \frac{1}{2} R_{2}\\ \phantom{x} \end{array} \\ \\ 
    \longrightarrow 
    & \left( \begin{array}{cc|cc} 
         1 & 0 & - \frac{2}{11} & \frac{1}{11} \\ 
         0 & 1 & \frac{3}{22} & \frac{2}{11} \\ 
    \end{array} \right) 
\end{align*} \end{split}
\end{equation*}
\sphinxAtStartPar
So \(A^{-1} = \left(\begin{matrix}- \frac{2}{11} & \frac{1}{11}\\\frac{3}{22} & \frac{2}{11}\end{matrix}\right)\). Checking this is correct
\begin{equation*}
\begin{split} \begin{align*} 
    \left(\begin{matrix}- \frac{2}{11} & \frac{1}{11}\\\frac{3}{22} & \frac{2}{11}\end{matrix}\right)\left(\begin{matrix}-4 & 2\\3 & 4\end{matrix}\right) = \left(\begin{matrix}1 & 0\\0 & 1\end{matrix}\right) \quad \checkmark\end{align*} \end{split}
\end{equation*}
\sphinxAtStartPar
(b)
\begin{equation*}
\begin{split} \begin{align*} 
    & \left( \begin{array}{cc|cc} 
         -4 & 2 & 1 & 0 \\ 
         -1 & -3 & 0 & 1 \\ 
    \end{array} \right) 
    \begin{array}{l} - \frac{1}{4} R_{1}\\ \phantom{x} \end{array} & 
    \longrightarrow 
    & \left( \begin{array}{cc|cc} 
         1 & - \frac{1}{2} & - \frac{1}{4} & 0 \\ 
         -1 & -3 & 0 & 1 \\ 
    \end{array} \right) 
    \begin{array}{l} \phantom{x} \\ R_{2} + R_{1} \end{array} \\ \\ 
    \longrightarrow 
    & \left( \begin{array}{cc|cc} 
         1 & - \frac{1}{2} & - \frac{1}{4} & 0 \\ 
         0 & - \frac{7}{2} & - \frac{1}{4} & 1 \\ 
    \end{array} \right) 
    \begin{array}{l} \phantom{x} \\ - \frac{2}{7} R_{2}\end{array} & 
    \longrightarrow 
    & \left( \begin{array}{cc|cc} 
         1 & - \frac{1}{2} & - \frac{1}{4} & 0 \\ 
         0 & 1 & \frac{1}{14} & - \frac{2}{7} \\ 
    \end{array} \right) 
    \begin{array}{l} R_{1} + \frac{1}{2} R_{2}\\ \phantom{x} \end{array} \\ \\ 
    \longrightarrow 
    & \left( \begin{array}{cc|cc} 
         1 & 0 & - \frac{3}{14} & - \frac{1}{7} \\ 
         0 & 1 & \frac{1}{14} & - \frac{2}{7} \\ 
    \end{array} \right) 
\end{align*} \end{split}
\end{equation*}
\sphinxAtStartPar
So \(A^{-1} = \left(\begin{matrix}- \frac{3}{14} & - \frac{1}{7}\\\frac{1}{14} & - \frac{2}{7}\end{matrix}\right)\). Checking this is correct
\begin{equation*}
\begin{split} \begin{align*} 
    \left(\begin{matrix}- \frac{3}{14} & - \frac{1}{7}\\\frac{1}{14} & - \frac{2}{7}\end{matrix}\right)\left(\begin{matrix}-4 & 2\\-1 & -3\end{matrix}\right) = \left(\begin{matrix}1 & 0\\0 & 1\end{matrix}\right) \quad \checkmark\end{align*} \end{split}
\end{equation*}
\sphinxAtStartPar
(c)
\begin{equation*}
\begin{split} \begin{align*} 
    & \left( \begin{array}{ccc|ccc} 
         -4 & -4 & -2 & 1 & 0 & 0 \\ 
         3 & 0 & 4 & 0 & 1 & 0 \\ 
         1 & 0 & 0 & 0 & 0 & 1 \\ 
    \end{array} \right) 
    \begin{array}{l} - \frac{1}{4} R_{1}\\ \phantom{x} \\ \phantom{x} \end{array} \\ \\ 
    \longrightarrow 
    & \left( \begin{array}{ccc|ccc} 
         1 & 1 & \frac{1}{2} & - \frac{1}{4} & 0 & 0 \\ 
         3 & 0 & 4 & 0 & 1 & 0 \\ 
         1 & 0 & 0 & 0 & 0 & 1 \\ 
    \end{array} \right) 
    \begin{array}{l} \phantom{x} \\ R_{2} - 3 R_{1} \\ R_{3} - R_{1} \end{array} \\ \\ 
    \longrightarrow 
    & \left( \begin{array}{ccc|ccc} 
         1 & 1 & \frac{1}{2} & - \frac{1}{4} & 0 & 0 \\ 
         0 & -3 & \frac{5}{2} & \frac{3}{4} & 1 & 0 \\ 
         0 & -1 & - \frac{1}{2} & \frac{1}{4} & 0 & 1 \\ 
    \end{array} \right) 
    \begin{array}{l} \phantom{x} \\ - \frac{1}{3} R_{2}\\ \phantom{x} \end{array} \\ \\ 
    \longrightarrow 
    & \left( \begin{array}{ccc|ccc} 
         1 & 1 & \frac{1}{2} & - \frac{1}{4} & 0 & 0 \\ 
         0 & 1 & - \frac{5}{6} & - \frac{1}{4} & - \frac{1}{3} & 0 \\ 
         0 & -1 & - \frac{1}{2} & \frac{1}{4} & 0 & 1 \\ 
    \end{array} \right) 
    \begin{array}{l} R_{1} - R_{2} \\ \phantom{x} \\ R_{3} + R_{2} \end{array} \\ \\ 
    \longrightarrow 
    & \left( \begin{array}{ccc|ccc} 
         1 & 0 & \frac{4}{3} & 0 & \frac{1}{3} & 0 \\ 
         0 & 1 & - \frac{5}{6} & - \frac{1}{4} & - \frac{1}{3} & 0 \\ 
         0 & 0 & - \frac{4}{3} & 0 & - \frac{1}{3} & 1 \\ 
    \end{array} \right) 
    \begin{array}{l} \phantom{x} \\ \phantom{x} \\ - \frac{3}{4} R_{3}\end{array} \\ \\ 
    \longrightarrow 
    & \left( \begin{array}{ccc|ccc} 
         1 & 0 & \frac{4}{3} & 0 & \frac{1}{3} & 0 \\ 
         0 & 1 & - \frac{5}{6} & - \frac{1}{4} & - \frac{1}{3} & 0 \\ 
         0 & 0 & 1 & 0 & \frac{1}{4} & - \frac{3}{4} \\ 
    \end{array} \right) 
    \begin{array}{l} R_{1} - \frac{4}{3} R_{3} \\ R_{2} + \frac{5}{6} R_{3}\\ \phantom{x} \end{array} \\ \\ 
    \longrightarrow 
    & \left( \begin{array}{ccc|ccc} 
         1 & 0 & 0 & 0 & 0 & 1 \\ 
         0 & 1 & 0 & - \frac{1}{4} & - \frac{1}{8} & - \frac{5}{8} \\ 
         0 & 0 & 1 & 0 & \frac{1}{4} & - \frac{3}{4} \\ 
    \end{array} \right) 
\end{align*} \end{split}
\end{equation*}
\sphinxAtStartPar
So \(A^{-1} = \left(\begin{matrix}0 & 0 & 1\\- \frac{1}{4} & - \frac{1}{8} & - \frac{5}{8}\\0 & \frac{1}{4} & - \frac{3}{4}\end{matrix}\right)\). Checking this is correct
\begin{equation*}
\begin{split} \begin{align*} 
    \left(\begin{matrix}0 & 0 & 1\\- \frac{1}{4} & - \frac{1}{8} & - \frac{5}{8}\\0 & \frac{1}{4} & - \frac{3}{4}\end{matrix}\right)\left(\begin{matrix}-4 & -4 & -2\\3 & 0 & 4\\1 & 0 & 0\end{matrix}\right) = \left(\begin{matrix}1 & 0 & 0\\0 & 1 & 0\\0 & 0 & 1\end{matrix}\right) \quad \checkmark\end{align*} \end{split}
\end{equation*}
\sphinxAtStartPar
(d)
\begin{equation*}
\begin{split} \begin{align*} 
    & \left( \begin{array}{ccc|ccc} 
         4 & 0 & -4 & 1 & 0 & 0 \\ 
         4 & -1 & 1 & 0 & 1 & 0 \\ 
         3 & 1 & 2 & 0 & 0 & 1 \\ 
    \end{array} \right) 
    \begin{array}{l} \frac{1}{4} R_{1}\\ \phantom{x} \\ \phantom{x} \end{array} \\ \\ 
    \longrightarrow 
    & \left( \begin{array}{ccc|ccc} 
         1 & 0 & -1 & \frac{1}{4} & 0 & 0 \\ 
         4 & -1 & 1 & 0 & 1 & 0 \\ 
         3 & 1 & 2 & 0 & 0 & 1 \\ 
    \end{array} \right) 
    \begin{array}{l} \phantom{x} \\ R_{2} - 4 R_{1} \\ R_{3} - 3 R_{1} \end{array} \\ \\ 
    \longrightarrow 
    & \left( \begin{array}{ccc|ccc} 
         1 & 0 & -1 & \frac{1}{4} & 0 & 0 \\ 
         0 & -1 & 5 & -1 & 1 & 0 \\ 
         0 & 1 & 5 & - \frac{3}{4} & 0 & 1 \\ 
    \end{array} \right) 
    \begin{array}{l} \phantom{x} \\ -1 R_{2}\\ \phantom{x} \end{array} \\ \\ 
    \longrightarrow 
    & \left( \begin{array}{ccc|ccc} 
         1 & 0 & -1 & \frac{1}{4} & 0 & 0 \\ 
         0 & 1 & -5 & 1 & -1 & 0 \\ 
         0 & 1 & 5 & - \frac{3}{4} & 0 & 1 \\ 
    \end{array} \right) 
    \begin{array}{l} \phantom{x} \\ \phantom{x} \\ R_{3} - R_{2} \end{array} \\ \\ 
    \longrightarrow 
    & \left( \begin{array}{ccc|ccc} 
         1 & 0 & -1 & \frac{1}{4} & 0 & 0 \\ 
         0 & 1 & -5 & 1 & -1 & 0 \\ 
         0 & 0 & 10 & - \frac{7}{4} & 1 & 1 \\ 
    \end{array} \right) 
    \begin{array}{l} \phantom{x} \\ \phantom{x} \\ \frac{1}{10} R_{3}\end{array} \\ \\ 
    \longrightarrow 
    & \left( \begin{array}{ccc|ccc} 
         1 & 0 & -1 & \frac{1}{4} & 0 & 0 \\ 
         0 & 1 & -5 & 1 & -1 & 0 \\ 
         0 & 0 & 1 & - \frac{7}{40} & \frac{1}{10} & \frac{1}{10} \\ 
    \end{array} \right) 
    \begin{array}{l} R_{1} + R_{3} \\ R_{2} + 5 R_{3}\\ \phantom{x} \end{array} \\ \\ 
    \longrightarrow 
    & \left( \begin{array}{ccc|ccc} 
         1 & 0 & 0 & \frac{3}{40} & \frac{1}{10} & \frac{1}{10} \\ 
         0 & 1 & 0 & \frac{1}{8} & - \frac{1}{2} & \frac{1}{2} \\ 
         0 & 0 & 1 & - \frac{7}{40} & \frac{1}{10} & \frac{1}{10} \\ 
    \end{array} \right) 
\end{align*} \end{split}
\end{equation*}
\sphinxAtStartPar
So \(A^{-1} = \left(\begin{matrix}\frac{3}{40} & \frac{1}{10} & \frac{1}{10}\\\frac{1}{8} & - \frac{1}{2} & \frac{1}{2}\\- \frac{7}{40} & \frac{1}{10} & \frac{1}{10}\end{matrix}\right)\). Checking this is correct
\begin{equation*}
\begin{split} \begin{align*} 
    \left(\begin{matrix}\frac{3}{40} & \frac{1}{10} & \frac{1}{10}\\\frac{1}{8} & - \frac{1}{2} & \frac{1}{2}\\- \frac{7}{40} & \frac{1}{10} & \frac{1}{10}\end{matrix}\right)\left(\begin{matrix}4 & 0 & -4\\4 & -1 & 1\\3 & 1 & 2\end{matrix}\right) = \left(\begin{matrix}1 & 0 & 0\\0 & 1 & 0\\0 & 0 & 1\end{matrix}\right) \quad \checkmark\end{align*} \end{split}
\end{equation*}\end{sphinxadmonition}
\phantomsection \label{_pages/A2_Linear_systems_exercises_solutions:_pages/A2_Linear_systems_exercises_solutions-solution-6}

\begin{sphinxadmonition}{note}{Solution to Exercise 2.11.7}



\sphinxAtStartPar
(a)
\begin{equation*}
\begin{split} \begin{align*}
&    \left( \begin{array}{ccc|c}
         1 & -1 & 2 & 2 \\
         2 & 1 & 4 & 7 \\
         4 & 1 & 1 & 4 \\
    \end{array} \right)
    \begin{matrix} \phantom{x} \\ R_{2} - 2 R_{1} \\ R_{3} - 4 R_{1} \end{matrix} &
    \longrightarrow &
    \left( \begin{array}{ccc|c}
         1 & -1 & 2 & 2 \\
         0 & 3 & 0 & 3 \\
         0 & 5 & -7 & -4 \\
    \end{array} \right)
    \begin{matrix} \phantom{x} \\ \phantom{x} \\ R_{3} - \frac{5}{3} R_{2} \end{matrix} \\ \\
    \longrightarrow &
    \left( \begin{array}{ccc|c}
         1 & -1 & 2 & 2 \\
         0 & 3 & 0 & 3 \\
         0 & 0 & -7 & -9 \\
    \end{array} \right)
\end{align*} \end{split}
\end{equation*}
\sphinxAtStartPar
therefore \(\operatorname{rank}(A) = 3\) and \(\operatorname{rank}(A \mid \mathbf{b}) = 3\) so this system has a unique solution
\begin{equation*}
\begin{split} \begin{align*}
    x_{3} &= \frac{1}{-7} \left( -9\right) = \frac{9}{7}, \\
    x_{2} &= \frac{1}{3} \left( 3 - 0 \left( \frac{9}{7} \right)\right) = 1, \\
    x_{1} &= \frac{1}{1} \left( 2 - \left( -1 \right) \left( 1 \right) - 2 \left( \frac{9}{7} \right)\right) = \frac{3}{7}.
\end{align*} \end{split}
\end{equation*}
\sphinxAtStartPar
(b)
\begin{equation*}
\begin{split} \begin{align*}
&    \left( \begin{array}{ccc|c}
         1 & -1 & 2 & 3 \\
         2 & -3 & 7 & 4 \\
         -1 & 3 & -8 & 1 \\
    \end{array} \right)
    \begin{matrix} \phantom{x} \\ R_{2} - 2 R_{1} \\ R_{3} + R_{1} \end{matrix} &
    \longrightarrow &
    \left( \begin{array}{ccc|c}
         1 & -1 & 2 & 3 \\
         0 & -1 & 3 & -2 \\
         0 & 2 & -6 & 4 \\
    \end{array} \right)
    \begin{matrix} \phantom{x} \\ \phantom{x} \\ R_{3} + 2 R_{2} \end{matrix} \\ \\
    \longrightarrow &
    \left( \begin{array}{ccc|c}
         1 & -1 & 2 & 3 \\
         0 & -1 & 3 & -2 \\
         0 & 0 & 0 & 0 \\
    \end{array} \right)
\end{align*} \end{split}
\end{equation*}
\sphinxAtStartPar
therefore \(\operatorname{rank}(A) = 2\) and \(\operatorname{rank}(A \mid \mathbf{b}) = 2\). Since \(\operatorname{rank}(A) = \operatorname{rank}(A \mid \mathbf{b}) < 3\) then this system has infintely many solutons. Let \(r = x_3\)
\begin{equation*}
\begin{split} \begin{align*}
    x_2 &= \frac{1}{-1}(-2 - 3r) = 2 + 3r, \\
    x_1 &= 3 + 2 + 3r - 2r = 5 + r.
\end{align*} \end{split}
\end{equation*}
\sphinxAtStartPar
(c)
\begin{equation*}
\begin{split} \begin{align*}
&    \left( \begin{array}{ccc|c}
         1 & 1 & -2 & 1 \\
         2 & -1 & 1 & 9 \\
         1 & 4 & -7 & 2 \\
    \end{array} \right)
    \begin{matrix} \phantom{x} \\ R_{2} - 2 R_{1} \\ R_{3} - R_{1} \end{matrix} &
    \longrightarrow &
    \left( \begin{array}{ccc|c}
         1 & 1 & -2 & 1 \\
         0 & -3 & 5 & 7 \\
         0 & 3 & -5 & 1 \\
    \end{array} \right)
    \begin{matrix} \phantom{x} \\ \phantom{x} \\ R_{3} + R_{2} \end{matrix} \\ \\
    \longrightarrow &
    \left( \begin{array}{ccc|c}
         1 & 1 & -2 & 1 \\
         0 & -3 & 5 & 7 \\
         0 & 0 & 0 & 8 \\
    \end{array} \right)
\end{align*} \end{split}
\end{equation*}
\sphinxAtStartPar
therefore \(\operatorname{rank}(A) = 2\) and \(\operatorname{rank}(A \mid \mathbf{b}) = 3\). Since \(\operatorname{rank}(A) < \operatorname{rank}(A \mid \mathbf{b})\) then this is an inconsistent system and does not have a solution.
\end{sphinxadmonition}

\sphinxstepscope


\chapter{Vectors Exercise Solutions}
\label{\detokenize{_pages/A3_Vectors_exercises_solutions:vectors-exercise-solutions}}\label{\detokenize{_pages/A3_Vectors_exercises_solutions:vectors-exercises-solutions-section}}\label{\detokenize{_pages/A3_Vectors_exercises_solutions::doc}}\phantomsection \label{_pages/A3_Vectors_exercises_solutions:_pages/A3_Vectors_exercises_solutions-solution-0}

\begin{sphinxadmonition}{note}{Solution to Exercise 3.8.1}



\sphinxAtStartPar
(a)   \(2\mathbf{u} + \mathbf{w} = 2\begin{pmatrix} 2 \\ 3 \end{pmatrix} + \begin{pmatrix} 1 \\ 6 \end{pmatrix} 
= \begin{pmatrix} 4 \\ 6 \end{pmatrix} + \begin{pmatrix} 1 \\ 6 \end{pmatrix} 
= \begin{pmatrix} 5 \\ 12 \end{pmatrix}\)

\sphinxAtStartPar
(b)   \(\mathbf{w} - \mathbf{u} = \begin{pmatrix} 1 \\ 6 \end{pmatrix} - \begin{pmatrix} 2 \\ 3 \end{pmatrix} 
= \begin{pmatrix} 1 - 2 \\ 6 - 3 \end{pmatrix} = \begin{pmatrix} -1 \\ 3 \end{pmatrix}\)

\sphinxAtStartPar
(c)   \(\hat{\mathbf{u}} = \dfrac{\mathbf{u}}{\|\mathbf{u}\|} = \dfrac{1}{\sqrt{13}} \begin{pmatrix} 2 \\ 3 \end{pmatrix} 
= \begin{pmatrix} \frac{2}{\sqrt{13}} \\ \frac{3}{\sqrt{13}} \end{pmatrix}\)

\sphinxAtStartPar
(d)   \(-\hat{\mathbf{v}} = -\dfrac{\mathbf{v}}{\|\mathbf{v}\|} = -\dfrac{1}{\sqrt{13}} \begin{pmatrix} 3 \\ -2 \end{pmatrix} = \begin{pmatrix} -\frac{3}{\sqrt{13}} \\ \frac{2}{\sqrt{13}} \end{pmatrix}\)

\sphinxAtStartPar
(e)   \(\dfrac{1}{2}\mathbf{v} = \dfrac{1}{2} \begin{pmatrix} 3 \\ -2 \end{pmatrix} 
= \begin{pmatrix} 3 / 2 \\ -2 / 2 \end{pmatrix}\)

\sphinxAtStartPar
(f)   \(\mathbf{v} - \mathbf{u} = \begin{pmatrix} 3 \\ -2 \end{pmatrix} - \begin{pmatrix} 2 \\ 3 \end{pmatrix}= \begin{pmatrix} 1 \\ -5 \end{pmatrix}\)

\sphinxAtStartPar
(g)   \(\mathbf{w} - \mathbf{u} = \begin{pmatrix} 1 \\ 6 \end{pmatrix} - \begin{pmatrix} 2 \\ 3 \end{pmatrix} = \begin{pmatrix} -1 \\ 3 \end{pmatrix}\)

\sphinxAtStartPar
(h)   \(\mathbf{u} \cdot \mathbf{w} = \begin{pmatrix} 2 \\ 3 \end{pmatrix} \cdot \begin{pmatrix} 1 \\ 6 \end{pmatrix} = 2 \times 1 + 3 \times 6 = 20\)

\sphinxAtStartPar
(i)   Using equation \eqref{equation:_pages/3.3_Dot_and_cross_products:geometric-dot-product-equation}
\begin{equation*}
\begin{split} \begin{align*}
    (\mathbf{v} - \mathbf{u}) \cdot (\mathbf{w} - \mathbf{u}) &= \|(\mathbf{v} - \mathbf{u})\|\|(\mathbf{w} - \mathbf{u})\| \cos(\theta) \\
    \begin{pmatrix} 1 \\ -5 \end{pmatrix} \cdot \begin{pmatrix} -1 \\ 3 \end{pmatrix} &=
    \left\| \begin{pmatrix} 1 \\ -5 \end{pmatrix} \right\|
    \left\| \begin{pmatrix} -1 \\ 3 \end{pmatrix} \right\| \cos(\theta) \\
    -16 &= \sqrt{26} \sqrt{10} \cos(\theta) \\
    \theta &= \cos^{-1} \left( \frac{-16}{\sqrt{260}} \right) \approx 3.0172
\end{align*} \end{split}
\end{equation*}
\sphinxAtStartPar
(j)   \(\mathbf{u} \cdot \mathbf{v} = \begin{pmatrix} 2 \\ 3 \end{pmatrix} \cdot \begin{pmatrix} 3 \\ -2 \end{pmatrix} = 2 \times 3 + 3 \times (-2) = 0\)

\sphinxAtStartPar
(k)   \(\mathbf{v} \times \mathbf{w} = \begin{vmatrix} \mathbf{i} & \mathbf{j} & \mathbf{k} \\ 3 & -2 & 0 \\ 1 & 6 & 0 \end{vmatrix} = 0\mathbf{i} - 0 \mathbf{j} + 20 \mathbf{k} = \begin{pmatrix} 0 \\ 0 \\ 20 \end{pmatrix}\)
\end{sphinxadmonition}
\phantomsection \label{_pages/A3_Vectors_exercises_solutions:_pages/A3_Vectors_exercises_solutions-solution-1}

\begin{sphinxadmonition}{note}{Solution to Exercise 3.8.2}



\sphinxAtStartPar
(a)   \(\mathbf{u} = 2 \begin{pmatrix} 1 \\ 0 \\ 0 \end{pmatrix} + 7 \begin{pmatrix} 0 \\ 1 \\ 0 \end{pmatrix} + \begin{pmatrix} 0 \\ 0 \\ 1 \end{pmatrix} = 2 \mathbf{i} + 7 \mathbf{j} + \mathbf{k}\)

\sphinxAtStartPar
(b)
\begin{equation*}
\begin{split} \begin{align*}
    & \left( \begin{array}{ccc\|c}
    1 & 0 & 1 & 2 \\
    -1 & 2 & 0 & 7 \\
    0 & 0 & -1 & 1
    \end{array} \right)
    \begin{array}{l} \\ R_2 + R_1 \\ \phantom{x} \end{array} &
    \longrightarrow &
    \left( \begin{array}{ccc\|c}
        1 & 0 & 1 & 2 \\
        0 & 2 & 1 & 9 \\
        0 & 0 & -1 & 1
    \end{array} \right)
    \begin{array}{l} \\ \frac{1}{2} R_2 \\ -R_3 \phantom{x} \end{array} \\ \\
    \longrightarrow &
    \left( \begin{array}{ccc\|c}
        1 & 0 & 1 & 2 \\
        0 & 1 & 1/2 & 9/2 \\
        0 & 0 & 1 & -1
    \end{array} \right)
    \begin{array}{l} R_1 - R_3 \\ R_2 - \frac{1}{2} R_3 \\ \phantom{x} \end{array} &
    \longrightarrow &
    \left( \begin{array}{ccc\|c}
        1 & 0 & 0 & 3 \\
        0 & 1 & 0 & 5 \\
        0 & 0 & 1 & -1
    \end{array} \right)
    \begin{array}{l} R_1 - R_3 \\ R_2 - \frac{1}{2} R_3 \\ \phantom{x} \end{array} \\ \\
\end{align*} \end{split}
\end{equation*}
\sphinxAtStartPar
Therefore \(\mathbf{u} = 3 \mathbf{f}_1 + 5 \mathbf{f}_2 - \mathbf{f}_3\).
\end{sphinxadmonition}
\phantomsection \label{_pages/A3_Vectors_exercises_solutions:_pages/A3_Vectors_exercises_solutions-solution-2}

\begin{sphinxadmonition}{note}{Solution to Exercise 3.8.3}



\sphinxAtStartPar
(a)   If \(\mathbf{u}\) and \(\mathbf{v}\) are perpendicular then \(\mathbf{u} \cdot \mathbf{v} = 0\).
\begin{equation*}
\begin{split} \begin{align*}
    \mathbf{u} \cdot \mathbf{v} &= 0 \\
    \begin{pmatrix} 1 \\ k \\ -2 \end{pmatrix} \cdot
    \begin{pmatrix} 2 \\ -5 \\ 4 \end{pmatrix} &= 0 \\
    2 - 5 k - 8 &= 0 \\
    -5k &= 6 \\
    \therefore k &= -\frac{6}{5}.
\end{align*} \end{split}
\end{equation*}
\sphinxAtStartPar
(b)  
\begin{equation*}
\begin{split} \begin{align*}
    \mathbf{u} \cdot \mathbf{v} &= 0 \\
    \begin{pmatrix} 1 \\ 0 \\ k + 2 \\ -1 \\ 2 \end{pmatrix} \cdot 
    \begin{pmatrix} 1 \\ k \\ -2 \\ 1 \\ 2 \end{pmatrix} &= 0 \\
    1 - 2k - 2 - 1 + 2 &= 0 \\
    -2k &= 0 \\
    \therefore k &= 0.
\end{align*} \end{split}
\end{equation*}\end{sphinxadmonition}
\phantomsection \label{_pages/A3_Vectors_exercises_solutions:_pages/A3_Vectors_exercises_solutions-solution-3}

\begin{sphinxadmonition}{note}{Solution to Exercise 3.8.4}


\begin{equation*}
\begin{split} \begin{align*}
    \mathbf{u} \cdot \mathbf{v} &= \begin{pmatrix} 1 \\ 2 \\ 3 \end{pmatrix} \cdot
    \begin{pmatrix} -1 \\ 2 \\ -1 \end{pmatrix} = -1 + 4 - 3 = 0, \\
    \mathbf{u} \cdot \mathbf{w} &= \begin{pmatrix} 1 \\ 2 \\ 3 \end{pmatrix} \cdot
    \begin{pmatrix} 2 \\ -3 \\ 1 \end{pmatrix} = 2 - 6 + 3 = -1, \\
    \mathbf{v} \cdot \mathbf{w} &= \begin{pmatrix} -1 \\ 2 \\ -1 \end{pmatrix} \cdot
    \begin{pmatrix} 2 \\ -3 \\ 1 \end{pmatrix} = -2 -6 -1 =  -9.
\end{align*} \end{split}
\end{equation*}
\sphinxAtStartPar
Therefore \(\mathbf{u} \perp \mathbf{v}\). The angle between \(\mathbf{u}\) and \(\mathbf{w}\) is
\begin{equation*}
\begin{split} \begin{align*}
    \theta &= \cos^{-1} \left( \frac{\mathbf{u} \cdot \mathbf{w}}{\|\mathbf{u} \|\|\mathbf{w}\|} \right) \\
    &= \cos^{-1} \left( \frac{-1}{\sqrt{14}\sqrt{14}}\right) \\
    &= \cos^{-1}\left(\frac{-1}{14}\right) \approx 1.6423,
\end{align*} \end{split}
\end{equation*}
\sphinxAtStartPar
and the angle between \(\mathbf{v}\) and \(\mathbf{w}\) is
\begin{equation*}
\begin{split} \begin{align*}
    \theta &= \cos^{-1} \left( \frac{\mathbf{v} \cdot \mathbf{w}}{\|\mathbf{v} \|\|\mathbf{w}\|} \right) \\
    &= \cos^{-1} \left( \frac{-9}{\sqrt{6}\sqrt{14}} \right) \approx 2.9515.
\end{align*} \end{split}
\end{equation*}\end{sphinxadmonition}

\sphinxstepscope


\chapter{Co\sphinxhyphen{}ordinate Geometry Exercise Solutions}
\label{\detokenize{_pages/A4_Coordinate_geometry_exercises_solutions:co-ordinate-geometry-exercise-solutions}}\label{\detokenize{_pages/A4_Coordinate_geometry_exercises_solutions:co-ordinate-geometry-exercises-solutions-section}}\label{\detokenize{_pages/A4_Coordinate_geometry_exercises_solutions::doc}}\phantomsection \label{_pages/A4_Coordinate_geometry_exercises_solutions:_pages/A4_Coordinate_geometry_exercises_solutions-solution-0}

\begin{sphinxadmonition}{note}{Solution to Exercise 4.5.1}



\sphinxAtStartPar
(a)   Using the {\hyperref[\detokenize{_pages/4.1_Lines:vector-equation-of-a-line-definition}]{\sphinxcrossref{vector equation of a line}}} we have \(\mathbf{r} = \mathbf{a} + t\mathbf{d}_1\)
\begin{equation*}
\begin{split} \begin{align*}
    \mathbf{d}_1 &= \mathbf{b} - \mathbf{a} =
    \begin{pmatrix} 1 \\ 1 \\ 0 \end{pmatrix} -
    \begin{pmatrix} 2 \\ 1 \\ 0 \end{pmatrix} = 
    \begin{pmatrix} -1 \\ 0 \\ 0 \end{pmatrix}, \\
    \therefore \mathbf{r} &= \mathbf{a} + t\mathbf{d}_1 =
    \begin{pmatrix} 2 \\ 1 \\ 0 \end{pmatrix} + t 
    \begin{pmatrix} -1 \\ 0 \\ 0 \end{pmatrix} = 
    \begin{pmatrix} 2 - t \\ 1 \\ 0 \end{pmatrix}
\end{align*} \end{split}
\end{equation*}
\sphinxAtStartPar
(b)   Using the {\hyperref[\detokenize{_pages/4.1_Lines:vector-equation-of-a-line-definition}]{\sphinxcrossref{vector equation of a line}}} we have \(\mathbf{r} = \mathbf{c} + t\mathbf{d}_2\)
\begin{equation*}
\begin{split} \begin{align*}
    \mathbf{d}_2 &= \mathbf{d} - \mathbf{c} =
    \begin{pmatrix} 5 \\ 2 \\ 6 \end{pmatrix} -
    \begin{pmatrix} 3 \\ -1 \\ 4 \end{pmatrix} = 
    \begin{pmatrix} 2 \\ 3 \\ 2 \end{pmatrix}, \\
    \therefore \mathbf{r} &= \mathbf{c} + t\mathbf{d}_2 =
    \begin{pmatrix} 3 \\ -1 \\ 4 \end{pmatrix} + t 
    \begin{pmatrix} 2 \\ 3 \\ 2 \end{pmatrix} = 
    \begin{pmatrix} 3 + 2 t \\ -1 + 3 t \\ 4 + 2 t \end{pmatrix}
\end{align*} \end{split}
\end{equation*}
\sphinxAtStartPar
(c)   Calculate the normal vector to the plane
\begin{equation*}
\begin{split} \begin{align*}
    \mathbf{b} - \mathbf{a} &= \begin{pmatrix} 1 \\ 1 \\ 0 \end{pmatrix} - 
    \begin{pmatrix} 2 \\ 1 \\ 0 \end{pmatrix} = 
    \begin{pmatrix} -1 \\ 0 \\ 0 \end{pmatrix}, \\
    \mathbf{c} - \mathbf{a} &= \begin{pmatrix} 3 \\ -1 \\ 4 \end{pmatrix} -
    \begin{pmatrix} 2 \\ 1 \\ 0 \end{pmatrix} =
    \begin{pmatrix} 1 \\ -2 \\ 4 \end{pmatrix}, \\
    \therefore \mathbf{n} &= (\mathbf{b} - \mathbf{a}) \times (\mathbf{c} - \mathbf{a}) = 
    \begin{vmatrix} 
        \mathbf{i} & \mathbf{j} & \mathbf{k} \\
        -1 & 0 & 0 \\
        1 & -2 & 4 
    \end{vmatrix} =
    \begin{pmatrix} 0 \\ 4 \\ 2 \end{pmatrix},
\end{align*} \end{split}
\end{equation*}
\sphinxAtStartPar
Using the {\hyperref[\detokenize{_pages/4.2_Planes:point-normal-definition}]{\sphinxcrossref{point normal definition of a plane}}}
\begin{equation*}
\begin{split} \begin{align*}
    n_x(x - a_x) + n_y(y - a_y) + n_z(z - a_z) &= 0 \\
    0(x - 2) + 4(y - 1) + 2(z - 0) &= 0 \\
    4 y + 2 z - 4 &= 0.
\end{align*} \end{split}
\end{equation*}
\sphinxAtStartPar
(d)   Calculate the normal vector to the plane
\begin{equation*}
\begin{split} \begin{align*}
    \mathbf{c} - \mathbf{b} &= \begin{pmatrix} 3 \\ -1 \\ 4 \end{pmatrix} - 
    \begin{pmatrix} 1 \\ 1 \\ 0 \end{pmatrix} =
    \begin{pmatrix} 2 \\ -2 \\ 4 \end{pmatrix}, \\
    \mathbf{d} - \mathbf{b} &= \begin{pmatrix} 5 \\ 2 \\ 6 \end{pmatrix} -
    \begin{pmatrix} 1 \\ 1 \\ 0 \end{pmatrix} =
    \begin{pmatrix} 4 \\ 1 \\ 6 \end{pmatrix}, \\
    \mathbf{n} &= (\mathbf{c} - \mathbf{b}) \times (\mathbf{d} - \mathbf{b}) = 
    \begin{vmatrix}
        \mathbf{i} & \mathbf{j} & \mathbf{k} \\
        2 & -2 & 4 \\
        4 & 1 & 6
    \end{vmatrix} =
    \begin{pmatrix} -16 \\ 4 \\ 10 \end{pmatrix},
\end{align*} \end{split}
\end{equation*}
\sphinxAtStartPar
Using the {\hyperref[\detokenize{_pages/4.2_Planes:point-normal-definition}]{\sphinxcrossref{point normal definition of a plane}}}
\begin{equation*}
\begin{split} \begin{align*}
    n_x(x - b_x) + n_y(y - b_y) + n_z(z - b_z) &= 0 \\
    -16(x - 1) + 4(y - 1) + 10(z - 0) &= 0\\
    -16 x + 4 y + 10 z + 12 &= 0.
\end{align*} \end{split}
\end{equation*}\end{sphinxadmonition}
\phantomsection \label{_pages/A4_Coordinate_geometry_exercises_solutions:_pages/A4_Coordinate_geometry_exercises_solutions-solution-1}

\begin{sphinxadmonition}{note}{Solution to Exercise 4.5.2}


\begin{equation*}
\begin{split} \mathbf{r} = \begin{pmatrix} 3 \\ 2 \\ 1 \end{pmatrix} + t \begin{pmatrix} 2 \\ 1 \\ 3 \end{pmatrix}  = \begin{pmatrix} 3 + 2 t \\ 2 + t \\ 1 + 3 t \end{pmatrix} \end{split}
\end{equation*}\end{sphinxadmonition}
\phantomsection \label{_pages/A4_Coordinate_geometry_exercises_solutions:_pages/A4_Coordinate_geometry_exercises_solutions-solution-2}

\begin{sphinxadmonition}{note}{Solution to Exercise 4.5.3}



\sphinxAtStartPar
Using the {\hyperref[\detokenize{_pages/4.2_Planes:point-normal-definition}]{\sphinxcrossref{point normal definition of a plane}}}
\begin{equation*}
\begin{split} \begin{align*}
    \mathbf{n} \cdot \begin{pmatrix} x - x_0 \\ y - y_0 \\ z - z_0 \end{pmatrix} &= 0 \\
    \begin{pmatrix} 2 \\ 1 \\ 3 \end{pmatrix} \cdot
    \begin{pmatrix} x - 3 \\ y - 2 \\ z - 5 \end{pmatrix} \\
    2(x - 3) + (y - 2) + 3(z - 5) &= 0 \\
    2 x + y + 3 z - 23 &= 0
\end{align*} \end{split}
\end{equation*}\end{sphinxadmonition}
\phantomsection \label{_pages/A4_Coordinate_geometry_exercises_solutions:_pages/A4_Coordinate_geometry_exercises_solutions-solution-3}

\begin{sphinxadmonition}{note}{Solution to Exercise 4.5.4}



\sphinxAtStartPar
\(\mathbf{n} = (3, -2, 1)\).

\sphinxAtStartPar
Let \(x=0\) then \(3(0) - 2 y + 2 = 10\) so \(y = -4\) and a point on the plane has co\sphinxhyphen{}ordinates \((0, -4, 2)\).

\sphinxAtStartPar
Let \(x = 2\) then \(3(2) - 2 y + 2 = 10\) so \(y = -1\) and a point on the plane has co\sphinxhyphen{}ordinates \((2, -1, 2)\).
\end{sphinxadmonition}
\phantomsection \label{_pages/A4_Coordinate_geometry_exercises_solutions:_pages/A4_Coordinate_geometry_exercises_solutions-solution-4}

\begin{sphinxadmonition}{note}{Solution to Exercise 4.5.5}



\sphinxAtStartPar
(a)   Equating \(\ell_1\) and \(\ell_2\) and attempting to solve for \(t\)
\begin{equation*}
\begin{split} \begin{align*}
    1 + 2t_1 &= 1 + 2t_2 \\
    -t_1 &= 4 \\
    1 + 3t_1 &= 7 - t_2
\end{align*} \end{split}
\end{equation*}
\sphinxAtStartPar
From the second equation \(t_1 = -4\) which when substituted into the third equation gives \(t_2 = 18\). Substituting these into the first equation gives \(-7 = 37\) which is a contradiction so \(\ell_1\) and \(\ell_2\) do not intersect.

\sphinxAtStartPar
We also need to show that they are not parallel, i.e., there is no value \(k\) such that \(\mathbf{d}_1 = k \mathbf{d}_2\). The direction vectors for \(\ell_1\) and \(\ell_2\) are \(\mathbf{d}_1 = (2, -1, 3)\) and \(\mathbf{d}_2 = (2, 0, -1)\) so
\begin{equation*}
\begin{split} \begin{pmatrix} 2 \\ -1 \\ 3 \end{pmatrix} &= k \begin{pmatrix} 2 \\ 0 \\ -1 \end{pmatrix}, \end{split}
\end{equation*}
\sphinxAtStartPar
which gives the system
\begin{equation*}
\begin{split} \begin{align*}
    2 &= 2k ,\\
    -1 &= 0, \\
    3 &= -k.
\end{align*} \end{split}
\end{equation*}
\sphinxAtStartPar
The second equation is a contradiction so \(\ell_1\) and \(\ell_2\) are not parallel, and since they do not intersect then they must be skew.

\sphinxAtStartPar
(b)   Using the {\hyperref[\detokenize{_pages/4.3_Shortest_distance_problems:point-line-distance-theorem}]{\sphinxcrossref{shortest distance between a point and a line}}}
\begin{equation*}
\begin{split} \begin{align*}
    t &= \frac{(\mathbf{p} - \mathbf{p}_1)\cdot \mathbf{d}_1}{\mathbf{d}_1 \cdot \mathbf{d}_1} = \frac{ 
    \left( \begin{pmatrix} 0 \\ -1 \\ 3 \end{pmatrix} -
    \begin{pmatrix} 1 \\ 0 \\ 1 \end{pmatrix} \right) \cdot
    \begin{pmatrix} 2 \\ -1 \\ 3 \end{pmatrix}}{
        \begin{pmatrix} 2 \\ -1 \\ 3 \end{pmatrix} \cdot
        \begin{pmatrix} 2 \\ -1 \\ 3 \end{pmatrix}
    } 
    = \frac{5}{14}, \\
    \therefore \mathbf{r} &= \mathbf{p}_1 + t\mathbf{d}_1 = \begin{pmatrix} 1 \\ 0 \\ 1 \end{pmatrix} + \frac{5}{14} 
    \begin{pmatrix} 2 \\ -1 \\ 3 \end{pmatrix} = 
    \begin{pmatrix} \frac{12}{7} \\ - \frac{5}{14} \\ \frac{29}{14} \end{pmatrix}, \\
    \overrightarrow{\mathbf{rp}} &= \begin{pmatrix} 0 \\ -1 \\ 3 \end{pmatrix} - 
    \begin{pmatrix} \frac{12}{7} \\ -\frac{5}{14} \\ \frac{29}{14} \end{pmatrix} =
    \begin{pmatrix} -\frac{12}{7} \\ -\frac{9}{14} \\ \frac{13}{14} \end{pmatrix}, \\
    \therefore d &= \|\overrightarrow{\mathbf{rp}}\| = 
    \sqrt{\left(-\frac{12}{7}\right)^2 + \left(-\frac{9}{14}\right)^2 + \left(\frac{13}{14}\right)^2} 
    = \frac{\sqrt{826}}{14}.
\end{align*} \end{split}
\end{equation*}
\sphinxAtStartPar
(c)   The direction vectors for lines \(\ell_1\) and \(\ell_2\) are \(\mathbf{d}_1 = (2, -1, 3)^\mathsf{T}\) and \(\mathbf{d}_2 = (2, -, -1)^\mathsf{T}\) respectively. Calculating a vector perpendicular to both \(\mathbf{d}_1\) and \(\mathbf{d}_2\)
\begin{equation*}
\begin{split} \begin{align*}
    \mathbf{n} &= \mathbf{d}_1 \times \mathbf{d_1} = \begin{pmatrix} 2 \\ -1 \\ 3 \end{pmatrix} \times
    \begin{pmatrix} 2 \\ 0 \\ -1 \end{pmatrix} =
    \begin{vmatrix}
        \mathbf{i} & \mathbf{j} & \mathbf{k} \\
        2 & -1 & 3 \\
        2 & 0 & -1 
    \end{vmatrix} = 
    \begin{pmatrix} 1 \\ 8 \\ 2 \end{pmatrix}, \\
\end{align*} \end{split}
\end{equation*}
\sphinxAtStartPar
and normalising gives
\begin{equation*}
\begin{split} \begin{align*}
    \hat{\mathbf{n}} = \frac{\mathbf{n}}{\|\mathbf{n}\|} = \frac{1}{\sqrt{69}} 
    \begin{pmatrix} 1 \\ 8 \\ 2 \end{pmatrix} =
    \begin{pmatrix} \frac{\sqrt{69}}{69} \\ \frac{8\sqrt{69}}{69} \\ \frac{2\sqrt{69}}{69} \end{pmatrix}.
\end{align*} \end{split}
\end{equation*}
\sphinxAtStartPar
Using {\hyperref[\detokenize{_pages/4.3_Shortest_distance_problems:line-line-distance-theorem}]{\sphinxcrossref{the distance between two lines}}}
\begin{equation*}
\begin{split} \begin{align*}
    d &= (\mathbf{p}_2 - \mathbf{p}_1) \cdot \hat{\mathbf{n}} \\
    &= \left( \begin{pmatrix} 1 \\ 4 \\ 7 \end{pmatrix} -
    \begin{pmatrix} 1 \\ 0 \\ 1 \end{pmatrix} \right) \cdot
    \begin{pmatrix} \frac{\sqrt{69}}{69} \\ \frac{8\sqrt{69}}{69} \\ \frac{2\sqrt{69}}{69} \end{pmatrix} \\
    &= \begin{pmatrix} 0 \\ 4 \\ 6 \end{pmatrix} \cdot
    \begin{pmatrix} \frac{\sqrt{69}}{69} \\ \frac{8\sqrt{69}}{69} \\ \frac{2\sqrt{69}}{69} \end{pmatrix} \\
    &= \frac{44\sqrt{69}}{69}.
\end{align*} \end{split}
\end{equation*}\end{sphinxadmonition}
\phantomsection \label{_pages/A4_Coordinate_geometry_exercises_solutions:_pages/A4_Coordinate_geometry_exercises_solutions-solution-5}

\begin{sphinxadmonition}{note}{Solution to Exercise 4.5.6}



\sphinxAtStartPar
First we need to find the position vector of a point, \(\mathbf{r}\) say, that lies on the plane. Let \(x=0\) and \(y=1\) then \(z=-1\) so we know that \(\mathbf{r} = (0, 1, -1)^\mathsf{T}\) lies on the plane. Using the {\hyperref[\detokenize{_pages/4.2_Planes:point-normal-definition}]{\sphinxcrossref{point normal definition of a plane}}}
\begin{equation*}
\begin{split} \begin{align*}
    \mathbf{n} \cdot \begin{pmatrix} x - x_0 \\ y - y_0 \\ z - z_0 \end{pmatrix} &= 0 \\
    \begin{pmatrix} 6 \\ -1 \\ -4 \end{pmatrix} \cdot
    \begin{pmatrix} 0 - (1 + 2 t) \\ 1 - (2 + t) \\ -1 - (-1 + 4 t) \end{pmatrix} &= 0 \\
    \begin{pmatrix} 6 \\ -1 \\ -4 \end{pmatrix} \cdot
    \begin{pmatrix} - 1 - 2 t \\ -1 - t \\  -4 t \end{pmatrix} &= 0 \\
    -6 - 12 t + 1 + t + 16 t &= 0 \\
    5 t - 5 &= 0 \\
    \therefore t = 1.
\end{align*} \end{split}
\end{equation*}
\sphinxAtStartPar
So the line intersects with the plane at
\begin{equation*}
\begin{split} \begin{align*}
    \mathbf{p} + t \mathbf{d} = 
    \begin{pmatrix} 1 \\ 2 \\ -1 \end{pmatrix} + 
    \begin{pmatrix} 2 \\ 1 \\ 4 \end{pmatrix} =
    \begin{pmatrix} 3 \\ 3 \\ 3 \end{pmatrix}.
\end{align*} \end{split}
\end{equation*}\end{sphinxadmonition}
\phantomsection \label{_pages/A4_Coordinate_geometry_exercises_solutions:_pages/A4_Coordinate_geometry_exercises_solutions-solution-6}

\begin{sphinxadmonition}{note}{Solution to Exercise 4.5.7}



\sphinxAtStartPar
Using the {\hyperref[\detokenize{_pages/3.3_Dot_and_cross_products:dot-product-definition}]{\sphinxcrossref{geometric definition of a dot product}}}
\begin{equation*}
\begin{split} \begin{align*}
    (\mathbf{q} - \mathbf{p})\cdot \mathbf{n} &= \|\mathbf{n}\|\|\mathbf{q} - \mathbf{p}\| \cos(\theta).
\end{align*} \end{split}
\end{equation*}
\sphinxAtStartPar
Since \(d\) is the length of the adjacent side of the right\sphinxhyphen{}angled triangle then
\begin{equation*}
\begin{split} \begin{align*}
    \cos(\theta) &= \frac{d}{\|\mathbf{q} - \mathbf{p}\|},
\end{align*} \end{split}
\end{equation*}
\sphinxAtStartPar
so
\begin{equation*}
\begin{split} \begin{align*}
    (\mathbf{q} - \mathbf{p}) \cdot \mathbf{n} &= \|\mathbf{n}\| \|\mathbf{q} - \mathbf{p}\| \frac{d}{\|\mathbf{q} - \mathbf{p}\|} \\
    \therefore d &= (\mathbf{q} - \mathbf{p})\cdot \frac{\mathbf{n}}{\|\mathbf{n}\|}.
\end{align*} \end{split}
\end{equation*}
\sphinxAtStartPar
The equation of the plane is \(6 x-y-4 z=3\) so letting \(x=0\) and \(y=1\) then \(z = 1\) so we know that \(\mathbf{p} = (0, 1, -1)\) lies on the plane. Since \(\mathbf{q} = (2, 4, -3)^\mathsf{T}\) and \(\mathbf{n} = (6, -1, -4)^\mathsf{T}\) then applying the above formula gives
\begin{equation*}
\begin{split} \begin{align*}
    d &= \left( \begin{pmatrix} 2 \\ 4 \\ -3 \end{pmatrix} -
    \begin{pmatrix} 0 \\ 1 \\ -1 \end{pmatrix} \right) \cdot 
    \begin{pmatrix} 6 \\ -1 \\ -4 \end{pmatrix} / \left\|
    \begin{pmatrix} 6 \\ -1 \\ -4 \end{pmatrix} \right\| \\
    &= \begin{pmatrix} 2 \\ 3 \\ -2 \end{pmatrix} \cdot \frac{1}{\sqrt{53}} 
    \begin{pmatrix} 6 \\ -1 \\ -4 \end{pmatrix} \\
    &= \frac{12 - 3 + 8}{\sqrt{53}} = \frac{17}{\sqrt{53}}
\end{align*} \end{split}
\end{equation*}\end{sphinxadmonition}

\sphinxstepscope


\chapter{Vector Spaces Exercise Solutions}
\label{\detokenize{_pages/A5_Vector_spaces_exercises_solutions:vector-spaces-exercise-solutions}}\label{\detokenize{_pages/A5_Vector_spaces_exercises_solutions:vector-spaces-exercises-solutions-section}}\label{\detokenize{_pages/A5_Vector_spaces_exercises_solutions::doc}}\phantomsection \label{_pages/A5_Vector_spaces_exercises_solutions:_pages/A5_Vector_spaces_exercises_solutions-solution-0}

\begin{sphinxadmonition}{note}{Solution to Exercise 5.5.1}



\sphinxAtStartPar
Let \(\mathbf{u}, \mathbf{v}, \mathbf{w} \in \mathbb{R}^3\) and \(\alpha, \beta \in \mathbb{R}\) then
\begin{itemize}
\item {} 
\sphinxAtStartPar
A1: \(\mathbf{u} + (\mathbf{v} + \mathbf{w}) = (\mathbf{u} + \mathbf{v}) + \mathbf{w} \checkmark\)

\item {} 
\sphinxAtStartPar
A2: \(\mathbf{u} + \mathbf{v} = \mathbf{v} + \mathbf{u} \checkmark\)

\item {} 
\sphinxAtStartPar
A3: \(\mathbf{u} + \mathbf{0} = \mathbf{u} \checkmark\)

\item {} 
\sphinxAtStartPar
A4: \(\mathbf{u} + (-\mathbf{u}) = \mathbf{0} \checkmark\)

\item {} 
\sphinxAtStartPar
M1: \(\alpha(\beta \mathbf{u}) = (\alpha \beta) \mathbf{u} \checkmark\)

\item {} 
\sphinxAtStartPar
M2: \(1 \mathbf{u} = \mathbf{u} \checkmark\)

\item {} 
\sphinxAtStartPar
M3: \(\alpha(\mathbf{u} + \mathbf{v}) = \alpha\mathbf{u} + \alpha \mathbf{v} \checkmark\)

\item {} 
\sphinxAtStartPar
M4: \((\alpha + \beta) \mathbf{u} = \alpha \mathbf{u} + \beta \mathbf{u} \checkmark\)

\end{itemize}

\sphinxAtStartPar
All of the axioms of vector spaces hold for \(\mathbb{R}^3\).
\end{sphinxadmonition}
\phantomsection \label{_pages/A5_Vector_spaces_exercises_solutions:_pages/A5_Vector_spaces_exercises_solutions-solution-1}

\begin{sphinxadmonition}{note}{Solution to Exercise 5.5.2}



\sphinxAtStartPar
(a)   \(U\) is non\sphinxhyphen{}empty since \(\mathbf{0} \in U\). Let \(\mathbf{u} = (u_1, u_2, 0), \mathbf{v} = (v_1, v_2, 0) \in U\) and \(\alpha \in \mathbb{R}\) then
\begin{equation*}
\begin{split} \mathbf{u} + \alpha \mathbf{v} = 
    \begin{pmatrix} u_1 \\ u_2 \\ 0 \end{pmatrix} + \alpha 
    \begin{pmatrix} v_1 \\ v_2 \\ 0 \end{pmatrix} = 
    \begin{pmatrix} 
        u_1 + \alpha v_1 \\ 
        u_2 + \alpha v_2 \\ 
        0 
    \end{pmatrix} \in U, \end{split}
\end{equation*}
\sphinxAtStartPar
therefore \(U\) is a subspace.

\sphinxAtStartPar
(b)   \(V\) is non\sphinxhyphen{}empty since \((1,2,0) \in V\). However \(\alpha (1, 2, 0) = (\alpha , 2\alpha , 0) \notin V\) for \(\alpha \in \mathbb{R}\) so \(V\) is not a subspace.

\sphinxAtStartPar
(c)   \(W\) is non\sphinxhyphen{}empty since \(\mathbf{0} \in W\). Let \(\mathbf{u} = (0, u_2, 0), \mathbf{v} = (0, v_2, 0) \in U\) and \(\alpha \in \mathbb{R}\) then
\begin{equation*}
\begin{split} \mathbf{u} + \alpha \mathbf{v} = 
    \begin{pmatrix} 0 \\ u_2 \\ 0 \end{pmatrix} + \alpha 
    \begin{pmatrix} 0 \\ v_2 \\ 0 \end{pmatrix} = 
    \begin{pmatrix} 0 \\ u_2 + \alpha v_2 \\ 0 \end{pmatrix} \in W, \end{split}
\end{equation*}
\sphinxAtStartPar
therefore \(W\) is a subspace. Note that \(W \subseteq U\) so since we showed \(U\) is a subspace then \(W\) must also be a subspace.

\sphinxAtStartPar
(d)   \(X\) is not a subspace since if \(\mathbf{u} = (1, 1, 0), \mathbf{v} = (-1, 1, 0) \in X\) then \(\mathbf{u} + \mathbf{v} = (0, 2, 0) \notin X\).
\end{sphinxadmonition}
\phantomsection \label{_pages/A5_Vector_spaces_exercises_solutions:_pages/A5_Vector_spaces_exercises_solutions-solution-2}

\begin{sphinxadmonition}{note}{Solution to Exercise 5.5.3}



\sphinxAtStartPar
(a)   \(A\) is non\sphinxhyphen{}empty since \(\begin{pmatrix} 1 & 0 \\ 0 & 0 \end{pmatrix} \in A\). Let \(U = \begin{pmatrix} 1 & 0 \\ 0 & 0 \end{pmatrix} \in A\) then
\begin{equation*}
\begin{split} 2U = \begin{pmatrix} 2 & 0 \\ 0 & 0 \end{pmatrix} \notin A, \end{split}
\end{equation*}
\sphinxAtStartPar
so \(A\) is not a subspace.

\sphinxAtStartPar
(b)   \(B\) is non\sphinxhyphen{}empty since \(0_{2\times 2} \in B\). Let \(U = \begin{pmatrix} u_{11} & 0 \\ u_{11} & u_{11} \end{pmatrix}, V = \begin{pmatrix} v_{11} & 0 \\ v_{11} & v_{11} \end{pmatrix} \in B\) and \(\alpha \in \mathbb{R}\) then
\begin{equation*}
\begin{split} \begin{align*}
    U + \alpha V = 
    \begin{pmatrix} u_{11} & 0 \\ u_{11} & u_{11} \end{pmatrix} + \alpha
    \begin{pmatrix} v_{11} & 0 \\ v_{11} & v_{11}\end{pmatrix}
    = \begin{pmatrix} u_{11} + \alpha v_{11} & 0 \\ u_{11} + \alpha v_{11} & u_{11} + \alpha v_{11} \end{pmatrix} \in B,
\end{align*} \end{split}
\end{equation*}
\sphinxAtStartPar
so \(B\) is a subspace.

\sphinxAtStartPar
(c)   \(C\) is not a subspace since \(U = \begin{pmatrix} 1 & 0 \\ 0 & 0 \end{pmatrix} \in C\) but \(2U = \begin{pmatrix} 2 & 0 \\ 0 & 0 \end{pmatrix} \notin C\).
\end{sphinxadmonition}
\phantomsection \label{_pages/A5_Vector_spaces_exercises_solutions:_pages/A5_Vector_spaces_exercises_solutions-solution-3}

\begin{sphinxadmonition}{note}{Solution to Exercise 5.5.4}



\sphinxAtStartPar
We need to show that the vectors in the set are {\hyperref[\detokenize{_pages/5.3_Linear_dependence:linear-dependence-definition}]{\sphinxcrossref{linearly independent}}}.
\begin{equation*}
\begin{split} \begin{align*}
    & \left( \begin{array}{ccc|c}
        1 & 0 & -1 & 0 \\
        2 & 5 & 1 & 0 \\
        0 & 7 & 3 & 0
    \end{array} \right)
    \begin{matrix} \\ R_2 - 2R_1 \\ \phantom{x} \end{matrix} &
    \longrightarrow &
    \left( \begin{array}{ccc|c}
        1 & 0 & -1 & 0 \\
        0 & 5 & 3 & 0 \\
        0 & 7 & 3 & 0
    \end{array} \right)
    \begin{matrix} \\ \frac{1}{5}R_2 \\ \phantom{x} \end{matrix} \\ \\
    \longrightarrow &
    \left( \begin{array}{ccc|c}
        1 & 0 & -1 & 0 \\
        0 & 1 & 3/5 & 0 \\
        0 & 7 & 3 & 0
    \end{array} \right)
    \begin{matrix} \\ \\ R_3 - 7R_2 \end{matrix} &
    \longrightarrow &
    \left( \begin{array}{ccc|c}
        1 & 0 & -1 & 0 \\
        0 & 1 & 3/5 & 0 \\
        0 & 0 & -6/5 & 0
    \end{array} \right)
    \begin{matrix} \\ \\ -\frac{5}{6}R_3 \end{matrix}  \\ \\
    \longrightarrow &
    \left( \begin{array}{ccc|c}
        1 & 0 & -1 & 0 \\
        0 & 1 & 3/5 & 0 \\
        0 & 0 & 1 & 0
    \end{array} \right)
    \begin{matrix} R_1 + R_3 \\ R_2 - \frac{3}{5}R_3 \\ \phantom{x} \end{matrix} &
    \longrightarrow &
    \left( \begin{array}{ccc|c}
        1 & 0 & 0 & 0 \\
        0 & 1 & 0 & 0 \\
        0 & 0 & 1 & 0
    \end{array} \right) 
\end{align*} \end{split}
\end{equation*}
\sphinxAtStartPar
So this set of vectors is a basis for \(\mathbb{R}^3\), calculating the inverse of the coefficient matrix
\begin{equation*}
\begin{split} \begin{align*}
    \det 
    \begin{pmatrix} 1 & 0 & -1 \\ 2 & 5 & 1 \\ 0 & 7 & 3 \end{pmatrix} &=
    8 - 14 = -6, \\
    \operatorname{adj}
    \begin{pmatrix} 1 & 0 & -1 \\ 2 & 5 & 1 \\ 0 & 7 & 3 \end{pmatrix} &=
    \begin{pmatrix} 8 & -6 & 14 \\ -7 & 3 & -7 \\ 5 & -3 & 5 \end{pmatrix} =
    \begin{pmatrix} 8 & -7 & 5 \\ -6 & 3 & -3 \\ 14 & -7 & 5 \end{pmatrix}, \\
    \therefore
    \begin{pmatrix} 1 & 0 & -1 \\ 2 & 5 & 1 \\ 0 & 7 & 3 \end{pmatrix}^{-1} &=
    \frac{1}{6} \begin{pmatrix} -8 & 7 & -5 \\ 6 & -3 & 3 \\ -14 & 7 & -5 \end{pmatrix}.
\end{align*} \end{split}
\end{equation*}
\sphinxAtStartPar
Let \(U = \{(1, 2, 0), (0, 5, 7), (-1, 1, 3)\}\) then
\begin{equation*}
\begin{split} \begin{align*}
    \left[ \begin{pmatrix} 0 \\ 13 \\ 17 \end{pmatrix} \right]_U &= 
    \frac{1}{6} \begin{pmatrix} -8 & 7 & -5 \\ 6 & -3 & 3 \\ -14 & 7 & -5 \end{pmatrix}
    \begin{pmatrix} 0 \\ 13 \\ 17 \end{pmatrix} =
    \begin{pmatrix} 1 \\ 2 \\ 1 \end{pmatrix}, \\
    \left[ \begin{pmatrix} 2 \\ 3 \\ 1 \end{pmatrix} \right]_U &= 
    \frac{1}{6} \begin{pmatrix} -8 & 7 & -5 \\ 6 & -3 & 3 \\ -14 & 7 & -5 \end{pmatrix}
    \begin{pmatrix} 2 \\ 3 \\ 1 \end{pmatrix} =
    \begin{pmatrix} 0 \\ 1 \\ -2  \end{pmatrix}.
\end{align*} \end{split}
\end{equation*}\end{sphinxadmonition}
\phantomsection \label{_pages/A5_Vector_spaces_exercises_solutions:_pages/A5_Vector_spaces_exercises_solutions-solution-4}

\begin{sphinxadmonition}{note}{Solution to Exercise 5.5.5}



\sphinxAtStartPar
We need to find two vectors in \(\mathbb{R}^4\) that are linearly independent to \((1, 1, 2, 4)\) and \((2, -1, -5, 2)\) and one another. Let’s choose \((1, 0, 0, 0)\) and \((0, 1, 0, 0)\)  and check for linear dependence
\begin{equation*}
\begin{split} \begin{align*}
    & \left( \begin{array}{cccc|c}
        1 & 0 & 1 & 2 & 0 \\
        0 & 1 & 1 & -1 & 0 \\
        0 & 0 & 2 & -5 & 0 \\
        0 & 0 & 4 & 2 & 0
    \end{array} \right)
    \begin{matrix} \\ \\ \frac{1}{2}R_3 \\ \phantom{x} \end{matrix} &
    \longrightarrow &
    \left( \begin{array}{cccc|c}
        1 & 0 & 1 & 2 & 0 \\
        0 & 1 & 1 & -1 & 0 \\
        0 & 0 & 1 & -5/2 & 0 \\
        0 & 0 & 4 & 2 & 0
    \end{array} \right)
    \begin{matrix} R_1 - R_3 \\ R_2 - R_3 \\ \\ R_4 - 4R_1 \end{matrix} \\ \\
    \longrightarrow &
    \left( \begin{array}{cccc|c}
        1 & 0 & 0 & 9/2 & 0 \\
        0 & 1 & 0 & 3/2 & 0 \\
        0 & 0 & 1 & -5/2 & 0 \\
        0 & 0 & 0 & 12 & 0
    \end{array} \right)
    \begin{matrix} \\ \\ \\ \frac{1}{12}R_4 \end{matrix}  &
    \longrightarrow &
    \left( \begin{array}{cccc|c}
        1 & 0 & 0 & 9/2 & 0 \\
        0 & 1 & 0 & 3/2 & 0 \\
        0 & 0 & 1 & -5/2 & 0 \\
        0 & 0 & 0 & 1 & 0
    \end{array} \right)
    \begin{matrix} R_1 - \frac{9}{2}R_4 \\ R_2 - \frac{3}{2}R_4 \\ R_3 + \frac{5}{2}R_4 \\ \phantom{x} \end{matrix} \\ \\
    \longrightarrow &
    \left( \begin{array}{cccc|c}
        1 & 0 & 0 & 0 & 0 \\
        0 & 1 & 0 & 0 & 0 \\
        0 & 0 & 1 & 0 & 0 \\
        0 & 0 & 0 & 1 & 0
    \end{array} \right)
\end{align*} \end{split}
\end{equation*}
\sphinxAtStartPar
Therefore \(\{(1, 1, 2, 4), (2, -1, -5, 2), \mathbf{e}_1, \mathbf{e}_2 \}\) is a basis for \(\mathbb{R}^4\). Note that we could have used any two vectors in \(\mathbb{R}^4\) that form a linearly independent set of vectors.
\end{sphinxadmonition}
\phantomsection \label{_pages/A5_Vector_spaces_exercises_solutions:_pages/A5_Vector_spaces_exercises_solutions-solution-5}

\begin{sphinxadmonition}{note}{Solution to Exercise 5.5.6}



\sphinxAtStartPar
We need to find which of the vectors \(\mathbf{u}\), \(\mathbf{v}\), \(\mathbf{w}\), \(\mathbf{x}\) and \(\mathbf{y}\) are linearly dependent (and therefore remove them to from the basis).
\begin{equation*}
\begin{split} \begin{align*}
    & \left( \begin{array}{ccccc|c}
        1 & 1 & 2 & 1 & 1 & 0 \\
        2 & -1 & 1 & -1 & -1 & 0 \\
        3 & 2 & 5 & 0 & 0 & 0 \\
        4 & 0 & 3 & 3 & 4 & 0
    \end{array} \right)
    \begin{matrix} \\ R_2 - 2R_1 \\ R_3 - 3R_1 \\ R_4 - 4R_1 \end{matrix} &
    \longrightarrow &
    \left( \begin{array}{ccccc|c}
        1 & 1 & 2 & 1 & 1 & 0 \\
        0 & -3 & -3 & -3 & -3 & 0 \\
        0 & -1 & -1 & -3 & -3 & 0 \\
        0 & -4 & -5 & -1 & 0 & 0
    \end{array} \right)
    \begin{matrix} \\ -\frac{1}{3}R_2 \\ \phantom{x} \\ \phantom{x} \end{matrix} \\ \\
    \longrightarrow &
    \left( \begin{array}{ccccc|c}
        1 & 1 & 2 & 1 & 1 & 0 \\
        0 & 1 & 1 & 1 & 1 & 0 \\
        0 & -1 & -1 & -3 & -3 & 0 \\
        0 & -4 & -5 & -1 & 0 & 0
    \end{array} \right)
    \begin{matrix} R_1 - R_2 \\ \\ R_3 + R_2 \\ R_4 + 4R_2 \end{matrix} &
    \longrightarrow &
    \left( \begin{array}{ccccc|c}
        1 & 0 & 1 & 0 & 0 & 0 \\
        0 & 1 & 1 & 1 & 1 & 0 \\
        0 & 0 & 0 & -2 & -2 & 0 \\
        0 & 0 & -1 & 3 & 4 & 0
    \end{array} \right)
    \begin{matrix} \\ \\ R_3 \leftrightarrow R_4 \\ \phantom{x} \end{matrix} \\ \\
    \longrightarrow &
    \left( \begin{array}{ccccc|c}
        1 & 0 & 1 & 0 & 0 & 0 \\
        0 & 1 & 1 & 1 & 1 & 0 \\
        0 & 0 & -1 & 3 & 4 & 0 \\
        0 & 0 & 0 & -2 & -2 & 0 
    \end{array} \right)
    \begin{matrix} \\ \\ -R_3 \\ \phantom{x} \end{matrix} &
    \longrightarrow &
    \left( \begin{array}{ccccc|c}
        1 & 0 & 1 & 0 & 0 & 0 \\
        0 & 1 & 1 & 1 & 1 & 0 \\
        0 & 0 & 1 & -3 & -4 & 0 \\
        0 & 0 & 0 & -2 & -2 & 0 
    \end{array} \right)
    \begin{matrix} R_1 - R_3 \\ R_2 - R_3 \\ \phantom{x} \\ \phantom{x} \end{matrix} \\ \\
    \longrightarrow &
    \left( \begin{array}{ccccc|c}
        1 & 0 & 0 & 3 & 4 & 0 \\
        0 & 1 & 0 & 4 & 5 & 0 \\
        0 & 0 & 1 & -3 & -4 & 0 \\
        0 & 0 & 0 & -2 & -2 & 0 
    \end{array} \right) 
    \begin{matrix} \\  \\  \\ -\frac{1}{2}R_4 \end{matrix}  &
    \longrightarrow &
    \left( \begin{array}{ccccc|c}
        1 & 0 & 0 & 3 & 4 & 0 \\
        0 & 1 & 0 & 4 & 5 & 0 \\
        0 & 0 & 1 & -3 & -4 & 0 \\
        0 & 0 & 0 & 1 & 1 & 0 
    \end{array} \right)
    \begin{matrix} R_1 - 3 R_4 \\ R_2 - 4R_4 \\ R_3 + 3R_4 \\ \phantom{x} \end{matrix} \\ \\
    \longrightarrow &
    \left( \begin{array}{ccccc|c}
        1 & 0 & 0 & 0 & 1 & 0 \\
        0 & 1 & 0 & 0 & 1 & 0 \\
        0 & 0 & 1 & 0 & -1 & 0 \\
        0 & 0 & 0 & 1 & 1 & 0 
    \end{array} \right)
\end{align*} \end{split}
\end{equation*}
\sphinxAtStartPar
The fifth column does not have a pivot element so \(\mathbf{y}\) is linearly dependent on the other vectors, therefore a basis for \(W\) is \(\{ \mathbf{u}, \mathbf{v}, \mathbf{w}, \mathbf{x}\}\) and \(\dim(W) = 4\).
\end{sphinxadmonition}

\sphinxstepscope


\chapter{Linear Transformations Exercise Solutions}
\label{\detokenize{_pages/A6_Linear_transformations_exercises_solutions:linear-transformations-exercise-solutions}}\label{\detokenize{_pages/A6_Linear_transformations_exercises_solutions:transformations-exercise-solutions-section}}\label{\detokenize{_pages/A6_Linear_transformations_exercises_solutions::doc}}\phantomsection \label{_pages/A6_Linear_transformations_exercises_solutions:_pages/A6_Linear_transformations_exercises_solutions-solution-0}

\begin{sphinxadmonition}{note}{Solution to Exercise 6.5.1}



\sphinxAtStartPar
(a) Let \(\mathbf{u} = (u_1, u_2), \mathbf{v} = (v_1, v_2) \in \mathbb{R}^2\) and \(\alpha \in \mathbb{R}\)
\begin{equation*}
\begin{split} \begin{align*}
    T(\mathbf{u} + \alpha \mathbf{v}) 
    &= T\begin{pmatrix} u_1 + \alpha v_1 \\ u_2 + \alpha v_2 \end{pmatrix} 
    = \begin{pmatrix} 0 \\ u_2 + \alpha v_2 \end{pmatrix}, \\
    T(\mathbf{u}) + \alpha T(\mathbf{v})
    &= T\begin{pmatrix} u_1 \\ u_2 \end{pmatrix} + \alpha T \begin{pmatrix} v_1 \\ v_2 \end{pmatrix}
    = \begin{pmatrix} 0 \\ u_2 + \alpha v_2 \end{pmatrix},
\end{align*} \end{split}
\end{equation*}
\sphinxAtStartPar
therefore \(T\) is a linear transformation.

\sphinxAtStartPar
(b) \(T\) is not a linear transformation since
\begin{equation*}
\begin{split} \begin{align*}
    T\left( \begin{pmatrix} 1 \\ 0 \end{pmatrix} + \begin{pmatrix} 1 \\ 1 \end{pmatrix} \right) = \begin{pmatrix} 2 \\ 5 \end{pmatrix}, \\
    T\begin{pmatrix} 1 \\ 0 \end{pmatrix} + T\begin{pmatrix} 1 \\ 1 \end{pmatrix} = \begin{pmatrix} 2 \\ 10 \end{pmatrix}.
\end{align*} \end{split}
\end{equation*}
\sphinxAtStartPar
(c) Let \(\mathbf{u} = (u_1, u_2), \mathbf{v} = (v_1, v_2) \in \mathbb{R}^2\) and \(\alpha \in \mathbb{R}\)
\begin{equation*}
\begin{split} \begin{align*}
    T(\mathbf{u} + \alpha \mathbf{v})
    &= T\begin{pmatrix} u_1 + \alpha v_1 \\ u_2 + \alpha v_2 \end{pmatrix}
    = \begin{pmatrix} u_1 + \alpha v_1 \\ u_1 - u_2 + \alpha v_1 - \alpha v_2 \end{pmatrix}, \\
    T(\mathbf{u}) + \alpha T(\mathbf{v})
    &= T\begin{pmatrix} u_1 \\ u_2 \end{pmatrix} + \alpha T\begin{pmatrix} v_1 \\ v_2 \end{pmatrix}
    = \begin{pmatrix} u_1 + \alpha v_1 \\ u_1 + \alpha v_1 - u_2 - \alpha v_2\end{pmatrix},
\end{align*} \end{split}
\end{equation*}
\sphinxAtStartPar
therefore \(T\) is a linear transformation.

\sphinxAtStartPar
(d) Let \(\mathbf{u} = (u_1, u_2, v_3),\mathbf{v} = (v_1, v_2, v_3)\in \mathbb{R}^3\) and \(\alpha \in \mathbb{R}\)
\begin{equation*}
\begin{split} \begin{align*}
    T(\mathbf{u} + \alpha \mathbf{v}) 
    &= T\begin{pmatrix} u_1 + \alpha v_1 \\ u_2 + \alpha v_2 \\ u_3 + \alpha v_3 \end{pmatrix}
    = \begin{pmatrix} u_1 + u_2 + \alpha v_1 + \alpha v_2 \\  u_3 + \alpha v_3 \end{pmatrix} , \\
    T(\mathbf{u}) + \alpha T(\mathbf{v}) 
    &= T\begin{pmatrix} u_1 \\ u_2 \\ u_3 \end{pmatrix} + \alpha \begin{pmatrix} v_1 \\ v_2 \\ v_3 \end{pmatrix}
    = \begin{pmatrix} u_1 + \alpha v_1 + u_2 + \alpha v_2 \\  u_3 + \alpha v_3 \end{pmatrix}
\end{align*} \end{split}
\end{equation*}
\sphinxAtStartPar
therefore \(T\) is a linear transformation.

\sphinxAtStartPar
(e) \(T\) is not a linear transformation since
\begin{equation*}
\begin{split} \begin{align*}
    T\left( \begin{pmatrix} 1 \\ 0 \end{pmatrix} + \begin{pmatrix} 1 \\ 1 \end{pmatrix} \right)
    &= T \begin{pmatrix} 2 \\ 1 \end{pmatrix} 
    = \begin{pmatrix} 7 \\ 1 \end{pmatrix}, \\
    T\begin{pmatrix} 1 \\ 0 \end{pmatrix} + T \begin{pmatrix} 1 \\ 1 \end{pmatrix} 
    &= \begin{pmatrix} 4 \\ 0 \end{pmatrix} + \begin{pmatrix} 4 \\ 1 \end{pmatrix} 
    = \begin{pmatrix} 8 \\ 1 \end{pmatrix}.
\end{align*} \end{split}
\end{equation*}
\sphinxAtStartPar
(f) Let \(u = f(x), v = g(x) \in P(\mathbb{R})\) and \(\alpha \in \mathbb{R}\):
\begin{equation*}
\begin{split} \begin{align*}
    T(u + \alpha v) &= T(f(x) + \alpha g(x)) = \frac{\mathrm{d}}{\mathrm{d} x}(f(x) + \alpha g(x)) 
    = \frac{\mathrm{d}}{\mathrm{d} x}f(x) + \alpha \frac{\mathrm{d}}{\mathrm{d} x} g(x), \\
    T(u) + \alpha T(v) &= T(f(x)) + \alpha T(g(x)) 
    = \frac{\mathrm{d}}{\mathrm{d} x}f(x) + \alpha \frac{\mathrm{d}}{\mathrm{d} x}g(x),
\end{align*} \end{split}
\end{equation*}
\sphinxAtStartPar
therefore \(T\) is a linear transformation.

\sphinxAtStartPar
(g) Let \(u = f(x), v = g(x) \in P(\mathbb{R})\) and \(\alpha \in \mathbb{R}\):
\begin{equation*}
\begin{split} \begin{align*}
    T(u + \alpha v) &= T(f(x) + \alpha g(x)) = x(f(x) + \alpha g(x)) = xf(x) + \alpha x g(x), \\
    T(u) + \alpha T(v) &= T(f(x)) + \alpha T(g(x)) = xf(x) + \alpha x g(x),
\end{align*} \end{split}
\end{equation*}
\sphinxAtStartPar
therefore \(T\) is a linear transformation.
\end{sphinxadmonition}
\phantomsection \label{_pages/A6_Linear_transformations_exercises_solutions:_pages/A6_Linear_transformations_exercises_solutions-solution-1}

\begin{sphinxadmonition}{note}{Solution to Exercise 6.5.2}



\sphinxAtStartPar
The transformation matrix is
\begin{equation*}
\begin{split} A = \begin{pmatrix} -1 & 3 \\ 1 & -4 \end{pmatrix} \end{split}
\end{equation*}
\sphinxAtStartPar
Calculating \(T (2, 5)\)
\begin{equation*}
\begin{split} T 
\begin{pmatrix} 2 \\ 5 \end{pmatrix} = 
\begin{pmatrix} -1 & 3 \\ 1 & -4 \end{pmatrix} 
\begin{pmatrix} 2 \\ 5 \end{pmatrix}
= \begin{pmatrix} 13 \\ -18 \end{pmatrix}. \end{split}
\end{equation*}\end{sphinxadmonition}
\phantomsection \label{_pages/A6_Linear_transformations_exercises_solutions:_pages/A6_Linear_transformations_exercises_solutions-solution-2}

\begin{sphinxadmonition}{note}{Solution to Exercise 6.5.3}



\sphinxAtStartPar
The transformation matrix is
\begin{equation*}
\begin{split} A = \begin{pmatrix} 1 & -2 \\ 2 & 3 \end{pmatrix}, \end{split}
\end{equation*}
\sphinxAtStartPar
so the inverse is
\begin{equation*}
\begin{split}  A^{-1} = \frac{1}{7}
\begin{pmatrix} 
    3 & 2 \\
    -2 & 1
\end{pmatrix}. \end{split}
\end{equation*}
\sphinxAtStartPar
Therefore
\begin{equation*}
\begin{split} \mathbf{u} = A^{-1} 
\begin{pmatrix} -1 \\ 5 \end{pmatrix} =  \frac{1}{7}
\begin{pmatrix} 
    3 & 2 \\
    -2 & 1
\end{pmatrix}
\begin{pmatrix} -1 \\ 5 \end{pmatrix} = 
\begin{pmatrix} 1 \\ 1 \end{pmatrix}. \end{split}
\end{equation*}\end{sphinxadmonition}
\phantomsection \label{_pages/A6_Linear_transformations_exercises_solutions:_pages/A6_Linear_transformations_exercises_solutions-solution-3}

\begin{sphinxadmonition}{note}{Solution to Exercise 6.5.4}



\sphinxAtStartPar
The transformation matrix is determined using equation \eqref{equation:_pages/6.1_Transformation_matrices:determining-the-transformation-matrix} which is
\begin{equation*}
\begin{split}A = \begin{pmatrix} T(\mathbf{u}_1) & T(\mathbf{u}_2) & \cdots & T(\mathbf{u}_n) \end{pmatrix}
\begin{pmatrix} \mathbf{u}_1 & \mathbf{u}_2 & \cdots & \mathbf{u}_n \end{pmatrix}^{-1}.\end{split}
\end{equation*}
\sphinxAtStartPar
Using Gauss\sphinxhyphen{}Jordan elimination to calculate the inverse of \((\mathbf{u}_1, \mathbf{u}_2, \mathbf{u}_3)^{-1}\)
\begin{equation*}
\begin{split} \begin{align*}
    & \left( \begin{array}{rrr|rrr}
       1 & 0 & -1 & 1 & 0 & 0 \\
       -1 & 1 & 1 & 0 & 1 & 0 \\
       0 & 2 & 1 & 0 & 0 & 1
    \end{array} \right)
    \begin{array}{l} \\ R_2 + R_1 \\ \phantom{x} \end{array} \\ \\ 
    \longrightarrow \qquad 
    & \left( \begin{array}{rrr|rrr}
       1 & 0 & -1 & 1 & 0 & 0 \\
       0 & 1 & 0 & 1 & 1 & 0 \\
       0 & 2 & 1 & 0 & 0 & 1
    \end{array} \right)
    \begin{array}{l} \\ \\ R_3 - 2 R_2 \end{array} \\ \\ 
    \longrightarrow \qquad  
    & \left( \begin{array}{rrr|rrr}
       1 & 0 & -1 & 1 & 0 & 0 \\
       0 & 1 & 0 & 1 & 1 & 0 \\
       0 & 0 & 1 & -2 & -2 & 1
    \end{array} \right)
    \begin{array}{l} R_1 + R_3 \\ \phantom{x} \\ \phantom{x} \end{array} \\ \\ 
    \longrightarrow \qquad
    & \left( \begin{array}{rrr|rrr}
       1 & 0 & 0 & -1 & -2 & 1 \\
       0 & 1 & 0 & 1 & 1 & 0 \\
       0 & 0 & 1 & -2 & -2 & 1
    \end{array} \right)
\end{align*} \end{split}
\end{equation*}
\sphinxAtStartPar
So \((\mathbf{u}_1, \mathbf{u}_2, \mathbf{u}_3)^{-1} = \begin{pmatrix}  -1 & -2 & 1 \\ 1 & 1 & 0 \\ -2 & -2 & 1 \end{pmatrix}\) and
\begin{equation*}
\begin{split} \begin{align*}
    A &= \begin{pmatrix} 1 & 6 & 2 \\ -2 & 5 & 4 \\ -4 & 10 & 7 \end{pmatrix}
    \begin{pmatrix} 
        -1 & -2 & 1 \\
        1 & 1 & 0 \\
        -2 & -2 & 1
    \end{pmatrix}
    = \begin{pmatrix} 1 & 0 & 3 \\ -1 & 1 & 2 \\ 0 & 4 & 3 \end{pmatrix}.
\end{align*} \end{split}
\end{equation*}
\sphinxAtStartPar
Checking \(A\)
\begin{equation*}
\begin{split} T \begin{pmatrix} 1 \\ -1 \\ 0 \end{pmatrix} =
\begin{pmatrix} 1 & 0 & 3 \\ -1 & 1 & 2 \\ 0 & 4 & 3 \end{pmatrix}
\begin{pmatrix}1 \\ -1 \\ 0 \end{pmatrix} = 
\begin{pmatrix} 1 \\ -2 \\ - 4 \end{pmatrix} \quad \checkmark \end{split}
\end{equation*}\end{sphinxadmonition}
\phantomsection \label{_pages/A6_Linear_transformations_exercises_solutions:_pages/A6_Linear_transformations_exercises_solutions-solution-4}

\begin{sphinxadmonition}{note}{Solution to Exercise 6.5.5}



\sphinxAtStartPar
The transformation matrix is
\begin{equation*}
\begin{split} \begin{align*}
    Rot\left(\pi/6\right) &= \begin{pmatrix} 
        \cos(\pi/6) & -\sin(\pi/6) \\
        \sin(\pi/6) & \cos(\pi/6)
    \end{pmatrix}  \\
    &= \begin{pmatrix}
        \sqrt{3}/2 & -1/2 \\
        1/2 & \sqrt{3}/2
    \end{pmatrix}
\end{align*} \end{split}
\end{equation*}
\sphinxAtStartPar
therefore
\begin{equation*}
\begin{split} \begin{align*}
    Rot\left(\frac{\pi}{6}\right) \begin{pmatrix} 2 \\ 1 \end{pmatrix}
    &= \begin{pmatrix}
        \sqrt{3}/2 & -1/2 \\
        1/2 & \sqrt{3}/2
    \end{pmatrix}
    \begin{pmatrix} 2 \\ 1 \end{pmatrix} \\
    &= \begin{pmatrix} \sqrt{3} - 1/2 \\ 1 + \sqrt{3}/2 \end{pmatrix}
    \approx \begin{pmatrix} 1.2321 \\ 1.8660 \end{pmatrix}
\end{align*} \end{split}
\end{equation*}\end{sphinxadmonition}
\phantomsection \label{_pages/A6_Linear_transformations_exercises_solutions:_pages/A6_Linear_transformations_exercises_solutions-solution-5}

\begin{sphinxadmonition}{note}{Solution to Exercise 6.5.6}



\sphinxAtStartPar
The transformation matrix is
\begin{equation*}
\begin{split} Re\!f \left(\pi/3\right) =
\begin{pmatrix} 
    \cos(2\pi/3) & \sin(2\pi/3) \\
    \sin(2\pi/3) & -\cos(2\pi/3)
\end{pmatrix}
\begin{pmatrix}
    -1/2 & \sqrt{3}/2 \\
    \sqrt{3}/2 & 1/2
\end{pmatrix} \end{split}
\end{equation*}
\sphinxAtStartPar
therefore
\begin{equation*}
\begin{split} \begin{align*}
Re\!f \left(\pi/3\right)
\begin{pmatrix} 5 \\ 3 \end{pmatrix} 
&= \begin{pmatrix}
    -1/2 & \sqrt{3}/2 \\
    \sqrt{3}/2 & 1/2
\end{pmatrix} 
\begin{pmatrix} 5 \\ 3 \end{pmatrix} \\
&= 
\begin{pmatrix} 3\sqrt{3}/2 - 5/2 \\ 3/2 + 5\sqrt{3}/2 \end{pmatrix} 
\approx \begin{pmatrix} 0.0981 \\ 5.8301 \end{pmatrix}.
\end{align*} \end{split}
\end{equation*}\end{sphinxadmonition}
\phantomsection \label{_pages/A6_Linear_transformations_exercises_solutions:_pages/A6_Linear_transformations_exercises_solutions-solution-6}

\begin{sphinxadmonition}{note}{Solution to Exercise 6.5.7}



\sphinxAtStartPar
(a) \(\begin{pmatrix} 
    2 & 4 & 4 & 2 \\
    1 & 1 & 3 & 3 \\
    1 & 1 & 1 & 1 
\end{pmatrix} \)

\sphinxAtStartPar
(b) Translate by \((-3, -2)\) so that the centre of the square is at the origin:
\begin{equation*}
\begin{split} \begin{align*} 
    T \begin{pmatrix} -3 \\ -2 \end{pmatrix} = 
    \begin{pmatrix}
        1 & 0 & -3 \\
        0 & 1 & -2 \\
        0 & 0 & 1 
    \end{pmatrix}
\end{align*} \end{split}
\end{equation*}
\sphinxAtStartPar
Rotate by \(\pi/3\) clockwise about the origin:
\begin{equation*}
\begin{split} \begin{align*}
    Rot\left(-\frac{\pi}{3}\right) &=
    \begin{pmatrix}
        \cos(\pi/3) & \sin(\pi/3) & 0 \\
        -\sin(\pi/3) & \cos(\pi/3) & 0 \\
        0 & 0 & 1
    \end{pmatrix} \\
    &= \begin{pmatrix}
        1/2 & \sqrt{3}/2 & 0 \\
        -\sqrt{3}/2 & 1/2 & 0 \\
        0 & 0 & 1
    \end{pmatrix} \\
    &\approx \begin{pmatrix}
        0.5 & 0.8660 & 0 \\
        -0.8660 & 0.5 & 0 \\
        0 & 0 & 1
    \end{pmatrix}
\end{align*} \end{split}
\end{equation*}
\sphinxAtStartPar
Translate by \((3, 2)\) so that the centre of the square is back to \(\mathbf{c}\)
\begin{equation*}
\begin{split} \begin{align*}
    T \begin{pmatrix} 3 \\ 2 \end{pmatrix} &= 
    \begin{pmatrix}
        1 & 0 & 3 \\
        0 & 1 & 2 \\
        0 & 0 & 1
    \end{pmatrix}
\end{align*} \end{split}
\end{equation*}
\sphinxAtStartPar
(c) Calculate composite alignment matrix
\begin{equation*}
\begin{split} \begin{align*}
    A &= T \begin{pmatrix} 3 \\ 2 \end{pmatrix} \cdot Rot\left(-\frac{\pi}{3}\right) \cdot T \begin{pmatrix} -3 \\ -2 \end{pmatrix} \\
    &= \begin{pmatrix}
        1 & 0 & 3 \\
        0 & 1 & 2 \\
        0 & 0 & 1
    \end{pmatrix}
    \begin{pmatrix}
        1/2 & \sqrt{3}/2 & 0 \\
        -\sqrt{3}/2 & 1/2 & 0 \\
        0 & 0 & 1
    \end{pmatrix}
    \begin{pmatrix}
        1 & 0 & -3 \\
        0 & 1 & -2 \\
        0 & 0 & 1 
    \end{pmatrix} \\
    &= \begin{pmatrix}
        1/2 & \sqrt{3}/2 & 3/2 - \sqrt{3} \\
        -\sqrt{3}/2 & 1/2 & 1 + 3\sqrt{3}/2 \\
        0 & 0 & 1 
    \end{pmatrix} \\
    &\approx
    \begin{pmatrix}
        0.5 & 0.8660 & -0.2321 \\
        -0.8660 & 0.5 & 3.5981 \\
        0 & 0 & 1
    \end{pmatrix}.
\end{align*} \end{split}
\end{equation*}
\sphinxAtStartPar
Apply composite transformation matrix
\begin{equation*}
\begin{split} \begin{align*}
    AP &= \begin{pmatrix}
        1/2 & \sqrt{3}/2 & 3/2 - \sqrt{3} \\
        -\sqrt{3}/2 & 1/2 & 1 + 3\sqrt{3}/2 \\
        0 & 0 & 1 
    \end{pmatrix}
    \begin{pmatrix} 
        2 & 4 & 4 & 2 \\
        1 & 1 & 3 & 3 \\
        1 & 1 & 1 & 1 
    \end{pmatrix} \\
    &= \begin{pmatrix}
        5/2 - \sqrt{3}/2 & 7/2 - \sqrt{3}/2 & \sqrt{3}/2 + 7/2 & \sqrt{3}/2 + 5/2 \\
        \sqrt{3}/2 + 3/2 & 3/2 - \sqrt{3}/2 & 5/2 - \sqrt{3}/2 & \sqrt{3}/2 + 5/2 \\
        1 & 1 & 1 & 1
    \end{pmatrix} \\
    &\approx \begin{pmatrix}
        1.634 & 2.634 & 4.366 & 3.366 \\
        2.366 & 0.634 & 1.634 & 3.366 \\
        1 & 1 & 1 & 1 
    \end{pmatrix}
\end{align*} \end{split}
\end{equation*}
\begin{figure}[htbp]
\centering

\noindent\sphinxincludegraphics{{a8d3db2cb174f98244734da319c7bba2f78b74192befe388dbb30a454f8b261f}.png}
\end{figure}
\end{sphinxadmonition}

\sphinxstepscope


\part{Part}

\sphinxstepscope


\chapter{Index}
\label{\detokenize{genindex:index}}\label{\detokenize{genindex::doc}}





\renewcommand{\indexname}{Proof Index}
\begin{sphinxtheindex}
\let\bigletter\sphinxstyleindexlettergroup
\bigletter{2x2\sphinxhyphen{}determinant\sphinxhyphen{}definition}
\item\relax\sphinxstyleindexentry{2x2\sphinxhyphen{}determinant\sphinxhyphen{}definition}\sphinxstyleindexextra{\_pages/1.4\_Determinants}\sphinxstyleindexpageref{_pages/1.4_Determinants:\detokenize{2x2-determinant-definition}}
\indexspace
\bigletter{2x2\sphinxhyphen{}determinant\sphinxhyphen{}example}
\item\relax\sphinxstyleindexentry{2x2\sphinxhyphen{}determinant\sphinxhyphen{}example}\sphinxstyleindexextra{\_pages/1.4\_Determinants}\sphinxstyleindexpageref{_pages/1.4_Determinants:\detokenize{2x2-determinant-example}}
\indexspace
\bigletter{4x4\sphinxhyphen{}determinant\sphinxhyphen{}example}
\item\relax\sphinxstyleindexentry{4x4\sphinxhyphen{}determinant\sphinxhyphen{}example}\sphinxstyleindexextra{\_pages/1.4\_Determinants}\sphinxstyleindexpageref{_pages/1.4_Determinants:\detokenize{4x4-determinant-example}}
\indexspace
\bigletter{adjoint\sphinxhyphen{}definition}
\item\relax\sphinxstyleindexentry{adjoint\sphinxhyphen{}definition}\sphinxstyleindexextra{\_pages/1.5\_Inverse\_matrix}\sphinxstyleindexpageref{_pages/1.5_Inverse_matrix:\detokenize{adjoint-definition}}
\indexspace
\bigletter{adjoint\sphinxhyphen{}determinant\sphinxhyphen{}formula\sphinxhyphen{}theorem}
\item\relax\sphinxstyleindexentry{adjoint\sphinxhyphen{}determinant\sphinxhyphen{}formula\sphinxhyphen{}theorem}\sphinxstyleindexextra{\_pages/1.5\_Inverse\_matrix}\sphinxstyleindexpageref{_pages/1.5_Inverse_matrix:\detokenize{adjoint-determinant-formula-theorem}}
\indexspace
\bigletter{adjoint\sphinxhyphen{}example}
\item\relax\sphinxstyleindexentry{adjoint\sphinxhyphen{}example}\sphinxstyleindexextra{\_pages/1.5\_Inverse\_matrix}\sphinxstyleindexpageref{_pages/1.5_Inverse_matrix:\detokenize{adjoint-example}}
\indexspace
\bigletter{augmented\sphinxhyphen{}matrix\sphinxhyphen{}definition}
\item\relax\sphinxstyleindexentry{augmented\sphinxhyphen{}matrix\sphinxhyphen{}definition}\sphinxstyleindexextra{\_pages/2.3\_Gaussian\_elimination}\sphinxstyleindexpageref{_pages/2.3_Gaussian_elimination:\detokenize{augmented-matrix-definition}}
\indexspace
\bigletter{basis\sphinxhyphen{}definition}
\item\relax\sphinxstyleindexentry{basis\sphinxhyphen{}definition}\sphinxstyleindexextra{\_pages/5.4\_Basis}\sphinxstyleindexpageref{_pages/5.4_Basis:\detokenize{basis-definition}}
\indexspace
\bigletter{basis\sphinxhyphen{}example}
\item\relax\sphinxstyleindexentry{basis\sphinxhyphen{}example}\sphinxstyleindexextra{\_pages/5.4\_Basis}\sphinxstyleindexpageref{_pages/5.4_Basis:\detokenize{basis-example}}
\indexspace
\bigletter{change\sphinxhyphen{}of\sphinxhyphen{}basis\sphinxhyphen{}example}
\item\relax\sphinxstyleindexentry{change\sphinxhyphen{}of\sphinxhyphen{}basis\sphinxhyphen{}example}\sphinxstyleindexextra{\_pages/5.4\_Basis}\sphinxstyleindexpageref{_pages/5.4_Basis:\detokenize{change-of-basis-example}}
\indexspace
\bigletter{change\sphinxhyphen{}of\sphinxhyphen{}basis\sphinxhyphen{}matrix\sphinxhyphen{}definition}
\item\relax\sphinxstyleindexentry{change\sphinxhyphen{}of\sphinxhyphen{}basis\sphinxhyphen{}matrix\sphinxhyphen{}definition}\sphinxstyleindexextra{\_pages/5.4\_Basis}\sphinxstyleindexpageref{_pages/5.4_Basis:\detokenize{change-of-basis-matrix-definition}}
\indexspace
\bigletter{cofactor\sphinxhyphen{}definition}
\item\relax\sphinxstyleindexentry{cofactor\sphinxhyphen{}definition}\sphinxstyleindexextra{\_pages/1.4\_Determinants}\sphinxstyleindexpageref{_pages/1.4_Determinants:\detokenize{cofactor-definition}}
\indexspace
\bigletter{composite\sphinxhyphen{}transformation\sphinxhyphen{}definition}
\item\relax\sphinxstyleindexentry{composite\sphinxhyphen{}transformation\sphinxhyphen{}definition}\sphinxstyleindexextra{\_pages/6.2\_Composite\_transformations}\sphinxstyleindexpageref{_pages/6.2_Composite_transformations:\detokenize{composite-transformation-definition}}
\indexspace
\bigletter{composite\sphinxhyphen{}transformation\sphinxhyphen{}matrices\sphinxhyphen{}theorem}
\item\relax\sphinxstyleindexentry{composite\sphinxhyphen{}transformation\sphinxhyphen{}matrices\sphinxhyphen{}theorem}\sphinxstyleindexextra{\_pages/6.2\_Composite\_transformations}\sphinxstyleindexpageref{_pages/6.2_Composite_transformations:\detokenize{composite-transformation-matrices-theorem}}
\indexspace
\bigletter{composite\sphinxhyphen{}transformation\sphinxhyphen{}matrix\sphinxhyphen{}example}
\item\relax\sphinxstyleindexentry{composite\sphinxhyphen{}transformation\sphinxhyphen{}matrix\sphinxhyphen{}example}\sphinxstyleindexextra{\_pages/6.2\_Composite\_transformations}\sphinxstyleindexpageref{_pages/6.2_Composite_transformations:\detokenize{composite-transformation-matrix-example}}
\indexspace
\bigletter{consistent\sphinxhyphen{}inconsistent\sphinxhyphen{}and\sphinxhyphen{}indeterminate\sphinxhyphen{}systems\sphinxhyphen{}definition}
\item\relax\sphinxstyleindexentry{consistent\sphinxhyphen{}inconsistent\sphinxhyphen{}and\sphinxhyphen{}indeterminate\sphinxhyphen{}systems\sphinxhyphen{}definition}\sphinxstyleindexextra{\_pages/2.6\_Consistent\_systems}\sphinxstyleindexpageref{_pages/2.6_Consistent_systems:\detokenize{consistent-inconsistent-and-indeterminate-systems-definition}}
\indexspace
\bigletter{consistent\sphinxhyphen{}system\sphinxhyphen{}theorem}
\item\relax\sphinxstyleindexentry{consistent\sphinxhyphen{}system\sphinxhyphen{}theorem}\sphinxstyleindexextra{\_pages/2.6\_Consistent\_systems}\sphinxstyleindexpageref{_pages/2.6_Consistent_systems:\detokenize{consistent-system-theorem}}
\indexspace
\bigletter{cramers\sphinxhyphen{}rule\sphinxhyphen{}example}
\item\relax\sphinxstyleindexentry{cramers\sphinxhyphen{}rule\sphinxhyphen{}example}\sphinxstyleindexextra{\_pages/2.2\_Cramers\_rule}\sphinxstyleindexpageref{_pages/2.2_Cramers_rule:\detokenize{cramers-rule-example}}
\indexspace
\bigletter{cramers\sphinxhyphen{}rule\sphinxhyphen{}theorem}
\item\relax\sphinxstyleindexentry{cramers\sphinxhyphen{}rule\sphinxhyphen{}theorem}\sphinxstyleindexextra{\_pages/2.2\_Cramers\_rule}\sphinxstyleindexpageref{_pages/2.2_Cramers_rule:\detokenize{cramers-rule-theorem}}
\indexspace
\bigletter{cross\sphinxhyphen{}product\sphinxhyphen{}definition}
\item\relax\sphinxstyleindexentry{cross\sphinxhyphen{}product\sphinxhyphen{}definition}\sphinxstyleindexextra{\_pages/3.3\_Dot\_and\_cross\_products}\sphinxstyleindexpageref{_pages/3.3_Dot_and_cross_products:\detokenize{cross-product-definition}}
\indexspace
\bigletter{cross\sphinxhyphen{}product\sphinxhyphen{}example}
\item\relax\sphinxstyleindexentry{cross\sphinxhyphen{}product\sphinxhyphen{}example}\sphinxstyleindexextra{\_pages/3.3\_Dot\_and\_cross\_products}\sphinxstyleindexpageref{_pages/3.3_Dot_and_cross_products:\detokenize{cross-product-example}}
\indexspace
\bigletter{cross\sphinxhyphen{}product\sphinxhyphen{}properties\sphinxhyphen{}theorem}
\item\relax\sphinxstyleindexentry{cross\sphinxhyphen{}product\sphinxhyphen{}properties\sphinxhyphen{}theorem}\sphinxstyleindexextra{\_pages/3.3\_Dot\_and\_cross\_products}\sphinxstyleindexpageref{_pages/3.3_Dot_and_cross_products:\detokenize{cross-product-properties-theorem}}
\indexspace
\bigletter{definition\sphinxhyphen{}0}
\item\relax\sphinxstyleindexentry{definition\sphinxhyphen{}0}\sphinxstyleindexextra{\_pages/0.3\_Mathematical\_preliminaries}\sphinxstyleindexpageref{_pages/0.3_Mathematical_preliminaries:\detokenize{definition-0}}
\indexspace
\bigletter{dot\sphinxhyphen{}product\sphinxhyphen{}definition}
\item\relax\sphinxstyleindexentry{dot\sphinxhyphen{}product\sphinxhyphen{}definition}\sphinxstyleindexextra{\_pages/3.3\_Dot\_and\_cross\_products}\sphinxstyleindexpageref{_pages/3.3_Dot_and_cross_products:\detokenize{dot-product-definition}}
\indexspace
\bigletter{dot\sphinxhyphen{}product\sphinxhyphen{}example}
\item\relax\sphinxstyleindexentry{dot\sphinxhyphen{}product\sphinxhyphen{}example}\sphinxstyleindexextra{\_pages/3.3\_Dot\_and\_cross\_products}\sphinxstyleindexpageref{_pages/3.3_Dot_and_cross_products:\detokenize{dot-product-example}}
\indexspace
\bigletter{dot\sphinxhyphen{}product\sphinxhyphen{}properties\sphinxhyphen{}theorem}
\item\relax\sphinxstyleindexentry{dot\sphinxhyphen{}product\sphinxhyphen{}properties\sphinxhyphen{}theorem}\sphinxstyleindexextra{\_pages/3.3\_Dot\_and\_cross\_products}\sphinxstyleindexpageref{_pages/3.3_Dot_and_cross_products:\detokenize{dot-product-properties-theorem}}
\indexspace
\bigletter{elementary\sphinxhyphen{}matrices\sphinxhyphen{}definition}
\item\relax\sphinxstyleindexentry{elementary\sphinxhyphen{}matrices\sphinxhyphen{}definition}\sphinxstyleindexextra{\_pages/2.5\_Gauss\_Jordan\_elimination}\sphinxstyleindexpageref{_pages/2.5_Gauss_Jordan_elimination:\detokenize{elementary-matrices-definition}}
\indexspace
\bigletter{elementary\sphinxhyphen{}matrix\sphinxhyphen{}multiplication\sphinxhyphen{}theorem}
\item\relax\sphinxstyleindexentry{elementary\sphinxhyphen{}matrix\sphinxhyphen{}multiplication\sphinxhyphen{}theorem}\sphinxstyleindexextra{\_pages/2.5\_Gauss\_Jordan\_elimination}\sphinxstyleindexpageref{_pages/2.5_Gauss_Jordan_elimination:\detokenize{elementary-matrix-multiplication-theorem}}
\indexspace
\bigletter{ero\sphinxhyphen{}definition}
\item\relax\sphinxstyleindexentry{ero\sphinxhyphen{}definition}\sphinxstyleindexextra{\_pages/2.3\_Gaussian\_elimination}\sphinxstyleindexpageref{_pages/2.3_Gaussian_elimination:\detokenize{ero-definition}}
\indexspace
\bigletter{example\sphinxhyphen{}2}
\item\relax\sphinxstyleindexentry{example\sphinxhyphen{}2}\sphinxstyleindexextra{\_pages/0.3\_Mathematical\_preliminaries}\sphinxstyleindexpageref{_pages/0.3_Mathematical_preliminaries:\detokenize{example-2}}
\indexspace
\bigletter{field\sphinxhyphen{}definition}
\item\relax\sphinxstyleindexentry{field\sphinxhyphen{}definition}\sphinxstyleindexextra{\_pages/5.1\_Vector\_spaces\_definitions}\sphinxstyleindexpageref{_pages/5.1_Vector_spaces_definitions:\detokenize{field-definition}}
\indexspace
\bigletter{finding\sphinxhyphen{}transformation\sphinxhyphen{}matrix\sphinxhyphen{}theorem}
\item\relax\sphinxstyleindexentry{finding\sphinxhyphen{}transformation\sphinxhyphen{}matrix\sphinxhyphen{}theorem}\sphinxstyleindexextra{\_pages/6.1\_Transformation\_matrices}\sphinxstyleindexpageref{_pages/6.1_Transformation_matrices:\detokenize{finding-transformation-matrix-theorem}}
\indexspace
\bigletter{gauss\sphinxhyphen{}jordan\sphinxhyphen{}inverse\sphinxhyphen{}example}
\item\relax\sphinxstyleindexentry{gauss\sphinxhyphen{}jordan\sphinxhyphen{}inverse\sphinxhyphen{}example}\sphinxstyleindexextra{\_pages/2.5\_Gauss\_Jordan\_elimination}\sphinxstyleindexpageref{_pages/2.5_Gauss_Jordan_elimination:\detokenize{gauss-jordan-inverse-example}}
\indexspace
\bigletter{ge\sphinxhyphen{}algorithm}
\item\relax\sphinxstyleindexentry{ge\sphinxhyphen{}algorithm}\sphinxstyleindexextra{\_pages/2.3\_Gaussian\_elimination}\sphinxstyleindexpageref{_pages/2.3_Gaussian_elimination:\detokenize{ge-algorithm}}
\indexspace
\bigletter{ge\sphinxhyphen{}example}
\item\relax\sphinxstyleindexentry{ge\sphinxhyphen{}example}\sphinxstyleindexextra{\_pages/2.3\_Gaussian\_elimination}\sphinxstyleindexpageref{_pages/2.3_Gaussian_elimination:\detokenize{ge-example}}
\indexspace
\bigletter{ge\sphinxhyphen{}pp\sphinxhyphen{}algorithm}
\item\relax\sphinxstyleindexentry{ge\sphinxhyphen{}pp\sphinxhyphen{}algorithm}\sphinxstyleindexextra{\_pages/2.4\_Partial\_pivoting}\sphinxstyleindexpageref{_pages/2.4_Partial_pivoting:\detokenize{ge-pp-algorithm}}
\indexspace
\bigletter{gje\sphinxhyphen{}algorithm}
\item\relax\sphinxstyleindexentry{gje\sphinxhyphen{}algorithm}\sphinxstyleindexextra{\_pages/2.5\_Gauss\_Jordan\_elimination}\sphinxstyleindexpageref{_pages/2.5_Gauss_Jordan_elimination:\detokenize{gje-algorithm}}
\indexspace
\bigletter{gje\sphinxhyphen{}example}
\item\relax\sphinxstyleindexentry{gje\sphinxhyphen{}example}\sphinxstyleindexextra{\_pages/2.5\_Gauss\_Jordan\_elimination}\sphinxstyleindexpageref{_pages/2.5_Gauss_Jordan_elimination:\detokenize{gje-example}}
\indexspace
\bigletter{homogeneous\sphinxhyphen{}co\sphinxhyphen{}ordinates\sphinxhyphen{}definition}
\item\relax\sphinxstyleindexentry{homogeneous\sphinxhyphen{}co\sphinxhyphen{}ordinates\sphinxhyphen{}definition}\sphinxstyleindexextra{\_pages/6.4\_Translation}\sphinxstyleindexpageref{_pages/6.4_Translation:\detokenize{homogeneous-co-ordinates-definition}}
\indexspace
\bigletter{homogeneous\sphinxhyphen{}system\sphinxhyphen{}definition}
\item\relax\sphinxstyleindexentry{homogeneous\sphinxhyphen{}system\sphinxhyphen{}definition}\sphinxstyleindexextra{\_pages/2.7\_Homogeneous\_systems}\sphinxstyleindexpageref{_pages/2.7_Homogeneous_systems:\detokenize{homogeneous-system-definition}}
\indexspace
\bigletter{homogeneous\sphinxhyphen{}systems\sphinxhyphen{}example}
\item\relax\sphinxstyleindexentry{homogeneous\sphinxhyphen{}systems\sphinxhyphen{}example}\sphinxstyleindexextra{\_pages/2.7\_Homogeneous\_systems}\sphinxstyleindexpageref{_pages/2.7_Homogeneous_systems:\detokenize{homogeneous-systems-example}}
\indexspace
\bigletter{inconsistent\sphinxhyphen{}system\sphinxhyphen{}theorem}
\item\relax\sphinxstyleindexentry{inconsistent\sphinxhyphen{}system\sphinxhyphen{}theorem}\sphinxstyleindexextra{\_pages/2.6\_Consistent\_systems}\sphinxstyleindexpageref{_pages/2.6_Consistent_systems:\detokenize{inconsistent-system-theorem}}
\indexspace
\bigletter{indeterminate\sphinxhyphen{}system\sphinxhyphen{}theorem}
\item\relax\sphinxstyleindexentry{indeterminate\sphinxhyphen{}system\sphinxhyphen{}theorem}\sphinxstyleindexextra{\_pages/2.6\_Consistent\_systems}\sphinxstyleindexpageref{_pages/2.6_Consistent_systems:\detokenize{indeterminate-system-theorem}}
\indexspace
\bigletter{intersecting\sphinxhyphen{}planes\sphinxhyphen{}example}
\item\relax\sphinxstyleindexentry{intersecting\sphinxhyphen{}planes\sphinxhyphen{}example}\sphinxstyleindexextra{\_pages/4.2\_Planes}\sphinxstyleindexpageref{_pages/4.2_Planes:\detokenize{intersecting-planes-example}}
\indexspace
\bigletter{inverse\sphinxhyphen{}matrix\sphinxhyphen{}definition}
\item\relax\sphinxstyleindexentry{inverse\sphinxhyphen{}matrix\sphinxhyphen{}definition}\sphinxstyleindexextra{\_pages/1.5\_Inverse\_matrix}\sphinxstyleindexpageref{_pages/1.5_Inverse_matrix:\detokenize{inverse-matrix-definition}}
\indexspace
\bigletter{inverse\sphinxhyphen{}matrix\sphinxhyphen{}properties\sphinxhyphen{}theorem}
\item\relax\sphinxstyleindexentry{inverse\sphinxhyphen{}matrix\sphinxhyphen{}properties\sphinxhyphen{}theorem}\sphinxstyleindexextra{\_pages/1.5\_Inverse\_matrix}\sphinxstyleindexpageref{_pages/1.5_Inverse_matrix:\detokenize{inverse-matrix-properties-theorem}}
\indexspace
\bigletter{inverse\sphinxhyphen{}transformation\sphinxhyphen{}definition}
\item\relax\sphinxstyleindexentry{inverse\sphinxhyphen{}transformation\sphinxhyphen{}definition}\sphinxstyleindexextra{\_pages/6.1\_Transformation\_matrices}\sphinxstyleindexpageref{_pages/6.1_Transformation_matrices:\detokenize{inverse-transformation-definition}}
\indexspace
\bigletter{inverse\sphinxhyphen{}transformation\sphinxhyphen{}example}
\item\relax\sphinxstyleindexentry{inverse\sphinxhyphen{}transformation\sphinxhyphen{}example}\sphinxstyleindexextra{\_pages/6.1\_Transformation\_matrices}\sphinxstyleindexpageref{_pages/6.1_Transformation_matrices:\detokenize{inverse-transformation-example}}
\indexspace
\bigletter{line\sphinxhyphen{}line\sphinxhyphen{}distance\sphinxhyphen{}example}
\item\relax\sphinxstyleindexentry{line\sphinxhyphen{}line\sphinxhyphen{}distance\sphinxhyphen{}example}\sphinxstyleindexextra{\_pages/4.3\_Shortest\_distance\_problems}\sphinxstyleindexpageref{_pages/4.3_Shortest_distance_problems:\detokenize{line-line-distance-example}}
\indexspace
\bigletter{line\sphinxhyphen{}line\sphinxhyphen{}distance\sphinxhyphen{}theorem}
\item\relax\sphinxstyleindexentry{line\sphinxhyphen{}line\sphinxhyphen{}distance\sphinxhyphen{}theorem}\sphinxstyleindexextra{\_pages/4.3\_Shortest\_distance\_problems}\sphinxstyleindexpageref{_pages/4.3_Shortest_distance_problems:\detokenize{line-line-distance-theorem}}
\indexspace
\bigletter{line\sphinxhyphen{}line\sphinxhyphen{}intersection\sphinxhyphen{}definition}
\item\relax\sphinxstyleindexentry{line\sphinxhyphen{}line\sphinxhyphen{}intersection\sphinxhyphen{}definition}\sphinxstyleindexextra{\_pages/4.1\_Lines}\sphinxstyleindexpageref{_pages/4.1_Lines:\detokenize{line-line-intersection-definition}}
\indexspace
\bigletter{line\sphinxhyphen{}line\sphinxhyphen{}intersection\sphinxhyphen{}example}
\item\relax\sphinxstyleindexentry{line\sphinxhyphen{}line\sphinxhyphen{}intersection\sphinxhyphen{}example}\sphinxstyleindexextra{\_pages/4.1\_Lines}\sphinxstyleindexpageref{_pages/4.1_Lines:\detokenize{line-line-intersection-example}}
\indexspace
\bigletter{line\sphinxhyphen{}vector\sphinxhyphen{}equation\sphinxhyphen{}example}
\item\relax\sphinxstyleindexentry{line\sphinxhyphen{}vector\sphinxhyphen{}equation\sphinxhyphen{}example}\sphinxstyleindexextra{\_pages/4.1\_Lines}\sphinxstyleindexpageref{_pages/4.1_Lines:\detokenize{line-vector-equation-example}}
\indexspace
\bigletter{linear\sphinxhyphen{}combination\sphinxhyphen{}of\sphinxhyphen{}vectors\sphinxhyphen{}definition}
\item\relax\sphinxstyleindexentry{linear\sphinxhyphen{}combination\sphinxhyphen{}of\sphinxhyphen{}vectors\sphinxhyphen{}definition}\sphinxstyleindexextra{\_pages/3.4\_Linear\_combinations}\sphinxstyleindexpageref{_pages/3.4_Linear_combinations:\detokenize{linear-combination-of-vectors-definition}}
\indexspace
\bigletter{linear\sphinxhyphen{}combination\sphinxhyphen{}of\sphinxhyphen{}vectors\sphinxhyphen{}example}
\item\relax\sphinxstyleindexentry{linear\sphinxhyphen{}combination\sphinxhyphen{}of\sphinxhyphen{}vectors\sphinxhyphen{}example}\sphinxstyleindexextra{\_pages/3.4\_Linear\_combinations}\sphinxstyleindexpageref{_pages/3.4_Linear_combinations:\detokenize{linear-combination-of-vectors-example}}
\indexspace
\bigletter{linear\sphinxhyphen{}dependence\sphinxhyphen{}definition}
\item\relax\sphinxstyleindexentry{linear\sphinxhyphen{}dependence\sphinxhyphen{}definition}\sphinxstyleindexextra{\_pages/5.3\_Linear\_dependence}\sphinxstyleindexpageref{_pages/5.3_Linear_dependence:\detokenize{linear-dependence-definition}}
\indexspace
\bigletter{linear\sphinxhyphen{}dependence\sphinxhyphen{}equation}
\item\relax\sphinxstyleindexentry{linear\sphinxhyphen{}dependence\sphinxhyphen{}equation}\sphinxstyleindexextra{\_pages/5.3\_Linear\_dependence}\sphinxstyleindexpageref{_pages/5.3_Linear_dependence:\detokenize{linear-dependence-equation}}
\indexspace
\bigletter{linear\sphinxhyphen{}transformation\sphinxhyphen{}condition \sphinxhyphen{}definition}
\item\relax\sphinxstyleindexentry{linear\sphinxhyphen{}transformation\sphinxhyphen{}condition \sphinxhyphen{}definition}\sphinxstyleindexextra{\_pages/6.0\_Linear\_transformations}\sphinxstyleindexpageref{_pages/6.0_Linear_transformations:\detokenize{linear-transformation-condition -definition}}
\indexspace
\bigletter{linear\sphinxhyphen{}transformation\sphinxhyphen{}definition}
\item\relax\sphinxstyleindexentry{linear\sphinxhyphen{}transformation\sphinxhyphen{}definition}\sphinxstyleindexextra{\_pages/6.0\_Linear\_transformations}\sphinxstyleindexpageref{_pages/6.0_Linear_transformations:\detokenize{linear-transformation-definition}}
\indexspace
\bigletter{linear\sphinxhyphen{}transformation\sphinxhyphen{}example}
\item\relax\sphinxstyleindexentry{linear\sphinxhyphen{}transformation\sphinxhyphen{}example}\sphinxstyleindexextra{\_pages/6.0\_Linear\_transformations}\sphinxstyleindexpageref{_pages/6.0_Linear_transformations:\detokenize{linear-transformation-example}}
\indexspace
\bigletter{magnitude\sphinxhyphen{}definition}
\item\relax\sphinxstyleindexentry{magnitude\sphinxhyphen{}definition}\sphinxstyleindexextra{\_pages/3.2\_Vector\_magnitude}\sphinxstyleindexpageref{_pages/3.2_Vector_magnitude:\detokenize{magnitude-definition}}
\indexspace
\bigletter{magnitude\sphinxhyphen{}example}
\item\relax\sphinxstyleindexentry{magnitude\sphinxhyphen{}example}\sphinxstyleindexextra{\_pages/3.2\_Vector\_magnitude}\sphinxstyleindexpageref{_pages/3.2_Vector_magnitude:\detokenize{magnitude-example}}
\indexspace
\bigletter{matrix\sphinxhyphen{}addition\sphinxhyphen{}definition}
\item\relax\sphinxstyleindexentry{matrix\sphinxhyphen{}addition\sphinxhyphen{}definition}\sphinxstyleindexextra{\_pages/1.1\_Matrix\_operations}\sphinxstyleindexpageref{_pages/1.1_Matrix_operations:\detokenize{matrix-addition-definition}}
\indexspace
\bigletter{matrix\sphinxhyphen{}addition\sphinxhyphen{}example}
\item\relax\sphinxstyleindexentry{matrix\sphinxhyphen{}addition\sphinxhyphen{}example}\sphinxstyleindexextra{\_pages/1.1\_Matrix\_operations}\sphinxstyleindexpageref{_pages/1.1_Matrix_operations:\detokenize{matrix-addition-example}}
\indexspace
\bigletter{matrix\sphinxhyphen{}algebra\sphinxhyphen{}example}
\item\relax\sphinxstyleindexentry{matrix\sphinxhyphen{}algebra\sphinxhyphen{}example}\sphinxstyleindexextra{\_pages/1.6\_Matrix\_algebra}\sphinxstyleindexpageref{_pages/1.6_Matrix_algebra:\detokenize{matrix-algebra-example}}
\indexspace
\bigletter{matrix\sphinxhyphen{}dimension\sphinxhyphen{}definition}
\item\relax\sphinxstyleindexentry{matrix\sphinxhyphen{}dimension\sphinxhyphen{}definition}\sphinxstyleindexextra{\_pages/1.0\_Matrices}\sphinxstyleindexpageref{_pages/1.0_Matrices:\detokenize{matrix-dimension-definition}}
\indexspace
\bigletter{matrix\sphinxhyphen{}equality\sphinxhyphen{}definition}
\item\relax\sphinxstyleindexentry{matrix\sphinxhyphen{}equality\sphinxhyphen{}definition}\sphinxstyleindexextra{\_pages/1.1\_Matrix\_operations}\sphinxstyleindexpageref{_pages/1.1_Matrix_operations:\detokenize{matrix-equality-definition}}
\indexspace
\bigletter{matrix\sphinxhyphen{}exponents\sphinxhyphen{}definition}
\item\relax\sphinxstyleindexentry{matrix\sphinxhyphen{}exponents\sphinxhyphen{}definition}\sphinxstyleindexextra{\_pages/1.2\_Matrix\_multiplication}\sphinxstyleindexpageref{_pages/1.2_Matrix_multiplication:\detokenize{matrix-exponents-definition}}
\indexspace
\bigletter{matrix\sphinxhyphen{}exponents\sphinxhyphen{}example}
\item\relax\sphinxstyleindexentry{matrix\sphinxhyphen{}exponents\sphinxhyphen{}example}\sphinxstyleindexextra{\_pages/1.2\_Matrix\_multiplication}\sphinxstyleindexpageref{_pages/1.2_Matrix_multiplication:\detokenize{matrix-exponents-example}}
\indexspace
\bigletter{matrix\sphinxhyphen{}inverse\sphinxhyphen{}example\sphinxhyphen{}2}
\item\relax\sphinxstyleindexentry{matrix\sphinxhyphen{}inverse\sphinxhyphen{}example\sphinxhyphen{}2}\sphinxstyleindexextra{\_pages/1.5\_Inverse\_matrix}\sphinxstyleindexpageref{_pages/1.5_Inverse_matrix:\detokenize{matrix-inverse-example-2}}
\indexspace
\bigletter{matrix\sphinxhyphen{}multiplication\sphinxhyphen{}definition}
\item\relax\sphinxstyleindexentry{matrix\sphinxhyphen{}multiplication\sphinxhyphen{}definition}\sphinxstyleindexextra{\_pages/1.2\_Matrix\_multiplication}\sphinxstyleindexpageref{_pages/1.2_Matrix_multiplication:\detokenize{matrix-multiplication-definition}}
\indexspace
\bigletter{matrix\sphinxhyphen{}multiplication\sphinxhyphen{}example}
\item\relax\sphinxstyleindexentry{matrix\sphinxhyphen{}multiplication\sphinxhyphen{}example}\sphinxstyleindexextra{\_pages/1.2\_Matrix\_multiplication}\sphinxstyleindexpageref{_pages/1.2_Matrix_multiplication:\detokenize{matrix-multiplication-example}}
\indexspace
\bigletter{matrix\sphinxhyphen{}transpose\sphinxhyphen{}definition}
\item\relax\sphinxstyleindexentry{matrix\sphinxhyphen{}transpose\sphinxhyphen{}definition}\sphinxstyleindexextra{\_pages/1.1\_Matrix\_operations}\sphinxstyleindexpageref{_pages/1.1_Matrix_operations:\detokenize{matrix-transpose-definition}}
\indexspace
\bigletter{matrix\sphinxhyphen{}transpose\sphinxhyphen{}example}
\item\relax\sphinxstyleindexentry{matrix\sphinxhyphen{}transpose\sphinxhyphen{}example}\sphinxstyleindexextra{\_pages/1.1\_Matrix\_operations}\sphinxstyleindexpageref{_pages/1.1_Matrix_operations:\detokenize{matrix-transpose-example}}
\indexspace
\bigletter{minor\sphinxhyphen{}definition}
\item\relax\sphinxstyleindexentry{minor\sphinxhyphen{}definition}\sphinxstyleindexextra{\_pages/1.4\_Determinants}\sphinxstyleindexpageref{_pages/1.4_Determinants:\detokenize{minor-definition}}
\indexspace
\bigletter{minor\sphinxhyphen{}example}
\item\relax\sphinxstyleindexentry{minor\sphinxhyphen{}example}\sphinxstyleindexextra{\_pages/1.4\_Determinants}\sphinxstyleindexpageref{_pages/1.4_Determinants:\detokenize{minor-example}}
\indexspace
\bigletter{non\sphinxhyphen{}subspace\sphinxhyphen{}example}
\item\relax\sphinxstyleindexentry{non\sphinxhyphen{}subspace\sphinxhyphen{}example}\sphinxstyleindexextra{\_pages/5.2\_Subspaces}\sphinxstyleindexpageref{_pages/5.2_Subspaces:\detokenize{non-subspace-example}}
\indexspace
\bigletter{non\sphinxhyphen{}vector\sphinxhyphen{}space\sphinxhyphen{}example}
\item\relax\sphinxstyleindexentry{non\sphinxhyphen{}vector\sphinxhyphen{}space\sphinxhyphen{}example}\sphinxstyleindexextra{\_pages/5.1\_Vector\_spaces\_definitions}\sphinxstyleindexpageref{_pages/5.1_Vector_spaces_definitions:\detokenize{non-vector-space-example}}
\indexspace
\bigletter{normal\sphinxhyphen{}vector\sphinxhyphen{}definition}
\item\relax\sphinxstyleindexentry{normal\sphinxhyphen{}vector\sphinxhyphen{}definition}\sphinxstyleindexextra{\_pages/4.2\_Planes}\sphinxstyleindexpageref{_pages/4.2_Planes:\detokenize{normal-vector-definition}}
\indexspace
\bigletter{normalising\sphinxhyphen{}a\sphinxhyphen{}vector\sphinxhyphen{}example}
\item\relax\sphinxstyleindexentry{normalising\sphinxhyphen{}a\sphinxhyphen{}vector\sphinxhyphen{}example}\sphinxstyleindexextra{\_pages/3.2\_Vector\_magnitude}\sphinxstyleindexpageref{_pages/3.2_Vector_magnitude:\detokenize{normalising-a-vector-example}}
\indexspace
\bigletter{normalising\sphinxhyphen{}a\sphinxhyphen{}vector\sphinxhyphen{}proposition}
\item\relax\sphinxstyleindexentry{normalising\sphinxhyphen{}a\sphinxhyphen{}vector\sphinxhyphen{}proposition}\sphinxstyleindexextra{\_pages/3.2\_Vector\_magnitude}\sphinxstyleindexpageref{_pages/3.2_Vector_magnitude:\detokenize{normalising-a-vector-proposition}}
\indexspace
\bigletter{nxn\sphinxhyphen{}determinant\sphinxhyphen{}definition}
\item\relax\sphinxstyleindexentry{nxn\sphinxhyphen{}determinant\sphinxhyphen{}definition}\sphinxstyleindexextra{\_pages/1.4\_Determinants}\sphinxstyleindexpageref{_pages/1.4_Determinants:\detokenize{nxn-determinant-definition}}
\indexspace
\bigletter{nxn\sphinxhyphen{}determinant\sphinxhyphen{}example}
\item\relax\sphinxstyleindexentry{nxn\sphinxhyphen{}determinant\sphinxhyphen{}example}\sphinxstyleindexextra{\_pages/1.4\_Determinants}\sphinxstyleindexpageref{_pages/1.4_Determinants:\detokenize{nxn-determinant-example}}
\indexspace
\bigletter{orthogonal\sphinxhyphen{}basis\sphinxhyphen{}definition}
\item\relax\sphinxstyleindexentry{orthogonal\sphinxhyphen{}basis\sphinxhyphen{}definition}\sphinxstyleindexextra{\_pages/5.4\_Basis}\sphinxstyleindexpageref{_pages/5.4_Basis:\detokenize{orthogonal-basis-definition}}
\indexspace
\bigletter{orthonormal\sphinxhyphen{}basis\sphinxhyphen{}definition}
\item\relax\sphinxstyleindexentry{orthonormal\sphinxhyphen{}basis\sphinxhyphen{}definition}\sphinxstyleindexextra{\_pages/5.4\_Basis}\sphinxstyleindexpageref{_pages/5.4_Basis:\detokenize{orthonormal-basis-definition}}
\indexspace
\bigletter{parallel\sphinxhyphen{}lines\sphinxhyphen{}definition}
\item\relax\sphinxstyleindexentry{parallel\sphinxhyphen{}lines\sphinxhyphen{}definition}\sphinxstyleindexextra{\_pages/4.1\_Lines}\sphinxstyleindexpageref{_pages/4.1_Lines:\detokenize{parallel-lines-definition}}
\indexspace
\bigletter{parallel\sphinxhyphen{}lines\sphinxhyphen{}example}
\item\relax\sphinxstyleindexentry{parallel\sphinxhyphen{}lines\sphinxhyphen{}example}\sphinxstyleindexextra{\_pages/4.1\_Lines}\sphinxstyleindexpageref{_pages/4.1_Lines:\detokenize{parallel-lines-example}}
\indexspace
\bigletter{partial\sphinxhyphen{}pivoting\sphinxhyphen{}example}
\item\relax\sphinxstyleindexentry{partial\sphinxhyphen{}pivoting\sphinxhyphen{}example}\sphinxstyleindexextra{\_pages/2.4\_Partial\_pivoting}\sphinxstyleindexpageref{_pages/2.4_Partial_pivoting:\detokenize{partial-pivoting-example}}
\indexspace
\bigletter{perpendicular\sphinxhyphen{}lines\sphinxhyphen{}definition}
\item\relax\sphinxstyleindexentry{perpendicular\sphinxhyphen{}lines\sphinxhyphen{}definition}\sphinxstyleindexextra{\_pages/4.1\_Lines}\sphinxstyleindexpageref{_pages/4.1_Lines:\detokenize{perpendicular-lines-definition}}
\indexspace
\bigletter{perpendicular\sphinxhyphen{}lines\sphinxhyphen{}example}
\item\relax\sphinxstyleindexentry{perpendicular\sphinxhyphen{}lines\sphinxhyphen{}example}\sphinxstyleindexextra{\_pages/4.1\_Lines}\sphinxstyleindexpageref{_pages/4.1_Lines:\detokenize{perpendicular-lines-example}}
\indexspace
\bigletter{plane\sphinxhyphen{}vector\sphinxhyphen{}equation}
\item\relax\sphinxstyleindexentry{plane\sphinxhyphen{}vector\sphinxhyphen{}equation}\sphinxstyleindexextra{\_pages/4.2\_Planes}\sphinxstyleindexpageref{_pages/4.2_Planes:\detokenize{plane-vector-equation}}
\indexspace
\bigletter{point\sphinxhyphen{}line\sphinxhyphen{}distance\sphinxhyphen{}example}
\item\relax\sphinxstyleindexentry{point\sphinxhyphen{}line\sphinxhyphen{}distance\sphinxhyphen{}example}\sphinxstyleindexextra{\_pages/4.3\_Shortest\_distance\_problems}\sphinxstyleindexpageref{_pages/4.3_Shortest_distance_problems:\detokenize{point-line-distance-example}}
\indexspace
\bigletter{point\sphinxhyphen{}line\sphinxhyphen{}distance\sphinxhyphen{}theorem}
\item\relax\sphinxstyleindexentry{point\sphinxhyphen{}line\sphinxhyphen{}distance\sphinxhyphen{}theorem}\sphinxstyleindexextra{\_pages/4.3\_Shortest\_distance\_problems}\sphinxstyleindexpageref{_pages/4.3_Shortest_distance_problems:\detokenize{point-line-distance-theorem}}
\indexspace
\bigletter{point\sphinxhyphen{}normal\sphinxhyphen{}definition}
\item\relax\sphinxstyleindexentry{point\sphinxhyphen{}normal\sphinxhyphen{}definition}\sphinxstyleindexextra{\_pages/4.2\_Planes}\sphinxstyleindexpageref{_pages/4.2_Planes:\detokenize{point-normal-definition}}
\indexspace
\bigletter{point\sphinxhyphen{}normal\sphinxhyphen{}example}
\item\relax\sphinxstyleindexentry{point\sphinxhyphen{}normal\sphinxhyphen{}example}\sphinxstyleindexextra{\_pages/4.2\_Planes}\sphinxstyleindexpageref{_pages/4.2_Planes:\detokenize{point-normal-example}}
\indexspace
\bigletter{properties\sphinxhyphen{}of\sphinxhyphen{}determinants\sphinxhyphen{}theorem}
\item\relax\sphinxstyleindexentry{properties\sphinxhyphen{}of\sphinxhyphen{}determinants\sphinxhyphen{}theorem}\sphinxstyleindexextra{\_pages/1.4\_Determinants}\sphinxstyleindexpageref{_pages/1.4_Determinants:\detokenize{properties-of-determinants-theorem}}
\indexspace
\bigletter{properties\sphinxhyphen{}of\sphinxhyphen{}matrix\sphinxhyphen{}addition\sphinxhyphen{}theorem}
\item\relax\sphinxstyleindexentry{properties\sphinxhyphen{}of\sphinxhyphen{}matrix\sphinxhyphen{}addition\sphinxhyphen{}theorem}\sphinxstyleindexextra{\_pages/1.1\_Matrix\_operations}\sphinxstyleindexpageref{_pages/1.1_Matrix_operations:\detokenize{properties-of-matrix-addition-theorem}}
\indexspace
\bigletter{properties\sphinxhyphen{}of\sphinxhyphen{}matrix\sphinxhyphen{}transpose\sphinxhyphen{}theorem}
\item\relax\sphinxstyleindexentry{properties\sphinxhyphen{}of\sphinxhyphen{}matrix\sphinxhyphen{}transpose\sphinxhyphen{}theorem}\sphinxstyleindexextra{\_pages/1.1\_Matrix\_operations}\sphinxstyleindexpageref{_pages/1.1_Matrix_operations:\detokenize{properties-of-matrix-transpose-theorem}}
\indexspace
\bigletter{properties\sphinxhyphen{}of\sphinxhyphen{}scalar\sphinxhyphen{}multiplication\sphinxhyphen{}of\sphinxhyphen{}vectors}
\item\relax\sphinxstyleindexentry{properties\sphinxhyphen{}of\sphinxhyphen{}scalar\sphinxhyphen{}multiplication\sphinxhyphen{}of\sphinxhyphen{}vectors}\sphinxstyleindexextra{\_pages/3.1\_Vector\_arithmetic}\sphinxstyleindexpageref{_pages/3.1_Vector_arithmetic:\detokenize{properties-of-scalar-multiplication-of-vectors}}
\indexspace
\bigletter{properties\sphinxhyphen{}of\sphinxhyphen{}scalar\sphinxhyphen{}multiplication\sphinxhyphen{}theorem}
\item\relax\sphinxstyleindexentry{properties\sphinxhyphen{}of\sphinxhyphen{}scalar\sphinxhyphen{}multiplication\sphinxhyphen{}theorem}\sphinxstyleindexextra{\_pages/1.1\_Matrix\_operations}\sphinxstyleindexpageref{_pages/1.1_Matrix_operations:\detokenize{properties-of-scalar-multiplication-theorem}}
\indexspace
\bigletter{properties\sphinxhyphen{}of\sphinxhyphen{}vector\sphinxhyphen{}spaces\sphinxhyphen{}theorem}
\item\relax\sphinxstyleindexentry{properties\sphinxhyphen{}of\sphinxhyphen{}vector\sphinxhyphen{}spaces\sphinxhyphen{}theorem}\sphinxstyleindexextra{\_pages/5.1\_Vector\_spaces\_definitions}\sphinxstyleindexpageref{_pages/5.1_Vector_spaces_definitions:\detokenize{properties-of-vector-spaces-theorem}}
\indexspace
\bigletter{rank\sphinxhyphen{}definition}
\item\relax\sphinxstyleindexentry{rank\sphinxhyphen{}definition}\sphinxstyleindexextra{\_pages/2.6\_Consistent\_systems}\sphinxstyleindexpageref{_pages/2.6_Consistent_systems:\detokenize{rank-definition}}
\indexspace
\bigletter{rank\sphinxhyphen{}example}
\item\relax\sphinxstyleindexentry{rank\sphinxhyphen{}example}\sphinxstyleindexextra{\_pages/2.6\_Consistent\_systems}\sphinxstyleindexpageref{_pages/2.6_Consistent_systems:\detokenize{rank-example}}
\indexspace
\bigletter{real\sphinxhyphen{}numbers\sphinxhyphen{}vector\sphinxhyphen{}space\sphinxhyphen{}example}
\item\relax\sphinxstyleindexentry{real\sphinxhyphen{}numbers\sphinxhyphen{}vector\sphinxhyphen{}space\sphinxhyphen{}example}\sphinxstyleindexextra{\_pages/5.1\_Vector\_spaces\_definitions}\sphinxstyleindexpageref{_pages/5.1_Vector_spaces_definitions:\detokenize{real-numbers-vector-space-example}}
\indexspace
\bigletter{ref\sphinxhyphen{}definition}
\item\relax\sphinxstyleindexentry{ref\sphinxhyphen{}definition}\sphinxstyleindexextra{\_pages/2.3\_Gaussian\_elimination}\sphinxstyleindexpageref{_pages/2.3_Gaussian_elimination:\detokenize{ref-definition}}
\indexspace
\bigletter{reflection\sphinxhyphen{}definition}
\item\relax\sphinxstyleindexentry{reflection\sphinxhyphen{}definition}\sphinxstyleindexextra{\_pages/6.3\_Rotation\_reflection\_and\_translation}\sphinxstyleindexpageref{_pages/6.3_Rotation_reflection_and_translation:\detokenize{reflection-definition}}
\indexspace
\bigletter{reflection\sphinxhyphen{}example}
\item\relax\sphinxstyleindexentry{reflection\sphinxhyphen{}example}\sphinxstyleindexextra{\_pages/6.3\_Rotation\_reflection\_and\_translation}\sphinxstyleindexpageref{_pages/6.3_Rotation_reflection_and_translation:\detokenize{reflection-example}}
\indexspace
\bigletter{reflection\sphinxhyphen{}theorem}
\item\relax\sphinxstyleindexentry{reflection\sphinxhyphen{}theorem}\sphinxstyleindexextra{\_pages/6.3\_Rotation\_reflection\_and\_translation}\sphinxstyleindexpageref{_pages/6.3_Rotation_reflection_and_translation:\detokenize{reflection-theorem}}
\indexspace
\bigletter{rotation\sphinxhyphen{}definition}
\item\relax\sphinxstyleindexentry{rotation\sphinxhyphen{}definition}\sphinxstyleindexextra{\_pages/6.3\_Rotation\_reflection\_and\_translation}\sphinxstyleindexpageref{_pages/6.3_Rotation_reflection_and_translation:\detokenize{rotation-definition}}
\indexspace
\bigletter{rotation\sphinxhyphen{}example}
\item\relax\sphinxstyleindexentry{rotation\sphinxhyphen{}example}\sphinxstyleindexextra{\_pages/6.3\_Rotation\_reflection\_and\_translation}\sphinxstyleindexpageref{_pages/6.3_Rotation_reflection_and_translation:\detokenize{rotation-example}}
\indexspace
\bigletter{rotation\sphinxhyphen{}in\sphinxhyphen{}R2\sphinxhyphen{}theorem}
\item\relax\sphinxstyleindexentry{rotation\sphinxhyphen{}in\sphinxhyphen{}R2\sphinxhyphen{}theorem}\sphinxstyleindexextra{\_pages/6.3\_Rotation\_reflection\_and\_translation}\sphinxstyleindexpageref{_pages/6.3_Rotation_reflection_and_translation:\detokenize{rotation-in-R2-theorem}}
\indexspace
\bigletter{rotation\sphinxhyphen{}scaling\sphinxhyphen{}and\sphinxhyphen{}translating\sphinxhyphen{}example}
\item\relax\sphinxstyleindexentry{rotation\sphinxhyphen{}scaling\sphinxhyphen{}and\sphinxhyphen{}translating\sphinxhyphen{}example}\sphinxstyleindexextra{\_pages/6.4\_Translation}\sphinxstyleindexpageref{_pages/6.4_Translation:\detokenize{rotation-scaling-and-translating-example}}
\indexspace
\bigletter{rref\sphinxhyphen{}definition}
\item\relax\sphinxstyleindexentry{rref\sphinxhyphen{}definition}\sphinxstyleindexextra{\_pages/2.5\_Gauss\_Jordan\_elimination}\sphinxstyleindexpageref{_pages/2.5_Gauss_Jordan_elimination:\detokenize{rref-definition}}
\indexspace
\bigletter{scalar\sphinxhyphen{}multiplication\sphinxhyphen{}definition}
\item\relax\sphinxstyleindexentry{scalar\sphinxhyphen{}multiplication\sphinxhyphen{}definition}\sphinxstyleindexextra{\_pages/1.1\_Matrix\_operations}\sphinxstyleindexpageref{_pages/1.1_Matrix_operations:\detokenize{scalar-multiplication-definition}}
\indexspace
\bigletter{scalar\sphinxhyphen{}multiplication\sphinxhyphen{}example}
\item\relax\sphinxstyleindexentry{scalar\sphinxhyphen{}multiplication\sphinxhyphen{}example}\sphinxstyleindexextra{\_pages/1.1\_Matrix\_operations}\sphinxstyleindexpageref{_pages/1.1_Matrix_operations:\detokenize{scalar-multiplication-example}}
\indexspace
\bigletter{scalar\sphinxhyphen{}multiplication\sphinxhyphen{}of\sphinxhyphen{}a\sphinxhyphen{}vector\sphinxhyphen{}definition}
\item\relax\sphinxstyleindexentry{scalar\sphinxhyphen{}multiplication\sphinxhyphen{}of\sphinxhyphen{}a\sphinxhyphen{}vector\sphinxhyphen{}definition}\sphinxstyleindexextra{\_pages/3.1\_Vector\_arithmetic}\sphinxstyleindexpageref{_pages/3.1_Vector_arithmetic:\detokenize{scalar-multiplication-of-a-vector-definition}}
\indexspace
\bigletter{scaling\sphinxhyphen{}definition}
\item\relax\sphinxstyleindexentry{scaling\sphinxhyphen{}definition}\sphinxstyleindexextra{\_pages/6.3\_Rotation\_reflection\_and\_translation}\sphinxstyleindexpageref{_pages/6.3_Rotation_reflection_and_translation:\detokenize{scaling-definition}}
\indexspace
\bigletter{scaling\sphinxhyphen{}example}
\item\relax\sphinxstyleindexentry{scaling\sphinxhyphen{}example}\sphinxstyleindexextra{\_pages/6.3\_Rotation\_reflection\_and\_translation}\sphinxstyleindexpageref{_pages/6.3_Rotation_reflection_and_translation:\detokenize{scaling-example}}
\indexspace
\bigletter{scaling\sphinxhyphen{}theorem}
\item\relax\sphinxstyleindexentry{scaling\sphinxhyphen{}theorem}\sphinxstyleindexextra{\_pages/6.3\_Rotation\_reflection\_and\_translation}\sphinxstyleindexpageref{_pages/6.3_Rotation_reflection_and_translation:\detokenize{scaling-theorem}}
\indexspace
\bigletter{skew\sphinxhyphen{}lines\sphinxhyphen{}definition}
\item\relax\sphinxstyleindexentry{skew\sphinxhyphen{}lines\sphinxhyphen{}definition}\sphinxstyleindexextra{\_pages/4.1\_Lines}\sphinxstyleindexpageref{_pages/4.1_Lines:\detokenize{skew-lines-definition}}
\indexspace
\bigletter{skew\sphinxhyphen{}lines\sphinxhyphen{}example}
\item\relax\sphinxstyleindexentry{skew\sphinxhyphen{}lines\sphinxhyphen{}example}\sphinxstyleindexextra{\_pages/4.1\_Lines}\sphinxstyleindexpageref{_pages/4.1_Lines:\detokenize{skew-lines-example}}
\indexspace
\bigletter{solution\sphinxhyphen{}by\sphinxhyphen{}inverse\sphinxhyphen{}example}
\item\relax\sphinxstyleindexentry{solution\sphinxhyphen{}by\sphinxhyphen{}inverse\sphinxhyphen{}example}\sphinxstyleindexextra{\_pages/2.1\_Solving\_using\_inverse\_matrix}\sphinxstyleindexpageref{_pages/2.1_Solving_using_inverse_matrix:\detokenize{solution-by-inverse-example}}
\indexspace
\bigletter{solution\sphinxhyphen{}using\sphinxhyphen{}inverse\sphinxhyphen{}matrix\sphinxhyphen{}theorem}
\item\relax\sphinxstyleindexentry{solution\sphinxhyphen{}using\sphinxhyphen{}inverse\sphinxhyphen{}matrix\sphinxhyphen{}theorem}\sphinxstyleindexextra{\_pages/2.1\_Solving\_using\_inverse\_matrix}\sphinxstyleindexpageref{_pages/2.1_Solving_using_inverse_matrix:\detokenize{solution-using-inverse-matrix-theorem}}
\indexspace
\bigletter{spanning\sphinxhyphen{}set\sphinxhyphen{}definition}
\item\relax\sphinxstyleindexentry{spanning\sphinxhyphen{}set\sphinxhyphen{}definition}\sphinxstyleindexextra{\_pages/5.4\_Basis}\sphinxstyleindexpageref{_pages/5.4_Basis:\detokenize{spanning-set-definition}}
\indexspace
\bigletter{subspace\sphinxhyphen{}condition\sphinxhyphen{}theorem}
\item\relax\sphinxstyleindexentry{subspace\sphinxhyphen{}condition\sphinxhyphen{}theorem}\sphinxstyleindexextra{\_pages/5.2\_Subspaces}\sphinxstyleindexpageref{_pages/5.2_Subspaces:\detokenize{subspace-condition-theorem}}
\indexspace
\bigletter{subspace\sphinxhyphen{}definition}
\item\relax\sphinxstyleindexentry{subspace\sphinxhyphen{}definition}\sphinxstyleindexextra{\_pages/5.2\_Subspaces}\sphinxstyleindexpageref{_pages/5.2_Subspaces:\detokenize{subspace-definition}}
\indexspace
\bigletter{subspace\sphinxhyphen{}example}
\item\relax\sphinxstyleindexentry{subspace\sphinxhyphen{}example}\sphinxstyleindexextra{\_pages/5.2\_Subspaces}\sphinxstyleindexpageref{_pages/5.2_Subspaces:\detokenize{subspace-example}}
\indexspace
\bigletter{system\sphinxhyphen{}of\sphinxhyphen{}linear\sphinxhyphen{}equation\sphinxhyphen{}definition}
\item\relax\sphinxstyleindexentry{system\sphinxhyphen{}of\sphinxhyphen{}linear\sphinxhyphen{}equation\sphinxhyphen{}definition}\sphinxstyleindexextra{\_pages/2.0\_Linear\_systems}\sphinxstyleindexpageref{_pages/2.0_Linear_systems:\detokenize{system-of-linear-equation-definition}}
\indexspace
\bigletter{theorem\sphinxhyphen{}0}
\item\relax\sphinxstyleindexentry{theorem\sphinxhyphen{}0}\sphinxstyleindexextra{\_pages/1.3\_Special\_matrices}\sphinxstyleindexpageref{_pages/1.3_Special_matrices:\detokenize{theorem-0}}
\indexspace
\bigletter{theorem\sphinxhyphen{}1}
\item\relax\sphinxstyleindexentry{theorem\sphinxhyphen{}1}\sphinxstyleindexextra{\_pages/0.3\_Mathematical\_preliminaries}\sphinxstyleindexpageref{_pages/0.3_Mathematical_preliminaries:\detokenize{theorem-1}}
\indexspace
\bigletter{theorem\sphinxhyphen{}2}
\item\relax\sphinxstyleindexentry{theorem\sphinxhyphen{}2}\sphinxstyleindexextra{\_pages/1.2\_Matrix\_multiplication}\sphinxstyleindexpageref{_pages/1.2_Matrix_multiplication:\detokenize{theorem-2}}
\indexspace
\bigletter{theorem\sphinxhyphen{}5}
\item\relax\sphinxstyleindexentry{theorem\sphinxhyphen{}5}\sphinxstyleindexextra{\_pages/2.5\_Gauss\_Jordan\_elimination}\sphinxstyleindexpageref{_pages/2.5_Gauss_Jordan_elimination:\detokenize{theorem-5}}
\indexspace
\bigletter{transformation\sphinxhyphen{}matrix\sphinxhyphen{}definition}
\item\relax\sphinxstyleindexentry{transformation\sphinxhyphen{}matrix\sphinxhyphen{}definition}\sphinxstyleindexextra{\_pages/6.1\_Transformation\_matrices}\sphinxstyleindexpageref{_pages/6.1_Transformation_matrices:\detokenize{transformation-matrix-definition}}
\indexspace
\bigletter{transformation\sphinxhyphen{}matrix\sphinxhyphen{}example}
\item\relax\sphinxstyleindexentry{transformation\sphinxhyphen{}matrix\sphinxhyphen{}example}\sphinxstyleindexextra{\_pages/6.1\_Transformation\_matrices}\sphinxstyleindexpageref{_pages/6.1_Transformation_matrices:\detokenize{transformation-matrix-example}}
\indexspace
\bigletter{transformation\sphinxhyphen{}matrix\sphinxhyphen{}example\sphinxhyphen{}2}
\item\relax\sphinxstyleindexentry{transformation\sphinxhyphen{}matrix\sphinxhyphen{}example\sphinxhyphen{}2}\sphinxstyleindexextra{\_pages/6.1\_Transformation\_matrices}\sphinxstyleindexpageref{_pages/6.1_Transformation_matrices:\detokenize{transformation-matrix-example-2}}
\indexspace
\bigletter{translation\sphinxhyphen{}definition}
\item\relax\sphinxstyleindexentry{translation\sphinxhyphen{}definition}\sphinxstyleindexextra{\_pages/6.4\_Translation}\sphinxstyleindexpageref{_pages/6.4_Translation:\detokenize{translation-definition}}
\indexspace
\bigletter{translation\sphinxhyphen{}example}
\item\relax\sphinxstyleindexentry{translation\sphinxhyphen{}example}\sphinxstyleindexextra{\_pages/6.4\_Translation}\sphinxstyleindexpageref{_pages/6.4_Translation:\detokenize{translation-example}}
\indexspace
\bigletter{translation\sphinxhyphen{}theorem}
\item\relax\sphinxstyleindexentry{translation\sphinxhyphen{}theorem}\sphinxstyleindexextra{\_pages/6.4\_Translation}\sphinxstyleindexpageref{_pages/6.4_Translation:\detokenize{translation-theorem}}
\indexspace
\bigletter{unit\sphinxhyphen{}vector\sphinxhyphen{}definition}
\item\relax\sphinxstyleindexentry{unit\sphinxhyphen{}vector\sphinxhyphen{}definition}\sphinxstyleindexextra{\_pages/3.2\_Vector\_magnitude}\sphinxstyleindexpageref{_pages/3.2_Vector_magnitude:\detokenize{unit-vector-definition}}
\indexspace
\bigletter{vector\sphinxhyphen{}addition\sphinxhyphen{}definition}
\item\relax\sphinxstyleindexentry{vector\sphinxhyphen{}addition\sphinxhyphen{}definition}\sphinxstyleindexextra{\_pages/3.1\_Vector\_arithmetic}\sphinxstyleindexpageref{_pages/3.1_Vector_arithmetic:\detokenize{vector-addition-definition}}
\indexspace
\bigletter{vector\sphinxhyphen{}equality\sphinxhyphen{}definition}
\item\relax\sphinxstyleindexentry{vector\sphinxhyphen{}equality\sphinxhyphen{}definition}\sphinxstyleindexextra{\_pages/3.1\_Vector\_arithmetic}\sphinxstyleindexpageref{_pages/3.1_Vector_arithmetic:\detokenize{vector-equality-definition}}
\indexspace
\bigletter{vector\sphinxhyphen{}equation\sphinxhyphen{}of\sphinxhyphen{}a\sphinxhyphen{}line\sphinxhyphen{}definition}
\item\relax\sphinxstyleindexentry{vector\sphinxhyphen{}equation\sphinxhyphen{}of\sphinxhyphen{}a\sphinxhyphen{}line\sphinxhyphen{}definition}\sphinxstyleindexextra{\_pages/4.1\_Lines}\sphinxstyleindexpageref{_pages/4.1_Lines:\detokenize{vector-equation-of-a-line-definition}}
\indexspace
\bigletter{vector\sphinxhyphen{}magnitude\sphinxhyphen{}properties}
\item\relax\sphinxstyleindexentry{vector\sphinxhyphen{}magnitude\sphinxhyphen{}properties}\sphinxstyleindexextra{\_pages/3.2\_Vector\_magnitude}\sphinxstyleindexpageref{_pages/3.2_Vector_magnitude:\detokenize{vector-magnitude-properties}}
\indexspace
\bigletter{vector\sphinxhyphen{}space\sphinxhyphen{}axioms}
\item\relax\sphinxstyleindexentry{vector\sphinxhyphen{}space\sphinxhyphen{}axioms}\sphinxstyleindexextra{\_pages/5.1\_Vector\_spaces\_definitions}\sphinxstyleindexpageref{_pages/5.1_Vector_spaces_definitions:\detokenize{vector-space-axioms}}
\indexspace
\bigletter{vector\sphinxhyphen{}space\sphinxhyphen{}definition}
\item\relax\sphinxstyleindexentry{vector\sphinxhyphen{}space\sphinxhyphen{}definition}\sphinxstyleindexextra{\_pages/5.1\_Vector\_spaces\_definitions}\sphinxstyleindexpageref{_pages/5.1_Vector_spaces_definitions:\detokenize{vector-space-definition}}
\indexspace
\bigletter{vector\sphinxhyphen{}space\sphinxhyphen{}dimension\sphinxhyphen{}definition}
\item\relax\sphinxstyleindexentry{vector\sphinxhyphen{}space\sphinxhyphen{}dimension\sphinxhyphen{}definition}\sphinxstyleindexextra{\_pages/5.4\_Basis}\sphinxstyleindexpageref{_pages/5.4_Basis:\detokenize{vector-space-dimension-definition}}
\end{sphinxtheindex}

\renewcommand{\indexname}{Index}
\printindex
\end{document}